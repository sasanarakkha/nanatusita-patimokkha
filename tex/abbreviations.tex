
\chapter{Abbreviations}
\label{abbreviations}

{\raggedright

For a full explanation of these abbreviations, please see Ñāṇatusita's \textit{Analysis of the Pātimokkha.}

\smallskip

Be - Burmese Edition (Burmese Chaṭṭha Saṅgāyana)\\
BhPm - Bhikkhupātimokkhaṁ, a 1875 Sinhla script edition based on Siyam manuscripts\\
C - National Museum Library Manuscript 6 E 9. No. 10 in De Silva’s Catalogue\\
Ce - Ceylon Edition (Buddha Jayanti Tipiṭaka Series)\\
D - The Pātimokkha, a text based on the manuscripts found at the Malwatta Vihāra\\
Dhp - Dhammapada
Dm - Dve Mātikāpāḷi\\
Ee - European edition (Pāli Text Society P.T.S.), Oxford\\
G - Gannoruwa Rājamahāvihāra Manuscript\\
Kkh - Kaṅkhāvitaraṇī\\
MJG - Mahā-jaya-maṅgala-gāthā (Sri Lanka)
MN - Majjhima Nikāya
Mi Se - Royal Thai edition\\
Mm Se - Mahā Makuṭ Academy Siamese edition\\
Ñd - Ñāṇadassana's Bhikkhu-Pātimokkhaṁ\\
Mv - Mahāvagga (Vinaya-piṭaka)\\
P - Perādeniya University Library manuscript\\
Pg - Bhikkhupātimokkhagaṇṭhidīpanī\\
Ra - Ratanārthasudanī-namvu-bhikṣubhikṣuṇī-prātimokṣa-varṇanāva\\
SK - Sri Kaḷyāṇa Yogāshrama Saṃsṭhāva\\
SN - Saṁyutta Nikāya
SVibh/Vibh - Suttavibhaṅga\\
Sanne - Bhikṣuprātimokṣa-padartha\\
TP - The Pātimokkha by K. R. Norman and W. Pruitt\\
UP - Ubhaya Pratimokṣaya\\
Um - Ubhayamātikāpāḷi\\
V - Siamese Braḥ Pāṭimokkhaṃ manuscript in the Viyayasundarārāmaya vihāra\\
W - Watärakapansala manuscript\\

}

