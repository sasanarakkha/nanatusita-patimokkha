
\chapter{Abbreviations}
\label{abbreviations}

%TODO - for ninebythirteen a table doesn't work, first try line breaking

\ifninebythirteenversion

\vspace{-0.6em}
\begin{tabular}{@{}llll@{}}
  Be & Burmese Chaṭṭha Saṅgāyana ed. (As on CSCD.)\\
  BhPm & Bhikkhupātimokkhaṁ.\\
  Ce & “Ceylon edition,” Buddha Jayanti Tipiṭaka Series; Colombo.\\
  Dm & Dve Mātikāpāḷi.\\
  Ee & European edition. Pāli Text Society (P.T.S.), Oxford.\\
  Kkh & Kaṅkhāvitaraṇī\\
  Mi Se & Royal Thai edition.\\
  Mm Se & Mahā Makuṭ Academy Siamese edition of Pātimokkha.\\
  Mv & Mahāvagga (Vinaya-piṭaka)\\
  Ñd & Ñāṇadassana. Bhikkhu-Pātimokkhaṁ.\\
  Pg & Bhikkhupātimokkhagaṇṭhidīpanī\\
  Sanne & Bhikṣuprātimokṣa-padartha\\
  SVibh & Suttavibhaṅga\\
  SN    & Saṁyutta Nikāya\\
  TP & The Pātimokkha; K. R. Norman and W. Pruitt.\\
  Vibh & Vibhaṅga\\
  UP & Ubhaya Pratimokṣaya.\\
  Um & Ubhayamātikāpāḷi.\\
\end{tabular}

\else

\begin{tabular}{@{}llll@{}}
  Be & Burmese Chaṭṭha Saṅgāyana ed. (As on CSCD.)\\
  BhPm & Bhikkhupātimokkhaṁ.\\
  Ce & “Ceylon edition,” Buddha Jayanti Tipiṭaka Series; Colombo.\\
  Dm & Dve Mātikāpāḷi.\\
  Ee & European edition. Pāli Text Society (P.T.S.), Oxford.\\
  Kkh & Kaṅkhāvitaraṇī\\
  Mi Se & Royal Thai edition.\\
  Mm Se & Mahā Makuṭ Academy Siamese edition of Pātimokkha.\\
  Ñd & Ñāṇadassana. Bhikkhu-Pātimokkhaṁ.\\
  Pg & Bhikkhupātimokkhagaṇṭhidīpanī\\
  Sanne & Bhikṣuprātimokṣa-padartha\\
  SN    & Saṁyutta Nikāya\\
  TP & The Pātimokkha; K. R. Norman and W. Pruitt.\\
  UP & Ubhaya Pratimokṣaya.\\
  Um & Ubhayamātikāpāḷi.\\
\end{tabular}

\medskip

\fi

D The Pātimokkha, being the Buddhist Office of the Confession of Priests, Dickson, J.F. Journal of the Royal Asiatic Society, New Series VIII pp. 62–130, 1876.

W is the Watärakapansala Pātimokkha manuscript. It is almost identical with the preceding Colombo Museum manuscript.

Ra: Pātimokkha edition as given in the Sri Lankan Vinaya-manual called the Ratanārthasudanī-namvu-bhikṣubhikṣuṇī-prātimokṣa-varṇanāva, edited by Soṇuttara Jinaratana thera and Ratgama Pragnāśekhara, Colombo, 1946.

SK Bhikkhupātimokkhapāḷi, Sri Kaḷyāṇa Yogāshrama Saṃsṭhāva, 1981.

V is the abbreviation used for the Siamese Braḥ Pāṭimokkhaṃ manuscript in the Viyayasundarārāmaya vihāra in Asgiriya, Kandy.

P is the abbreviation used for the Perādeniya University Library (Perādeniya) manuscript, i.e., the second Pātimokkha MS in Ms 277637 called Vinayagaṇṭhi which contains various texts (Mūlasikkha,Kammavācā, etc.) in one bundle.

G is the abbreviation used for the Pātimokkha manuscript found in a plain hardwood cover in the small collection of the Gannoruwa Rājamahāvihāra at Ganoruwa near Perādeniya.

C is the abbreviation used for the Bhikkhu Pātimokkha Pāḷi manuscript. National Museum Library Manuscript 6 E 9. No. 10 in De Silva’s Catalogue of the National Museum Library. National
Museum, Sir Marcus Fernando Mawatha, Colombo 7.

