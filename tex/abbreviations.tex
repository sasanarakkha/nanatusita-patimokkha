
\chapter{Abbreviations}
\label{abbreviations}

{\raggedright

Be - Burmese Edition (Burmese Chaṭṭha Saṅgāyana)\\
BhPm - Bhikkhupātimokkhaṁ from where?\\
Ce - Ceylon Edition (Buddha Jayanti Tipiṭaka Series)\\
Dm - Dve Mātikāpāḷi from where?\\
Ee - European edition (Pāli Text Society P.T.S.), Oxford\\
Kkh - Kaṅkhāvitaraṇī\\
Mi Se - Royal Thai edition\\
Mm Se - Mahā Makuṭ Academy Siamese Edition\\
Mv - Mahāvagga (Vinaya-piṭaka)\\
Ñd - Ñāṇadassana's Bhikkhu-Pātimokkhaṁ\\
Pg - Bhikkhupātimokkhagaṇṭhidīpanī\\
Sanne - Bhikṣuprātimokṣa-padartha\\
SVibha - Suttavibhaṅga\\
TP - The Pātimokkha by K. R. Norman and W. Pruitt\\
Vibh - Vibhaṅga\\
UP - Ubhaya Pratimokṣaya\\
Um - Ubhayamātikāpāḷi\\
D - The Pātimokkha, being the Buddhist Office of the Confession of Priests, Dickson, J.F. Journal of the Royal Asiatic Society, New Series VIII pp. 62–130, 1876.\\
W - Watärakapansala Pātimokkha Manuscript\\
Ra - Ratanārthasudanī-namvu-bhikṣubhikṣuṇī-prātimokṣa-varṇanāva\\
SK - Sri Kaḷyāṇa Yogāshrama Saṃsṭhāva\\
V - Siamese Braḥ Pāṭimokkhaṃ manuscript in the Viyayasundarārāmaya vihāra\\
P - Perādeniya University Library Manuscript\\
G - Gannoruwa Rājamahāvihāra Manuscript\\
C - National Museum Library Manuscript 6 E 9. No. 10 in De Silva’s Catalogue\\

\smallskip

For an explanation for the aforementioned abbreviations, please see Ñāṇatusita's full Analysis of the Pātimokkha.
}

