
\section{Tāyana-gāthā}
\label{tayana-gatha}

\begin{intro}
	\anglebracketleft\ \hspace{-0.5mm}Handa mayaṁ tāyana-gāthāyo bhaṇāmase \hspace{-0.5mm}\anglebracketright\
\end{intro}

%% \medskip

\begin{hanging-indent}
Chinda sotaṁ parakkamma\\
Kāme panūda brāhmaṇa
\end{hanging-indent}

\begin{hanging-indent}
Nappahāya muni kāme\\
N'ekattam'upapajjati
\end{hanging-indent}

\begin{hanging-indent}
Kayirā ce kayirāth'enaṁ\\
Daḷham'enaṁ parakkame
\end{hanging-indent}

\begin{hanging-indent}
Sithilo hi paribbājo\\
Bhiyyo ākirate rajaṁ
\end{hanging-indent}

\begin{hanging-indent}
Akataṁ dukkaṭaṁ seyyo\\
Pacchā tappati dukkaṭaṁ
\end{hanging-indent}

\begin{hanging-indent}
Katañ'ca sukataṁ seyyo\\
Yaṁ katvā n'ānutappati
\end{hanging-indent}

\begin{hanging-indent}
Kuso yathā duggahito\\
Hattham'ev'ānukantati
\end{hanging-indent}

\ifninebythirteenversion \clearpage \fi

\begin{hanging-indent}
Sāmaññaṁ dupparāmaṭṭhaṁ\\
Nirayāy'ūpakaḍḍhati
\end{hanging-indent}

\begin{hanging-indent}
Yaṅ'kiñci sithilaṁ kammaṁ\\
Saṅkiliṭṭhañ'ca yaṁ vataṁ
\end{hanging-indent}

\begin{hanging-indent}
Saṅkassaraṁ brahmacariyaṁ\\
Na taṁ hoti, mahapphalan'ti
\end{hanging-indent}

\suttaRef{[SN 2.8]}

\clearpage

