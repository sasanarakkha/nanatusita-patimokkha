
\section{Saṅkhittapātimokkh'uddesa-niṭṭhāna}
\label{sankhittapatimokkh'uddesa-nitthana}

\ifafiveversion\vspace{0.15cm}\fi
\ifninebythirteenversion\vspace{0.15cm}\fi

Uddiṭṭhaṁ kho āyasmanto nidānaṁ, uddiṭṭhā cattāro pārājikā dhammā, uddiṭṭhā terasa saṅghādisesā dhammā, uddiṭṭhā dve aniyatā dhammā. Sutā kho pan'āyasmantehi terasa saṅghādisesā dhammā, dve aniyatā dhammā, tiṁsa nissaggiyā pācittiyā dhammā, dvenavuti pācittiyā dhammā, cattāro pāṭidesanīyā dhammā, pañcasattati sekhiyā dhammā, satta adhikaraṇasamathā dhammā, ettakaṁ tassa bhagavato sutt'āgataṁ suttapariyāpannaṁ, anvaḍḍhamāsaṁ uddesaṁ āgacchati, tattha sabbeh'eva samaggehi sammodamānehi avivadamānehi sikkhitabban'ti.

Vitthār'uddeso pañcamo.

In the introduction and conclusion to each of the sections the number of rules recited are mentioned. The only exception is the Sekhiya section, which in all editions of the Pātimokkha and the Suttavibhaṅga does not give the number of rules in the section, i.e., there is no pañcasattati (“seventy-five”) before sekhiyā dhammā (“training rules”). The only exceptions are Mi Se, a Siamese manuscript (V), and the saṅkhittapātimokkhuddesa (the recitation of the Pātimokkha in brief) that is given in a few printed editions and manuscripts and is also found in the Parivāra. As this would be the only section introduction where the number of rules are omitted, it could seem that the Thai tra- dition has preserved an older tradition, however, the reason for the omission might be different. When comparing the Pali Pātimokkha and the Prātimokṣasūtras of other major early traditions, it is clear that the divergence between them—both in the number of rules and in rules not found in other Prātimokṣasūtras—occurs in this section. In other sec- tions the difference is only in the order of rules and in the wording. In contrast to the fixed content of the other sections of the Pātimokkha/ Prātimokṣasūta, the Sekhiya section would have been seen as an open- ended appendix section in which minor training rules related to ordi- nary etiquette could be added by different traditions and perhaps even major monastic centers. Nowadays too different monasteries have dif- ferent sets of monastery rules. Thai forest monasteries are known espe- cially for their refined rules regarding etiquette. In the Cullavagga, sekhiya rules are referred to as vatta or “observances”; see Sekhiya sec- tion introduction. The other major early traditions also did not number their śaikṣa sections: the Ma-L has sātirekapañcāśaccaikṣā (PrMoMā-L 30, 34 ; cf. Kar I I 80, II 77: sātirekapaṃcāśa śaikṣakā), “more than fifty” (it has 67 śaikṣa rules), while the Mū and Sa Pm have saṃbahulāḥ śaikṣā (PrMoSa 241, 255, Chandra 11), “many training rules” (Sa has 113 śaikṣa rules, Mū has 108).

\begin{outro}
	Bhikkhupātimokkhaṁ niṭṭhitaṁ
\end{outro}

\clearpage



\section{The Pātimokkha Recitation in Brief Conclusion}
\label{patimokkha-in-brief-conclusion}

\ifafiveversion\vspace{0.15cm}\fi
\ifninebythirteenversion\vspace{0.15cm}\fi

Venerables, the introduction has been recited, the four cases involving disqualification have been recited, the thirteen cases [involving] the community in the beginning and in the rest have been recited, the two indefinite cases have been recited. Heard by the venerables have been the thirty cases involving expiation with forfeiture, the ninety-two cases involving expiation, the four cases that are to be acknowledged, the cases related to the training, the seven cases that are settlements of legal issues. This much [training-rule] of the Fortunate One has been handed down in the Sutta, has been included in the Sutta, comes up for recitation half-monthly. By all who are united, who are on friendly terms, who are not disputing, is to be trained herein.

\begin{outro}
	The Bhikkhu Disciplinary Code \ifninebythirteenversion\\\fi has been finished
\end{outro}

\clearpage

