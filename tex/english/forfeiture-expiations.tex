
\section{Forfeiture and Expiations}
\label{forf-exp}

\begin{intro}
	Venerables, these thirty cases involving expiation with forfeiture come up for recitation.
\end{intro}

\setsubsecheadstyle{\subsectionFmt}
\subsection{The Section on Robes}
\label{forf-exp-robes}
% \vspace{0.2cm}

\pdfbookmark[3]{Forfeiture and Expiation 1}{forf-exp1}
\subsubsection*{\hyperref[np1]{Forfeiture and Expiation 1: The training precept on the kaṭhina}}
\label{forf-exp1}

When the robe [-cloth] has been finished by a bhikkhu, when the kaṭhina [-frame-privileges] have been withdrawn, [then] extra robe [-cloth] is to be kept for ten days at the most. For one who lets it pass beyond [the ten days], [this is a case] involving expiation with forfeiture.



\pdfbookmark[3]{Forfeiture and Expiation 2}{forf-exp2}
\subsubsection*{\hyperref[np2]{Forfeiture and Expiation 2: The training precept on the storehouse}}
\label{forf-exp2}

When the robe [-cloth] has been finished by a bhikkhu, when the kaṭhina [-frame-privileges] have been withdrawn, if even for a single night a bhikkhu should stay apart from the three robes, except with the authorization of bhikkhus, [this is a case] involving expiation with forfeiture.



\pdfbookmark[3]{Forfeiture and Expiation 3}{forf-exp3}
\subsubsection*{\hyperref[np3]{Forfeiture and Expiation 3: The training precept on the out-of-season (robe)-cloth}}
\label{forf-exp3}

When the robe [-cloth] has been finished by a bhikkhu, when the kaṭhina [-frame-privileges] have been withdrawn, if out-of-season robe [cloth] should become available to a bhikkhu, by a bhikkhu who is wishing [so, it] can be accepted; having accepted [it, it] is to be made very quickly. If [the robe-cloth] should not be [enough for] the completion [of the robe], [then] for a month at the most that robe [-cloth] can be put aside by that bhikkhu for the completion of the deficiency [of robe-cloth], when there is an expectation [that he will get more robe-cloth]; if he should put [it] aside more than that, even when there is an expectation [that he will get more robe-cloth], [this is a case] involving expiation with forfeiture.



\pdfbookmark[3]{Forfeiture and Expiation 4}{forf-exp4}
\subsubsection*{\hyperref[np4]{Forfeiture and Expiation 4: The training precept on the used (robe)-cloth}}
\label{forf-exp4}

If any bhikkhu should have a used robe [-cloth] washed, dyed, or beaten by an unrelated bhikkhunī, [this is a case] involving expiation with forfeiture.



\pdfbookmark[3]{Forfeiture and Expiation 5}{forf-exp5}
\subsubsection*{\hyperref[np5]{Forfeiture and Expiation 5: The training precept on the acceptance of robe(-cloth)}}
\label{forf-exp5}

If any bhikkhu should accept a robe [-cloth] from the hand of an unrelated bhikkhunī, except in an exchange [of robes], [this is a case] involving expiation with forfeiture.



\pdfbookmark[3]{Forfeiture and Expiation 6}{forf-exp6}
\subsubsection*{\hyperref[np6]{Forfeiture and Expiation 6: The training precept on making a suggestion to someone who is not related}}
\label{forf-exp6}

If any bhikkhu should request a robe [-cloth] to an unrelated male householder or female householder, except at the [right] occasion, [this is a case] involving expiation with forfeiture. Here the occasion is this: he is a bhikkhu whose robe has been robbed or whose robe has been lost; this is the occasion here.



\pdfbookmark[3]{Forfeiture and Expiation 7}{forf-exp7}
\subsubsection*{\hyperref[np7]{Forfeiture and Expiation 7: The training precept on (accepting) more than that}}
\label{forf-exp7}

If the unrelated male householder or female householder should invite him to take [as many] robe [-cloth]s [as he likes], [then] robe [-cloths for] an upper [robe] together with an inner [robe] can be accepted at the most from that robe [-cloth] by that bhikkhu; if he should accept more from that [robe-cloth], [this is a case] involving expiation with forfeiture.



\pdfbookmark[3]{Forfeiture and Expiation 8}{forf-exp8}
\subsubsection*{\hyperref[np8]{Forfeiture and Expiation 8: The first training precept on setting up (a robe-fund)}}
\label{forf-exp8}

Now, if a robe-fund has been set up for a bhikkhu by an unrelated male householder or female householder [thinking]: ``Having traded this robe-fund for a robe, I shall clothe the bhikkhu named so and so with a robe,'' and then if that bhikkhu, previously uninvited, having approached [the householder], should make a suggestion about the robe [cloth] [saying]: ``It would be good indeed, Sir, [if you] having traded this robe-fund for a such and such a robe, were to clothe me [with a robe],'' [if the suggestion is made] out of a liking for what is fine, [this is a case] involving expiation with forfeiture.



\pdfbookmark[3]{Forfeiture and Expiation 9}{forf-exp9}
\subsubsection*{\hyperref[np9]{Forfeiture and Expiation 9: The second training precept on setting up (a robe-fund)}}
\label{forf-exp9}

Now, if separate robe-funds have been set up for a bhikkhu by both unrelated male householders or female householders [thinking]: ``Having traded these separate robe-funds for separate robes, we shall clothe the bhikkhu named so and so with robes,'' and then if that bhikkhu, previously uninvited, having approached [the householders], should make a suggestion about the robe [saying]: ``It would be good indeed, Sirs, [if you] having traded these separate robe-funds for a such and such a robe, were to clothe me [with a robe], [you] both being one [donor],'' [if the suggestion is made] out of a liking for what is fine, [this is a case] involving expiation with forfeiture.



\pdfbookmark[3]{Forfeiture and Expiation 10}{forf-exp10}
\subsubsection*{\hyperref[np10]{Forfeiture and Expiation 10: The training precept on the king}}
\label{forf-exp10}

Now, if a king or a kings' official or a brahmin or a male householder should convey by messenger a robe-fund for a bhikkhu [saying]: ``Having traded this robe-fund for a robe, clothe the bhikkhu named so and so with a robe,'' and if that messenger, having approached that bhikkhu, should say so: ``Venerable Sir, this robe-fund has been brought for the venerable one. Let the venerable one accept the robe-fund!'' [then] that messenger should be spoken to thus by that bhikkhu: ``Sir, we do not accept a robe-fund, but we do accept a robe at the right time [when it is] allowable.'' If that messenger should say thus to that bhikkhu: ``Is there, perhaps, someone who is the steward of the venerable one?'' [then,] bhikkhus, by a bhikkhu who is in need of a robe a steward can be appointed: a monastery attendant or a male lay-follower [saying]: ``Sir, this is the bhikkhus' steward.'' If that messenger having instructed that steward, having approached that bhikkhu, should say so: ``Venerable Sir, the steward whom the venerable one has appointed has been instructed by me. Let the venerable one approach [him] at the right time [and] he will clothe you with a robe,'' [then] bhikkhus, having approached the steward, [the steward] can be prompted [and] can be reminded two or three times by the bhikkhu who is in need of a robe [saying]: ``Sir, I am in need of a robe.'' [If through] prompting [and] reminding [him] two or three times, he should have [him] bring forth that robe, it is good. If he should not have [him] bring [it] forth, [then] four times, five times, six times at the most, [it] can be stood [for] by [a bhikkhu] who has become silent. [If through] standing silently for [it] four times, five times, six times at the most, he should have [him] bring forth that robe, it is good; if [through] making effort more than that, he should have [him] produce that robe, [this is a case] involving expiation with forfeiture. If he should not have [him] produce [it], [then] from wherever [that] the robe-fund may have been brought, there [he] himself can go, or a messenger can be sent [saying]: ``Sirs, that robe-fund which you conveyed for the bhikkhu does not fulfil any need of that bhikkhu. Let the sirs endeavour for [what is their] own. Let not [what is their] own get lost.'' This is the proper procedure here.

\begin{center}
	The section [starting with the rule] on robes is finished
\end{center}



\setsubsecheadstyle{\subsectionFmt}
\subsection{The Section on Sheepwool}
% \vspace{0.2cm}

\pdfbookmark[3]{Forfeiture and Expiation 11}{forf-exp11}
\subsubsection*{\hyperref[np11]{Forfeiture and Expiation 11: The training precept on silk}}
\label{forf-exp11}

If any bhikkhu should have a rug mixed with silk made, [this is a case] involving expiation with forfeiture.



\pdfbookmark[3]{Forfeiture and Expiation 12}{forf-exp12}
\subsubsection*{\hyperref[np12]{Forfeiture and Expiation 12: The training precept on pure black wool}}
\label{forf-exp12}

If any bhikkhu should have a rug made of pure black sheep's wool; [this is a case] involving expiation with forfeiture.



\pdfbookmark[3]{Forfeiture and Expiation 13}{forf-exp13}
\subsubsection*{\hyperref[np13]{Forfeiture and Expiation 13: The training precept on [using] two parts}}
\label{forf-exp13}

By a bhikkhu who is having a new rug made, two parts of pure black sheep-wool are to be taken, [and] a third [part] of white, a fourth [part] of ruddy brown. If a bhikkhu should have a rug made, without having taken two parts of pure black sheep's hair, [and] a third [part] of white, a fourth [part] of ruddy brown, [this is a case] involving expiation with forfeiture.



\pdfbookmark[3]{Forfeiture and Expiation 14}{forf-exp14}
\subsubsection*{\hyperref[np14]{Forfeiture and Expiation 14: The training precept on (keeping a rug for) six years}}
\label{forf-exp14}

By a bhikkhu who has had a new rug made, it is to be kept for six years [at least]. If within less than six years, having given up or not having given up that rug, he should have another new rug made, except with the authorisation of bhikkhus, [this is a case] involving expiation with forfeiture.



\pdfbookmark[3]{Forfeiture and Expiation 15}{forf-exp15}
\subsubsection*{\hyperref[np15]{Forfeiture and Expiation 15: The training precept on the sitting cloth}}
\label{forf-exp15}

By a bhikkhu who is having a sitting-rug made, a sugata-span from the border of an old rug is to be taken for making [it] stained. If a bhikkhu, without having taken a sugata-span from the border of an old rug, should have a new sitting cloth made, [this is a case] involving expiation with forfeiture.



\pdfbookmark[3]{Forfeiture and Expiation 16}{forf-exp16}
\subsubsection*{\hyperref[np16]{Forfeiture and Expiation 16: The training precept on sheep wool}}
\label{forf-exp16}

Now, if sheep-wool should become available to a bhikkhu who is travelling on a main road, by a bhikkhu who is wishing [so, it] can be accepted, having accepted [it, it] can be carried with his own hand for three yojanas at the most when there is no one present who can carry it; if he should carry it more than that, even when there is no one present who can carry it, [this is a case] involving expiation with forfeiture.



\pdfbookmark[3]{Forfeiture and Expiation 17}{forf-exp17}
\subsubsection*{\hyperref[np17]{Forfeiture and Expiation 17: The training precept on having sheep-wool washed}}
\label{forf-exp17}

If any bhikkhu should have sheep-wool washed, dyed, or carded by an unrelated bhikkhunī, [this is a case] involving expiation with forfeiture.



\pdfbookmark[3]{Forfeiture and Expiation 18}{forf-exp18}
\subsubsection*{\hyperref[np18]{Forfeiture and Expiation 18: The training precept on silver}}
\label{forf-exp18}

If any bhikkhu should take gold and silver, or should have [it] taken, or should consent to [it] being deposited [for him], [this is a case] involving expiation with forfeiture.



\pdfbookmark[3]{Forfeiture and Expiation 19}{forf-exp19}
\subsubsection*{\hyperref[np19]{Forfeiture and Expiation 19: The training precept on trading in money}}
\label{forf-exp19}

If any bhikkhu should engage in the various kinds of trading in money, [this is a case] involving expiation with forfeiture.



\pdfbookmark[3]{Forfeiture and Expiation 20}{forf-exp20}
\subsubsection*{\hyperref[np20]{Forfeiture and Expiation 20: The training precept on bartering}}
\label{forf-exp20}

If any bhikkhu should engage in the various kinds of bartering, [this is a case] involving expiation with forfeiture.

The section on sheepwool is second.



\setsubsecheadstyle{\subsectionFmt}
\subsection{The Section on Bowls}
% \vspace{0.2cm}

\pdfbookmark[3]{Forfeiture and Expiation 21}{forf-exp21}
\subsubsection*{\hyperref[np21]{Forfeiture and Expiation 21: The training precept on bowls}}
\label{forf-exp21}

An extra bowl can be kept for ten days at the most. For one who lets it pass beyond [the ten days]; [this is a case] involving expiation with forfeiture.



\pdfbookmark[3]{Forfeiture and Expiation 22}{forf-exp22}
\subsubsection*{\hyperref[np22]{Forfeiture and Expiation 22: The training precept on [a bowl with] less than five mends}}
\label{forf-exp22}

If any bhikkhu should exchange a bowl with less than five mends for another new bowl, [this is a case] involving expiation with forfeiture. That bowl is to be relinquished by that bhikkhu to the assembly of bhikkhus, and whichever [bowl] is the last bowl of that assembly of bhikkhus, that [bowl] is to be bestowed on that bhikkhu [thus]: ``Bhikkhu, this bowl is for you, it is to be kept until breaking.'' This is the proper procedure here.



\pdfbookmark[3]{Forfeiture and Expiation 23}{forf-exp23}
\subsubsection*{\hyperref[np23]{Forfeiture and Expiation 23: The training precept on medicine}}
\label{forf-exp23}

Now, [there are] those medicines which are permissable for sick bhikkhus, namely: ghee, butter, oil, [and] honey and molasses— having been accepted, they can be partaken of [while] being kept in store for seven days at the most. For one who lets it pass beyond [the seven days], [this is a case] involving expiation with forfeiture.



\pdfbookmark[3]{Forfeiture and Expiation 24}{forf-exp24}
\subsubsection*{\hyperref[np24]{Forfeiture and Expiation 24: The training precept on the rain's bathing-cloth}}
\label{forf-exp24}

[Thinking:] ``One month is what remains of the hot season,'' [then] the robe-cloth for the rain's bathing cloth can be sought by a bhikkhu. [Thinking:] ``A half month is what remains of the hot season,'' [after] having made [it, it] can be worn. If earlier than [what is reckoned as] ``One month is what remains of the hot season,'' he should seek robe-cloth for the rain's bathing-cloth, [and] [if] earlier than [what is reckoned as] ``A half month is what remains of the hot season,'' he should wear [it], [this is a case] involving expiation with forfeiture.



\pdfbookmark[3]{Forfeiture and Expiation 25}{forf-exp25}
\subsubsection*{\hyperref[np25]{Forfeiture and Expiation 25: The training precept on snatching robes}}
\label{forf-exp25}

If any bhikkhu, having himself given a robe to a bhikkhu, should, being resentful [and] displeased, snatch [it] away or should have it snatched away [from the bhikkhu], [this is a case] involving expiation with forfeiture.



\pdfbookmark[3]{Forfeiture and Expiation 26}{forf-exp26}
\subsubsection*{\hyperref[np26]{Forfeiture and Expiation 26: The training precept on requesting thread}}
\label{forf-exp26}

If any bhikkhu, having himself requested the thread [to be used], should have a robe-cloth woven by cloth-weavers, [this is a case] involving expiation with forfeiture.



\pdfbookmark[3]{Forfeiture and Expiation 27}{forf-exp27}
\subsubsection*{\hyperref[np27]{Forfeiture and Expiation 27: The greater training precept about weavers}}
\label{forf-exp27}

Now, if an unrelated male householder or female householder should have a robe-cloth woven for a bhikkhu by cloth-weavers, and then if that bhikkhu, uninvited beforehand, having approached the clothweavers, should make a suggestion about the robe-cloth [saying]: ``Friends, this robe-cloth which is being woven for me: make [it] long, wide, thick, well woven, well diffused, well scraped, and well plucked! Certainly we will also [then] present a little something to the sirs,'' and if that bhikkhu, having said so, should present a little something, even just a little alms-food, [this is a case] involving expiation with forfeiture.



\pdfbookmark[3]{Forfeiture and Expiation 28}{forf-exp28}
\subsubsection*{\hyperref[np28]{Forfeiture and Expiation 28: The training precept on extra-ordinary robes}}
\label{forf-exp28}

For the ten days coming up to the three-month Kattikā full moon: if extraordinary robe [-cloth] should become available to a bhikkhu, [then] after considering [it as] extraordinary [robe-cloth, it] can be accepted by a bhikkhu, having been accepted, [it] is to be put aside until the occasion of the robe-season; if he should put [it] aside for more than that, [this is a case] involving expiation with forfeiture.



\pdfbookmark[3]{Forfeiture and Expiation 29}{forf-exp29}
\subsubsection*{\hyperref[np29]{Forfeiture and Expiation 29: The training precept on risks}}
\label{forf-exp29}

Now, the Kattika-full-moon has been observed. [There are] those wilderness lodgings which are considered risky, which are dangerous. A bhikkhu dwelling in such kind of lodgings, who is wishing [to do so], may put aside one of the three robes inside an inhabited area. And if there may be any reason for that bhikkhu for dwelling apart from that robe, the bhikkhu can dwell apart from that robe for six days at the most; if he should dwell apart for more than that, except with the authorization of bhikkhus, [this is a case] involving expiation with forfeiture.



\pdfbookmark[3]{Forfeiture and Expiation 30}{forf-exp30}
\subsubsection*{\hyperref[np30]{Forfeiture and Expiation 30: The training precept on allocation}}
\label{forf-exp30}

If any bhikkhu should knowingly allocate for himself a gain belonging to [and] allocated to the community, [this is a case] involving expiation with forfeiture.

\begin{center}
	The section [starting with the rule] on bowls is third
\end{center}



\medskip

\begin{center}
	Venerables, the thirty cases involving expiation with forfeiture have been recited.

	\smallskip

	Concerning that I ask the Venerables: [Are you] pure in this?\\
	A second time again I ask: [Are you] pure in this?\\
	A third time again I ask: [Are you] pure in this?

	\smallskip

	The Venerables are pure in this, therefore there is silence, thus I keep this [in mind].
\end{center}

\begin{outro}
	The cases involving expiation with forfeiture are finished
\end{outro}

\clearpage

