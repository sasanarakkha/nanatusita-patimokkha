\section{The Recitation of the Introduction}
\label{introduction}

Venerable Sir, let the Community listen to me! Today is a fifteenth / fourteenth / unity [day] Observance. If it is suitable to the Community, [then] the Community should do the Observance [and] should recite the Disciplinary Code.

What is the preliminary for the Community?

Venerables, announce the purity, [for] I shall recite the Disciplinary Code. Let us all [who are] present listen to it carefully [and] let us take it to mind. Whoever may have an offence, he should disclose [it]. When there is no offence, [then it] is to be silent. By the silence I shall know the Venerables [with the thought]: ``[They are] pure.''

As an answer occurs to [a bhikkhu] who is asked individually, just so in such an assembly [as this one] there is the announcement up to the third time. But if any bhikkhu, [who is] remembering [an offence] when the announcement is being made up to the third time, should not disclose the existing offence, there is [a further offence of] deliberate false speech for him. But, venerables, deliberate false speech has been called an obstructive act by the Fortunate One. Therefore, by a bhikkhu who is remembering, who has committed [an offence], who is desiring purification, an existing offence is to be disclosed; because, [after] having disclosed [it], there is comfort for him.

\medskip

\begin{center}
Venerables, the introduction has been recited.

\smallskip

Concerning that I ask the Venerables: [Are you] pure in this?\\
A second time again I ask: [Are you] pure in this?\\
A third time again I ask: [Are you] pure in this?

\smallskip

The venerables are pure in this, therefore there is silence, so do I bear this [in mind].
\end{center}

\begin{outro}
  The recitation of the introduction is finished
\end{outro}

\clearpage
