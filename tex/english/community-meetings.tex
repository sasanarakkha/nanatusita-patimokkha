\section{Cases Concering the Community}
\label{comm}

\begin{intro}
  Venerables, these thirteen cases [concerning] the community in the beginning and in the rest [of the procedure] come up for recitation.
\end{intro}

\pdfbookmark[3]{Community Meeting 1}{comm1}
\subsection*{\hyperref[sd1]{Community Meeting 1: The training precept on emission of semen}}
\label{comm1}
The intentional emission of semen, except in a dream: [this is a case concerning] the community in the beginning and in the rest [of the procedure].

\pdfbookmark[3]{Community Meeting 2}{comm2}
\subsection*{\hyperref[sd2]{Community Meeting 2: The training precept on physical contact}}
\label{comm2}
If any bhikkhu, under the influence of an altered mind, should engage in [intimate] physical contact together with a woman [such as]: the holding of a hand, or holding a braid [of hair], or caressing any limb: [this is a case concerning] the community in the beginning and in the rest [of the procedure].

\pdfbookmark[3]{Community Meeting 3}{comm3}
\subsection*{\hyperref[sd3]{Community Meeting 3: The training precept on depraved words}}
\label{comm3}
If any bhikkhu, under the influence of an altered mind, should speak suggestively with depraved words to a woman, like a young man to a young woman, [with words] concerned with sexual intercourse: [this is a case concerning] the community in the beginning and in the rest [of the procedure].

\pdfbookmark[3]{Community Meeting 4}{comm4}
\subsection*{\hyperref[sd4]{Community Meeting 4: The training precept on (ministering) to himself with love}}
\label{comm4}
If any bhikkhu, under the influence of an altered mind, [and] in the presence of a woman, should speak praise about the ministering to himself with sex: ``Sister, this is the best of ministerings: she who would minister to a virtuous, good natured celibate like me with this act!,'' [which is something] connected with sexual intercourse: [this is a case concerning] the community in the beginning and in the rest [of the procedure].

\pdfbookmark[3]{Community Meeting 5}{comm5}
\subsection*{\hyperref[sd5]{Community Meeting 5: The training precept on mediating}}
\label{comm5}
If any bhikkhu should engage in mediating a man's intention to a woman, or a woman's intention to a man, for being a wife or for being a mistress, even for being one on [just] that occasion: [this is a case concerning] the community in the beginning and in the rest [of the procedure].

\pdfbookmark[3]{Community Meeting 6}{comm6}
\subsection*{\hyperref[sd6]{Community Meeting 6: The training precept on making a hut}}
\label{comm6}
By a bhikkhu who is having a hut, which is without an owner, [and] is designated for himself, built by means of his own begged requisites, [that hut] is to be built according to the [proper] measure. This is the measure here: twelve spans of the sugata-span in length, [and] inside seven [spans] across. Bhikkhus are to be brought to [it] for appointing the site. By those bhikkhus a site is to be appointed which is not entailing harm [to creatures and which is] having a surrounding space. If a bhikkhu, having requested it himself, should have a hut built on a site entailing harm [to creatures], [and] not having a surrounding space, or if he should not bring bhikkhus to [it] for appointing the site, or if he should let [it] exceed the measure: [this is a case concerning] the community in the beginning and in the rest [of the procedure].

\pdfbookmark[3]{Community Meeting 1}{comm7}
\subsection*{\hyperref[sd7]{Community Meeting 7: The training precept on making a dwelling}}
\label{comm7}
By a bhikkhu who is having a large dwelling built, which has an owner, [and] is designated for himself, bhikkhus are to be brought to [it] for appointing the site. By those bhikkhus a site not entailing harm [to any creatures] [and] having a surrounding space is to be appointed. If a bhikkhu should have a hut built on a site entailing harm [to creatures], [and] not having a surrounding space, or if he should not bring bhikkhus to [it] for appointing the site, [this is a case concerning] the community in the beginning and in the rest [of the procedure].

\pdfbookmark[3]{Community Meeting 8}{comm8}
\subsection*{\hyperref[sd8]{Community Meeting 8: The training precept on being corrupted by malice}}
\label{comm8}
If any bhikkhu, corrupted by malice [and] upset, should accuse a bhikkhu with a groundless case involving disqualification [thinking]: ``If only I could make him fall away from this holy life!,'' [and] then, on another occasion, [whether] being interrogated or not being interrogated, if that legal issue is really groundless, and if the bhikkhu stands firm in malice: [this is a case concerning] the community in the beginning and in the rest [of the procedure].

\pdfbookmark[3]{Community Meeting 9}{comm9}
\subsection*{\hyperref[sd9]{Community Meeting 9: The training precept on (an issue) belonging to another class}}
\label{comm9}
If any bhikkhu, corrupted by malice [and] upset, should accuse a bhikkhu with a case involving disqualification, having taken [it] up [with] some point, which is a mere pretext, of a legal issue belonging to another class [thinking]: ``If only I could make him fall away from this holy life!,'' [and] then, on another occasion, [whether] being interrogated or not being interrogated, if that legal issue is really belonging to another class, [and] some point, which a mere pretext, has been taken up, and if the bhikkhu stands firm in malice: [this is a case concerning] the community in the beginning and in the rest [of the procedure].

\pdfbookmark[3]{Community Meeting 10}{comm10}
\subsection*{\hyperref[sd10]{Community Meeting 10: The training precept on the schism of a community}}
\label{comm10}
If any bhikkhu should endeavor for the schism of a united community, or having undertaken, should persist in upholding a legal issue conducive to schism, [then] that bhikkhu should be spoken to thus by the bhikkhus: ``Let the venerable one not endeavor for the schism of the united community, or having undertaken, persist in upholding a legal issue conducive to schism. Let the venerable one convene with the community, for a united community, which is on friendly terms, which is not disputing, which has a single recitation, dwells in comfort,'' and [if] that bhikkhu being spoken to thus by the bhikkhus should persist in the same way [as before], [then] that bhikkhu is to be argued with up to three times by the bhikkhus for the relinquishing of that [course], [and if that bhikkhu,] being argued with up to three times, should relinquish that [course], then this is good, [but] if he should not relinquish [it]: [this is a case concerning] the community in the beginning and in the rest [of the procedure].

\pdfbookmark[3]{Community Meeting 11}{comm11}
\subsection*{\hyperref[sd11]{Community Meeting 11: The training precept on the followers of the schism}}
\label{comm11}
Now, there are bhikkhus who are followers of that same bhikkhu, [and] who are speaking for [his] faction: one, or two, or three, [and] they should say so: ``Venerables, don't say anything to this bhikkhu! This bhikkhu is one who speaks in accordance with the Teaching and this bhikkhu is one who speaks in accordance the Discipline; this [bhikkhu], having received [our] consent and favour defines [the Teaching \& Discipline]. Knowing us, he speaks, [and] this suits us too.'' [Then] those bhikkhus should be spoken to thus by the bhikkhus: ``Venerables, don't say so! This bhikkhu does not speak in accordance with the Teaching, and this bhikkhu does not speak in accordance with the Discipline! Don't let the venerables too favour the schism of the community. Let there be convening with the community for the venerables, for a united community, which is on friendly terms, which is not disputing, which has a single recitation, dwells in comfort,'' and [if] those bhikkhus being spoken to thus by the bhikkhus should persist in the same way [as before], [then] those bhikkhus are to be argued with up to three times by the bhikkhus for the relinquishing of that [course], [and if those bhikkhus] being argued with up to three times, should relinquish that [course], then this is good, [but] if they should not relinquish [it]: [this is a case concerning] the community in the beginning and in the rest [of the procedure].

\pdfbookmark[3]{Community Meeting 12}{comm12}
\subsection*{\hyperref[sd12]{Community Meeting 12: The training precept on being of a nature difficult to be spoken to}}
\label{comm12}
Now, a bhikkhu is of a nature difficult to be spoken to, [and when] being righteously spoken to by the bhikkhus about the training precepts included in the recitation, he makes himself [one] who can not be spoken to [saying]: ``Venerables, don't say anything good or bad to me, and I too shall not say anything good or bad to the venerables! Venerables, refrain from speaking to me!'' [Then] that bhikkhu should be spoken to thus by the bhikkhus: ``Let the venerable one one not make himself [one] who cannot be spoken to. Let the venerable one make himself [one] who can be spoken to. Let the venerable one speak to the bhikkhus with righteousness and the monks too will speak to the venerable one with righteousness. For the Blessed One's assembly has grown thus, that is, by the speaking of one to another, by the rehabilitating of one another,'' and [if] that bhikkhu being spoken to thus by the bhikkhus should persist in the same way [as before], [then] that bhikkhu is to be argued with up to three times by the bhikkhus for the relinquishing of that [course], [and if that bhikkhu,] being argued with up to three times, should relinquish that [course], then this is good, [but] if he should not relinquish [it]: [this is a case concerning] the community in the beginning and in the rest [of the procedure].

\pdfbookmark[3]{Community Meeting 13}{comm13}
\subsection*{\hyperref[sd13]{Community Meeting 13: The training precept on the spoiler of families}}
\label{comm13}
Now, a bhikkhu lives dependent upon a certain village or town who is a spoiler of families, who is of bad behaviour. His bad behaviour is seen and is heard about, and the families spoilt by him are seen and heard about. That bhikkhu is to be spoken to thus by the bhikkhus: ``The venerable one is a spoiler of families, one who is of bad behaviour. The bad behaviour of the venerable one is seen and is heard about, and the families spoilt by the venerable one are seen and are heard about. Let the venerable one depart from this dwelling-place! Enough of you dwelling here!'' and [if] that bhikkhu being spoken to thus by the bhikkhus should say thus to those bhikkhus: ``The bhikkhus are driven by desire; the bhikkhus are driven by anger; the bhikkhus are driven by delusion; the bhikkhus are driven by fear. They banish someone because of this kind of offence, [but] another one they do not banish.'' [Then] that bhikkhu is to be spoken to thus by the bhikkhus: ``Let the venerable one not speak thus! The bhikkhus are not driven by desire; and the bhikkhus are not driven by anger; and the bhikkhus are not driven by delusion; and the bhikkhus are not driven by fear. The venerable one is a spoiler of families, one who is of bad behaviour. The bad behaviour of the venerable one is seen and is heard about, and the families spoilt by the venerable one are seen and are heard about. Let the venerable one depart from this dwelling-place! Enough of you dwelling here!'' and [if] that bhikkhu being spoken to thus by the bhikkhus should persist in the same way [as before], [then] that bhikkhu is to be argued with up to three times by the bhikkhus for the relinquishing of that [course], [and if that bhikkhu,] being argued with up to three times, should relinquish that [course], then this is good, [but] if he should not relinquish [it]: [this is a case concerning] the community in the beginning and in the rest [of the procedure].

\medskip

\begin{center}
Venerables, the thirteen cases [concerning] the community in the beginning and in the rest [of the procedure] have been recited, nine [cases] are of the offence-at-once [-class], four [cases] are of the up-to the-third [time admonition-class]. A bhikkhu who has committed any one of [these offenses], has to stay on probation with no choice [in the matter] for as many days as he knowingly conceals [it]. Moreover, by a bhikkhu who has stayed on the probation, a six-night state of deference to [other] bhikkhus has to be entered upon. [When] the bhikkhu [is one by whom] the deference has been performed: wherever there may be a community of bhikkhus, which is a group of twenty [or more bhikkhus], there that bhikkhu should be reinstated. If a community of bhikkhus, which is a group of twenty deficient by even one [bhikkhu], should reinstate that bhikkhu [then] that bhikkhu is not reinstated, and those monks are blameworthy. This is the proper procedure here.

\smallskip

Concerning that I ask the Venerables: [Are you] pure in this?\\
A second time again I ask: [Are you] pure in this?\\
A third time again I ask: [Are you] pure in this?

\smallskip

The venerables are pure in this, therefore there is silence, so do I bear this [in mind].
\end{center}

\begin{outro}
  The recitation concerning the community in the beginning and the rest [of the procedure] is finished
\end{outro}

\clearpage
