
\section{Expiations}
\label{exp}

\begin{center}
	Venerables, these ninety-two cases involving expiation come up for recitation.
\end{center}

\setsubsecheadstyle{\subsectionFmt}
\subsection{The Section on False Speech}
% \vspace{0.2cm}

\pdfbookmark[3]{Expitation 1}{exp1}
\subsubsection*{\hyperref[pac1]{Expiation 1: The training precept on false speech}}
\label{exp1}

In deliberate false speech, [there is a case] involving expiation.



\pdfbookmark[3]{Expitation 2}{exp2}
\subsubsection*{\hyperref[pac2]{Expiation 2: The training precept on abusive speech}}
\label{exp2}

In abusive speech, [there is a case] involving expiation.



\pdfbookmark[3]{Expitation 3}{exp3}
\subsubsection*{\hyperref[pac3]{Expiation 3: The training precept on slandering}}
\label{exp3}

In the backbiting of a bhikkhu, [there is a case] involving expiation.



\pdfbookmark[3]{Expitation 4}{exp 4}
\subsubsection*{\hyperref[pac4]{Expiation 4: The training precept on teaching Dhamma line by line}}
\label{exp4}

If any bhikkhu should have one who has not been fully admitted [into the community] recite the Dhamma [line] by line, [this is a case] involving expiation.



\pdfbookmark[3]{Expitation 5}{exp 5}
\subsubsection*{\hyperref[pac5]{Expiation 5: The first training precept on (using a) sleeping place together with}}
\label{exp5}

If any bhikkhu should make use of a sleeping place for more than two nights or three nights together with one who has not been fully admitted [into the bhikkhu-community], [this is a case] involving expiation.



\pdfbookmark[3]{Expitation 6}{exp 6}
\subsubsection*{\hyperref[pac6]{Expiation 6: The second training precept on (using a) sleeping place together with}}
\label{exp6}

If any bhikkhu should make use of a sleeping place together with a woman, [this is a case] involving expiation.



\pdfbookmark[3]{Expitation 7}{exp 7}
\subsubsection*{\hyperref[pac7]{Expiation 7: The training precept on teaching Dhamma}}
\label{exp7}

If any bhikkhu should teach the Dhamma to a woman by [means of] more than five or six sentences, except [when being together] with a discerning male human being, [this is a case] involving expiation.



\pdfbookmark[3]{Expitation 8}{exp 8}
\subsubsection*{\hyperref[pac8]{Expiation 8: The training precept on factual announcing}}
\label{exp8}

If any bhikkhu should declare a superhuman state to one who has not been fully admitted [into the bhikkhu-community], [even] when it is a fact, [this is a case] involving expiation.



\pdfbookmark[3]{Expitation 9}{exp 9}
\subsubsection*{\hyperref[pac9]{Expiation 9: The training precept on the announcing of depraved (offences)}}
\label{exp9}

If any bhikkhu should declare the depraved offence of [another] bhikkhu to one who has not been fully admitted [into the bhikkhu-community], except with the authorisation of bhikkhus, [this is a case] involving expiation.



\pdfbookmark[3]{Expitation 10}{exp 10}
\subsubsection*{\hyperref[pac10]{Expiation 10: The training precept on digging earth}}
\label{exp10}
If any bhikkhu should dig the earth or should have it dug, [this is a case] involving expiation.

\begin{center}
	The section [starting with the rule] on false speech is first
\end{center}



\setsubsecheadstyle{\subsectionFmt}
\subsection{The Section on Vegetation}
% \vspace{0.2cm}

\pdfbookmark[3]{Expitation 11}{exp 11}
\subsubsection*{\hyperref[pac11]{Expiation 11: The training precept on vegetation}}
\label{exp11}
In the destroying of vegetation, [there is a case] involving expiation.



\pdfbookmark[3]{Expitation 12}{exp12}
\subsubsection*{\hyperref[pac12]{Expiation 12: The training precept on evading}}
\label{exp12}
In evading, in vexing, [there is a case] involving expiation.



\pdfbookmark[3]{Expitation 13}{exp13}
\subsubsection*{\hyperref[pac1]{Expiation 13: The training precept on making (a bhikkhu) find fault}}
\label{exp13}
In making [another bhikkhu] find fault, in criticising, [there is a case] involving expiation.



\pdfbookmark[3]{Expitation 14}{exp14}
\subsubsection*{\hyperref[pac14]{Expiation 14: The first training precept on sleeping places}}
\label{exp14}
If any bhikkhu, having [himself] put out or after having [someone else] put out in the open air, a bed or seat or mattress or stool belonging to the community, [and] then, when departing, should not take [it] away or should not have [it] taken away or should go without asking [someone to put it back], [this is a case] involving expiation.



\pdfbookmark[3]{Expitation 15}{exp15}
\subsubsection*{\hyperref[pac15]{Expiation 15: The second training precept on sleeping places}}
\label{exp15}
If any bhikkhu, having [himself] put out or having [someone else] put out, bedding in a dwelling belonging to the community, [and] then, when departing, should not take [it] away or should not have [it] taken away, or should go without asking [someone to put it back], [this is a case] involving expiation.



\pdfbookmark[3]{Expitation 16}{exp16}
\subsubsection*{\hyperref[pac16]{Expiation 16: The training precept on encroaching upon}}
\label{exp16}
If any bhikkhu, having encroached upon a bhikkhu who has arrived before, should knowingly use a sleeping place in a dwelling belonging to the community [saying]: ``He for whom it is [too] cramped, will leave,'' having done [it] for just this reason, [and] not another, [this is a case] involving expiation.



\pdfbookmark[3]{Expitation 17}{exp17}
\subsubsection*{\hyperref[pac17]{Expiation 17: The training precept on driving out}}
\label{exp17}
If any bhikkhu, being resentful and displeased, should drive out a bhikkhu or have [him] driven out from a dwelling belonging to the community, [this is a case] involving expiation.



\pdfbookmark[3]{Expitation 18}{exp18}
\subsubsection*{\hyperref[pac18]{Expiation 18: The training precept on the hut with an upper-floor}}
\label{exp18}
If any bhikkhu should [brusquely] sit down or lie down on a bed or seat with detachable legs in a hut with an upper-floor in a dwelling belonging to the community, [this is a case] involving expiation.



\pdfbookmark[3]{Expitation 19}{exp19}
\subsubsection*{\hyperref[pac19]{Expiation 19: The training precept on a large dwelling}}
\label{exp19}
By a bhikkhu who is having a large dwelling built, a surrounding-layer of two or three coverings can be ordered, by [a bhikkhu] standing on [a place which has] few crops, upto the frame of the door for [the purpose of] fixing the bolt, [and] for surrounding the window. If he should order more than that, even [when] standing on [a place which has] few crops, [this is a case] involving expiation.



\pdfbookmark[3]{Expitation 20}{exp 20}
\subsubsection*{\hyperref[pac20]{Expiation 20: The training precept on [water] containing living beings}}
\label{exp20}
If any bhikkhu should knowingly pour out, or should have [someone else] pour out, water containing living beings on grass or clay, [this is a case] involving expiation.

\begin{center}
	The section [starting with the rule] on vegetation is second
\end{center}



\setsubsecheadstyle{\subsectionFmt}
\subsection{The Section on Exhoration}
% \vspace{0.2cm}

\pdfbookmark[3]{Expitation 21}{exp 21}
\subsubsection*{\hyperref[pac21]{Expiation 21: The training precept on exhortation}}
\label{exp21}
If any bhikkhu who has not been authorised should exhort the bhikkhunīs, [this is a case] involving expiation.



\pdfbookmark[3]{Expitation 22}{exp 22}
\subsubsection*{\hyperref[pac22]{Expiation 22: The training precept on (after sun-)set}}
\label{exp22}
Even if a bhikkhu who has been authorised should exhort the bhikkhunīs after the sun has set, [this is a case] involving expiation.



\pdfbookmark[3]{Expitation 23}{exp 23}
\subsubsection*{\hyperref[pac23]{Expiation 23: The training precept on the bhikkhunī-quarters}}
\label{exp23}
If any bhikkhu, having approached the bhikkhunī-quarters, should exhort the bhikkhunīs, except at the [right] occasion, [this is a case] involving expiation. Here the occasion is this: a bhikkhunī is sick; this is the occasion here.



\pdfbookmark[3]{Expitation 24}{exp 24}
\subsubsection*{\hyperref[pac24]{Expiation 24: The training precept on worldly gain}}
\label{exp24}
If any bhikkhu should say so: ``The bhikkhus exhort bhikkhunīs for the sake of reward,'' [this is a case] involving expiation.



\pdfbookmark[3]{Expitation 25}{exp 25}
\subsubsection*{\hyperref[pac25]{Expiation 25: The training precept on giving robe(-cloth)}}
\label{exp25}
If any bhikkhu should give a robe [-cloth] to an unrelated bhikkhunī, except in an exchange, [this is a case] involving expiation.



\pdfbookmark[3]{Expitation 26}{exp 26}
\subsubsection*{\hyperref[pac26]{Expiation 26: The training precept on sewing a robe}}
\label{exp26}
If any bhikkhu should sew a robe or should have a robe sewn for an unrelated bhikkhunī, [this is a case] involving expiation.



\pdfbookmark[3]{Expitation 27}{exp 27}
\subsubsection*{\hyperref[pac27]{Expiation 27: The training precept on making arrangements}}
\label{exp27}
If any bhikkhu, having made an arrangement, should travel together with a bhikkhunī on the same main road, even [if] just the distance between villages, except at the [right] occasion, [this is a case] involving expiation. Here the occasion is this: the road, which is considered risky [and] which is dangerous, has to be gone with a company [of other travellers], this is the occasion here.



\pdfbookmark[3]{Expitation 28}{exp 28}
\subsubsection*{\hyperref[pac28]{Expiation 28: The training precept on embarking on a boat}}
\label{exp28}
If any bhikkhu, having made an arrangement, should embark [on a voyage] together with a bhikkhunī on the same boat, which is going up [-stream] or which is going down [-stream], except with [a boat which is] crossing over [a river], [this is a case] involving expiation.



\pdfbookmark[3]{Expitation 29}{exp 29}
\subsubsection*{\hyperref[pac29]{Expiation 29: The training precept on (alms-food) that has been prepared}}
\label{exp29}
If any bhikkhu should knowingly eat alms-food which a bhikkhunī has caused to be prepared, except through previous arrangement of householders, [this is a case] involving expiation.



\pdfbookmark[3]{Expitation 30}{exp 30}
\subsubsection*{\hyperref[pac30]{Expiation 30: The training precept on taking a seat privately}}
\label{exp30}
If any bhikkhu should sit down together with a bhikkhunī, privately, one [man] with one [woman], [this is a case] involving expiation.

\begin{center}
	The section [starting with the rule] on exhortation is third
\end{center}



\setsubsecheadstyle{\subsectionFmt}
\subsection{The Section on Eating}
% \vspace{0.2cm}

\pdfbookmark[3]{Expitation 31}{exp 31}
\subsubsection*{\hyperref[pac31]{Expiation 31: The training precept on the alms-meal in the resthouse}}
\label{exp31}
By a bhikkhu who is not ill one alms-meal in a resthouse can be eaten; if he should eat more than that, [this is a case] involving expiation.



\pdfbookmark[3]{Expitation 32}{exp 32}
\subsubsection*{\hyperref[pac32]{Expiation 32: The training precept on eating in a group}}
\label{exp32}
In eating [a meal] in a group, except at the [right] occasion, [there is a case] involving expiation. Here the occasion is this: the occasion of illness; the occasion of a giving of robe [-cloth]s; the occasion of a robe-making; the occasion of going on a [long] journey; the occasion of voyaging on a boat; the occasion of a great [gathering]; the occasion of a meal [made] by an ascetic; this is the occasion here.



\pdfbookmark[3]{Expitation 33}{exp 33}
\subsubsection*{\hyperref[pac33]{Expiation 33: The training precept on substituting a meal}}
\label{exp33}
In [taking] a meal before another [invitation-meal], except at the [right] occasion, [there is a case] involving expiation. Here the occasion is this: the occasion of illness; the occasion of a giving of robe [-cloth]s; the occasion of a robe-making; this is the occasion here.



\pdfbookmark[3]{Expitation 34}{exp 34}
\subsubsection*{\hyperref[pac34]{Expiation 34: The Kāṇa's mother training precept}}
\label{exp34}
Now, should a family invite a bhikkhu who has approached to take as many cakes and parched cakes [as he likes], by a bhikkhu who is wishing [so] two or three bowls full [of cakes] can be accepted; if he should accept more than that, [this is a case] involving expiation. Having accepted two or three bowls full, having taken [it] away from there, [it] is to be shared together with [other] bhikkhus. This is the proper procedure here.



\pdfbookmark[3]{Expitation 35}{exp 35}
\subsubsection*{\hyperref[pac35]{Expiation 35: The first training precept on invitation}}
\label{exp35}
If any bhikkhu who has eaten [a meal], who has been invited [to take more and refused], should chew uncooked food or eat cooked food which is not left over, [this is a case] involving expiation.



\pdfbookmark[3]{Expitation 36}{exp 36}
\subsubsection*{\hyperref[pac36]{Expiation 36: The second training precept on invitation}}
\label{exp36}
If any bhikkhu, knowingly [and] desiring to cause offence, should invite a bhikkhu, who has eaten [a meal and] who has been invited [to take more], to take uncooked food or cooked food which is not left over [saying]: ``Here, bhikkhu, chew and eat!,'' when [the bhikkhu] has eaten, [this is a case] involving expiation.



\pdfbookmark[3]{Expitation 37}{exp 37}
\subsubsection*{\hyperref[pac37]{Expiation 37: The training precept on eating at the wrong time}}
\label{exp37}
If any bhikkhu should chew uncooked food or eat cooked food at the wrong time, [this is a case] involving expiation.



\pdfbookmark[3]{Expitation 38}{exp 38}
\subsubsection*{\hyperref[pac38]{Expiation 38: The training precept on keeping (food) in store}}
\label{exp38}
If any bhikkhu should chew uncooked food or eat cooked food [while] keeping [it] in store, [this is a case] involving expiation.



\pdfbookmark[3]{Expitation 39}{exp 39}
\subsubsection*{\hyperref[pac39]{Expiation 39: The training precept on superior food}}
\label{exp39}
Those foods which are superior, namely: ghee, butter, oil, honey and molasses, fish, meat, milk, curd; whichever bhikkhu, who is not ill, having requested such superior foods for his own benefit, should eat [them], [this is a case] involving expiation.



\pdfbookmark[3]{Expitation 40}{exp 40}
\subsubsection*{\hyperref[pac40]{Expiation 40: The training precept on tooth-wood}}
\label{exp40}
If any bhikkhu should take into the mouth [any] nutriment that has not been given [to bhikkhus]; except water and tooth-wood, [this is a case] involving expiation.

\begin{center}
	The section [starting with the rule] on eating is fourth
\end{center}



\setsubsecheadstyle{\subsectionFmt}
\subsection{The Section on Naked Ascetics}
% \vspace{0.2cm}

\pdfbookmark[3]{Expitation 41}{exp 41}
\subsubsection*{\hyperref[pac41]{Expiation 41: The training precept on naked ascetics}}
\label{exp41}
If any bhikkhu should give with his own hand uncooked food or cooked food to a naked ascetic or to a male wanderer or to a female wanderer, [this is a case] involving expiation.



\pdfbookmark[3]{Expitation 42}{exp 42}
\subsubsection*{\hyperref[pac42]{Expiation 42: The training precept on dismissing}}
\label{exp42}
If any bhikkhu should say so to a bhikkhu, ``Come friend! We shall enter a village or town for alms,'' [then after] having had [food] given or not having had [food] given to him, should he dismiss [the bhikkhu saying], ``Go friend! There is no ease for me talking or sitting down together with you; there is ease for me talking or sitting down by myself;'' having made just this the reason, [and] not another, [this is a case] involving expiation.



\pdfbookmark[3]{Expitation 43}{exp 43}
\subsubsection*{\hyperref[pac43]{Expiation 43: The training precept on having a meal}}
\label{exp43}
If any bhikkhu, having intruded upon an family having a meal, should sit down, [this is a case] involving expiation.



\pdfbookmark[3]{Expitation 44}{exp 44}
\subsubsection*{\hyperref[pac44]{Expiation 44: The training precept on being privately and concealed}}
\label{exp44}
If any bhikkhu should sit down together with a woman, privately, on a concealed seat, [this is a case] involving expiation.



\pdfbookmark[3]{Expitation 45}{exp 45}
\subsubsection*{\hyperref[pac45]{Expiation 45: The training precept on taking a seat privately}}
\label{exp45}
If any bhikkhu sit down together with a woman, one [man] with one [woman], privately, [this is a case] involving expiation.



\pdfbookmark[3]{Expitation 46}{exp 46}
\subsubsection*{\hyperref[pac46]{Expiation 46: The training precept on visiting}}
\label{exp46}
If any bhikkhu who has been invited for a meal, not having asked [permission to] a bhikkhu who is present [in the monastery], should go visiting families before the meal or after the meal, except at the [right] occasion, [this is a case] involving expiation. Here the occasion is this: the occasion of a giving of robe[-cloth]s; the occasion of a making of robes; this is the occasion here.



\pdfbookmark[3]{Expitation 47}{exp 47}
\subsubsection*{\hyperref[pac47]{Expiation 47: The Mahānāma training precept}}
\label{exp47}
By a bhikkhu who is not ill a four-month invitation for requisites can be accepted; except with a repeated invitation, except with a permanent invitation; if he should accept more than that, [this is a case] involving expiation.



\pdfbookmark[3]{Expitation 48}{exp 48}
\subsubsection*{\hyperref[pac48]{Expiation 48: The training precept on the army in action}}
\label{exp48}
If any bhikkhu should should go to visit an army in action; except with an appropriate reason, [this is a case] involving expiation.



\pdfbookmark[3]{Expitation 49}{exp 49}
\subsubsection*{\hyperref[pac49]{Expiation 49: The training precept on staying in the army}}
\label{exp49}
And if there might be any reason for that bhikkhu for going to the army, two nights or three nights can be stayed within the army by that bhikkhu; if he should stay more than that, [this is a case] involving expiation.



\pdfbookmark[3]{Expitation 50}{exp 50}
\subsubsection*{\hyperref[pac50]{Expiation 50: The training precept on battle-fields}}
\label{exp50}
If a bhikkhu staying two nights or three nights within an army should go to a battle-field, or a review, or a massing of the army, or an inspection of units, [this is a case] involving expiation.

The section on naked ascetics is fifth.



\setsubsecheadstyle{\subsectionFmt}
\subsection{The Section on Alcoholic Drink}
% \vspace{0.2cm}

\pdfbookmark[3]{Expitation 51}{exp 51}
\subsubsection*{\hyperref[pac51]{Expiation 51: The training precept on alcoholic drink}}
\label{exp51}
In drinking alcoholic drink made of grain [products] or fruit [and/or flower products], [there is a case] involving expiation.



\pdfbookmark[3]{Expitation 52}{exp 52}
\subsubsection*{\hyperref[pac52]{Expiation 52: The training precept on tickling with the fingers}}
\label{exp52}
In tickling with the fingers, [there is a case] involving expiation.



\pdfbookmark[3]{Expitation 53}{exp 53}
\subsubsection*{\hyperref[pac53]{Expiation 53: The training precept on the act of playing}}
\label{exp53}
In the act of playing in water, [there is a case] involving expiation.



\pdfbookmark[3]{Expitation 54}{exp 54}
\subsubsection*{\hyperref[pac54]{Expiation 54: The training precept on disrespect}}
\label{exp54}
In disrespect, [there is a case] involving expiation.



\pdfbookmark[3]{Expitation 55}{exp 55}
\subsubsection*{\hyperref[pac55]{Expiation 55: The training precept on scaring}}
\label{exp55}
If any bhikkhu should scare [another] bhikkhu, [this is a case] involving expiation.



\pdfbookmark[3]{Expitation 56}{exp 56}
\subsubsection*{\hyperref[pac56]{Expiation 56: The training precept on (lighting) fires}}
\label{exp56}
If any bhikkhu who is not ill, desiring to warm [himself], should light a fire or should have [it] lit, except with an appropriate reason, [this is a case] involving expiation.



\pdfbookmark[3]{Expitation 57}{exp 57}
\subsubsection*{\hyperref[pac57]{Expiation 57: The training precept on bathing}}
\label{exp57}
If any bhikkhu should should bathe within less than half a month, except at the [right] occasion, [this is a case] involving expiation. Here the occasion is this [thinking]: ``one and a half month is what remains of the hot season,'' [and ``this is] the first month of the rainy season''—these two and a half months [are] the occasion of dry heat, [and] the occasion of humid heat—[also:] the occasion of being sick; the occasion of work; the occasion of going on a journey; the occasion of [dusty] wind and rain; this is the occasion here.



\pdfbookmark[3]{Expitation 58}{exp 58}
\subsubsection*{\hyperref[pac58]{Expiation 58: The training precept on stains}}
\label{exp58}
By a monk with the gain of a new robe a certain stain [from] amongst the three stains is to be applied: darkblue or muddy [-grey] or dark-brown. If a bhikkhu, not having applied a certain stain [from] amongst the three stains, should use a new robe, [this is a case] involving expiation.



\pdfbookmark[3]{Expitation 59}{exp 59}
\subsubsection*{\hyperref[pac59]{Expiation 59: The training precept on assigning}}
\label{exp59}
If any bhikkhu, having himself assigned a robe to a bhikkhu or a bhikkhunī or a male novice or a female novice, should use [it] without withdrawing [the assignment], [this is a case] involving expiation.



\pdfbookmark[3]{Expitation 60}{exp 60}
\subsubsection*{\hyperref[pac60]{Expiation 60: The training precept on hiding}}
\label{exp60}
If any bhikkhu should hide a bhikkhu's bowl or robe or sitting-cloth or needle case or body-belt, or have [it] hidden, even if just desiring amusement, [this is a case] involving expiation.

\begin{center}
	The section [starting with the rule] on alcoholic drink is sixth
\end{center}



\setsubsecheadstyle{\subsectionFmt}
\subsection{The Section on Living Beings}
% \vspace{0.2cm}

\pdfbookmark[3]{Expitation 61}{exp 61}
\subsubsection*{\hyperref[pac61]{Expiation 61: The training precept on intentionally (depriving a being of life)}}
\label{exp61}
If any bhikkhu should intentionally deprive a living being of life, [this is a case] involving expiation.



\pdfbookmark[3]{Expitation 62}{exp 62}
\subsubsection*{\hyperref[pac62]{Expiation 62: The training precept on (water) with living beings}}
\label{exp62}
If any bhikkhu should knowingly use water containing living beings, [this is a case] involving expiation.



\pdfbookmark[3]{Expitation 63}{exp 63}
\subsubsection*{\hyperref[pac63]{Expiation 63: The training precept on agitating}}
\label{exp63}
If any bhikkhu should knowingly agitate for further [legal] action a legal issue which has been disposed of according to the law, [this is a case] involving expiation.



\pdfbookmark[3]{Expitation 64}{exp 64}
\subsubsection*{\hyperref[pac64]{Expiation 64: The training precept on depraved (offences)}}
\label{exp64}
If any bhikkhu should knowingly conceal a bhikkhu's depraved offence, [this is a case] involving expiation.



\pdfbookmark[3]{Expitation 65}{exp 65}
\subsubsection*{\hyperref[pac65]{Expiation 65: The training precept on (a person) less than twenty years (old)}}
\label{exp65}
If any bhikkhu should knowingly have a person who is less than twenty years [old] fully admitted [into the bhikkhu-community], then that person is one who has not been fully admitted and those bhikkhus are blameworthy. Because of that, this [is a case] involving expiation.



\pdfbookmark[3]{Expitation 66}{exp 66}
\subsubsection*{\hyperref[pac66]{Expiation 66: The training precept on a company (of travellers intent on) theft}}
\label{exp66}
If any bhikkhu, having made an arrangement, should knowingly travel together on the same main road with a company of thieves, even [if] just the distance between villages, [this is a case] involving expiation.



\pdfbookmark[3]{Expitation 67}{exp 67}
\subsubsection*{\hyperref[pac67]{Expiation 67: The training precept on making arrangements}}
\label{exp67}
If any bhikkhu, having made an arrangement, should travel together with a woman on the same main road, even [if] just the distance between villages, [this is a case] involving expiation.



\pdfbookmark[3]{Expitation 68}{exp 68}
\subsubsection*{\hyperref[pac68]{Expiation 68: The Ariṭṭha training precept}}
\label{exp68}
If any bhikkhu should say so, ``As I understand the Teaching taught by the Fortunate One, these obstructive acts which are spoken of by the Fortunate One: they are not enough to be an obstruction for the one who is being engaged in [them],'' [then] that bhikkhu is to be spoken to thus by the bhikkhus: ``Venerable, don't say so! Don't misrepresent the Fortunate One; for the misrepresentation of the Fortunate One is not good; for the Fortunate One would not say so; friend, [that] obstructive acts are [really] obstructive is spoken of in manifold ways by the Fortunate One and they are enough to be an obstruction for the one who is being engaged in [them],'' and [if] that bhikkhu being spoken to thus by the bhikkhus should persist in the same way [as before], [then] that bhikkhu is to be argued with up to three times by the bhikkhus for the relinquishing of that [view], [and if that bhikkhu,] being argued with up to three times, should relinquish that [view], then this is good, [but] if he should not relinquish [it]: [this is a case] involving expiation.



\pdfbookmark[3]{Expitation 69}{exp 69}
\subsubsection*{\hyperref[pac69]{Expiation 69: The training precept on boycotted food}}
\label{exp69}
If any bhikkhu knowingly should eat together with, or should live together with, or should use a sleeping place together with a bhikkhu who is speaking thus, who has not performed the normal procedure, who has not relinquished that view, [this is a case] involving expiation.



\pdfbookmark[3]{Expitation 70}{exp 70}
\subsubsection*{\hyperref[pac70]{Expiation 70: The Kaṇṭaka training precept}}
\label{exp70}
If a novice should say so too, ``As I understand the Teaching taught by the Fortunate One, these obstructive acts which are spoken of by the Fortunate One: they are not enough to be an obstruction for the one who is being engaged in [them],'' [then] that novice is to be spoken to thus by the bhikkhus, ``Friend novice, don't say so! Don't misrepresent the Fortunate One; for the misrepresentation of the Fortunate One is not good; for the Fortunate One would not say so; friend novice, [that] obstructive acts are [really] obstructive is spoken of in manifold ways by the Fortunate One and they are enough to be an obstruction for the one who is engaging [in them],'' and if that novice being spoken to thus by the bhikkhus should persist in the same way [as before], [then] that novice is to be spoken to thus by the bhikkhus, ``From today on, friend novice, the Fortunate One is not to be referred to as the teacher by you, and also the two or three nights sleeping together [in one room] with bhikkhus that other novices get, that too is not for you. Go away, disappear!'' If any bhikkhu knowingly should treat kindly such an expelled novice, or should make [him] attend [to himself], or should eat together with [him], or should use a sleeping place together with [him], [this is a case] involving expiation.

\begin{center}
	The section [starting with the rule] on living beings is seventh
\end{center}



\setsubsecheadstyle{\subsectionFmt}
\subsection{The Section on Righteous Speech}
% \vspace{0.2cm}

\pdfbookmark[3]{Expitation 71}{exp 71}
\subsubsection*{\hyperref[pac71]{Expiation 71: The training precept on (being spoken to) righteously}}
\label{exp71}
If any bhikkhu when being righteously spoken to by bhikkhus should say so, ``Friends, I shall not train in this training precept for as long as I can not question another bhikkhu [about it] who is a learned memoriser of the discipline,'' [this is a case] involving expiation. Bhikkhus, [the training precept] is to be understood, is to be questioned about, is to be investigated by a bhikkhu who is training [in it]. This is the proper procedure here.



\pdfbookmark[3]{Expitation 72}{exp 72}
\subsubsection*{\hyperref[pac72]{Expiation 72: The training precept on creating discomfort}}
\label{exp72}
If any bhikkhu, when the Disciplinary Code is being recited, should say so, ``But why these small and minute training precepts that are recited? They just lead to worry, annoyance, [and] discomfort.'' In the disparaging of training precepts, [there is a case] involving expiation.



\pdfbookmark[3]{Expitation 73}{exp 73}
\subsubsection*{\hyperref[pac73]{Expiation 73: The training precept on delusion}}
\label{exp73}
If any bhikkhu when the Disciplinary Code is being recited half-monthly should say so, ``Only now I know! This too, indeed, is a case which has been handed down in the Sutta, which has been included in the Sutta, which comes up for recitation half-monthly!'' [and] if other bhikkhus should know [about] that bhikkhu [thus], ``This bhikkhu has sat [in] two or three times previously when the Disciplinary Code was being recited. What to say about more [times than that]!'' [then] there is no release for that bhikkhu through not-knowing, and whatever the offence is that he has committed there, he is to be made to do according to that case and moreover his deluding is to be exposed, ``Because of that [there are] losses for you, because of that [it] has been ill-gained by you, that you, when the Disciplinary Code is being recited, do not take [it] to mind [after] having focussed carefully [on it].'' Because of that deluding, this [is a case] involving expiation.



\pdfbookmark[3]{Expitation 74}{exp 74}
\subsubsection*{\hyperref[pac74]{Expiation 74: The training precept on (giving) a blow}}
\label{exp74}
If any bhikkhu who is resentful [and] displeased should give a blow to a bhikkhu, [this is a case] involving expiation.



\pdfbookmark[3]{Expitation 75}{exp 75}
\subsubsection*{\hyperref[pac75]{Expiation 75: The training precept on (brandishing) the hand-palm like a dagger}}
\label{exp75}
If any bhikkhu should brandish the palm of the hand [threateningly] like [one holds] a dagger to a bhikkhu, [this is a case] involving expiation.



\pdfbookmark[3]{Expitation 76}{exp 76}
\subsubsection*{\hyperref[pac76]{Expiation 76: The training precept on being groundless}}
\label{exp76}
If any bhikkhu should should accuse a bhikkhu with a groundless [case concerning] the community in the beginning and in the rest [of the procedure], [this is a case] involving expiation.



\pdfbookmark[3]{Expitation 77}{exp 77}
\subsubsection*{\hyperref[pac77]{Expiation 77: The training precept on deliberately (provoking worry)}}
\label{exp77}
If any bhikkhu should deliberately provoke worry for a bhikkhu [thinking], ``Thus there will be discomfort for him, even [if only] for a short time,'' having made just this the reason, [and] not another, [this is a case] involving expiation.



\pdfbookmark[3]{Expitation 78}{exp 78}
\subsubsection*{\hyperref[pac78]{Expiation 78: The training precept on overhearing}}
\label{exp78}
If any bhikkhu should stand overhearing bhikkhus who are arguing, who are quarrelling, who are engaged in dispute [thinking], ``I shall hear what these ones will say,'' having made just this the reason, [and] not another, [this is a case] involving expiation.



\pdfbookmark[3]{Expitation 79}{exp 79}
\subsubsection*{\hyperref[pac79]{Expiation 79: The training precept on the averting of legal action}}
\label{exp79}
If any bhikkhu, having given consent to legitimate [legal] actions, should afterwards engage in the act of criticising, [this is a case] involving expiation.



\pdfbookmark[3]{Expitation 80}{exp 80}
\subsubsection*{\hyperref[pac80]{Expiation 80: The training precept on going without having given consent}}
\label{exp80}
If any bhikkhu, when investigatory discussion is going on in the community, not having given [his] consent, having gotten up from [his] seat, should depart, [this is a case] involving expiation.



\pdfbookmark[3]{Expitation 81}{exp 81}
\subsubsection*{\hyperref[pac81]{Expiation 81: The training precept on feeble (-robes)}}
\label{exp81}
If any bhikkhu, having given a robe[-cloth] [together] with a united community, should afterwards engage in criticising [saying]: ``The bhikkhus allocate communal gain according to familiarity,'' [this is a case] involving expiation.



\pdfbookmark[3]{Expitation 82}{exp 82}
\subsubsection*{\hyperref[pac82]{Expiation 82: The training precept on allocation}}
\label{exp82}
If any bhikkhu should knowingly allocate [already] allocated communal gain to a [lay-] person, [this is a case] involving expiation.

\begin{center}
	The section [starting with the rule] about [being spoken to] righteously is eighth
\end{center}



\setsubsecheadstyle{\subsectionFmt}
\subsection{The Section on Kings}
% \vspace{0.2cm}

\pdfbookmark[3]{Expitation 83}{exp 83}
\subsubsection*{\hyperref[pac83]{Expiation 83: The training precept on the harem}}
\label{exp83}
If any bhikkhu, without having been announced beforehand, should go beyond the boundary post of a noble consecrated king's [bed-room] when the king has not departed, [and] the [queen-] jewel has not withdrawn, [this is a case] involving expiation.



\pdfbookmark[3]{Expitation 84}{exp 84}
\subsubsection*{\hyperref[pac84]{Expiation 84: The training precept on treasures}}
\label{exp84}
If any bhikkhu should pick up, or should make [someone else] pick up, a treasure or what is considered a treasure, except within a monastery or within a dwelling, [this is a case] involving expiation. However, by a bhikkhu having picked up, or having had picked up, a treasure or what is considered a treasure within a monastery or within a dwelling, [it] is to be put aside [thinking]: ``He to whom it belongs will take it.'' This is the proper procedure here.



\pdfbookmark[3]{Expitation 85}{exp 85}
\subsubsection*{\hyperref[pac85]{Expiation 85: The training precept on entering at the wrong time}}
\label{exp85}
If any bhikkhu, not having asked (permission of) a bhikkhu who is present, should enter a village at the wrong time, except with an appropriate urgent duty, [this is a case] involving expiation.



\pdfbookmark[3]{Expitation 86}{exp 86}
\subsubsection*{\hyperref[pac86]{Expiation 86: The training precept on the needle case}}
\label{exp86}
If any bhikkhu should have a needle-case made, which is made of bone, or made of ivory, or made of horn, [this is a case] involving expiation with breaking up [the needle-case].



\pdfbookmark[3]{Expitation 87}{exp 87}
\subsubsection*{\hyperref[pac87]{Expiation 87: The training precept on beds or seats}}
\label{exp87}
By a bhikkhu who is having a new bed or seat made, [a bed or seat] which has legs of eight fingerbreadths is to be made, according to the Sugata-finger-breadth, except the lowermost [edge of the] frame. For one who lets it exceed [this measure], [this is a case] involving expiation with cutting [down the legs].



\pdfbookmark[3]{Expitation 88}{exp 88}
\subsubsection*{\hyperref[pac88]{Expiation 88: The training precept on covered with cotton}}
\label{exp88}
If any bhikkhu should have a bed or seat covered with cotton made, [this is a case] involving expiation with tearing off [the cotton].



\pdfbookmark[3]{Expitation 89}{exp 89}
\subsubsection*{\hyperref[pac89]{Expiation 89: The training precept on the sitting-cloth}}
\label{exp89}
By a bhikkhu who is having a sitting-cloth made, [a sitting-cloth] which has the [proper] measure is to be made. This measure here is: two spans of the sugata-span in length, one and a half across, [and] the border is a span. For one who lets it exceed [the measure], [this is a case] involving expiation with cutting [off the cloth].



\pdfbookmark[3]{Expitation 90}{exp 90}
\subsubsection*{\hyperref[pac90]{Expiation 90: The training precept on itch-coverings}}
\label{exp90}
y a bhikkhu who is having an itch-covering [-cloth] made, [an itch-covering] which has the [proper] measure is to be made. This measure here is: four spans of the Sugata-span in length, two spans across. For one who lets it exceed [the measure], [this is a case] involving expiation with cutting off [ the cloth].



\pdfbookmark[3]{Expitation 91}{exp 91}
\subsubsection*{\hyperref[pac91]{Expiation 91: The training precept on rain's bathing-cloth}}
\label{exp91}
By a bhikkhu who is having a rain's bathing-cloth made, [a bathing-cloth] which has the [proper] measure is to be made. This measure here is: six spans of the sugata-span in length, two and a half across. For one who lets it exceed [the measure], [this is a case] involving expiation with cutting [off the cloth].



\pdfbookmark[3]{Expitation 92}{exp 92}
\subsubsection*{\hyperref[pac92]{Expiation 92: The Nanda training precept}}
\label{exp92}
If any bhikkhu should have a robe made which has the sugata-robe measure or [one] which is more [than that], [this is a case] involving expiation with cutting [off the robe]. This is the Sugata's sugata-robe measure here: nine spans of the sugata-span in length, six spans across. This is the Sugata's sugata-robe measure.

\begin{center}
	The section [starting with the rule] on kings is ninth
\end{center}



\medskip

\begin{center}
	Venerables, the ninety-two cases involving expiation have been recited.

	\smallskip

	Concerning that I ask the Venerables: [Are you] pure in this?\\
	A second time again I ask: [Are you] pure in this?\\
	A third time again I ask: [Are you] pure in this?

	\smallskip

	The venerables are pure in this, therefore there is silence, thus I keep this [in mind].
\end{center}

\begin{outro}
	The [cases] involving expiation are finished.
\end{outro}

\clearpage

