
\section{Pārājik'uddeso}
\label{par}

\linkdest{endnote-body}
\begin{intro}
	Tatr'ime\makeatletter\hyperlink{endnote-appendix}\Hy@raisedlink{\hypertarget{endnote-body}{}{\pagenote{%
		\hypertarget{endnote-appendix}{\hyperlink{endnote-body}{}}}}}\makeatother V, BhPm 2: tatrīme (BhPm 2 gives tatrime as v.l.). cattāro pārājikā dhammā uddesaṁ āgacchanti.
\end{intro}

\setsubsecheadstyle{\subsubsectionFmt}
\pdfbookmark[2]{Pārājika 1}{par1}
\subsection*{\hyperref[disq1]{Pārājika 1: Methunadhammasikkhāpadaṁ}}
\label{par1}

\linkdest{endnote-body}
\linkdest{endnote-body}
\linkdest{endnote-body}
\linkdest{endnote-body}
Yo pana bhikkhu bhikkhūnaṁ\makeatletter\hyperlink{endnote-appendix}\Hy@raisedlink{\hypertarget{endnote-body}{}{\pagenote{%
		\hypertarget{endnote-appendix}{\hyperlink{endnote-body}{}}}}}\makeatother V: bhikkhūna-sikkhā- sikkhāsājīvasamāpanno sikkhaṁ appaccakkhāya\makeatletter\hyperlink{endnote-appendix}\Hy@raisedlink{\hypertarget{endnote-body}{}{\pagenote{%
		\hypertarget{endnote-appendix}{\hyperlink{endnote-body}{}}}}}\makeatother C, D, G, V, W, Vibh Ce, Ra, UP, Vibh Ee, Mi Se, BhPm 1 & 2, Pg: apaccakkhāya. dubbalyaṁ anāvikatvā\makeatletter\hyperlink{endnote-appendix}\Hy@raisedlink{\hypertarget{endnote-body}{}{\pagenote{%
		\hypertarget{endnote-appendix}{\hyperlink{endnote-body}{}}}}}\makeatother V, Vibh Ce, Um, Ra: anāvīkatvā. methunaṁ dhammaṁ paṭiseveyya\makeatletter\hyperlink{endnote-appendix}\Hy@raisedlink{\hypertarget{endnote-body}{}{\pagenote{%
		\hypertarget{endnote-appendix}{\hyperlink{endnote-body}{}}}}}\makeatother C, W, UP, Um, BhPm 1 & 2, Ra, Pg: patiseveyya., antamaso tiracchānagatāya'pi; pārājiko hoti, asaṁvāso.



\pdfbookmark[2]{Pārājika 2}{par2}
\subsection*{\hyperref[disq2]{Pārājika 2: Adinn'ādānasikkhāpadaṁ}}
\label{par2}

\linkdest{endnote-body}
Yo pana bhikkhu gāmā vā araññā vā adinnaṁ theyyasaṅkhātaṁ ādiyeyya, yathārūpe adinn'ādāne rājāno coraṁ gahetvā haneyyuṁ vā bandheyyuṁ vā pabbājeyyuṁ vā: ``Coro'si, bālo'si, mūḷho'si\makeatletter\hyperlink{endnote-appendix}\Hy@raisedlink{\hypertarget{endnote-body}{}{\pagenote{%
		\hypertarget{endnote-appendix}{\hyperlink{endnote-body}{}}}}}\makeatother Mm Se, BhPm 1, V: muḷho., theno'sī'ti,'' tathārūpaṁ bhikkhu adinnaṁ ādiyamāno; ayam'pi pārājiko hoti, asaṁvāso.



\pdfbookmark[2]{Pārājika 3}{par3}
\subsection*{\hyperref[disq3]{Pārājika 3: Manussaviggahasikkhāpadaṁ}}
\label{par3}

\linkdest{endnote-body}
Yo pana bhikkhu sañcicca manussaviggahaṁ jīvitā voropeyya, satthahārakaṁ vā'ssa pariyeseyya, maraṇavaṇṇaṁ vā saṁvaṇṇeyya, maraṇāya vā samādapeyya: ``Ambho purisa, kiṁ tuyh'iminā pāpakena dujjīvitena? Matan'te\makeatletter\hyperlink{endnote-appendix}\Hy@raisedlink{\hypertarget{endnote-body}{}{\pagenote{%
		\hypertarget{endnote-appendix}{\hyperlink{endnote-body}{}}}}}\makeatother Dm: mataṁ te. jīvitā seyyo'ti!'', iti cittamano cittasaṅkappo anekapariyāyena maraṇavaṇṇaṁ vā saṁvaṇṇeyya, maraṇāya vā samādapeyya; ayam'pi pārājiko hoti, asaṁvāso.



\pdfbookmark[2]{Pārājika 4}{par4}
\subsection*{\hyperref[disq4]{Pārājika 4: Uttarimanussadhammasikkhāpadaṁ}}

\label{par4}

\linkdest{endnote-body}
\linkdest{endnote-body}
\linkdest{endnote-body}
\linkdest{endnote-body}
Yo pana bhikkhu anabhijānaṁ uttarimanussadhammaṁ att'ūpanāyikaṁ alam'ariyañāṇadassanaṁ\makeatletter\hyperlink{endnote-appendix}\Hy@raisedlink{\hypertarget{endnote-body}{}{\pagenote{%
		\hypertarget{endnote-appendix}{\hyperlink{endnote-body}{}}}}}\makeatother G: -dassaṇaṁ. samudācareyya: ``Iti jānāmi, iti passāmī'ti!'', tato aparena samayena samanuggāhiyamāno\makeatletter\hyperlink{endnote-appendix}\Hy@raisedlink{\hypertarget{endnote-body}{}{\pagenote{%
		\hypertarget{endnote-appendix}{\hyperlink{endnote-body}{}}}}}\makeatother Dm: -ggahīya-.  vā asamanuggāhiyamāno\makeatletter\hyperlink{endnote-appendix}\Hy@raisedlink{\hypertarget{endnote-body}{}{\pagenote{%
		\hypertarget{endnote-appendix}{\hyperlink{endnote-body}{}}}}}\makeatother Dm: -ggahīya-.  vā āpanno visuddh'āpekkho evaṁ vadeyya: ``Ajānam'ev'āhaṁ āvuso\makeatletter\hyperlink{endnote-appendix}\Hy@raisedlink{\hypertarget{endnote-body}{}{\pagenote{%
		\hypertarget{endnote-appendix}{\hyperlink{endnote-body}{}}}}}\makeatother C, D, W, Dm, Mi Se, BhPm 1 & 2, Um, Ra: ajānamevaṁ āvuso. Vibh Ee, UP, Mm Se: ajānaṁ evaṁ āvuso.  avacaṁ: 'Jānāmi!' apassaṁ: 'Passāmi!' Tucchaṁ musā vilapin'ti'', aññatra adhimānā, ayam'pi pārājiko hoti, asaṁvāso.



\ifafiveversion \clearpage \else \medskip \fi

\begin{center}
	Uddiṭṭhā kho āyasmanto cattāro pārājikā dhammā. Yesaṁ bhikkhu aññataraṁ vā aññataraṁ vā āpajjitvā na labhati bhikkhūhi saddhiṁ saṁvāsaṁ. Yathā pure, tathā pacchā, pārājiko hoti, asaṁvāso.

	\smallskip

	Tatth'āyasmante pucchāmi: Kacci'ttha parisuddhā?\\
	Dutiyam'pi pucchāmi: Kacci'ttha parisuddhā?\\
	Tatiyam'pi pucchāmi: Kacci'ttha parisuddhā?

	\smallskip

	Parisuddh'etth'āyasmanto, tasmā tuṇhī, evam'etaṁ dhārayāmi.
\end{center}

\linkdest{endnote10-body}
\linkdest{endnote-body}
\begin{outro}
	Cattāro pārājkā dhammā niṭṭhitā\makeatletter\hyperlink{endnote10-appendix}\Hy@raisedlink{\hypertarget{endnote10-body}{}{\pagenote{%
				\hypertarget{endnote10-appendix}{\hyperlink{endnote10-body}{Not in any edition or manuscript, but if a conclusion is to be recited then this one as given in the Parivāra would be the suitable one.\\
						When reciting in brief use: pārājik'uddeso niṭṭhito.\makeatletter\hyperlink{endnote-appendix}\Hy@raisedlink{\hypertarget{endnote-body}{}{\pagenote{%
		\hypertarget{endnote-appendix}{\hyperlink{endnote-body}{}}}}}\makeatother Dm: Pārājikaṁ niṭṭhitaṁ. Ñd Ce, UP, Um, Mi Se: Pārājikuddeso dutiyo.}}}}}\makeatother
\end{outro}

\clearpage

