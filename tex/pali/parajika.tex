
\section{Pārājik'uddeso}
\label{par}

\linkdest{endnote26-body}
\begin{intro}
	Tatr'ime\makeatletter\hyperlink{endnote26-appendix}\Hy@raisedlink{\hypertarget{endnote26-body}{}{\pagenote{%
		\hypertarget{endnote26-appendix}{\hyperlink{endnote26-body}{V, BhPm 2: \textit{tatrīme} (BhPm 2 gives \textit{tatrime} as v.l.).}}}}}\makeatother \thinspace cattāro pārājikā dhammā uddesaṁ āgacchanti.
\end{intro}

\setsubsecheadstyle{\subsubsectionFmtalt}
\pdfbookmark[2]{Pārājika 1}{par1}
\subsection*{\hyperref[disq1]{Pārājika 1: Methunadhammasikkhāpadaṁ}}
\label{par1}

\linkdest{endnote27-body}
\linkdest{endnote28-body}
\linkdest{endnote29-body}
\linkdest{endnote30-body}
Yo pana bhikkhu bhikkhūnaṁ\makeatletter\hyperlink{endnote27-appendix}\Hy@raisedlink{\hypertarget{endnote27-body}{}{\pagenote{%
		\hypertarget{endnote27-appendix}{\hyperlink{endnote27-body}{V: \textit{bhikkhūna-sikkhā-}}}}}}\makeatother \thinspace sikkhāsājīvasamāpanno sikkhaṁ appaccakkhāya\makeatletter\hyperlink{endnote28-appendix}\Hy@raisedlink{\hypertarget{endnote28-body}{}{\pagenote{%
		\hypertarget{endnote28-appendix}{\hyperlink{endnote28-body}{C, D, G, V, W, SVibh Ce, Ra, UP, SVibh Ee, Mi Se, BhPm 1 \& 2, Pg: \textit{apaccakkhāya}.}}}}}\makeatother \thinspace dubbalyaṁ anāvikatvā\makeatletter\hyperlink{endnote29-appendix}\Hy@raisedlink{\hypertarget{endnote29-body}{}{\pagenote{%
		\hypertarget{endnote29-appendix}{\hyperlink{endnote29-body}{V, SVibh Ce, Um, Ra: \textit{anāvīkatvā}.}}}}}\makeatother \thinspace methunaṁ dhammaṁ paṭiseveyya,\makeatletter\hyperlink{endnote30-appendix}\Hy@raisedlink{\hypertarget{endnote30-body}{}{\pagenote{%
		\hypertarget{endnote30-appendix}{\hyperlink{endnote30-body}{C, W, UP, Um, BhPm 1 \& 2, Ra, Pg: \textit{patiseveyya}.}}}}}\makeatother \thinspace antamaso tiracchānagatāya'pi; pārājiko hoti, asaṁvāso.



\pdfbookmark[2]{Pārājika 2}{par2}
\subsection*{\hyperref[disq2]{Pārājika 2: Adinn'ādānasikkhāpadaṁ}}
\label{par2}

\linkdest{endnote31-body}
Yo pana bhikkhu gāmā vā araññā vā adinnaṁ theyyasaṅkhātaṁ ādiyeyya, yathārūpe adinn'ādāne rājāno coraṁ gahetvā haneyyuṁ vā bandheyyuṁ vā pabbājeyyuṁ vā: ``Coro'si, bālo'si, mūḷho'si,\makeatletter\hyperlink{endnote31-appendix}\Hy@raisedlink{\hypertarget{endnote31-body}{}{\pagenote{%
		\hypertarget{endnote31-appendix}{\hyperlink{endnote-body}{Mm Se, BhPm 1, V: \textit{muḷho}.}}}}}\makeatother \thinspace theno'sī'ti,'' tathārūpaṁ bhikkhu adinnaṁ ādiyamāno; ayam'pi pārājiko hoti, asaṁvāso.



\pdfbookmark[2]{Pārājika 3}{par3}
\subsection*{\hyperref[disq3]{Pārājika 3: Manussaviggahasikkhāpadaṁ}}
\label{par3}

\linkdest{endnote32-body}
Yo pana bhikkhu sañcicca manussaviggahaṁ jīvitā voropeyya, satthahārakaṁ vā'ssa pariyeseyya, maraṇavaṇṇaṁ vā saṁvaṇṇeyya, maraṇāya vā samādapeyya: ``Ambho purisa, kiṁ tuyh'iminā pāpakena dujjīvitena? Matan'te\makeatletter\hyperlink{endnote-appendix}\Hy@raisedlink{\hypertarget{endnote-body}{}{\pagenote{%
		\hypertarget{endnote-appendix}{\hyperlink{endnote-body}{Dm: \textit{mataṁ te}.}}}}}\makeatother \thinspace jīvitā seyyo'ti!'', iti cittamano cittasaṅkappo anekapariyāyena maraṇavaṇṇaṁ vā saṁvaṇṇeyya, maraṇāya vā samādapeyya; ayam'pi pārājiko hoti, asaṁvāso.



\pdfbookmark[2]{Pārājika 4}{par4}
\subsection*{\hyperref[disq4]{Pārājika 4: Uttarimanussadhammasikkhāpadaṁ}}

\label{par4}

\linkdest{endnote33-body}
\linkdest{endnote34-body}
\linkdest{endnote35-body}
\linkdest{endnote36-body}
Yo pana bhikkhu anabhijānaṁ uttarimanussadhammaṁ att'ūpanāyikaṁ alam'ariyañāṇadassanaṁ\makeatletter\hyperlink{endnote33-appendix}\Hy@raisedlink{\hypertarget{endnote33-body}{}{\pagenote{%
		\hypertarget{endnote33-appendix}{\hyperlink{endnote33-body}{G: \textit{-dassaṇaṁ}.}}}}}\makeatother \thinspace samudācareyya: ``Iti jānāmi, iti passāmī'ti!'', tato aparena samayena samanuggāhiyamāno\makeatletter\hyperlink{endnote34-appendix}\Hy@raisedlink{\hypertarget{endnote34-body}{}{\pagenote{%
		\hypertarget{endnote34-appendix}{\hyperlink{endnote34-body}{Dm: \textit{-ggahīya-}.}}}}}\makeatother \thinspace  vā asamanuggāhiyamāno\makeatletter\hyperlink{endnote35-appendix}\Hy@raisedlink{\hypertarget{endnote35-body}{}{\pagenote{%
		\hypertarget{endnote35-appendix}{\hyperlink{endnote35-body}{Dm: \textit{-ggahīya-}.}}}}}\makeatother \thinspace  vā āpanno visuddh'āpekkho evaṁ vadeyya: ``Ajānam'ev'āhaṁ āvuso\makeatletter\hyperlink{endnote36-appendix}\Hy@raisedlink{\hypertarget{endnote36-body}{}{\pagenote{%
		\hypertarget{endnote36-appendix}{\hyperlink{endnote36-body}{C, D, W, Dm, Mi Se, BhPm 1 \& 2, Um, Ra: \textit{ajānamevaṁ āvuso}. SVibh Ee, UP, Mm Se: \textit{ajānaṁ evaṁ āvuso}.}}}}}\makeatother \thinspace  avacaṁ: 'Jānāmi!' apassaṁ: 'Passāmi!' Tucchaṁ musā vilapin'ti'', aññatra adhimānā, ayam'pi pārājiko hoti, asaṁvāso.



\ifafiveversion \clearpage \else \medskip \fi

\begin{center}
	Uddiṭṭhā kho āyasmanto cattāro pārājikā dhammā. Yesaṁ bhikkhu aññataraṁ vā aññataraṁ vā āpajjitvā na labhati bhikkhūhi saddhiṁ saṁvāsaṁ. Yathā pure, tathā pacchā, pārājiko hoti, asaṁvāso.

	\smallskip

	Tatth'āyasmante pucchāmi: Kacci'ttha parisuddhā?\\
	Dutiyam'pi pucchāmi: Kacci'ttha parisuddhā?\\
	Tatiyam'pi pucchāmi: Kacci'ttha parisuddhā?

	\smallskip

	Parisuddh'etth'āyasmanto, tasmā tuṇhī, evam'etaṁ dhārayāmi.
\end{center}

\linkdest{endnote10-body}
\linkdest{endnote37-body}
\begin{outro}
	Cattāro pārājkā dhammā niṭṭhitā\makeatletter\hyperlink{endnote10-appendix}\Hy@raisedlink{\hypertarget{endnote10-body}{}{\pagenote{%
				\hypertarget{endnote10-appendix}{\hyperlink{endnote10-body}{Not in any edition or manuscript, but if a conclusion is to be recited then this one as given in the Parivāra would be the suitable one.\\
						When reciting in brief use: \textit{pārājik'uddeso niṭṭhito}.}}}}}\makeatother \thinspace\makeatletter\hyperlink{endnote37-appendix}\Hy@raisedlink{\hypertarget{endnote37-body}{}{\pagenote{%
		\hypertarget{endnote37-appendix}{\hyperlink{endnote37-body}{Dm: \textit{Pārājikaṁ niṭṭhitaṁ}. Ñd Ce, UP, Um, Mi Se: \textit{Pārājikuddeso dutiyo}.}}}}}\makeatother \thinspace
\end{outro}

\clearpage

