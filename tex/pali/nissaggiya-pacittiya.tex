
\section{Nissaggiyapācittiyā}
\label{np}

\begin{intro}
	Ime kho pan'āyasmanto tiṁsa nissaggiyā pācittiyā dhammā uddesaṁ āgacchanti.
\end{intro}

\setsubsecheadstyle{\subsectionFmt}
\subsection{Cīvaravaggo}
% \vspace{0.2cm}

\pdfbookmark[3]{Nissaggiya Pācittiya 1}{np1}
\subsubsection*{\hyperref[forf-exp1]{Nissaggiya Pācittiya 1: Kaṭhinasikkhāpadaṁ}}
\label{np1}

\linkdest{endnote-body}
\linkdest{endnote-body}
Niṭṭhitacīvarasmiṁ\makeatletter\hyperlink{endnote-appendix}\Hy@raisedlink{\hypertarget{endnote-body}{}{\pagenote{%
		\hypertarget{endnote-appendix}{\hyperlink{endnote-body}{}}}}}\makeatother BhPm 1 & 2, C, D, W, Ra, UP v.l.: niṭṭhitacīvarasmiṁ pana. bhikkhunā ubbhatasmiṁ kaṭhine\makeatletter\hyperlink{endnote-appendix}\Hy@raisedlink{\hypertarget{endnote-body}{}{\pagenote{%
		\hypertarget{endnote-appendix}{\hyperlink{endnote-body}{}}}}}\makeatother Dm: kathine, das'āhaparamaṁ atirekacīvaraṁ dhāretabbaṁ. Taṁ atikkāmayato, nissaggiyaṁ pācittiyaṁ.



\pdfbookmark[3]{Nissaggiya Pācittiya 2}{np2}
\subsubsection*{\hyperref[forf-exp2]{Nissaggiya Pācittiya 2: Uddositasikkhāpadaṁ}}
\label{np2}

\linkdest{endnote-body}
\linkdest{endnote-body}
\linkdest{endnote-body}
\linkdest{endnote-body}
Niṭṭhitacīvarasmiṁ\makeatletter\hyperlink{endnote-appendix}\Hy@raisedlink{\hypertarget{endnote-body}{}{\pagenote{%
		\hypertarget{endnote-appendix}{\hyperlink{endnote-body}{}}}}}\makeatother BhPm 1 & 2, C, D, W, Ra, UP v.l.: niṭṭhitacīvarasmiṁ pana. bhikkhunā ubbhatasmiṁ kaṭhine\makeatletter\hyperlink{endnote-appendix}\Hy@raisedlink{\hypertarget{endnote-body}{}{\pagenote{%
		\hypertarget{endnote-appendix}{\hyperlink{endnote-body}{}}}}}\makeatother Dm: kathine., ekarattam'pi\makeatletter\hyperlink{endnote-appendix}\Hy@raisedlink{\hypertarget{endnote-body}{}{\pagenote{%
		\hypertarget{endnote-appendix}{\hyperlink{endnote-body}{}}}}}\makeatother D, G, P: -rattim-pi.  ce bhikkhu ticīvarena vippavaseyya, aññatra bhikkhusammutiyā\makeatletter\hyperlink{endnote-appendix}\Hy@raisedlink{\hypertarget{endnote-body}{}{\pagenote{%
		\hypertarget{endnote-appendix}{\hyperlink{endnote-body}{}}}}}\makeatother Mi & Mm Se, BhPm 1 v.l.: sammatiyā. (BhPm 2 has -sammatiyā at NP 14) (Pg: -sammutiyā.), nissaggiyaṁ pācittiyaṁ.



\pdfbookmark[3]{Nissaggiya Pācittiya 3}{np3}
\subsubsection*{\hyperref[forf-exp3]{Nissaggiya Pācittiya 3: Akālacīvarasikkhāpadaṁ}}
\label{np3}

Niṭṭhitacīvarasmiṁ\makeatletter\hyperlink{endnote-appendix}\Hy@raisedlink{\hypertarget{endnote-body}{}{\pagenote{%
		\hypertarget{endnote-appendix}{\hyperlink{endnote-body}{}}}}}\makeatother BhPm 1 & 2, C, D, W, Ra, UP v.l.: niṭṭhitacīvarasmiṁ pana bhikkhunā ubbhatasmiṁ kaṭhine\makeatletter\hyperlink{endnote-appendix}\Hy@raisedlink{\hypertarget{endnote-body}{}{\pagenote{%
		\hypertarget{endnote-appendix}{\hyperlink{endnote-body}{}}}}}\makeatother Dm: kathine., bhikkhuno pan'eva akālacīvaraṁ uppajjeyya, ākaṅkhamānena bhikkhunā paṭiggahetabbaṁ, paṭiggahetvā khippam'eva kāretabbaṁ. No c'assa pāripūri, māsaparamaṁ tena\makeatletter\hyperlink{endnote-appendix}\Hy@raisedlink{\hypertarget{endnote-body}{}{\pagenote{%
		\hypertarget{endnote-appendix}{\hyperlink{endnote-body}{}}}}}\makeatother Bh Pm 1 & 2, C, G, V, W, Vibh Ee, Mi & Mm Se: paraman-tena bhikkhunā taṁ cīvaraṁ nikkhipitabbaṁ ūnassa\makeatletter\hyperlink{endnote-appendix}\Hy@raisedlink{\hypertarget{endnote-body}{}{\pagenote{%
		\hypertarget{endnote-appendix}{\hyperlink{endnote-body}{}}}}}\makeatother G: ūṇassa. V: onassa. pāripūriyā, satiyā paccāsāya; tato ce uttariṁ\makeatletter\hyperlink{endnote-appendix}\Hy@raisedlink{\hypertarget{endnote-body}{}{\pagenote{%
		\hypertarget{endnote-appendix}{\hyperlink{endnote-body}{}}}}}\makeatother Dm, Um, UP: uttari. (Be reads uttariṁ in the Sp to Pāc 19, see Pāc 19. Pg: uttariṁ.) nikkhipeyya, satiyā'pi paccāsāya, nissaggiyaṁ pācittiyaṁ.



\pdfbookmark[3]{Nissaggiya Pācittiya 4}{np4}
\subsubsection*{\hyperref[forf-exp4]{Nissaggiya Pācittiya 4: Purāṇacīvarasikkhāpadaṁ}}
\label{np4}

\linkdest{endnote-body}
Yo pana bhikkhu aññātikāya\makeatletter\hyperlink{endnote-appendix}\Hy@raisedlink{\hypertarget{endnote-body}{}{\pagenote{%
		\hypertarget{endnote-appendix}{\hyperlink{endnote-body}{}}}}}\makeatother BhPm 2 (Syāma) v.l.: añātikāya. bhikkhuniyā purāṇacīvaraṁ dhovāpeyya vā rajāpeyya vā ākoṭāpeyya vā, nissaggiyaṁ pācittiyaṁ.



\pdfbookmark[3]{Nissaggiya Pācittiya 5}{np5}
\subsubsection*{\hyperref[forf-exp5]{Nissaggiya Pācittiya 5: Cīvarappaṭiggahaṇasikkhāpadaṁ}}
\label{np5}

\linkdest{endnote-body}
\linkdest{endnote-body}
\linkdest{endnote-body}
Yo pana bhikkhu aññātikāya\makeatletter\hyperlink{endnote-appendix}\Hy@raisedlink{\hypertarget{endnote-body}{}{\pagenote{%
		\hypertarget{endnote-appendix}{\hyperlink{endnote-body}{}}}}}\makeatother BhPm 2 (Syāma) v.l.: añātikāya. bhikkhuniyā hatthato cīvaraṁ paṭiggaṇheyya\makeatletter\hyperlink{endnote-appendix}\Hy@raisedlink{\hypertarget{endnote-body}{}{\pagenote{%
		\hypertarget{endnote-appendix}{\hyperlink{endnote-body}{}}}}}\makeatother BhPm 1: paṭigaṇheyya. BhPm 2, C, D, W, Ra: patigaṇheyya. (Cf NP 10, Pāc 34.), aññatra pārivattakā\makeatletter\hyperlink{endnote-appendix}\Hy@raisedlink{\hypertarget{endnote-body}{}{\pagenote{%
		\hypertarget{endnote-appendix}{\hyperlink{endnote-body}{}}}}}\makeatother Mi & Mm Se, Vibh Ce, UP, Ra, BhPm 1 & 2, C, D, G, V, W, Um, Pg: -vaṭṭakā., nissaggiyaṁ pācittiyaṁ.



\pdfbookmark[3]{Nissaggiya Pācittiya 6}{np6}
\subsubsection*{\hyperref[forf-exp6]{Nissaggiya Pācittiya 6: Aññātakaviññattisikkhāpadaṁ}}
\label{np6}

\linkdest{endnote-body}
Yo pana bhikkhu aññātakaṁ\makeatletter\hyperlink{endnote-appendix}\Hy@raisedlink{\hypertarget{endnote-body}{}{\pagenote{%
		\hypertarget{endnote-appendix}{\hyperlink{endnote-body}{}}}}}\makeatother G: aññātikaṁ. gahapatiṁ vā gahapatāniṁ vā cīvaraṁ viññāpeyya, aññatra samayā, nissaggiyaṁ pācittiyaṁ. Tatth'āyaṁ samayo: acchinnacīvaro vā hoti bhikkhu naṭṭhacīvaro vā; ayaṁ tattha samayo.



\pdfbookmark[3]{Nissaggiya Pācittiya 7}{np7}
\subsubsection*{\hyperref[forf-exp7]{Nissaggiya Pācittiya 7: Tat'uttarisikkhāpadaṁ}}
\label{np7}

\linkdest{endnote-body}
\linkdest{endnote-body}
\linkdest{endnote-body}
Tañ'ce aññātako gahapati vā gahapatānī vā bahūhi cīvarehi abhihaṭṭhuṁ pavāreyya\makeatletter\hyperlink{endnote-appendix}\Hy@raisedlink{\hypertarget{endnote-body}{}{\pagenote{%
		\hypertarget{endnote-appendix}{\hyperlink{endnote-body}{}}}}}\makeatother Mi Se, G: abhihaṭṭhum-pavāreyya; so at Pāc 34. V: abhihaṭṭham-pavāreyya. Mi Se v.l.: abhihaṭuṁ. , santar'uttaraparamaṁ\makeatletter\hyperlink{endnote-appendix}\Hy@raisedlink{\hypertarget{endnote-body}{}{\pagenote{%
		\hypertarget{endnote-appendix}{\hyperlink{endnote-body}{}}}}}\makeatother BhPm 1 & 2, C, D, G, Vibh Ee, Ra, Mi & Mm Se, V: -paraman-tena. tena bhikkhunā tato cīvaraṁ sāditabbaṁ; tato ce uttariṁ\makeatletter\hyperlink{endnote-appendix}\Hy@raisedlink{\hypertarget{endnote-body}{}{\pagenote{%
		\hypertarget{endnote-appendix}{\hyperlink{endnote-body}{}}}}}\makeatother NP3 sādiyeyya, nissaggiyaṁ pācittiyaṁ.



\pdfbookmark[3]{Nissaggiya Pācittiya 8}{np8}
\subsubsection*{\hyperref[forf-exp8]{Nissaggiya Pācittiya 8: Paṭhama-upakkhaṭasikkhāpadaṁ}}
\label{np8}

\linkdest{endnote-body}
\linkdest{endnote-body}
\linkdest{endnote-body}
\linkdest{endnote-body}
\linkdest{endnote-body}
\linkdest{endnote-body}
Bhikkhuṁ pan'eva uddissa aññātakassa\makeatletter\hyperlink{endnote-appendix}\Hy@raisedlink{\hypertarget{endnote-body}{}{\pagenote{%
		\hypertarget{endnote-appendix}{\hyperlink{endnote-body}{}}}}}\makeatother G: aññātikassa. C, W: aññātakagahapatissa. (In C corrected to aññātakassa.) gahapatissa vā gahapatāniyā vā cīvaracetāpanaṁ\makeatletter\hyperlink{endnote-appendix}\Hy@raisedlink{\hypertarget{endnote-body}{}{\pagenote{%
		\hypertarget{endnote-appendix}{\hyperlink{endnote-body}{}}}}}\makeatother BhPm 1 & 2, D, C, G, V, W, Dm, Um, UP, Ra, Vibh Ce: -cetāpann-. (Pg: -cetāpan- but has -cetāpann- in the next two rules...) upakkhaṭaṁ hoti: ``Iminā cīvaracetāpanena\makeatletter\hyperlink{endnote-appendix}\Hy@raisedlink{\hypertarget{endnote-body}{}{\pagenote{%
		\hypertarget{endnote-appendix}{\hyperlink{endnote-body}{}}}}}\makeatother See previous note.  cīvaraṁ cetāpetvā itthan'nāmaṁ\makeatletter\hyperlink{endnote-appendix}\Hy@raisedlink{\hypertarget{endnote-body}{}{\pagenote{%
		\hypertarget{endnote-appendix}{\hyperlink{endnote-body}{}}}}}\makeatother W, Um, Pg: itthaṁ nāmaṁ. bhikkhuṁ cīvarena acchādessāmī'ti.'' Tatra ce so bhikkhu pubbe appavārito upasaṅkamitvā cīvare vikappaṁ āpajjeyya: ``Sādhu vata maṁ āyasmā iminā cīvaracetāpanena\makeatletter\hyperlink{endnote-appendix}\Hy@raisedlink{\hypertarget{endnote-body}{}{\pagenote{%
		\hypertarget{endnote-appendix}{\hyperlink{endnote-body}{}}}}}\makeatother 270 See n. 135. evarūpaṁ vā evarūpaṁ vā cīvaraṁ cetāpetvā acchādehī'ti,'' kalyāṇakamyataṁ\makeatletter\hyperlink{endnote-appendix}\Hy@raisedlink{\hypertarget{endnote-body}{}{\pagenote{%
		\hypertarget{endnote-appendix}{\hyperlink{endnote-body}{}}}}}\makeatother G: kammyataṁ. The -y- seems to be a correction as it is cramped in between the -m- and -t-. upādāya, nissaggiyaṁ pācittiyaṁ.



\pdfbookmark[3]{Nissaggiya Pācittiya 9}{np9}
\subsubsection*{\hyperref[forf-exp9]{Nissaggiya Pācittiya 9: Dutiya-upakkhaṭasikkhāpadaṁ}}
\label{np9}

\linkdest{endnote-body}
\linkdest{endnote-body}
\linkdest{endnote-body}
\linkdest{endnote-body}
\linkdest{endnote-body}
\linkdest{endnote-body}
\linkdest{endnote-body}
Bhikkhuṁ pan'eva uddissa ubhinnaṁ aññātakānaṁ\makeatletter\hyperlink{endnote-appendix}\Hy@raisedlink{\hypertarget{endnote-body}{}{\pagenote{%
		\hypertarget{endnote-appendix}{\hyperlink{endnote-body}{}}}}}\makeatother G: aññātikānaṁ. gahapatīnaṁ\makeatletter\hyperlink{endnote-appendix}\Hy@raisedlink{\hypertarget{endnote-body}{}{\pagenote{%
		\hypertarget{endnote-appendix}{\hyperlink{endnote-body}{}}}}}\makeatother W: aññātakagahapatīnaṁ. vā gahapatānīnaṁ vā paccekacīvaracetāpanā upakkhaṭā\makeatletter\hyperlink{endnote-appendix}\Hy@raisedlink{\hypertarget{endnote-body}{}{\pagenote{%
		\hypertarget{endnote-appendix}{\hyperlink{endnote-body}{}}}}}\makeatother Dm, Um, UP, Ra: -cetāpannāni upakkhaṭāni. C, D, V, W, Vibh Ce, BhPm 1 & 2,
Pg: -cetāpannā upakkhaṭā. honti: ``Imehi mayaṁ paccekacīvaracetāpanehi\makeatletter\hyperlink{endnote-appendix}\Hy@raisedlink{\hypertarget{endnote-body}{}{\pagenote{%
		\hypertarget{endnote-appendix}{\hyperlink{endnote-body}{}}}}}\makeatother BhPm 1 & 2, C, D, W, Dm, UP, Ra, Vibh Ce, Pg: -cetāpannehi. paccekacīvarāni cetāpetvā itthan'nāmaṁ\makeatletter\hyperlink{endnote-appendix}\Hy@raisedlink{\hypertarget{endnote-body}{}{\pagenote{%
		\hypertarget{endnote-appendix}{\hyperlink{endnote-body}{}}}}}\makeatother W, Um: itthaṁ nāmaṁ. bhikkhuṁ cīvarehi acchādessāmā'ti.'' Tatra ce so bhikkhu pubbe appavārito upasaṅkamitvā cīvare vikappaṁ āpajjeyya: ``Sādhu vata maṁ āyasmanto imehi paccekacīvaracetāpanehi\makeatletter\hyperlink{endnote-appendix}\Hy@raisedlink{\hypertarget{endnote-body}{}{\pagenote{%
		\hypertarget{endnote-appendix}{\hyperlink{endnote-body}{}}}}}\makeatother BhPm 1 & 2, C, D, V, W, Dm, UP, Ra, Vibh Ce, Pg: -cetāpannehi. evarūpaṁ vā evarūpaṁ vā cīvaraṁ cetāpetvā acchādetha, ubho'va santā ekenā'ti,'' kalyāṇakamyataṁ\makeatletter\hyperlink{endnote-appendix}\Hy@raisedlink{\hypertarget{endnote-body}{}{\pagenote{%
		\hypertarget{endnote-appendix}{\hyperlink{endnote-body}{}}}}}\makeatother G: kammyataṁ. The -y- seems to be a correction as it is cramped in between the -m- and -t-. upādāya, nissaggiyaṁ pācittiyaṁ.



\pdfbookmark[3]{Nissaggiya Pācittiya 10}{np10}
\subsubsection*{\hyperref[forf-exp10]{Nissaggiya Pācittiya 10: Rājasikkhāpadaṁ}}
\label{np10}

\linkdest{endnote-body}
\linkdest{endnote-body}
\linkdest{endnote-body}
\linkdest{endnote-body}
\linkdest{endnote-body}
\linkdest{endnote-body}
\linkdest{endnote-body}
\linkdest{endnote-body}
\linkdest{endnote-body}
\linkdest{endnote-body}
\linkdest{endnote-body}
\linkdest{endnote-body}
\linkdest{endnote-body}
\linkdest{endnote-body}
\linkdest{endnote-body}
\linkdest{endnote-body}
\linkdest{endnote-body}
\linkdest{endnote-body}
\linkdest{endnote-body}
\linkdest{endnote-body}
\linkdest{endnote-body}
\linkdest{endnote-body}
\linkdest{endnote-body}
\linkdest{endnote-body}
\linkdest{endnote-body}
Bhikkhuṁ pan'eva uddissa rājā vā rājabhoggo\makeatletter\hyperlink{endnote-appendix}\Hy@raisedlink{\hypertarget{endnote-body}{}{\pagenote{%
		\hypertarget{endnote-appendix}{\hyperlink{endnote-body}{}}}}}\makeatother V: -bhogo.  vā brāhmaṇo vā gahapatiko vā dūtena cīvaracetāpanaṁ\makeatletter\hyperlink{endnote-appendix}\Hy@raisedlink{\hypertarget{endnote-body}{}{\pagenote{%
		\hypertarget{endnote-appendix}{\hyperlink{endnote-body}{}}}}}\makeatother BhPm 1 & 2, D, C, G, V, W, Dm, Um, UP, Ra, Vibh Ce: -cetāpann-.  pahiṇeyya: ``Iminā cīvaracetāpanena\makeatletter\hyperlink{endnote-appendix}\Hy@raisedlink{\hypertarget{endnote-body}{}{\pagenote{%
		\hypertarget{endnote-appendix}{\hyperlink{endnote-body}{}}}}}\makeatother See Previous cīvaraṁ cetāpetvā itthan'nāmaṁ\makeatletter\hyperlink{endnote-appendix}\Hy@raisedlink{\hypertarget{endnote-body}{}{\pagenote{%
		\hypertarget{endnote-appendix}{\hyperlink{endnote-body}{}}}}}\makeatother W, Um: itthaṁ nāmaṁ. bhikkhuṁ cīvarena acchādehī'ti.'' So ce dūto taṁ bhikkhuṁ upasaṅkamitvā evaṁ vadeyya: ``Idaṁ kho bhante āyasmantaṁ uddissa cīvaracetāpanaṁ ābhataṁ, paṭiggaṇhātu\makeatletter\hyperlink{endnote-appendix}\Hy@raisedlink{\hypertarget{endnote-body}{}{\pagenote{%
		\hypertarget{endnote-appendix}{\hyperlink{endnote-body}{}}}}}\makeatother BhPm 1 & 2, C, D, W, Dm, UP, Ra: patigaṇh-. āyasmā cīvaracetāpanan''ti\makeatletter\hyperlink{endnote-appendix}\Hy@raisedlink{\hypertarget{endnote-body}{}{\pagenote{%
		\hypertarget{endnote-appendix}{\hyperlink{endnote-body}{}}}}}\makeatother BhPm 1 & 2, D, C, G, V, W, Dm, Um, UP, Ra, Vibh Ce: -cetāpann-., tena bhikkhunā so dūto evam'assa vacanīyo: ``Na kho mayaṁ āvuso cīvaracetāpanaṁ\makeatletter\hyperlink{endnote-appendix}\Hy@raisedlink{\hypertarget{endnote-body}{}{\pagenote{%
		\hypertarget{endnote-appendix}{\hyperlink{endnote-body}{}}}}}\makeatother See Previous paṭiggaṇhāma\makeatletter\hyperlink{endnote-appendix}\Hy@raisedlink{\hypertarget{endnote-body}{}{\pagenote{%
		\hypertarget{endnote-appendix}{\hyperlink{endnote-body}{}}}}}\makeatother Vibh Ce: paṭigaṇh-. BhPm 1 & 2, C, D, W, Dm, UP, Ra: patigaṇh-., cīvarañ'ca kho mayaṁ paṭiggaṇhāma\makeatletter\hyperlink{endnote-appendix}\Hy@raisedlink{\hypertarget{endnote-body}{}{\pagenote{%
		\hypertarget{endnote-appendix}{\hyperlink{endnote-body}{}}}}}\makeatother Previous kālena kappiyan'ti.'' So ce dūto taṁ bhikkhuṁ evaṁ vadeyya: ``Atthi pan'āyasmato koci veyyāvaccakaro'ti,'' cīvar'atthikena, bhikkhave, bhikkhunā veyyāvaccakaro niddisitabbo ārāmiko vā upāsako vā: ``Eso kho āvuso bhikkhūnaṁ veyyāvaccakaro'ti.'' So ce dūto taṁ veyyāvaccakaraṁ saññāpetvā taṁ bhikkhuṁ upasaṅkamitvā evaṁ vadeyya: ``Yaṁ kho bhante āyasmā veyyāvaccakaraṁ niddisi, saññatto so mayā. Upasaṅkamatu\makeatletter\hyperlink{endnote-appendix}\Hy@raisedlink{\hypertarget{endnote-body}{}{\pagenote{%
		\hypertarget{endnote-appendix}{\hyperlink{endnote-body}{}}}}}\makeatother Dm, Um: upasaṅkamatāyasmā. āyasmā kālena, cīvarena taṁ acchādessatī''ti, cīvar'atthikena bhikkhave bhikkhunā veyyāvaccakaro upasaṅkamitvā dvattikkhattuṁ\makeatletter\hyperlink{endnote-appendix}\Hy@raisedlink{\hypertarget{endnote-body}{}{\pagenote{%
		\hypertarget{endnote-appendix}{\hyperlink{endnote-body}{}}}}}\makeatother Vibh Ee, Mi & Mm Se, Pg: dvi-. (Cf Pāc 19 & 34: dvitti-/dvatti-.) codetabbo sāretabbo: ``Attho me āvuso cīvarenā'ti.'' Dvattikkhattuṁ\makeatletter\hyperlink{endnote-appendix}\Hy@raisedlink{\hypertarget{endnote-body}{}{\pagenote{%
		\hypertarget{endnote-appendix}{\hyperlink{endnote-body}{}}}}}\makeatother Prev codayamāno sārayamāno\makeatletter\hyperlink{endnote-appendix}\Hy@raisedlink{\hypertarget{endnote-body}{}{\pagenote{%
		\hypertarget{endnote-appendix}{\hyperlink{endnote-body}{}}}}}\makeatother D, G, Vibh Ee, Um, V: codiyamāno sāriyamāno. C, W: codiyamāno sārayamāno taṁ cīvaraṁ abhinipphādeyya, icc'etaṁ kusalaṁ. No ce abhinipphādeyya, catukkhattuṁ pañcakkhattuṁ chakkhattu'paramaṁ\makeatletter\hyperlink{endnote-appendix}\Hy@raisedlink{\hypertarget{endnote-body}{}{\pagenote{%
		\hypertarget{endnote-appendix}{\hyperlink{endnote-body}{}}}}}\makeatother \makeatletter\hyperlink{endnote-appendix}\Hy@raisedlink{\hypertarget{endnote-body}{}{\pagenote{%
		\hypertarget{endnote-appendix}{\hyperlink{endnote-body}{}}}}}\makeatother BhPm 2, C, D, G, V, W, Vibh Ce: chakkhattuṁ paramaṁ.  tuṇhībhūtena uddissa ṭhātabbaṁ\makeatletter\hyperlink{endnote-appendix}\Hy@raisedlink{\hypertarget{endnote-body}{}{\pagenote{%
		\hypertarget{endnote-appendix}{\hyperlink{endnote-body}{}}}}}\makeatother V: thātabbaṁ.. Catukkhattuṁ pañcakkhattuṁ chakkhattu'paramaṁ tuṇhībhūto\makeatletter\hyperlink{endnote-appendix}\Hy@raisedlink{\hypertarget{endnote-body}{}{\pagenote{%
		\hypertarget{endnote-appendix}{\hyperlink{endnote-body}{}}}}}\makeatother  Vibh Ee, Ra: tuṇhi-. (Um illegible.) uddissa tiṭṭhamāno taṁ cīvaraṁ abhinipphādeyya, icc'etaṁ kusalaṁ\makeatletter\hyperlink{endnote-appendix}\Hy@raisedlink{\hypertarget{endnote-body}{}{\pagenote{%
		\hypertarget{endnote-appendix}{\hyperlink{endnote-body}{}}}}}\makeatother Mm, Mi Se, D, G, Ra, V: .”… kusalaṁ. No ce abhinipphādeyya. Tato ce uttariṁ....” Other eds.: .”… kusalaṁ. Tato ce uttariṁ...” (Um: tato ca uttari ...)
(Pg: ..”. kusalaṁ. Tato ... uttariṁ vāyamamāno ...” The Sannē also leaves out no ce abhinipphādeyya.); tato ce uttariṁ\makeatletter\hyperlink{endnote-appendix}\Hy@raisedlink{\hypertarget{endnote-body}{}{\pagenote{%
		\hypertarget{endnote-appendix}{\hyperlink{endnote-body}{}}}}}\makeatother Dm, Um, UP: uttari. See NP 3. vāyamamāno\makeatletter\hyperlink{endnote-appendix}\Hy@raisedlink{\hypertarget{endnote-body}{}{\pagenote{%
		\hypertarget{endnote-appendix}{\hyperlink{endnote-body}{}}}}}\makeatother C, D, G, V: vāyamāno. taṁ cīvaraṁ abhinipphādeyya, nissaggiyaṁ pācittiyaṁ. No ce abhinipphādeyya, yat'assa\makeatletter\hyperlink{endnote-appendix}\Hy@raisedlink{\hypertarget{endnote-body}{}{\pagenote{%
		\hypertarget{endnote-appendix}{\hyperlink{endnote-body}{}}}}}\makeatother  G: yaṁ tassa. cīvaracetāpanaṁ\makeatletter\hyperlink{endnote-appendix}\Hy@raisedlink{\hypertarget{endnote-body}{}{\pagenote{%
		\hypertarget{endnote-appendix}{\hyperlink{endnote-body}{}}}}}\makeatother BhPm 1 & 2, D, C, G, V, W, Dm, Um, UP, Ra, Vibh Ce: -cetāpann-. ābhataṁ, tattha sāmaṁ vā gantabbaṁ dūto vā pāhetabbo: ``Yaṁ kho tumhe āyasmanto bhikkhuṁ uddissa cīvaracetāpanaṁ pahiṇittha\makeatletter\hyperlink{endnote-appendix}\Hy@raisedlink{\hypertarget{endnote-body}{}{\pagenote{%
		\hypertarget{endnote-appendix}{\hyperlink{endnote-body}{}}}}}\makeatother G: pahinittha., na taṁ tassa\makeatletter\hyperlink{endnote-appendix}\Hy@raisedlink{\hypertarget{endnote-body}{}{\pagenote{%
		\hypertarget{endnote-appendix}{\hyperlink{endnote-body}{}}}}}\makeatother Mi & Mm Se, G, P: tan-tassa. bhikkhuno kiñci atthaṁ anubhoti, yuñjant'āyasmanto sakaṁ, mā vo sakaṁ vinassā'ti.\makeatletter\hyperlink{endnote-appendix}\Hy@raisedlink{\hypertarget{endnote-body}{}{\pagenote{%
		\hypertarget{endnote-appendix}{\hyperlink{endnote-body}{}}}}}\makeatother Mm & Mi Se: vinassī. (Pg: vinassā.)'' Ayaṁ tattha sāmīci.

\linkdest{endnote-body}
\linkdest{endnote-body}
\begin{center}
	Cīvaravaggo paṭhamo\makeatletter\hyperlink{endnote-appendix}\Hy@raisedlink{\hypertarget{endnote-body}{}{\pagenote{%
		\hypertarget{endnote-appendix}{\hyperlink{endnote-body}{}}}}}\makeatother Vibh Ee: kaṭhinavaggo. Dm: kathinavaggo.\makeatletter\hyperlink{endnote-appendix}\Hy@raisedlink{\hypertarget{endnote-body}{}{\pagenote{%
		\hypertarget{endnote-appendix}{\hyperlink{endnote-body}{}}}}}\makeatother V: pathamo.
\end{center}



\subsection{Eḷakalomavaggo}
% \vspace{0.2cm}

\pdfbookmark[3]{Nissaggiya Pācittiya 11}{np11}
\subsubsection*{\hyperref[forf-exp11]{Nissaggiya Pācittiya 11: Kosiyasikkhāpadaṁ}}
\label{np11}

Yo pana bhikkhu kosiyamissakaṁ santhataṁ kārāpeyya, nissaggiyaṁ pācittiyaṁ.



\pdfbookmark[3]{Nissaggiya Pācittiya 12}{np12}
\subsubsection*{\hyperref[forf-exp12]{Nissaggiya Pācittiya 12: Suddhakāḷakasikkhāpadaṁ}}
\label{np12}

\linkdest{endnote-body}
Yo pana bhikkhu suddhakāḷakānaṁ eḷakalomānaṁ santhataṁ\makeatletter\hyperlink{endnote-appendix}\Hy@raisedlink{\hypertarget{endnote-body}{}{\pagenote{%
		\hypertarget{endnote-appendix}{\hyperlink{endnote-body}{}}}}}\makeatother V: saṇṭhataṁ. kārāpeyya, nissaggiyaṁ pācittiyaṁ.



\pdfbookmark[3]{Nissaggiya Pācittiya 13}{np13}
\subsubsection*{\hyperref[forf-exp13]{Nissaggiya Pācittiya 13: Dvebhāgasikkhāpadaṁ}}
\label{np13}

\linkdest{endnote-body}
\linkdest{endnote-body}
Navaṁ pana***Mi Se, C, G, V, W: navam-pana. bhikkhunā santhataṁ\makeatletter\hyperlink{endnote-appendix}\Hy@raisedlink{\hypertarget{endnote-body}{}{\pagenote{%
		\hypertarget{endnote-appendix}{\hyperlink{endnote-body}{}}}}}\makeatother V: saṇṭhataṁ. kārayamānena dve bhāgā suddhakāḷakānaṁ eḷakalomānaṁ ādātabbā, tatiyaṁ odātānaṁ catutthaṁ gocariyānaṁ. Anādā ce bhikkhu dve bhāge suddhakāḷakānaṁ eḷakalomānaṁ tatiyaṁ odātānaṁ catutthaṁ gocariyānaṁ navaṁ santhataṁ\makeatletter\hyperlink{endnote-appendix}\Hy@raisedlink{\hypertarget{endnote-body}{}{\pagenote{%
		\hypertarget{endnote-appendix}{\hyperlink{endnote-body}{}}}}}\makeatother V: saṇṭhataṁ. kārāpeyya, nissaggiyaṁ pācittiyaṁ.



\pdfbookmark[3]{Nissaggiya Pācittiya 14}{np14}
\subsubsection*{\hyperref[forf-exp14]{Nissaggiya Pācittiya 14: Chabbassasikkhāpadaṁ}}
\label{np14}

\linkdest{endnote-body}
\linkdest{endnote-body}
\linkdest{endnote-body}
\linkdest{endnote-body}
\linkdest{endnote-body}
\linkdest{endnote-body}
\linkdest{endnote-body}
\linkdest{endnote-body}
Navaṁ pana\makeatletter\hyperlink{endnote-appendix}\Hy@raisedlink{\hypertarget{endnote-body}{}{\pagenote{%
		\hypertarget{endnote-appendix}{\hyperlink{endnote-body}{}}}}}\makeatother Mi Se, C, G, V, W: navam-pana. bhikkhunā santhataṁ\makeatletter\hyperlink{endnote-appendix}\Hy@raisedlink{\hypertarget{endnote-body}{}{\pagenote{%
		\hypertarget{endnote-appendix}{\hyperlink{endnote-body}{}}}}}\makeatother V: saṇṭhataṁ. kārāpetvā chabbassāni dhāretabbaṁ. Orena ce\makeatletter\hyperlink{endnote-appendix}\Hy@raisedlink{\hypertarget{endnote-body}{}{\pagenote{%
		\hypertarget{endnote-appendix}{\hyperlink{endnote-body}{}}}}}\makeatother BhPm 1 & 2, C, W, Ra, UP v.l., Vibh Ce v.l. (& correction in G): orena ce bhikkhu. D: orena ca channaṁ. channaṁ vassānaṁ taṁ santhataṁ\makeatletter\hyperlink{endnote-appendix}\Hy@raisedlink{\hypertarget{endnote-body}{}{\pagenote{%
		\hypertarget{endnote-appendix}{\hyperlink{endnote-body}{}}}}}\makeatother V: saṇṭhataṁ. visajjetvā\makeatletter\hyperlink{endnote-appendix}\Hy@raisedlink{\hypertarget{endnote-body}{}{\pagenote{%
		\hypertarget{endnote-appendix}{\hyperlink{endnote-body}{}}}}}\makeatother V: visajjetvā. Other eds.: vissajjetvā. vā avisajjetvā\makeatletter\hyperlink{endnote-appendix}\Hy@raisedlink{\hypertarget{endnote-body}{}{\pagenote{%
		\hypertarget{endnote-appendix}{\hyperlink{endnote-body}{}}}}}\makeatother V: avisajjetvā. vā aññaṁ navaṁ santhataṁ\makeatletter\hyperlink{endnote-appendix}\Hy@raisedlink{\hypertarget{endnote-body}{}{\pagenote{%
		\hypertarget{endnote-appendix}{\hyperlink{endnote-body}{}}}}}\makeatother V: saṇṭhataṁ. kārāpeyya, aññatra bhikkhusammutiyā\makeatletter\hyperlink{endnote-appendix}\Hy@raisedlink{\hypertarget{endnote-body}{}{\pagenote{%
		\hypertarget{endnote-appendix}{\hyperlink{endnote-body}{}}}}}\makeatother Mi & Mm Se, BhPm 1 & 2: sammatiyā. See NP 3. (Pg: -sammutiyā.), nissaggiyaṁ pācittiyaṁ.



\pdfbookmark[3]{Nissaggiya Pācittiya 15}{np15}
\subsubsection*{\hyperref[forf-exp15]{Nissaggiya Pācittiya 15: Nisīdanasanthatasikkhāpadaṁ}}
\label{np15}

\linkdest{endnote-body}
\linkdest{endnote-body}
\linkdest{endnote-body}
\linkdest{endnote-body}
\linkdest{endnote-body}
Nisīdanasanthataṁ\makeatletter\hyperlink{endnote-appendix}\Hy@raisedlink{\hypertarget{endnote-body}{}{\pagenote{%
		\hypertarget{endnote-appendix}{\hyperlink{endnote-body}{}}}}}\makeatother V: -saṇṭhatam-. pana\makeatletter\hyperlink{endnote-appendix}\Hy@raisedlink{\hypertarget{endnote-body}{}{\pagenote{%
		\hypertarget{endnote-appendix}{\hyperlink{endnote-body}{}}}}}\makeatother Mi Se, G: nisīdanasanthatam-pana. V: nisīdanasaṇṭhatam-pana. bhikkhunā kārayamānena purāṇasanthatassa\makeatletter\hyperlink{endnote-appendix}\Hy@raisedlink{\hypertarget{endnote-body}{}{\pagenote{%
		\hypertarget{endnote-appendix}{\hyperlink{endnote-body}{}}}}}\makeatother V: -saṇṭhata-. sāmantā sugatavidatthi\makeatletter\hyperlink{endnote-appendix}\Hy@raisedlink{\hypertarget{endnote-body}{}{\pagenote{%
		\hypertarget{endnote-appendix}{\hyperlink{endnote-body}{}}}}}\makeatother Vibh Ce: -vidatthī. ādātabbā dubbaṇṇakaraṇāya. Anādā ce bhikkhu purāṇasanthatassa sāmantā sugatavidatthiṁ navaṁ nisīdanasanthataṁ\makeatletter\hyperlink{endnote-appendix}\Hy@raisedlink{\hypertarget{endnote-body}{}{\pagenote{%
		\hypertarget{endnote-appendix}{\hyperlink{endnote-body}{}}}}}\makeatother V: -saṇṭhataṁ. kārāpeyya, nissaggiyaṁ pācittiyaṁ.



\pdfbookmark[3]{Nissaggiya Pācittiya 16}{np16}
\subsubsection*{\hyperref[forf-exp16]{Nissaggiya Pācittiya 16: Eḷakalomasikkhāpadaṁ}}
\label{np16}

\linkdest{endnote-body}
\linkdest{endnote-body}
\linkdest{endnote-body}
Bhikkhuno pan'eva addhānamaggappaṭipannassa\makeatletter\hyperlink{endnote-appendix}\Hy@raisedlink{\hypertarget{endnote-body}{}{\pagenote{%
		\hypertarget{endnote-appendix}{\hyperlink{endnote-body}{}}}}}\makeatother Mi & Mm Se, BhPm 1 & 2, C, D, V, W, Um, UP, Ra, Vibh Ee: maggapaṭi-. Vibh Ce, Dm: -maggappaṭi-. G: addhānamaggaṁ
paṭipannassa. eḷakalomāni uppajjeyyuṁ, ākaṅkhamānena bhikkhunā paṭiggahetabbāni, paṭiggahetvā tiyojanaparamaṁ sahatthā haritabbāni\makeatletter\hyperlink{endnote-appendix}\Hy@raisedlink{\hypertarget{endnote-body}{}{\pagenote{%
		\hypertarget{endnote-appendix}{\hyperlink{endnote-body}{}}}}}\makeatother BhPm 1, C, D, G, V, W, Um, Vibh Ee, Mi & Mm Se:  hāretabbāni. Pg has  hāritabbāni in its explanation, but states that
hāretabbāni is a v.l., asante hārake; tato ce uttariṁ\makeatletter\hyperlink{endnote-appendix}\Hy@raisedlink{\hypertarget{endnote-body}{}{\pagenote{%
		\hypertarget{endnote-appendix}{\hyperlink{endnote-body}{}}}}}\makeatother Dm, Um, UP: uttari. See NP 3. hareyya asante'pi hārake, nissaggiyaṁ pācittiyaṁ.



\pdfbookmark[3]{Nissaggiya Pācittiya 17}{np17}
\subsubsection*{\hyperref[forf-exp17]{Nissaggiya Pācittiya 17: Eḷakalomadhovāpanasikkhāpadaṁ}}
\label{np17}

Yo pana bhikkhu aññātikāya bhikkhuniyā eḷakalomāni dhovāpeyya vā rajāpeyya vā vijaṭāpeyya vā, nissaggiyaṁ pācittiyaṁ.



\pdfbookmark[3]{Nissaggiya Pācittiya 18}{np18}
\subsubsection*{\hyperref[forf-exp18]{Nissaggiya Pācittiya 18: Rūpiyasikkhāpadaṁ}}
\label{np18}

Yo pana bhikkhu jātarūparajataṁ uggaṇheyya vā uggaṇhāpeyya vā upanikkhittaṁ vā sādiyeyya, nissaggiyaṁ pācittiyaṁ.



\pdfbookmark[3]{Nissaggiya Pācittiya 19}{np19}
\subsubsection*{\hyperref[forf-exp19]{Nissaggiya Pācittiya 19: Rūpiyasaṁvohārasikkhāpadaṁ}}
\label{np19}

Yo pana bhikkhu nānappakārakaṁ rūpiyasaṁvohāraṁ samāpajjeyya, nissaggiyaṁ pācittiyaṁ.



\pdfbookmark[3]{Nissaggiya Pācittiya 20}{np20}
\subsubsection*{\hyperref[forf-exp20]{Nissaggiya Pācittiya 20: Kayavikkayasikkhāpadaṁ}}
\label{np20}

Yo pana bhikkhu nānappakārakaṁ kayavikkayaṁ samāpajjeyya, nissaggiyaṁ pācittiyaṁ.

\linkdest{endnote-body}
\begin{center}
	Eḷakalomavaggo dutiyo\makeatletter\hyperlink{endnote-appendix}\Hy@raisedlink{\hypertarget{endnote-body}{}{\pagenote{%
		\hypertarget{endnote-appendix}{\hyperlink{endnote-body}{}}}}}\makeatother D, Dm, G, Mi & Mm Se, V, Vibh Ce, Vibh Ee: kosiyavaggo.
UP, BhPm 1 & 2, C, W, Um, Ra, Mi Se v.l. & UP sīhala v.l., Burmese v.l. in TP (from a 1904 Burmese printed edition):
eḷakalomavaggo. (This reading is also found in the Kkh [Be, Ce, Ee] and the  Sanna.) Pg:  santhatavaggo. (The editor of the
Sinhalese Pg edition says in a footnote that eḷakalomavagga is in the PāḷI, i.e., the Pātimokkha.) See the note on the chapter
titles in the Analysis.
\end{center}



\subsection{Pattavaggo}
% \vspace{0.2cm}

\pdfbookmark[3]{Nissaggiya Pācittiya 21}{np21}
\subsubsection*{\hyperref[forf-exp21]{Nissaggiya Pācittiya 21: Pattasikkhāpadaṁ}}
\label{np21}

Das'āhaparamaṁ atirekapatto dhāretabbo. Taṁ atikkāmayato, nissaggiyaṁ pācittiyaṁ.



\pdfbookmark[3]{Nissaggiya Pācittiya 22}{np22}
\subsubsection*{\hyperref[forf-exp22]{Nissaggiya Pācittiya 22: Ūnapañcabandhanasikkhāpadaṁ}}
\label{np22}

\linkdest{endnote-body}
\linkdest{endnote-body}
\linkdest{endnote-body}
\linkdest{endnote-body}
Yo pana bhikkhu ūnapañcabandhanena\makeatletter\hyperlink{endnote-appendix}\Hy@raisedlink{\hypertarget{endnote-body}{}{\pagenote{%
		\hypertarget{endnote-appendix}{\hyperlink{endnote-body}{}}}}}\makeatother G: ūṇa-. V: ona-. pattena aññaṁ navaṁ pattaṁ cetāpeyya, nissaggiyaṁ pācittiyaṁ. Tena bhikkhunā so patto bhikkhuparisāya nissajitabbo\makeatletter\hyperlink{endnote-appendix}\Hy@raisedlink{\hypertarget{endnote-body}{}{\pagenote{%
		\hypertarget{endnote-appendix}{\hyperlink{endnote-body}{}}}}}\makeatother BhPm 1 & 2, C, D, W, Ra, Vibh Ce, UP sīhala v.l.: nissajitabbo. Other eds.: nissajjitabbo., yo ca tassā bhikkhuparisāya pattapariyanto, so\makeatletter\hyperlink{endnote-appendix}\Hy@raisedlink{\hypertarget{endnote-body}{}{\pagenote{%
		\hypertarget{endnote-appendix}{\hyperlink{endnote-body}{}}}}}\makeatother Mi & Mm Se, G, V, D: ”… so ca tassa …” tassa bhikkhuno padātabbo: ``Ayaṁ te\makeatletter\hyperlink{endnote-appendix}\Hy@raisedlink{\hypertarget{endnote-body}{}{\pagenote{%
		\hypertarget{endnote-appendix}{\hyperlink{endnote-body}{}}}}}\makeatother Mi & Mm Se, C, G, V, W: ayan-te. bhikkhu patto, yāva bhedanāya dhāretabbo'ti. Ayaṁ tattha sāmīci.



\pdfbookmark[3]{Nissaggiya Pācittiya 23}{np23}
\subsubsection*{\hyperref[forf-exp23]{Nissaggiya Pācittiya 23: Bhesajjasikkhāpadaṁ}}
\label{np23}

\linkdest{endnote-body}
\linkdest{endnote-body}
\linkdest{endnote-body}
\linkdest{endnote-body}
Yāni kho pana tāni gilānānaṁ\makeatletter\hyperlink{endnote-appendix}\Hy@raisedlink{\hypertarget{endnote-body}{}{\pagenote{%
		\hypertarget{endnote-appendix}{\hyperlink{endnote-body}{}}}}}\makeatother V: gīlānānaṁ. bhikkhūnaṁ paṭisāyanīyāni bhesajjāni, seyyath'īdaṁ\makeatletter\hyperlink{endnote-appendix}\Hy@raisedlink{\hypertarget{endnote-body}{}{\pagenote{%
		\hypertarget{endnote-appendix}{\hyperlink{endnote-body}{}}}}}\makeatother Dm, UP: seyyathidaṁ. Cf Pāc 39.: sappi, navanītaṁ\makeatletter\hyperlink{endnote-appendix}\Hy@raisedlink{\hypertarget{endnote-body}{}{\pagenote{%
		\hypertarget{endnote-appendix}{\hyperlink{endnote-body}{}}}}}\makeatother V: navanitaṁ., telaṁ, madhuphāṇitaṁ\makeatletter\hyperlink{endnote-appendix}\Hy@raisedlink{\hypertarget{endnote-body}{}{\pagenote{%
		\hypertarget{endnote-appendix}{\hyperlink{endnote-body}{}}}}}\makeatother C: madhupphāṇitaṁ. G: madhuphāṇītaṁ; later (i.e., uninked) corrected to madhupphāṇītaṁ. Cf Pāc 39., tāni paṭiggahetvā satt'āhaparamaṁ sannidhikārakaṁ paribhuñjitabbāni. Taṁ atikkāmayato, nissaggiyaṁ pācittiyaṁ.



\pdfbookmark[3]{Nissaggiya Pācittiya 24}{np24}
\subsubsection*{\hyperref[forf-exp24]{Nissaggiya Pācittiya 24: Vassikasāṭikasikkhāpadaṁ}}
\label{np24}

\linkdest{endnote-body}
\linkdest{endnote-body}
``Māso seso gimhānan''ti, bhikkhunā vassikasāṭikacīvaraṁ pariyesitabbaṁ. ``Aḍḍhamāso\makeatletter\hyperlink{endnote-appendix}\Hy@raisedlink{\hypertarget{endnote-body}{}{\pagenote{%
		\hypertarget{endnote-appendix}{\hyperlink{endnote-body}{}}}}}\makeatother C, D, W, Dm, Vibh Ce, BhPm 1 & 2, Um, UP, Vibh Ee: addha-. In Pāc 57 the same editions have the same readings as in this
rule. (Pg: aḍḍha-.) V: aḍha- as in Pāc 57. seso gimhānan''ti, katvā nivāsetabbaṁ. ``Orena ce māso seso gimhānan''ti, vassikasāṭikacīvaraṁ pariyeseyya, ``Oren'aḍḍhamāso\makeatletter\hyperlink{endnote-appendix}\Hy@raisedlink{\hypertarget{endnote-body}{}{\pagenote{%
		\hypertarget{endnote-appendix}{\hyperlink{endnote-body}{}}}}}\makeatother C, D, W, Dm, Vibh Ce, BhPm 1 & 2, Um, UP, Vibh Ee: addha-.  seso gimhānan''ti, katvā nivāseyya, nissaggiyaṁ pācittiyaṁ.



\pdfbookmark[3]{Nissaggiya Pācittiya 25}{np25}
\subsubsection*{\hyperref[forf-exp25]{Nissaggiya Pācittiya 25: Cīvara-acchindanasikkhāpadaṁ}}
\label{np25}

\linkdest{endnote-body}
Yo pana bhikkhu bhikkhussa sāmaṁ cīvaraṁ datvā kupito\makeatletter\hyperlink{endnote-appendix}\Hy@raisedlink{\hypertarget{endnote-body}{}{\pagenote{%
		\hypertarget{endnote-appendix}{\hyperlink{endnote-body}{}}}}}\makeatother V: kuppito. (Cf NP Pāc 17 & 74.)
Bh Pm 1 & 2, C, D, W, Ra, UP sīhala v.l.: pacchā kupito.  anattamano acchindeyya vā acchindāpeyya vā, nissaggiyaṁ pācittiyaṁ.



\pdfbookmark[3]{Nissaggiya Pācittiya 26}{np26}
\subsubsection*{\hyperref[forf-exp26]{Nissaggiya Pācittiya 26: Suttaviññattisikkhāpadaṁ}}
\label{np26}

Yo pana bhikkhu sāmaṁ suttaṁ viññāpetvā tantavāyehi cīvaraṁ vāyāpeyya, nissaggiyaṁ pācittiyaṁ.



\pdfbookmark[3]{Nissaggiya Pācittiya 27}{np27}
\subsubsection*{\hyperref[forf-exp27]{Nissaggiya Pācittiya 27: Mahāpesakārasikkhāpadaṁ}}
\label{np27}

\linkdest{endnote-body}
\linkdest{endnote-body}
\linkdest{endnote-body}
\linkdest{endnote-body}
\linkdest{endnote-body}
\linkdest{endnote-body}
Bhikkhuṁ pan'eva uddissa aññātako gahapati vā gahapatānī vā tantavāyehi cīvaraṁ vāyāpeyya, tatra ce so bhikkhu pubbe appavārito tantavāye upasaṅkamitvā cīvare vikappaṁ āpajjeyya: ``Idaṁ kho āvuso cīvaraṁ maṁ uddissa viyyati\makeatletter\hyperlink{endnote-appendix}\Hy@raisedlink{\hypertarget{endnote-body}{}{\pagenote{%
		\hypertarget{endnote-appendix}{\hyperlink{endnote-body}{}}}}}\makeatother Mi & Mm Se, Bh Pm 1 & 2, C, D, W, Ra, UP v.l.: vīyati. Um: vīyyati. āyatañ'ca karotha, vitthatañ'ca appitañ'ca\makeatletter\hyperlink{endnote-appendix}\Hy@raisedlink{\hypertarget{endnote-body}{}{\pagenote{%
		\hypertarget{endnote-appendix}{\hyperlink{endnote-body}{}}}}}\makeatother Bh Pm 2, Um: appīta-. suvītañ'ca\makeatletter\hyperlink{endnote-appendix}\Hy@raisedlink{\hypertarget{endnote-body}{}{\pagenote{%
		\hypertarget{endnote-appendix}{\hyperlink{endnote-body}{}}}}}\makeatother  Mi & Mm Se, G, V: suvita-. suppavāyitañ'ca\makeatletter\hyperlink{endnote-appendix}\Hy@raisedlink{\hypertarget{endnote-body}{}{\pagenote{%
		\hypertarget{endnote-appendix}{\hyperlink{endnote-body}{}}}}}\makeatother Mi & Mm Se, V: supavāyita-. suvilekhitañ'ca\makeatletter\hyperlink{endnote-appendix}\Hy@raisedlink{\hypertarget{endnote-body}{}{\pagenote{%
		\hypertarget{endnote-appendix}{\hyperlink{endnote-body}{}}}}}\makeatother Mi & Mm Se, G, V: suvilekkhita-. suvitacchitañ'ca karotha; app'eva nāma mayam'pi\makeatletter\hyperlink{endnote-appendix}\Hy@raisedlink{\hypertarget{endnote-body}{}{\pagenote{%
		\hypertarget{endnote-appendix}{\hyperlink{endnote-body}{}}}}}\makeatother D, Vibh Ee: mayaṁ pi. āyasmantānaṁ kiñcimattaṁ anupadajjeyyāmā''ti, evañ'ca so bhikkhu vatvā kiñcimattaṁ anupadajjeyya, antamaso piṇḍapātamattam'pi, nissaggiyaṁ pācittiyaṁ.



\pdfbookmark[3]{Nissaggiya Pācittiya 28}{np28}
\subsubsection*{\hyperref[forf-exp28]{Nissaggiya Pācittiya 28: Accekacīvarasikkhāpadaṁ}}
\label{np28}

\linkdest{endnote-body}
\linkdest{endnote-body}
Das'āh'ānāgataṁ kattikatemāsikapuṇṇamaṁ,\makeatletter\hyperlink{endnote-appendix}\Hy@raisedlink{\hypertarget{endnote-body}{}{\pagenote{%
		\hypertarget{endnote-appendix}{\hyperlink{endnote-body}{}}}}}\makeatother Mi & Mm Se, Bh Pm 1 & 2, D, G, V, W, Ra, Vibh Ee, Pg: -māsi-. C, P, Dm & Vibh Ce: -māsika-. (In the Be Vimativinodani-
ṭīka (Be I 356, 360) on NP 24 and 28 there is also the reading -māsi-). bhikkhuno pan'eva accekacīvaraṁ uppajjeyya, accekaṁ maññamānena bhikkhunā paṭiggahetabbaṁ, paṭiggahetvā yāva cīvarakālasamayaṁ nikkhipitabbaṁ; tato ce uttariṁ\makeatletter\hyperlink{endnote-appendix}\Hy@raisedlink{\hypertarget{endnote-body}{}{\pagenote{%
		\hypertarget{endnote-appendix}{\hyperlink{endnote-body}{}}}}}\makeatother Dm, Um, UP: uttari. See NP 3. nikkhipeyya, nissaggiyaṁ pācittiyaṁ.



\pdfbookmark[3]{Nissaggiya Pācittiya 29}{np29}
\subsubsection*{\hyperref[forf-exp29]{Nissaggiya Pācittiya 29: Sāsaṅkasikkhāpadaṁ}}
\label{np29}

\linkdest{endnote-body}
\linkdest{endnote-body}
\linkdest{endnote-body}
\linkdest{endnote-body}
Upavassaṁ kho pana kattikapuṇṇamaṁ. Yāni kho pana tāni āraññakāni sen'āsanāni sāsaṅkasammatāni\makeatletter\hyperlink{endnote-appendix}\Hy@raisedlink{\hypertarget{endnote-body}{}{\pagenote{%
		\hypertarget{endnote-appendix}{\hyperlink{endnote-body}{}}}}}\makeatother C, G, W: -saṁka-. sappaṭibhayāni. Tathārūpesu bhikkhu sen'āsanesu viharanto ākaṅkhamāno tiṇṇaṁ cīvarānaṁ aññataraṁ cīvaraṁ antaraghare nikkhipeyya, siyā ca tassa bhikkhuno koci'd'eva paccayo tena cīvarena vippavāsāya, chārattaparamaṁ\makeatletter\hyperlink{endnote-appendix}\Hy@raisedlink{\hypertarget{endnote-body}{}{\pagenote{%
		\hypertarget{endnote-appendix}{\hyperlink{endnote-body}{}}}}}\makeatother Mi Se, G, V, W,  : chārattaparamantena. tena bhikkhunā tena cīvarena vippavasitabbaṁ; tato ce uttariṁ\makeatletter\hyperlink{endnote-appendix}\Hy@raisedlink{\hypertarget{endnote-body}{}{\pagenote{%
		\hypertarget{endnote-appendix}{\hyperlink{endnote-body}{}}}}}\makeatother Dm, Um, UP: uttari. See NP 3. vippavaseyya, aññatra bhikkhusammutiyā\makeatletter\hyperlink{endnote-appendix}\Hy@raisedlink{\hypertarget{endnote-body}{}{\pagenote{%
		\hypertarget{endnote-appendix}{\hyperlink{endnote-body}{}}}}}\makeatother Mi & Mm Se, BhPm 1 v.l.: sammatiyā. , nissaggiyaṁ pācittiyaṁ.



\pdfbookmark[3]{Nissaggiya Pācittiya 30}{np30}
\subsubsection*{\hyperref[forf-exp30]{Nissaggiya Pācittiya 30: Pariṇatasikkhāpadaṁ}}
\label{np30}

\linkdest{endnote-body}
\linkdest{endnote-body}
\linkdest{endnote-body}
Yo pana bhikkhu jānaṁ saṅghikaṁ\makeatletter\hyperlink{endnote-appendix}\Hy@raisedlink{\hypertarget{endnote-body}{}{\pagenote{%
		\hypertarget{endnote-appendix}{\hyperlink{endnote-body}{}}}}}\makeatother BhPm 1, C, V, W: saṁghikaṁ. lābhaṁ pariṇataṁ\makeatletter\hyperlink{endnote-appendix}\Hy@raisedlink{\hypertarget{endnote-body}{}{\pagenote{%
		\hypertarget{endnote-appendix}{\hyperlink{endnote-body}{}}}}}\makeatother C, D, W: -nataṁ.  attano pariṇāmeyya,\makeatletter\hyperlink{endnote-appendix}\Hy@raisedlink{\hypertarget{endnote-body}{}{\pagenote{%
		\hypertarget{endnote-appendix}{\hyperlink{endnote-body}{}}}}}\makeatother D, W: -nāmeyya. nissaggiyaṁ pācittiyaṁ.

\begin{center}
	Pattavaggo tatiyo
\end{center}



\medskip

\begin{center}
	Uddiṭṭhā kho āyasmanto tiṁsa nissaggiyā pācittiyā dhammā.

	\smallskip

	Tatth'āyasmante pucchāmi: Kacci'ttha parisuddhā?\\
	Dutiyam'pi pucchāmi: Kacci'ttha parisuddhā?\\
	Tatiyam'pi pucchāmi: Kacci'ttha parisuddhā?

	\smallskip

\linkdest{endnote-body}
	Parisuddh'etth'āyasmanto, tasmā tuṇhī, evam'etaṁ dhārayāmi.\makeatletter\hyperlink{endnote-appendix}\Hy@raisedlink{\hypertarget{endnote-body}{}{\pagenote{%
		\hypertarget{endnote-appendix}{\hyperlink{endnote-body}{}}}}}\makeatother Dm, UP, Ra, Um: dhārayāmī ti.
\end{center}

\linkdest{endnote-body}
\begin{outro}
	Tiṁsa nissaggiyā pācittiyā dhammā niṭṭhitā\makeatletter\hyperlink{endnote-appendix}\Hy@raisedlink{\hypertarget{endnote-body}{}{\pagenote{%
		\hypertarget{endnote-appendix}{\hyperlink{endnote-body}{}}}}}\makeatother Mm Se, Ra. Ñd Ce & Mi Se: Tiṁsa nissaggiyā pācittiyā dhammā niṭṭhitā. Bh Pm 1 & 2, UP, V: Nissaggiyā pācittiyā niṭṭhitā.
Dm,  Um:  Nissaggiyapācittiyā  niṭṭhitā.  C,  W:  Nissaggiyā  niṭṭhitā.  D  (also  Wae  Uda  Pm):  Nissaggiyaṁ  niṭṭhitaṁ. G:
Nissaggiyapācittiyaṁ niṭṭhitaṁ.
\end{outro}

\clearpage

