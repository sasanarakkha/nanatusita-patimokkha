
\section{Sekhiyā}
\label{sekh}

\linkdest{endnote469-body}
\begin{intro}
	Ime kho pan'āyasmanto sekhiyā\makeatletter\hyperlink{endnote469-appendix}\Hy@raisedlink{\hypertarget{endnote469-body}{}{\pagenote{%
		\hypertarget{endnote469-appendix}{\hyperlink{endnote469-body}{Mi Se, V: \textit{pañcasattati sekhiyā dhammā}.}}}}}\makeatother \thinspace dhammā uddesaṁ āgacchanti.
\end{intro}

\setsubsecheadstyle{\subsectionFmt}
\subsection{Parimaṇḍalavaggo}

\pdfbookmark[3]{Sekhiya 1 \& 2}{sekh1-2}
\subsubsection*{\hyperref[training1-2]{Sekhiya 1 \& 2: Parimaṇḍalasikkhāpadaṁ}}
\label{sekh1-2}

\linkdest{endnote470-body}
\linkdest{endnote471-body}
Parimaṇḍalaṁ nivāsessāmī'ti,\makeatletter\hyperlink{endnote470-appendix}\Hy@raisedlink{\hypertarget{endnote470-body}{}{\pagenote{%
		\hypertarget{endnote470-appendix}{\hyperlink{endnote470-body}{C, P: \textit{nivāsissāmī}.}}}}}\makeatother \thinspace sikkhā karaṇīyā.\makeatletter\hyperlink{endnote471-appendix}\Hy@raisedlink{\hypertarget{endnote471-body}{}{\pagenote{%
		\hypertarget{endnote471-appendix}{\hyperlink{endnote471-body}{V: \textit{karaṇiyā} throughout the Sekhiyā section.}}}}}\makeatother \thinspace\\
Parimaṇḍalaṁ pārupissāmī'ti sikkhā karaṇīyā.



\pdfbookmark[3]{Sekhiya 3 \& 4}{sekh3-4}
\subsubsection*{\hyperref[training3-4]{Sekhiya 3 \& 4: Suppaṭicchannasikkhāpadaṁ}}
\label{sekh3-4}

\linkdest{endnote472-body}
Supaṭicchanno\makeatletter\hyperlink{endnote472-appendix}\Hy@raisedlink{\hypertarget{endnote472-body}{}{\pagenote{%
		\hypertarget{endnote472-appendix}{\hyperlink{endnote472-body}{Dm: \textit{suppaṭicchanno}. (Pg: \textit{supāṭi-})}}}}}\makeatother \thinspace antaraghare gamissāmī'ti, sikkhā karaṇīyā.\\
Supaṭicchanno antaraghare nisīdissāmī'ti, sikkhā karaṇīyā.



\pdfbookmark[3]{Sekhiya 5 \& 6}{sekh5-6}
\subsubsection*{\hyperref[training5-6]{Sekhiya 5 \& 6: Susaṁvutasikkhāpadaṁ}}
\label{sekh5-6}

Susaṁvuto antaraghare gamissāmī'ti sikkhā karaṇīyā.\\
Susaṁvuto antaraghare nisīdissāmī'ti sikkhā karaṇīyā.



\pdfbookmark[3]{Sekhiya 7 \& 8}{sekh7-8}
\subsubsection*{\hyperref[training7-8]{Sekhiya 7 \& 8: Okkhittacakkhusikkhāpadaṁ}}
\label{sekh7-8}

Okkhittacakkhu antaraghare gamissāmī'ti sikkhā karaṇīyā.\\
Okkhittacakkhu antaraghare nisīdissāmī'ti sikkhā karaṇīyā.



\pdfbookmark[3]{Sekhiya 9 \& 10}{sekh9-10}
\subsubsection*{\hyperref[training9-10]{Sekhiya 9 \& 10: Ukkhittakasikkhāpadaṁ}}
\label{sekh9-10}

\linkdest{endnote473-body}
Na ukkhittakāya\makeatletter\hyperlink{endnote473-appendix}\Hy@raisedlink{\hypertarget{endnote473-body}{}{\pagenote{%
		\hypertarget{endnote473-appendix}{\hyperlink{endnote473-body}{D, P: \textit{-kāyaṁ}.}}}}}\makeatother \thinspace antaraghare gamissāmī'ti, sikkhā karaṇīyā.\\
Na ukkhittakāya antaraghare nisīdissāmī'ti, sikkhā karaṇīyā.

\linkdest{endnote474-body}
\begin{center}
	Parimaṇḍalavaggo paṭhamo\makeatletter\hyperlink{endnote474-appendix}\Hy@raisedlink{\hypertarget{endnote474-body}{}{\pagenote{%
		\hypertarget{endnote474-appendix}{\hyperlink{endnote474-body}{V: pathamo. Bh Pm 1 \& 2, C, D, W: \textit{Paṭhamo vaggo}. Nothing in Mm Se.}}}}}\makeatother \thinspace
\end{center}



\subsection{Ujjagghikavaggo}
% \vspace{0.2cm}

\pdfbookmark[3]{Sekhiya 11 \& 12}{sekh11-12}
\subsubsection*{\hyperref[training11-12]{Sekhiya 11 \& 12: Ujjagghikasikkhāpadaṁ}}
\label{sekh11-12}

\linkdest{endnote475-body}
Na ujjagghikāya\makeatletter\hyperlink{endnote475-appendix}\Hy@raisedlink{\hypertarget{endnote475-body}{}{\pagenote{%
		\hypertarget{endnote475-appendix}{\hyperlink{endnote475-body}{C, D, G, Um, V, SVibh Ee, W: \textit{ujjhaggi-}.}}}}}\makeatother \thinspace antaraghare gamissāmī'ti, sikkhā karaṇīyā.\\
Na ujjagghikāya antaraghare nisīdissāmī'ti, sikkhā karaṇīyā.



\pdfbookmark[3]{Sekhiya 13 \& 14}{sekh13-14}
\subsubsection*{\hyperref[training13-14]{Sekhiya 13 \& 14: Uccasaddasikkhāpadaṁ}}
\label{sekh13-14}

Appasaddo antaraghare gamissāmī'ti sikkhā karaṇīyā.\\
Appasaddo antaraghare nisīdissāmī'ti sikkhā karaṇīyā.



\pdfbookmark[3]{Sekhiya 15 \& 16}{sekh15-16}
\subsubsection*{\hyperref[training15-16]{Sekhiya 15 \& 16: Kāyappacālakasikkhāpadaṁ}}
\label{sekh15-16}

Na kāyappacālakaṁ antaraghare gamissāmī'ti sikkhā karaṇīyā.\\
Na kāyappacālakaṁ antaraghare nisīdissāmī'ti sikkhā karaṇīyā.



\pdfbookmark[3]{Sekhiya 17 \& 18}{sekh17-18}
\subsubsection*{\hyperref[training17-18]{Sekhiya 17 \& 18: Bāhuppacālakasikkhāpadaṁ}}
\label{sekh17-18}

Na bāhuppacālakaṁ antaraghare gamissāmī'ti sikkhā karaṇīyā.\\
Na bāhuppacālakaṁ antaraghare nisīdissāmī'ti sikkhā karaṇīyā.



\pdfbookmark[3]{Sekhiya 19 \& 20}{sekh19-20}
\subsubsection*{\hyperref[training19-20]{Sekhiya 19 \& 20: Sīsappacālakasikkhāpadaṁ}}
\label{sekh19-20}

Na sīsappacālakaṁ antaraghare gamissāmī'ti sikkhā karaṇīyā.\\
Na sīsappacālakaṁ antaraghare nisīdissāmī'ti sikkhā karaṇīyā.

\linkdest{endnote476-body}
\linkdest{endnote477-body}
\begin{center}
	Ujjagghikavaggo\makeatletter\hyperlink{endnote476-appendix}\Hy@raisedlink{\hypertarget{endnote476-body}{}{\pagenote{%
		\hypertarget{endnote476-appendix}{\hyperlink{endnote476-body}{Um: ujjhaggika-. G, V: \textit{na-ujjhaggikavaggo}.}}}}}\makeatother \thinspace dutiyo\makeatletter\hyperlink{endnote477-appendix}\Hy@raisedlink{\hypertarget{endnote477-body}{}{\pagenote{%
		\hypertarget{endnote477-appendix}{\hyperlink{endnote477-body}{Bh Pm 1 \& 2, C, D, W: \textit{Dutiyo vaggo}. No section conclusion here in Mm Se.}}}}}\makeatother \thinspace
\end{center}



\subsection{Khambhakatavaggo}
% \vspace{0.2cm}

\pdfbookmark[3]{Sekhiya 21 \& 22}{sekh21-22}
\subsubsection*{\hyperref[training21-22]{Sekhiya 21 \& 22: Khambhakatasikkhāpadaṁ}}
\label{sekh21-22}

\linkdest{endnote478-body}
Na khambhakato\makeatletter\hyperlink{endnote478-appendix}\Hy@raisedlink{\hypertarget{endnote478-body}{}{\pagenote{%
		\hypertarget{endnote478-appendix}{\hyperlink{endnote478-body}{C, D, W: \textit{nakkhambhakato}. G: \textit{na khambhagato}. Corrected to \textit{nakkhambhagato}.}}}}}\makeatother \thinspace antaraghare gamissāmī'ti, sikkhā karaṇīyā.\\
Na khambhakato antaraghare nisīdissāmī'ti, sikkhā karaṇīyā.



\pdfbookmark[3]{Sekhiya 23 \& 24}{sekh23-24}
\subsubsection*{\hyperref[training23-24]{Sekhiya 23 \& 24: Oguṇṭhitasikkhāpadaṁ}}
\label{sekh23-24}

Na oguṇṭhito antaraghare gamissāmī'ti sikkhā karaṇīyā.\\
Na oguṇṭhito antaraghare nisīdissāmī'ti sikkhā karaṇīyā.



\pdfbookmark[3]{Sekhiya 25}{sekh25}
\subsubsection*{\hyperref[training25]{Sekhiya 25: Ukkuṭikasikkhāpadaṁ}}
\label{sekh25}

Na ukkuṭikāya antaraghare gamissāmī'ti, sikkhā karaṇīyā.



\pdfbookmark[3]{Sekhiya 26}{sekh26}
\subsubsection*{\hyperref[training26]{Sekhiya 26: Pallatthikasikkhāpadaṁ}}
\label{sekh26}

\linkdest{endnote479-body}
Na pallatthikāya antaraghare nisīdissāmī'ti, sikkhā karaṇīyā.\makeatletter\hyperlink{endnote479-appendix}\Hy@raisedlink{\hypertarget{endnote479-body}{}{\pagenote{%
		\hypertarget{endnote479-appendix}{\hyperlink{endnote479-body}{Mi \& Mm Mm Se have section-conclusions here. Mi Se: \textit{Chabbīsati sāruppā niṭṭhitā}. Mm Se: \textit{Chabbīsati sāruppā}.}}}}}\makeatother \thinspace



\pdfbookmark[3]{Sekhiya 27}{sekh27}
\subsubsection*{\hyperref[training27]{Sekhiya 27: Sakkaccapaṭiggahaṇasikkhāpadaṁ}}
\label{sekh27}

Sakkaccaṁ piṇḍapātaṁ paṭiggahessāmī'ti sikkhā karaṇīyā.



\pdfbookmark[3]{Sekhiya 28}{sekh28}
\subsubsection*{\hyperref[training28]{Sekhiya 28: Pattasaññīpaṭiggahaṇasikkhāpadaṁ}}
\label{sekh28}

Pattasaññī piṇḍapātaṁ paṭiggahessāmī'ti sikkhā karaṇīyā.



\pdfbookmark[3]{Sekhiya 29}{sekh29}
\subsubsection*{\hyperref[training29]{Sekhiya 29: Samasūpakapaṭiggahaṇasikkhāpadaṁ}}
\label{sekh29}

Samasūpakaṁ piṇḍapātaṁ paṭiggahessāmī'ti sikkhā karaṇīyā.



\pdfbookmark[3]{Sekhiya 30}{sekh30}
\subsubsection*{\hyperref[training30]{Sekhiya 30: Samatitthikasikkhāpadaṁ}}
\label{sekh30}

\linkdest{endnote480-body}
Samatitthikaṁ\makeatletter\hyperlink{endnote480-appendix}\Hy@raisedlink{\hypertarget{endnote480-body}{}{\pagenote{%
		\hypertarget{endnote480-appendix}{\hyperlink{endnote480-body}{SVibh Ee: \textit{-titthi-} (cf v.l.l. at Vin IV 364). Dm, UP, Mi \& Mm Se, Bh Pm 1 \& 2, C, D, G, Um, V, W, SVibh Ce, Pg, Ra: \textit{-titti-}.}}}}}\makeatother \thinspace piṇḍapātaṁ paṭiggahessāmī'ti, sikkhā karaṇīyā.

\linkdest{endnote481-body}
\linkdest{endnote482-body}
\begin{center}
	Khambhakatavaggo\makeatletter\hyperlink{endnote481-appendix}\Hy@raisedlink{\hypertarget{endnote481-body}{}{\pagenote{%
		\hypertarget{endnote481-appendix}{\hyperlink{endnote481-body}{G, V: \textit{nakhambhakavaggo}.}}}}}\makeatother \thinspace tatiyo\makeatletter\hyperlink{endnote482-appendix}\Hy@raisedlink{\hypertarget{endnote482-body}{}{\pagenote{%
		\hypertarget{endnote482-appendix}{\hyperlink{endnote482-body}{Bh Pm 1 \& 2, C, D, W: \textit{Tatiyo vaggo}. Nothing in Mm Se.}}}}}\makeatother \thinspace
\end{center}



\subsection{Sakkaccavaggo}
% \vspace{0.2cm}

\pdfbookmark[3]{Sekhiya 31}{sekh31}
\subsubsection*{\hyperref[training31]{Sekhiya 31: Sakkaccabhuñjanasikkhāpadaṁ}}
\label{sekh31}

Sakkaccaṁ piṇḍapātaṁ bhuñjissāmī'ti sikkhā karaṇīyā.



\pdfbookmark[3]{Sekhiya 32}{sekh32}
\subsubsection*{\hyperref[training32]{Sekhiya 32: Pattasaññībhuñjanasikkhāpadaṁ}}
\label{sekh32}

\linkdest{endnote483-body}
Pattasaññī\makeatletter\hyperlink{endnote483-appendix}\Hy@raisedlink{\hypertarget{endnote483-body}{}{\pagenote{%
		\hypertarget{endnote483-appendix}{\hyperlink{endnote483-body}{V: \textit{-saññi}. Not so in Sekh. 28.}}}}}\makeatother \thinspace piṇḍapātaṁ bhuñjissāmī'ti, sikkhā karaṇīyā.



\pdfbookmark[3]{Sekhiya 33}{sekh33}
\subsubsection*{\hyperref[training33]{Sekhiya 33: Sapadānasikkhāpadaṁ}}
\label{sekh33}

\linkdest{endnote484-body}
Sapadānaṁ\makeatletter\hyperlink{endnote484-appendix}\Hy@raisedlink{\hypertarget{endnote484-body}{}{\pagenote{%
		\hypertarget{endnote484-appendix}{\hyperlink{endnote484-body}{V: \textit{samadānaṁ}.}}}}}\makeatother \thinspace piṇḍapātaṁ bhuñjissāmī'ti, sikkhā karaṇīyā.



\pdfbookmark[3]{Sekhiya 34}{sekh34}
\subsubsection*{\hyperref[training34]{Sekhiya 34: Samasūpakasikkhāpadaṁ}}
\label{sekh34}

Samasūpakaṁ piṇḍapātaṁ bhuñjissāmī'ti sikkhā karaṇīyā.



\pdfbookmark[3]{Sekhiya 35}{sekh35}
\subsubsection*{\hyperref[training35]{Sekhiya 35: Na-thūpakatasikkhāpadaṁ}}
\label{sekh35}

\linkdest{endnote485-body}
Na thūpakato\makeatletter\hyperlink{endnote485-appendix}\Hy@raisedlink{\hypertarget{endnote485-body}{}{\pagenote{%
		\hypertarget{endnote485-appendix}{\hyperlink{endnote485-body}{Mm Se, Bh Pm 1 \& 2, C, D, W, Ra, SVibh Ce, SVibh Ee, Mi Se v.l., Sannē: \textit{thūpato}. Dm, UP, Mi Se (and the parallel rule at Ee Vin II 214) read \textit{thūpakato} (although in the \textit{uddāna} (Vin II 232) Be also reads \textit{thūpato}.) G, V: \textit{thūpikato}. SVibh Ee Burmese ms.v.l.l. at Vin IV 364: \textit{dhūpakato, thupato, dhūpato}.}}}}}\makeatother \thinspace omadditvā piṇḍapātaṁ bhuñjissāmī'ti, sikkhā karaṇīyā.



\pdfbookmark[3]{Sekhiya 36}{sekh36}
\subsubsection*{\hyperref[training36]{Sekhiya 36: Odanappaṭicchādanasikkhāpadaṁ}}
\label{sekh36}

\linkdest{endnote486-body}
\linkdest{endnote487-body}
\linkdest{endnote488-body}
\linkdest{endnote489-body}
Na sūpaṁ vā byañjanaṁ\makeatletter\hyperlink{endnote486-appendix}\Hy@raisedlink{\hypertarget{endnote486-body}{}{\pagenote{%
		\hypertarget{endnote486-appendix}{\hyperlink{endnote486-body}{Bh Pm 1 \& 2, C, D, Um: \textit{vyañjanaṁ}.}}}}}\makeatother \thinspace vā odanena paṭicchādessāmi\makeatletter\hyperlink{endnote487-appendix}\Hy@raisedlink{\hypertarget{endnote487-body}{}{\pagenote{%
		\hypertarget{endnote487-appendix}{\hyperlink{endnote487-body}{D, W, Ra, Wae UdaPm: \textit{paṭicchādessāmī ti}.}}}}}\makeatother \thinspace bhiyyokamyataṁ\makeatletter\hyperlink{endnote488-appendix}\Hy@raisedlink{\hypertarget{endnote488-body}{}{\pagenote{%
		\hypertarget{endnote488-appendix}{\hyperlink{endnote488-body}{Bh Pm 1 \& 2: \textit{bhīyyo-}. Ra: \textit{bhīyo-}.}}}}}\makeatother \thinspace upādāyā'ti,\makeatletter\hyperlink{endnote489-appendix}\Hy@raisedlink{\hypertarget{endnote489-body}{}{\pagenote{%
		\hypertarget{endnote489-appendix}{\hyperlink{endnote489-body}{D: ...\textit{upādāya, sikkhā}...}}}}}\makeatother \thinspace sikkhā karaṇīyā.



\pdfbookmark[3]{Sekhiya 37}{sekh37}
\subsubsection*{\hyperref[training37]{Sekhiya 37: Sūpodanaviññattisikkhāpadaṁ}}
\label{sekh37}

\linkdest{endnote490-body}
\linkdest{endnote491-body}
Na sūpaṁ vā\makeatletter\hyperlink{endnote490-appendix}\Hy@raisedlink{\hypertarget{endnote490-body}{}{\pagenote{%
		\hypertarget{endnote490-appendix}{\hyperlink{endnote490-body}{G and V add: \textit{byañjanaṁ vā}.}}}}}\makeatother \thinspace odanaṁ vā agilāno\makeatletter\hyperlink{endnote491-appendix}\Hy@raisedlink{\hypertarget{endnote491-body}{}{\pagenote{%
		\hypertarget{endnote491-appendix}{\hyperlink{endnote491-body}{V: \textit{agīlāno}.}}}}}\makeatother \thinspace attano atthāya viññāpetvā bhuñjissāmī'ti, sikkhā karaṇīyā.



\pdfbookmark[3]{Sekhiya 38}{sekh38}
\subsubsection*{\hyperref[training38]{Sekhiya 38: Ujjhānasaññīsikkhāpadaṁ}}
\label{sekh38}

Na ujjhānasaññī paresaṁ pattaṁ olokessāmī'ti, sikkhā karaṇīyā.



\pdfbookmark[3]{Sekhiya 39}{sekh39}
\subsubsection*{\hyperref[training39]{Sekhiya 39: Kabaḷasikkhāpadaṁ}}
\label{sekh39}

\linkdest{endnote492-body}
N'ātimahantaṁ kabaḷaṁ\makeatletter\hyperlink{endnote492-appendix}\Hy@raisedlink{\hypertarget{endnote492-body}{}{\pagenote{%
		\hypertarget{endnote492-appendix}{\hyperlink{endnote492-body}{Mi \& Mm Se, V: \textit{kavaḷaṁ}. C, D, W: \textit{kabalaṁ}.}}}}}\makeatother \thinspace karissāmī'ti, sikkhā karaṇīyā.



\pdfbookmark[3]{Sekhiya 40}{sekh40}
\subsubsection*{\hyperref[training40]{Sekhiya 40: Ālopasikkhāpadaṁ}}
\label{sekh40}

Parimaṇḍalaṁ ālopaṁ karissāmī'ti sikkhā karaṇīyā.

\linkdest{endnote493-body}
\begin{center}
	Sakkaccavaggo catuttho\makeatletter\hyperlink{endnote493-appendix}\Hy@raisedlink{\hypertarget{endnote493-body}{}{\pagenote{%
		\hypertarget{endnote493-appendix}{\hyperlink{endnote493-body}{Bh Pm 1 \& 2, C, D, W: \textit{Catuttho vaggo}. Nothing in Mm Se.}}}}}\makeatother \thinspace
\end{center}



\subsection{Anāhaṭavaggo}
% \vspace{0.2cm}

\pdfbookmark[3]{Sekhiya 41}{sekh41}
\subsubsection*{\hyperref[training41]{Sekhiya 41: Anāhaṭasikkhāpadaṁ}}
\label{sekh41}

\linkdest{endnote494-body}
\linkdest{endnote495-body}
Na anāhaṭe\makeatletter\hyperlink{endnote494-appendix}\Hy@raisedlink{\hypertarget{endnote494-body}{}{\pagenote{%
		\hypertarget{endnote494-appendix}{\hyperlink{endnote494-body}{Bh Pm 1 \& 2, C, D, W, Ra: \textit{nānāhaṭe}. W: \textit{anāhate}.}}}}}\makeatother \thinspace kabaḷe\makeatletter\hyperlink{endnote495-appendix}\Hy@raisedlink{\hypertarget{endnote495-body}{}{\pagenote{%
		\hypertarget{endnote495-appendix}{\hyperlink{endnote495-body}{Mi \& Mm Se, V: \textit{kavaḷe}. C, D, W: \textit{kabale}.}}}}}\makeatother \thinspace mukhadvāraṁ vivarissāmī'ti, sikkhā karaṇīyā.



\pdfbookmark[3]{Sekhiya 42}{sekh42}
\subsubsection*{\hyperref[training42]{Sekhiya 42: Bhuñjamānasikkhāpadaṁ}}
\label{sekh42}

Na bhuñjamāno sabbaṁ hatthaṁ mukhe pakkhipissāmī'ti sikkhā karaṇīyā.



\pdfbookmark[3]{Sekhiya 43}{sekh43}
\subsubsection*{\hyperref[training43]{Sekhiya 43: Sakabaḷasikkhāpadaṁ}}
\label{sekh43}

\linkdest{endnote496-body}
\linkdest{endnote497-body}
Na sakabaḷena\makeatletter\hyperlink{endnote496-appendix}\Hy@raisedlink{\hypertarget{endnote496-body}{}{\pagenote{%
		\hypertarget{endnote496-appendix}{\hyperlink{endnote496-body}{Mi \& Mm Se, V: \textit{-kavaḷena}. C, D, W: \textit{-kabalena}.}}}}}\makeatother \thinspace mukhena byāharissāmī'ti,\makeatletter\hyperlink{endnote497-appendix}\Hy@raisedlink{\hypertarget{endnote497-body}{}{\pagenote{%
		\hypertarget{endnote497-appendix}{\hyperlink{endnote497-body}{G, P: \textit{vyāharissāmī}.}}}}}\makeatother \thinspace sikkhā karaṇīyā.



\pdfbookmark[3]{Sekhiya 44}{sekh44}
\subsubsection*{\hyperref[training44]{Sekhiya 44: Piṇḍukkhepakasikkhāpadaṁ}}
\label{sekh44}

Na piṇḍ'ukkhepakaṁ bhuñjissāmī'ti sikkhā karaṇīyā.



\pdfbookmark[3]{Sekhiya 45}{sekh45}
\subsubsection*{\hyperref[training45]{Sekhiya 45: Kabaḷāvacchedakasikkhāpadaṁ}}
\label{sekh45}

\linkdest{endnote498-body}
Na kabaḷ'āvacchedakaṁ\makeatletter\hyperlink{endnote498-appendix}\Hy@raisedlink{\hypertarget{endnote498-body}{}{\pagenote{%
		\hypertarget{endnote498-appendix}{\hyperlink{endnote498-body}{Mi \& Mm Se, V: \textit{kavaḷ-}. C, D: \textit{kabal-}.}}}}}\makeatother \thinspace bhuñjissāmī'ti, sikkhā karaṇīyā.



\pdfbookmark[3]{Sekhiya 46}{sekh46}
\subsubsection*{\hyperref[training46]{Sekhiya 46: Avagaṇḍakārakasikkhāpadaṁ}}
\label{sekh46}

Na avagaṇḍakārakaṁ bhuñjissāmī'ti sikkhā karaṇīyā.



\pdfbookmark[3]{Sekhiya 47}{sekh47}
\subsubsection*{\hyperref[training47]{Sekhiya 47: Hatthaniddhunakasikkhāpadaṁ}}
\label{sekh47}

\linkdest{endnote499-body}
Na hatthaniddhunakaṁ\makeatletter\hyperlink{endnote499-appendix}\Hy@raisedlink{\hypertarget{endnote499-body}{}{\pagenote{%
		\hypertarget{endnote499-appendix}{\hyperlink{endnote499-body}{Mi \& Mm Se, Bh Pm 1 \& 2, C, D, G, V, W, Pg: \textit{-niddhūnakaṁ}.}}}}}\makeatother \thinspace bhuñjissāmī'ti, sikkhā karaṇīyā.



\pdfbookmark[3]{Sekhiya 48}{sekh48}
\subsubsection*{\hyperref[training48]{Sekhiya 48: Sitthāvakārakasikkhāpadaṁ}}
\label{sekh48}

\linkdest{endnote500-body}
Na sitth'āvakārakaṁ\makeatletter\hyperlink{endnote500-appendix}\Hy@raisedlink{\hypertarget{endnote500-body}{}{\pagenote{%
		\hypertarget{endnote500-appendix}{\hyperlink{endnote500-body}{V: \textit{siṭṭh-}.}}}}}\makeatother \thinspace bhuñjissāmī'ti, sikkhā karaṇīyā.



\pdfbookmark[3]{Sekhiya 49}{sekh49}
\subsubsection*{\hyperref[training49]{Sekhiya 49: Jivhānicchārakasikkhāpadaṁ}}
\label{sekh49}

Na jivhānicchārakaṁ bhuñjissāmī'ti sikkhā karaṇīyā.



\pdfbookmark[3]{Sekhiya 50}{sekh50}
\subsubsection*{\hyperref[training50]{Sekhiya 50: Capucapukārakasikkhāpadaṁ}}
\label{sekh50}

Na capucapukārakaṁ bhuñjissāmī'ti sikkhā karaṇīyā.

\linkdest{endnote501-body}
\linkdest{endnote502-body}
\begin{center}
	Kabaḷavaggo\makeatletter\hyperlink{endnote501-appendix}\Hy@raisedlink{\hypertarget{endnote501-body}{}{\pagenote{%
		\hypertarget{endnote501-appendix}{\hyperlink{endnote501-body}{Mi Se: \textit{anāhaṭavaggo}. G: \textit{na anāhaṭavaggo}. V: \textit{na anāhatavaggo}.}}}}}\makeatother \thinspace pañcamo\makeatletter\hyperlink{endnote502-appendix}\Hy@raisedlink{\hypertarget{endnote502-body}{}{\pagenote{%
		\hypertarget{endnote502-appendix}{\hyperlink{endnote502-body}{Bh Pm 1 \& 2, C, D, W: \textit{Pañcamo vaggo}. Nothing in Mm Se.}}}}}\makeatother \thinspace
\end{center}



\subsection{Surusuruvaggo}
% \vspace{0.2cm}

\pdfbookmark[3]{Sekhiya 51}{sekh51}
\subsubsection*{\hyperref[training51]{Sekhiya 51: Surusurukārakasikkhāpadaṁ}}
\label{sekh51}

Na surusurukārakaṁ bhuñjissāmī'ti sikkhā karaṇīyā.



\pdfbookmark[3]{Sekhiya 52}{sekh52}
\subsubsection*{\hyperref[training52]{Sekhiya 52: Hatthanillehakasikkhāpadaṁ}}
\label{sekh52}

Na hatthanillehakaṁ bhuñjissāmī'ti sikkhā karaṇīyā.



\pdfbookmark[3]{Sekhiya 53}{sekh53}
\subsubsection*{\hyperref[training53]{Sekhiya 53: Pattanillehakasikkhāpadaṁ}}
\label{sekh53}

Na pattanillehakaṁ bhuñjissāmī'ti sikkhā karaṇīyā.



\pdfbookmark[3]{Sekhiya 54}{sekh54}
\subsubsection*{\hyperref[training54]{Sekhiya 54: Oṭṭhanillehakasikkhāpadaṁ}}
\label{sekh54}

\linkdest{endnote503-body}
Na oṭṭhanillehakaṁ\makeatletter\hyperlink{endnote503-appendix}\Hy@raisedlink{\hypertarget{endnote503-body}{}{\pagenote{%
		\hypertarget{endnote503-appendix}{\hyperlink{endnote503-body}{W: \textit{uṭṭha-} (Probably based on a corruption based on the Khom script as the Sinhala characters \textit{o} and \textit{u} can't be confused easily; see note on ūna at Sd conclusion.)}}}}}\makeatother \thinspace bhuñjissāmī'ti, sikkhā karaṇīyā.



\pdfbookmark[3]{Sekhiya 55}{sekh55}
\subsubsection*{\hyperref[training55]{Sekhiya 55: Sāmisasikkhāpadaṁ}}
\label{sekh55}

\linkdest{endnote504-body}
Na sāmisena hatthena pānīyathālakaṁ\makeatletter\hyperlink{endnote504-appendix}\Hy@raisedlink{\hypertarget{endnote504-body}{}{\pagenote{%
		\hypertarget{endnote504-appendix}{\hyperlink{endnote504-body}{V: \textit{pāṇiya-}.}}}}}\makeatother \thinspace paṭiggahessāmī'ti, sikkhā karaṇīyā.



\pdfbookmark[3]{Sekhiya 56}{sekh56}
\subsubsection*{\hyperref[training56]{Sekhiya 56: Sasitthakasikkhāpadaṁ}}
\label{sekh56}

\linkdest{endnote505-body}
\linkdest{endnote506-body}
\linkdest{endnote507-body}
Na sasitthakaṁ\makeatletter\hyperlink{endnote505-appendix}\Hy@raisedlink{\hypertarget{endnote505-body}{}{\pagenote{%
		\hypertarget{endnote505-appendix}{\hyperlink{endnote505-body}{V: \textit{sasiṭṭhakaṁ}. (Cf Sekh 48.) G: \textit{na sitthakaṁ}.}}}}}\makeatother \thinspace pattadhovanaṁ antaraghare chaḍḍessāmī'ti,\makeatletter\hyperlink{endnote506-appendix}\Hy@raisedlink{\hypertarget{endnote506-body}{}{\pagenote{%
		\hypertarget{endnote506-appendix}{\hyperlink{endnote506-body}{V: \textit{chaddessāmī}.}}}}}\makeatother \thinspace sikkhā karaṇīyā.\makeatletter\hyperlink{endnote507-appendix}\Hy@raisedlink{\hypertarget{endnote507-body}{}{\pagenote{%
		\hypertarget{endnote507-appendix}{\hyperlink{endnote507-body}{Mm Se: \textit{Samatiṁsa bhojanapaṭisaṁyuttā}. Mi Se: \textit{Samatiṁsa bhojanapaṭisaṁyuttā
niṭṭhitā}.}}}}}\makeatother \thinspace



\pdfbookmark[3]{Sekhiya 57}{sekh57}
\subsubsection*{\hyperref[training57]{Sekhiya 57: Chattapāṇisikkhāpadaṁ}}
\label{sekh57}

\linkdest{endnote508-body}
\linkdest{endnote509-body}
Na chattapāṇissa agilānassa\makeatletter\hyperlink{endnote508-appendix}\Hy@raisedlink{\hypertarget{endnote508-body}{}{\pagenote{%
		\hypertarget{endnote508-appendix}{\hyperlink{endnote508-body}{V: \textit{agīlānassa}.}}}}}\makeatother \thinspace dhammaṁ desessāmī'ti,\makeatletter\hyperlink{endnote509-appendix}\Hy@raisedlink{\hypertarget{endnote509-body}{}{\pagenote{%
		\hypertarget{endnote509-appendix}{\hyperlink{endnote509-body}{Mm Se, Bh Pm 1 \& 2, C, D, G, V, W, Mi Se v.l., Ra: \textit{desissāmī ti} throughout. (Pg: \textit{desessāmī ti}.)}}}}}\makeatother \thinspace sikkhā karaṇīyā.



\pdfbookmark[3]{Sekhiya 58}{sekh58}
\subsubsection*{\hyperref[training58]{Sekhiya 58: Daṇḍapāṇisikkhāpadaṁ}}
\label{sekh58}

Na daṇḍapāṇissa agilānassa dhammaṁ desessāmī'ti sikkhā karaṇīyā.



\pdfbookmark[3]{Sekhiya 59}{sekh59}
\subsubsection*{\hyperref[training59]{Sekhiya 59: Satthapāṇisikkhāpadaṁ}}
\label{sekh59}

\linkdest{endnote510-body}
Na satthapāṇissa agilānassa\makeatletter\hyperlink{endnote510-appendix}\Hy@raisedlink{\hypertarget{endnote510-body}{}{\pagenote{%
		\hypertarget{endnote510-appendix}{\hyperlink{endnote510-body}{V: \textit{agīlānassa}.}}}}}\makeatother \thinspace dhammaṁ desessāmī'ti, sikkhā karaṇīyā.



\pdfbookmark[3]{Sekhiya 60}{sekh60}
\subsubsection*{\hyperref[training60]{Sekhiya 60: Āvudhapāṇisikkhāpadaṁ}}
\label{sekh60}

\linkdest{endnote511-body}
Na āvudhapāṇissa\makeatletter\hyperlink{endnote511-appendix}\Hy@raisedlink{\hypertarget{endnote511-body}{}{\pagenote{%
		\hypertarget{endnote511-appendix}{\hyperlink{endnote511-body}{Bh Pm 1 \& 2, Um, Ra, Pg, SVibh Ce: \textit{āyudha}.}}}}}\makeatother \thinspace agilānassa dhammaṁ desessāmī'ti, sikkhā karaṇīyā.

\linkdest{endnote512-body}
\linkdest{endnote513-body}
\begin{center}
	Surusuruvaggo\makeatletter\hyperlink{endnote512-appendix}\Hy@raisedlink{\hypertarget{endnote512-body}{}{\pagenote{%
		\hypertarget{endnote512-appendix}{\hyperlink{endnote512-body}{G, V: \textit{nasurusuruvaggo}.}}}}}\makeatother \thinspace chaṭṭho\makeatletter\hyperlink{endnote513-appendix}\Hy@raisedlink{\hypertarget{endnote513-body}{}{\pagenote{%
		\hypertarget{endnote513-appendix}{\hyperlink{endnote513-body}{Bh Pm 1 \& 2, C, D, W: \textit{Chaṭṭho vaggo}. Nothing in Mm Se.}}}}}\makeatother \thinspace
\end{center}



\subsection{Pādukavaggo}
% \vspace{0.2cm}

\pdfbookmark[3]{Sekhiya 61}{sekh61}
\subsubsection*{\hyperref[training61]{Sekhiya 61: Pādukasikkhāpadaṁ}}
\label{sekh61}

\linkdest{endnote514-body}
Na pāduk'ārūḷhassa\makeatletter\hyperlink{endnote514-appendix}\Hy@raisedlink{\hypertarget{endnote514-body}{}{\pagenote{%
		\hypertarget{endnote514-appendix}{\hyperlink{endnote514-body}{Bh Pm 1 \& 2, Dm, V: \textit{-ruḷhassa}.}}}}}\makeatother \thinspace agilānassa dhammaṁ desessāmī'ti, sikkhā karaṇīyā.



\pdfbookmark[3]{Sekhiya 62}{sekh62}
\subsubsection*{\hyperref[training62]{Sekhiya 62: Upāhanasikkhāpadaṁ}}
\label{sekh62}

\linkdest{endnote515-body}
\linkdest{endnote516-body}
Na upāhan'ārūḷhassa\makeatletter\hyperlink{endnote515-appendix}\Hy@raisedlink{\hypertarget{endnote515-body}{}{\pagenote{%
		\hypertarget{endnote515-appendix}{\hyperlink{endnote515-body}{Bh Pm 1 \& 2, Dm, V: \textit{-ruḷhassa}.}}}}}\makeatother \thinspace agilānassa\makeatletter\hyperlink{endnote-appendix}\Hy@raisedlink{\hypertarget{endnote516-body}{}{\pagenote{%
		\hypertarget{endnote516-appendix}{\hyperlink{endnote516-body}{V: \textit{agilānassa} throughout the section.}}}}}\makeatother \thinspace dhammaṁ desessāmī'ti, sikkhā karaṇīyā.



\pdfbookmark[3]{Sekhiya 63}{sekh63}
\subsubsection*{\hyperref[training63]{Sekhiya 63: Yānasikkhāpadaṁ}}
\label{sekh63}

Na yānagatassa agilānassa dhammaṁ desessāmī'ti sikkhā karaṇīyā.



\pdfbookmark[3]{Sekhiya 64}{sekh64}
\subsubsection*{\hyperref[training64]{Sekhiya 64: Sayanasikkhāpadaṁ}}
\label{sekh64}

Na sayanagatassa agilānassa dhammaṁ desessāmī'ti sikkhā karaṇīyā.



\pdfbookmark[3]{Sekhiya 65}{sekh65}
\subsubsection*{\hyperref[training65]{Sekhiya 65: Pallatthikasikkhāpadaṁ}}
\label{sekh65}

Na pallatthikāya nisinnassa agilānassa dhammaṁ desessāmī'ti sikkhā karaṇīyā.



\pdfbookmark[3]{Sekhiya 66}{sekh66}
\subsubsection*{\hyperref[training66]{Sekhiya 66: Veṭhitasikkhāpadaṁ}}
\label{sekh66}

\linkdest{endnote517-body}
Na veṭhitasīsassa\makeatletter\hyperlink{endnote517-appendix}\Hy@raisedlink{\hypertarget{endnote517-body}{}{\pagenote{%
		\hypertarget{endnote517-appendix}{\hyperlink{endnote517-body}{Mi \& Mm Se, G, V: \textit{veṭṭhita-}.}}}}}\makeatother \thinspace agilānassa dhammaṁ desessāmī'ti, sikkhā karaṇīyā.



\pdfbookmark[3]{Sekhiya 67}{sekh67}
\subsubsection*{\hyperref[training67]{Sekhiya 67: Oguṇṭhitasikkhāpadaṁ}}
\label{sekh67}

Na oguṇṭhitasīsassa agilānassa dhammaṁ desessāmī'ti sikkhā karaṇīyā.



\pdfbookmark[3]{Sekhiya 68}{sekh68}
\subsubsection*{\hyperref[training68]{Sekhiya 68: Chamāsikkhāpadaṁ}}
\label{sekh68}

\linkdest{endnote518-body}
Na chamāyaṁ\makeatletter\hyperlink{endnote518-appendix}\Hy@raisedlink{\hypertarget{endnote518-body}{}{\pagenote{%
		\hypertarget{endnote518-appendix}{\hyperlink{endnote518-body}{Bh Pm 1 \& 2, C, D, W, SVibh Ce, SVibh Ee: \textit{chamāya}.}}}}}\makeatother \thinspace nisīditvā āsane nisinnassa agilānassa dhammaṁ desessāmī'ti, sikkhā karaṇīyā.



\pdfbookmark[3]{Sekhiya 69}{sekh69}
\subsubsection*{\hyperref[training69]{Sekhiya 69: Nīcāsanasikkhāpadaṁ}}
\label{sekh69}

\linkdest{endnote519-body}
Na nīce\makeatletter\hyperlink{endnote519-appendix}\Hy@raisedlink{\hypertarget{endnote519-body}{}{\pagenote{%
		\hypertarget{endnote519-appendix}{\hyperlink{endnote519-body}{V: \textit{nice}.}}}}}\makeatother \thinspace āsane nisīditvā ucce āsane nisinnassa agilānassa dhammaṁ desessāmī'ti, sikkhā karaṇīyā.



\pdfbookmark[3]{Sekhiya 70}{sekh70}
\subsubsection*{\hyperref[training70]{Sekhiya 70: Ṭhitasikkhāpadaṁ}}
\label{sekh70}

\linkdest{endnote520-body}
Na ṭhito nisinnassa agilānassa dhammaṁ desessāmī'ti, sikkhā karaṇīyā.\makeatletter\hyperlink{endnote520-appendix}\Hy@raisedlink{\hypertarget{endnote520-body}{}{\pagenote{%
		\hypertarget{endnote520-appendix}{\hyperlink{endnote520-body}{C, D, W: \textit{Sattamo vaggo}. G, V: \textit{Napādukavaggo sattamo}.}}}}}\makeatother \thinspace



\pdfbookmark[3]{Sekhiya 71}{sekh71}
\subsubsection*{\hyperref[training71]{Sekhiya 71: Pacchatogamanasikkhāpadaṁ}}
\label{sekh71}

\linkdest{endnote521-body}
Na pacchato gacchanto purato\makeatletter\hyperlink{endnote521-appendix}\Hy@raisedlink{\hypertarget{endnote521-body}{}{\pagenote{%
		\hypertarget{endnote521-appendix}{\hyperlink{endnote521-body}{V: \textit{pūrato}.}}}}}\makeatother \thinspace gacchantassa agilānassa dhammaṁ desessāmī'ti, sikkhā karaṇīyā.



\pdfbookmark[3]{Sekhiya 72}{sekh72}
\subsubsection*{\hyperref[training72]{Sekhiya 72: Uppathenagamanasikkhāpadaṁ}}
\label{sekh72}

\linkdest{endnote522-body}
\linkdest{endnote523-body}
Na uppathena\makeatletter\hyperlink{endnote522-appendix}\Hy@raisedlink{\hypertarget{endnote522-body}{}{\pagenote{%
		\hypertarget{endnote522-appendix}{\hyperlink{endnote522-body}{V: \textit{upathena}.}}}}}\makeatother \thinspace gacchanto pathena gacchantassa agilānassa dhammaṁ desessāmī'ti, sikkhā karaṇīyā.\makeatletter\hyperlink{endnote523-appendix}\Hy@raisedlink{\hypertarget{endnote523-body}{}{\pagenote{%
		\hypertarget{endnote523-appendix}{\hyperlink{endnote523-body}{Mm Se: \textit{Soḷasa dhammadesanā-paṭisaṁyuttā}. Mi Se: \textit{Soḷasa dhammadesanāpaṭisaṁyuttā niṭṭhitā}.}}}}}\makeatother \thinspace



\pdfbookmark[3]{Sekhiya 73}{sekh73}
\subsubsection*{\hyperref[training73]{Sekhiya 73: Ṭhito-uccārasikkhāpadaṁ}}
\label{sekh73}

\linkdest{endnote524-body}
Na ṭhito agilāno\makeatletter\hyperlink{endnote524-appendix}\Hy@raisedlink{\hypertarget{endnote524-body}{}{\pagenote{%
		\hypertarget{endnote524-appendix}{\hyperlink{endnote524-body}{V: \textit{agīlāno} throughout the section.}}}}}\makeatother \thinspace uccāraṁ vā passāvaṁ vā karissāmī'ti, sikkhā karaṇīyā.



\pdfbookmark[3]{Sekhiya 74}{sekh74}
\subsubsection*{\hyperref[training74]{Sekhiya 74: Harite-uccārasikkhāpadaṁ}}
\label{sekh74}

Na harite agilāno uccāraṁ vā passāvaṁ vā kheḷaṁ vā karissāmī'ti sikkhā karaṇīyā.



\pdfbookmark[3]{Sekhiya 75}{sekh75}
\subsubsection*{\hyperref[training75]{Sekhiya 75: Udake-uccārasikkhāpadaṁ}}
\label{sekh75}

Na udake agilāno uccāraṁ vā passāvaṁ vā kheḷaṁ vā karissāmī'ti sikkhā karaṇīyā.

\linkdest{endnote525-body}
\begin{center}
	Pādukavaggo sattamo\makeatletter\hyperlink{endnote525-appendix}\Hy@raisedlink{\hypertarget{endnote525-body}{}{\pagenote{%
		\hypertarget{endnote525-appendix}{\hyperlink{endnote525-body}{Bh Pm 1 \& 2: \textit{Sattamo vaggo}. G, V: \textit{Napacchatovaggo aṭṭhamo}. Mm Se: \textit{Tayo pakiṇṇakā}. Mi Se: \textit{Tayo pakiṇṇakā niṭṭhitā}.}}}}}\makeatother \thinspace
\end{center}



\medskip

\linkdest{endnote526-body}
\begin{center}
	Uddiṭṭhā kho āyasmanto sekhiyā\makeatletter\hyperlink{endnote526-appendix}\Hy@raisedlink{\hypertarget{endnote526-body}{}{\pagenote{%
		\hypertarget{endnote526-appendix}{\hyperlink{endnote526-body}{Mi Se, V, P: \textit{pañcasattati sekhiyā}.}}}}}\makeatother \thinspace dhammā.

	\smallskip

	Tatth'āyasmante pucchāmi: Kacci'ttha parisuddhā?\\
	Dutiyam'pi pucchāmi: Kacci'ttha parisuddhā?\\
	Tatiyam'pi pucchāmi: Kacci'ttha parisuddhā?

	\smallskip

\linkdest{endnote527-body}
	Parisuddh'etth'āyasmanto, tasmā tuṇhī, evam'etaṁ dhārayāmi.\makeatletter\hyperlink{endnote527-appendix}\Hy@raisedlink{\hypertarget{endnote527-body}{}{\pagenote{%
		\hypertarget{endnote527-appendix}{\hyperlink{endnote527-body}{Dm, UP, Ra, Um: \textit{dhārayāmī ti}.}}}}}\makeatother \thinspace
\end{center}

\linkdest{endnote528-body}
\begin{outro}
	Sekhiyā dhammā niṭṭhitā\makeatletter\hyperlink{endnote528-appendix}\Hy@raisedlink{\hypertarget{endnote528-body}{}{\pagenote{%
		\hypertarget{endnote528-appendix}{\hyperlink{endnote528-body}{Dm, Bh Pm 1 \& 2, c, V, W, Mm Se, Um. Ñd Ce, P: Sekhiyā dhammā niṭṭhitā. Mi Se: \textit{Pañcasattati sekhiyā dhammā niṭṭhitā}. (N.B. The \textit{Katthapaññattivāra} chapter of the \textit{Parivāra} (Be, Ce, and Ee.) has \textit{Pañcasattati sekhiyā niṭṭhitā}.}}}}}\makeatother \thinspace

\end{outro}

\clearpage

