
\section{Sekhiyā}
\label{sekh}

\begin{intro}
	Ime kho pan'āyasmanto sekhiyā dhammā uddesaṁ āgacchanti.
\end{intro}

\setsubsecheadstyle{\subsectionFmt}
\subsection{Parimaṇḍalavaggo}
% \vspace{0.2cm}

\pdfbookmark[3]{Sekhiya 1 \& 2}{sekh1-2}
\subsubsection*{\hyperref[training1-2]{Sekhiya 1 \& 2: Parimaṇḍalasikkhāpadaṁ}}
\label{sekh1-2}

Parimaṇḍalaṁ nivāsessāmī'ti, sikkhā karaṇīyā.\\
Parimaṇḍalaṁ pārupissāmī'ti sikkhā karaṇīyā.



\pdfbookmark[3]{Sekhiya 3 \& 4}{sekh3-4}
\subsubsection*{\hyperref[training3-4]{Sekhiya 3 \& 4: Suppaṭicchannasikkhāpadaṁ}}
\label{sekh3-4}

Supaṭicchanno antaraghare gamissāmī'ti, sikkhā karaṇīyā.\\
Supaṭicchanno antaraghare nisīdissāmī'ti, sikkhā karaṇīyā.



\pdfbookmark[3]{Sekhiya 5 \& 6}{sekh5-6}
\subsubsection*{\hyperref[training5-6]{Sekhiya 5 \& 6: Susaṁvutasikkhāpadaṁ}}
\label{sekh5-6}

Susaṁvuto antaraghare gamissāmī'ti sikkhā karaṇīyā.\\
Susaṁvuto antaraghare nisīdissāmī'ti sikkhā karaṇīyā.



\pdfbookmark[3]{Sekhiya 7 \& 8}{sekh7-8}
\subsubsection*{\hyperref[training7-8]{Sekhiya 7 \& 8: Okkhittacakkhusikkhāpadaṁ}}
\label{sekh7-8}

Okkhittacakkhu antaraghare gamissāmī'ti sikkhā karaṇīyā.\\
Okkhittacakkhu antaraghare nisīdissāmī'ti sikkhā karaṇīyā.



\pdfbookmark[3]{Sekhiya 9 \& 10}{sekh9-10}
\subsubsection*{\hyperref[training9-10]{Sekhiya 9 \& 10: Ukkhittakasikkhāpadaṁ}}
\label{sekh9-10}

Na ukkhittakāya antaraghare gamissāmī'ti, sikkhā karaṇīyā.\\
Na ukkhittakāya antaraghare nisīdissāmī'ti, sikkhā karaṇīyā.

\begin{center}
	Parimaṇḍalavaggo paṭhamo
\end{center}



\subsection{Ujjagghikavaggo}
% \vspace{0.2cm}

\pdfbookmark[3]{Sekhiya 11 \& 12}{sekh11-12}
\subsubsection*{\hyperref[training11-12]{Sekhiya 11 \& 12: Ujjagghikasikkhāpadaṁ}}
\label{sekh11-12}

Na ujjagghikāya antaraghare gamissāmī'ti, sikkhā karaṇīyā.\\
Na ujjagghikāya antaraghare nisīdissāmī'ti, sikkhā karaṇīyā.



\pdfbookmark[3]{Sekhiya 13 \& 14}{sekh13-14}
\subsubsection*{\hyperref[training13-14]{Sekhiya 13 \& 14: Uccasaddasikkhāpadaṁ}}
\label{sekh13-14}

Appasaddo antaraghare gamissāmī'ti sikkhā karaṇīyā.\\
Appasaddo antaraghare nisīdissāmī'ti sikkhā karaṇīyā.



\pdfbookmark[3]{Sekhiya 15 \& 16}{sekh15-16}
\subsubsection*{\hyperref[training15-16]{Sekhiya 15 \& 16: Kāyappacālakasikkhāpadaṁ}}
\label{sekh15-16}

Na kāyappacālakaṁ antaraghare gamissāmī'ti sikkhā karaṇīyā.\\
Na kāyappacālakaṁ antaraghare nisīdissāmī'ti sikkhā karaṇīyā.



\pdfbookmark[3]{Sekhiya 17 \& 18}{sekh17-18}
\subsubsection*{\hyperref[training17-18]{Sekhiya 17 \& 18: Bāhuppacālakasikkhāpadaṁ}}
\label{sekh17-18}

Na bāhuppacālakaṁ antaraghare gamissāmī'ti sikkhā karaṇīyā.\\
Na bāhuppacālakaṁ antaraghare nisīdissāmī'ti sikkhā karaṇīyā.



\pdfbookmark[3]{Sekhiya 19 \& 20}{sekh19-20}
\subsubsection*{\hyperref[training19-20]{Sekhiya 19 \& 20: Sīsappacālakasikkhāpadaṁ}}
\label{sekh19-20}

Na sīsappacālakaṁ antaraghare gamissāmī'ti sikkhā karaṇīyā.\\
Na sīsappacālakaṁ antaraghare nisīdissāmī'ti sikkhā karaṇīyā.

\begin{center}
	Ujjagghikavaggo dutiyo
\end{center}



\subsection{Khambhakatavaggo}
% \vspace{0.2cm}

\pdfbookmark[3]{Sekhiya 21 \& 22}{sekh21-22}
\subsubsection*{\hyperref[training21-22]{Sekhiya 21 \& 22: Khambhakatasikkhāpadaṁ}}
\label{sekh21-22}

Na khambhakato antaraghare gamissāmī'ti, sikkhā karaṇīyā.\\
Na khambhakato antaraghare nisīdissāmī'ti, sikkhā karaṇīyā.



\pdfbookmark[3]{Sekhiya 23 \& 24}{sekh23-24}
\subsubsection*{\hyperref[training23-24]{Sekhiya 23 \& 24: Oguṇṭhitasikkhāpadaṁ}}
\label{sekh23-24}

Na oguṇṭhito antaraghare gamissāmī'ti sikkhā karaṇīyā.\\
Na oguṇṭhito antaraghare nisīdissāmī'ti sikkhā karaṇīyā.



\pdfbookmark[3]{Sekhiya 25}{sekh25}
\subsubsection*{\hyperref[training25]{Sekhiya 25: Ukkuṭikasikkhāpadaṁ}}
\label{sekh25}

Na ukkuṭikāya antaraghare gamissāmī'ti, sikkhā karaṇīyā.



\pdfbookmark[3]{Sekhiya 26}{sekh26}
\subsubsection*{\hyperref[training26]{Sekhiya 26: Pallatthikasikkhāpadaṁ}}
\label{sekh26}

Na pallatthikāya antaraghare nisīdissāmī'ti, sikkhā karaṇīyā.



\pdfbookmark[3]{Sekhiya 27}{sekh27}
\subsubsection*{\hyperref[training27]{Sekhiya 27: Sakkaccapaṭiggahaṇasikkhāpadaṁ}}
\label{sekh27}

Sakkaccaṁ piṇḍapātaṁ paṭiggahessāmī'ti sikkhā karaṇīyā.



\pdfbookmark[3]{Sekhiya 28}{sekh28}
\subsubsection*{\hyperref[training28]{Sekhiya 28: Pattasaññīpaṭiggahaṇasikkhāpadaṁ}}
\label{sekh28}

Pattasaññī piṇḍapātaṁ paṭiggahessāmī'ti sikkhā karaṇīyā.



\pdfbookmark[3]{Sekhiya 29}{sekh29}
\subsubsection*{\hyperref[training29]{Sekhiya 29: Samasūpakapaṭiggahaṇasikkhāpadaṁ}}
\label{sekh29}

Samasūpakaṁ piṇḍapātaṁ paṭiggahessāmī'ti sikkhā karaṇīyā.



\pdfbookmark[3]{Sekhiya 30}{sekh30}
\subsubsection*{\hyperref[training30]{Sekhiya 30: Samatitthikasikkhāpadaṁ}}
\label{sekh30}

Samatitthikaṁ piṇḍapātaṁ paṭiggahessāmī'ti, sikkhā karaṇīyā.

\begin{center}
	Khambhakatavaggo tatiyo
\end{center}



\subsection{Sakkaccavaggo}
% \vspace{0.2cm}

\pdfbookmark[3]{Sekhiya 31}{sekh31}
\subsubsection*{\hyperref[training31]{Sekhiya 31: Sakkaccabhuñjanasikkhāpadaṁ}}
\label{sekh31}

Sakkaccaṁ piṇḍapātaṁ bhuñjissāmī'ti sikkhā karaṇīyā.



\pdfbookmark[3]{Sekhiya 32}{sekh32}
\subsubsection*{\hyperref[training32]{Sekhiya 32: Pattasaññībhuñjanasikkhāpadaṁ}}
\label{sekh32}

Pattasaññī piṇḍapātaṁ bhuñjissāmī'ti, sikkhā karaṇīyā.



\pdfbookmark[3]{Sekhiya 33}{sekh33}
\subsubsection*{\hyperref[training33]{Sekhiya 33: Sapadānasikkhāpadaṁ}}
\label{sekh33}

Sapadānaṁ piṇḍapātaṁ bhuñjissāmī'ti, sikkhā karaṇīyā.



\pdfbookmark[3]{Sekhiya 34}{sekh34}
\subsubsection*{\hyperref[training34]{Sekhiya 34: Samasūpakasikkhāpadaṁ}}
\label{sekh34}

Samasūpakaṁ piṇḍapātaṁ bhuñjissāmī'ti sikkhā karaṇīyā.



\pdfbookmark[3]{Sekhiya 35}{sekh35}
\subsubsection*{\hyperref[training35]{Sekhiya 35: Na-thūpakatasikkhāpadaṁ}}
\label{sekh35}

Na thūpakato omadditvā piṇḍapātaṁ bhuñjissāmī'ti, sikkhā karaṇīyā.



\pdfbookmark[3]{Sekhiya 36}{sekh36}
\subsubsection*{\hyperref[training36]{Sekhiya 36: Odanappaṭicchādanasikkhāpadaṁ}}
\label{sekh36}

Na sūpaṁ vā byañjanaṁ vā odanena paṭicchādessāmi bhiyyokamyataṁ upādāyā'ti, sikkhā karaṇīyā.



\pdfbookmark[3]{Sekhiya 37}{sekh37}
\subsubsection*{\hyperref[training37]{Sekhiya 37: Sūpodanaviññattisikkhāpadaṁ}}
\label{sekh37}

Na sūpaṁ vā odanaṁ vā agilāno attano atthāya viññāpetvā bhuñjissāmī'ti, sikkhā karaṇīyā.



\pdfbookmark[3]{Sekhiya 38}{sekh38}
\subsubsection*{\hyperref[training38]{Sekhiya 38: Ujjhānasaññīsikkhāpadaṁ}}
\label{sekh38}

Na ujjhānasaññī paresaṁ pattaṁ olokessāmī'ti, sikkhā karaṇīyā.



\pdfbookmark[3]{Sekhiya 39}{sekh39}
\subsubsection*{\hyperref[training39]{Sekhiya 39: Kabaḷasikkhāpadaṁ}}
\label{sekh39}

N'ātimahantaṁ kabaḷaṁ karissāmī'ti, sikkhā karaṇīyā.



\pdfbookmark[3]{Sekhiya 40}{sekh40}
\subsubsection*{\hyperref[training40]{Sekhiya 40: Ālopasikkhāpadaṁ}}
\label{sekh40}

Parimaṇḍalaṁ ālopaṁ karissāmī'ti sikkhā karaṇīyā.

\begin{center}
	Sakkaccavaggo catuttho
\end{center}



\subsection{Anāhaṭavaggo}
% \vspace{0.2cm}

\pdfbookmark[3]{Sekhiya 41}{sekh41}
\subsubsection*{\hyperref[training41]{Sekhiya 41: Anāhaṭasikkhāpadaṁ}}
\label{sekh41}

Na anāhaṭe kabaḷe mukhadvāraṁ vivarissāmī'ti, sikkhā karaṇīyā.



\pdfbookmark[3]{Sekhiya 42}{sekh42}
\subsubsection*{\hyperref[training42]{Sekhiya 42: Bhuñjamānasikkhāpadaṁ}}
\label{sekh42}

Na bhuñjamāno sabbaṁ hatthaṁ mukhe pakkhipissāmī'ti sikkhā karaṇīyā.



\pdfbookmark[3]{Sekhiya 43}{sekh43}
\subsubsection*{\hyperref[training43]{Sekhiya 43: Sakabaḷasikkhāpadaṁ}}
\label{sekh43}

Na sakabaḷena mukhena byāharissāmī'ti, sikkhā karaṇīyā.



\pdfbookmark[3]{Sekhiya 44}{sekh44}
\subsubsection*{\hyperref[training44]{Sekhiya 44: Piṇḍukkhepakasikkhāpadaṁ}}
\label{sekh44}

Na piṇḍ'ukkhepakaṁ bhuñjissāmī'ti sikkhā karaṇīyā.



\pdfbookmark[3]{Sekhiya 45}{sekh45}
\subsubsection*{\hyperref[training45]{Sekhiya 45: Kabaḷāvacchedakasikkhāpadaṁ}}
\label{sekh45}

Na kabaḷ'āvacchedakaṁ bhuñjissāmī'ti, sikkhā karaṇīyā.



\pdfbookmark[3]{Sekhiya 46}{sekh46}
\subsubsection*{\hyperref[training46]{Sekhiya 46: Avagaṇḍakārakasikkhāpadaṁ}}
\label{sekh46}

Na avagaṇḍakārakaṁ bhuñjissāmī'ti sikkhā karaṇīyā.



\pdfbookmark[3]{Sekhiya 47}{sekh47}
\subsubsection*{\hyperref[training47]{Sekhiya 47: Hatthaniddhunakasikkhāpadaṁ}}
\label{sekh47}

Na hatthaniddhunakaṁ bhuñjissāmī'ti, sikkhā karaṇīyā.



\pdfbookmark[3]{Sekhiya 48}{sekh48}
\subsubsection*{\hyperref[training48]{Sekhiya 48: Sitthāvakārakasikkhāpadaṁ}}
\label{sekh48}

Na sitth'āvakārakaṁ bhuñjissāmī'ti, sikkhā karaṇīyā.



\pdfbookmark[3]{Sekhiya 49}{sekh49}
\subsubsection*{\hyperref[training49]{Sekhiya 49: Jivhānicchārakasikkhāpadaṁ}}
\label{sekh49}

Na jivhānicchārakaṁ bhuñjissāmī'ti sikkhā karaṇīyā.



\pdfbookmark[3]{Sekhiya 50}{sekh50}
\subsubsection*{\hyperref[training50]{Sekhiya 50: Capucapukārakasikkhāpadaṁ}}
\label{sekh50}

Na capucapukārakaṁ bhuñjissāmī'ti sikkhā karaṇīyā.

\begin{center}
	Kabaḷavaggo pañcamo
\end{center}



\subsection{Surusuruvaggo}
% \vspace{0.2cm}

\pdfbookmark[3]{Sekhiya 51}{sekh51}
\subsubsection*{\hyperref[training51]{Sekhiya 51: Surusurukārakasikkhāpadaṁ}}
\label{sekh51}

Na surusurukārakaṁ bhuñjissāmī'ti sikkhā karaṇīyā.



\pdfbookmark[3]{Sekhiya 52}{sekh52}
\subsubsection*{\hyperref[training52]{Sekhiya 52: Hatthanillehakasikkhāpadaṁ}}
\label{sekh52}

Na hatthanillehakaṁ bhuñjissāmī'ti sikkhā karaṇīyā.



\pdfbookmark[3]{Sekhiya 53}{sekh53}
\subsubsection*{\hyperref[training53]{Sekhiya 53: Pattanillehakasikkhāpadaṁ}}
\label{sekh53}

Na pattanillehakaṁ bhuñjissāmī'ti sikkhā karaṇīyā.



\pdfbookmark[3]{Sekhiya 54}{sekh54}
\subsubsection*{\hyperref[training54]{Sekhiya 54: Oṭṭhanillehakasikkhāpadaṁ}}
\label{sekh54}

Na oṭṭhanillehakaṁ bhuñjissāmī'ti, sikkhā karaṇīyā.



\pdfbookmark[3]{Sekhiya 55}{sekh55}
\subsubsection*{\hyperref[training55]{Sekhiya 55: Sāmisasikkhāpadaṁ}}
\label{sekh55}

Na sāmisena hatthena pānīyathālakaṁ paṭiggahessāmī'ti, sikkhā karaṇīyā.



\pdfbookmark[3]{Sekhiya 56}{sekh56}
\subsubsection*{\hyperref[training56]{Sekhiya 56: Sasitthakasikkhāpadaṁ}}
\label{sekh56}

Na sasitthakaṁ pattadhovanaṁ antaraghare chaḍḍessāmī'ti, sikkhā karaṇīyā.



\pdfbookmark[3]{Sekhiya 57}{sekh57}
\subsubsection*{\hyperref[training57]{Sekhiya 57: Chattapāṇisikkhāpadaṁ}}
\label{sekh57}

Na chattapāṇissa agilānassa dhammaṁ desessāmī'ti, sikkhā karaṇīyā.



\pdfbookmark[3]{Sekhiya 58}{sekh58}
\subsubsection*{\hyperref[training58]{Sekhiya 58: Daṇḍapāṇisikkhāpadaṁ}}
\label{sekh58}

Na daṇḍapāṇissa agilānassa dhammaṁ desessāmī'ti sikkhā karaṇīyā.



\pdfbookmark[3]{Sekhiya 59}{sekh59}
\subsubsection*{\hyperref[training59]{Sekhiya 59: Satthapāṇisikkhāpadaṁ}}
\label{sekh59}

Na satthapāṇissa agilānassa dhammaṁ desessāmī'ti, sikkhā karaṇīyā.



\pdfbookmark[3]{Sekhiya 60}{sekh60}
\subsubsection*{\hyperref[training60]{Sekhiya 60: Āvudhapāṇisikkhāpadaṁ}}
\label{sekh60}

Na āvudhapāṇissa agilānassa dhammaṁ desessāmī'ti, sikkhā karaṇīyā.

\begin{center}
	Surusuruvaggo chaṭṭho
\end{center}



\subsection{Pādukavaggo}
% \vspace{0.2cm}

\pdfbookmark[3]{Sekhiya 61}{sekh61}
\subsubsection*{\hyperref[training61]{Sekhiya 61: Pādukasikkhāpadaṁ}}
\label{sekh61}

Na pāduk'ārūḷhassa agilānassa dhammaṁ desessāmī'ti, sikkhā karaṇīyā.



\pdfbookmark[3]{Sekhiya 62}{sekh62}
\subsubsection*{\hyperref[training62]{Sekhiya 62: Upāhanasikkhāpadaṁ}}
\label{sekh62}

Na upāhan'ārūḷhassa agilānassa dhammaṁ desessāmī'ti, sikkhā karaṇīyā.



\pdfbookmark[3]{Sekhiya 63}{sekh63}
\subsubsection*{\hyperref[training63]{Sekhiya 63: Yānasikkhāpadaṁ}}
\label{sekh63}

Na yānagatassa agilānassa dhammaṁ desessāmī'ti sikkhā karaṇīyā.



\pdfbookmark[3]{Sekhiya 64}{sekh64}
\subsubsection*{\hyperref[training64]{Sekhiya 64: Sayanasikkhāpadaṁ}}
\label{sekh64}

Na sayanagatassa agilānassa dhammaṁ desessāmī'ti sikkhā karaṇīyā.



\pdfbookmark[3]{Sekhiya 65}{sekh65}
\subsubsection*{\hyperref[training65]{Sekhiya 65: Pallatthikasikkhāpadaṁ}}
\label{sekh65}

Na pallatthikāya nisinnassa agilānassa dhammaṁ desessāmī'ti sikkhā karaṇīyā.



\pdfbookmark[3]{Sekhiya 66}{sekh66}
\subsubsection*{\hyperref[training66]{Sekhiya 66: Veṭhitasikkhāpadaṁ}}
\label{sekh66}

Na veṭhitasīsassa agilānassa dhammaṁ desessāmī'ti, sikkhā karaṇīyā.



\pdfbookmark[3]{Sekhiya 67}{sekh67}
\subsubsection*{\hyperref[training67]{Sekhiya 67: Oguṇṭhitasikkhāpadaṁ}}
\label{sekh67}

Na oguṇṭhitasīsassa agilānassa dhammaṁ desessāmī'ti sikkhā karaṇīyā.



\pdfbookmark[3]{Sekhiya 68}{sekh68}
\subsubsection*{\hyperref[training68]{Sekhiya 68: Chamāsikkhāpadaṁ}}
\label{sekh68}

Na chamāyaṁ nisīditvā āsane nisinnassa agilānassa dhammaṁ desessāmī'ti, sikkhā karaṇīyā.



\pdfbookmark[3]{Sekhiya 69}{sekh69}
\subsubsection*{\hyperref[training69]{Sekhiya 69: Nīcāsanasikkhāpadaṁ}}
\label{sekh69}

Na nīce āsane nisīditvā ucce āsane nisinnassa agilānassa dhammaṁ desessāmī'ti, sikkhā karaṇīyā.



\pdfbookmark[3]{Sekhiya 70}{sekh70}
\subsubsection*{\hyperref[training70]{Sekhiya 70: Ṭhitasikkhāpadaṁ}}
\label{sekh70}

Na ṭhito nisinnassa agilānassa dhammaṁ desessāmī'ti, sikkhā karaṇīyā.



\pdfbookmark[3]{Sekhiya 71}{sekh71}
\subsubsection*{\hyperref[training71]{Sekhiya 71: Pacchatogamanasikkhāpadaṁ}}
\label{sekh71}

Na pacchato gacchanto purato gacchantassa agilānassa dhammaṁ desessāmī'ti, sikkhā karaṇīyā.



\pdfbookmark[3]{Sekhiya 72}{sekh72}
\subsubsection*{\hyperref[training72]{Sekhiya 72: Uppathenagamanasikkhāpadaṁ}}
\label{sekh72}

Na uppathena gacchanto pathena gacchantassa agilānassa dhammaṁ desessāmī'ti, sikkhā karaṇīyā.



\pdfbookmark[3]{Sekhiya 73}{sekh73}
\subsubsection*{\hyperref[training73]{Sekhiya 73: Ṭhito-uccārasikkhāpadaṁ}}
\label{sekh73}

Na ṭhito agilāno uccāraṁ vā passāvaṁ vā karissāmī'ti, sikkhā karaṇīyā.



\pdfbookmark[3]{Sekhiya 74}{sekh74}
\subsubsection*{\hyperref[training74]{Sekhiya 74: Harite-uccārasikkhāpadaṁ}}
\label{sekh74}

Na harite agilāno uccāraṁ vā passāvaṁ vā kheḷaṁ vā karissāmī'ti sikkhā karaṇīyā.



\pdfbookmark[3]{Sekhiya 75}{sekh75}
\subsubsection*{\hyperref[training75]{Sekhiya 75: Udake-uccārasikkhāpadaṁ}}
\label{sekh75}

Na udake agilāno uccāraṁ vā passāvaṁ vā kheḷaṁ vā karissāmī'ti sikkhā karaṇīyā.

\begin{center}
	Pādukavaggo sattamo
\end{center}



\medskip

\begin{center}
	Uddiṭṭhā kho āyasmanto sekhiyā dhammā.

	\smallskip

	Tatth'āyasmante pucchāmi: Kacci'ttha parisuddhā?\\
	Dutiyam'pi pucchāmi: Kacci'ttha parisuddhā?\\
	Tatiyam'pi pucchāmi: Kacci'ttha parisuddhā?

	\smallskip

	Parisuddh'etth'āyasmanto, tasmā tuṇhī, evam'etaṁ dhārayāmi.
\end{center}

\begin{outro}
	Sekhiyā niṭṭhitā
\end{outro}

\clearpage

