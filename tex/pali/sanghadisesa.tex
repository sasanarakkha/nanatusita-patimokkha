
\setsecheadstyle{\sectionFmt}
\section{Saṅghādises'uddeso}
\label{sd}

\begin{intro}
	Ime kho pan'āyasmanto terasa saṅghādisesā dhammā uddesaṁ āgacchanti.
\end{intro}

\pdfbookmark[2]{Saṅghādisesa 1}{sd1}
\subsection*{\hyperref[comm1]{Saṅghādisesa 1: Sukkavissaṭṭhisikkhāpadaṁ}}

\label{sd1}

\linkdest{endnote-body}
Sañcetanikā sukkavisaṭṭhi\makeatletter\hyperlink{endnote-appendix}\Hy@raisedlink{\hypertarget{endnote-body}{}{\pagenote{%
		\hypertarget{endnote-appendix}{\hyperlink{endnote-body}{}}}}}\makeatother C, G, V, W, Dm, Um, UP, Bh Pm 1 & 2, Pg, Ra, Vibh Ce, Vibh Ee: visaṭṭhi. Mm & Mi Se: vissaṭṭhi., aññatra supinantā, saṅghādiseso.



\pdfbookmark[2]{Saṅghādisesa 2}{sd2}
\subsection*{\hyperref[comm2]{Saṅghādisesa 2: Kāyasaṁsaggasikkhāpadaṁ}}
\label{sd2}

\linkdest{endnote-body}
\linkdest{endnote-body}
Yo pana bhikkhu otiṇṇo vipariṇatena cittena mātugāmena saddhiṁ kāyasaṁsaggaṁ samāpajjeyya, hatthagāhaṁ\makeatletter\hyperlink{endnote-appendix}\Hy@raisedlink{\hypertarget{endnote-body}{}{\pagenote{%
		\hypertarget{endnote-appendix}{\hyperlink{endnote-body}{}}}}}\makeatother Dm: hatthaggāhaṁ. vā veṇigāhaṁ\makeatletter\hyperlink{endnote-appendix}\Hy@raisedlink{\hypertarget{endnote-body}{}{\pagenote{%
		\hypertarget{endnote-appendix}{\hyperlink{endnote-body}{}}}}}\makeatother Dm: veṇiggāhaṁ. (Pg: venigāhaṁ) vā aññatarassa vā aññatarassa vā aṅgassa parāmasanaṁ, saṅghādiseso.



\pdfbookmark[2]{Saṅghādisesa 3}{sd3}
\subsection*{\hyperref[comm3]{Saṅghādisesa 3: Duṭṭhullavācāsikkhāpadaṁ}}
\label{sd3}

\linkdest{endnote-body}
\linkdest{endnote-body}
Yo pana bhikkhu otiṇṇo vipariṇatena cittena mātugāmaṁ duṭṭhullāhi vācāhi obhāseyya, yathā taṁ\makeatletter\hyperlink{endnote-appendix}\Hy@raisedlink{\hypertarget{endnote-body}{}{\pagenote{%
		\hypertarget{endnote-appendix}{\hyperlink{endnote-body}{}}}}}\makeatother All printed eds, except Mi Se, Um, Ra: yathā taṁ. yuvā yuvatiṁ, methun'ūpasaṁhitāhi\makeatletter\hyperlink{endnote-appendix}\Hy@raisedlink{\hypertarget{endnote-body}{}{\pagenote{%
		\hypertarget{endnote-appendix}{\hyperlink{endnote-body}{}}}}}\makeatother Dm, Vibh Ee: -upa-. Mi, Mm Se, V: -sañhitāhi, all other eds. -saṁhitāhi., saṅghādiseso.



\pdfbookmark[2]{Saṅghādisesa 4}{sd4}
\subsection*{\hyperref[comm4]{Saṅghādisesa 4: Attakāmapāricariyasikkhāpadaṁ}}
\label{sd4}

\linkdest{endnote-body}
\linkdest{endnote-body}
Yo pana bhikkhu otiṇṇo vipariṇatena cittena mātugāmassa santike attakāmapāricariyāya vaṇṇaṁ bhāseyya: ``Etad'aggaṁ bhagini pāricariyānaṁ yā mādisaṁ sīlavantaṁ kalyāṇadhammaṁ brahmacāriṁ etena dhammena paricareyyā'ti\makeatletter\hyperlink{endnote-appendix}\Hy@raisedlink{\hypertarget{endnote-body}{}{\pagenote{%
		\hypertarget{endnote-appendix}{\hyperlink{endnote-body}{}}}}}\makeatother Mm Se: pāri-.,'' methun'ūpasaṁhitena\makeatletter\hyperlink{endnote-appendix}\Hy@raisedlink{\hypertarget{endnote-body}{}{\pagenote{%
		\hypertarget{endnote-appendix}{\hyperlink{endnote-body}{}}}}}\makeatother Dm,Vibh Ee: methunupasaṁhitena. Mi & Mm Se, V: -ūpasañhitena., saṅghādiseso.



\pdfbookmark[2]{Saṅghādisesa 5}{sd5}
\subsection*{\hyperref[comm5]{Saṅghādisesa 5: Sañcarittasikkhāpadaṁ}}
\label{sd5}

\linkdest{endnote-body}
\linkdest{endnote-body}
Yo pana bhikkhu sañcarittaṁ samāpajjeyya, itthiyā vā purisamatiṁ purisassa vā itthimatiṁ\makeatletter\hyperlink{endnote-appendix}\Hy@raisedlink{\hypertarget{endnote-body}{}{\pagenote{%
		\hypertarget{endnote-appendix}{\hyperlink{endnote-body}{}}}}}\makeatother Mi & Mm Se: itthī-., jāyattane vā jārattane vā, antamaso taṁkhaṇikāya'pi\makeatletter\hyperlink{endnote-appendix}\Hy@raisedlink{\hypertarget{endnote-body}{}{\pagenote{%
		\hypertarget{endnote-appendix}{\hyperlink{endnote-body}{}}}}}\makeatother Be & Se Vibh: taṅkhaṇikāya., saṅghādiseso.



\pdfbookmark[2]{Saṅghādisesa 6}{sd6}
\subsection*{\hyperref[comm6]{Saṅghādisesa 6: Kuṭikārasikkhāpadaṁ}}
\label{sd6}

\linkdest{endnote-body}
\linkdest{endnote-body}
\linkdest{endnote-body}
\linkdest{endnote-body}
\linkdest{endnote-body}
\linkdest{endnote-body}
\linkdest{endnote-body}
\linkdest{endnote-body}
Saññācikāya\makeatletter\hyperlink{endnote-appendix}\Hy@raisedlink{\hypertarget{endnote-body}{}{\pagenote{%
		\hypertarget{endnote-appendix}{\hyperlink{endnote-body}{}}}}}\makeatother C, D, W: saṁyācikaya. pana bhikkhunā kuṭiṁ kārayamānena assāmikaṁ att'uddesaṁ, pamāṇikā kāretabbā. Tatr'idaṁ\makeatletter\hyperlink{endnote-appendix}\Hy@raisedlink{\hypertarget{endnote-body}{}{\pagenote{%
		\hypertarget{endnote-appendix}{\hyperlink{endnote-body}{}}}}}\makeatother V: tatrīdaṁ. pamāṇaṁ: dīghaso dvādasa vidatthiyo sugatavidatthiyā tiriyaṁ satt'antarā. Bhikkhū abhinetabbā vatthudesanāya. Tehi bhikkhūhi vatthuṁ\makeatletter\hyperlink{endnote-appendix}\Hy@raisedlink{\hypertarget{endnote-body}{}{\pagenote{%
		\hypertarget{endnote-appendix}{\hyperlink{endnote-body}{}}}}}\makeatother Dm, Um: vatthu (So UP in Sd 7). desetabbaṁ anārambhaṁ\makeatletter\hyperlink{endnote-appendix}\Hy@raisedlink{\hypertarget{endnote-body}{}{\pagenote{%
		\hypertarget{endnote-appendix}{\hyperlink{endnote-body}{}}}}}\makeatother Vibh Be v.l.: anārabbhaṁ. UP (sīhala) v.l. anārabhaṁ. saparikkamanaṁ\makeatletter\hyperlink{endnote-appendix}\Hy@raisedlink{\hypertarget{endnote-body}{}{\pagenote{%
		\hypertarget{endnote-appendix}{\hyperlink{endnote-body}{}}}}}\makeatother Ra, Um, Pg: -kamaṇaṁ.. Sārambhe\makeatletter\hyperlink{endnote-appendix}\Hy@raisedlink{\hypertarget{endnote-body}{}{\pagenote{%
		\hypertarget{endnote-appendix}{\hyperlink{endnote-body}{}}}}}\makeatother Vibh Be v.l.: sārabbhe. ce bhikkhu vatthusmiṁ aparikkamane\makeatletter\hyperlink{endnote-appendix}\Hy@raisedlink{\hypertarget{endnote-body}{}{\pagenote{%
		\hypertarget{endnote-appendix}{\hyperlink{endnote-body}{}}}}}\makeatother Ra, Pg: -kamaṇe. saññācikāya\makeatletter\hyperlink{endnote-appendix}\Hy@raisedlink{\hypertarget{endnote-body}{}{\pagenote{%
		\hypertarget{endnote-appendix}{\hyperlink{endnote-body}{}}}}}\makeatother C, D, W: saṁyācikaya. kuṭiṁ kāreyya, bhikkhū vā anabhineyya vatthudesanāya, pamāṇaṁ vā atikkāmeyya, saṅghādiseso.



\pdfbookmark[2]{Saṅghādisesa 7}{sd7}
\subsection*{\hyperref[comm7]{Saṅghādisesa 7: Vihārakārasikkhāpadaṁ}}
\label{sd7}

\linkdest{endnote-body}
\linkdest{endnote-body}
\linkdest{endnote-body}
\linkdest{endnote-body}
Mahallakaṁ pana\makeatletter\hyperlink{endnote-appendix}\Hy@raisedlink{\hypertarget{endnote-body}{}{\pagenote{%
		\hypertarget{endnote-appendix}{\hyperlink{endnote-body}{}}}}}\makeatother Mi Se, G, V: mahallakam-pana. bhikkhunā vihāraṁ kārayamānena sassāmikaṁ att'uddesaṁ bhikkhū abhinetabbā vatthudesanāya. Tehi bhikkhūhi vatthuṁ\makeatletter\hyperlink{endnote-appendix}\Hy@raisedlink{\hypertarget{endnote-body}{}{\pagenote{%
		\hypertarget{endnote-appendix}{\hyperlink{endnote-body}{}}}}}\makeatother Dm, UP, Um: vatthu. desetabbaṁ anārambhaṁ saparikkamanaṁ\makeatletter\hyperlink{endnote-appendix}\Hy@raisedlink{\hypertarget{endnote-body}{}{\pagenote{%
		\hypertarget{endnote-appendix}{\hyperlink{endnote-body}{}}}}}\makeatother Ra: -kamaṇaṁ.. Sārambhe ce bhikkhu vatthusmiṁ aparikkamane\makeatletter\hyperlink{endnote-appendix}\Hy@raisedlink{\hypertarget{endnote-body}{}{\pagenote{%
		\hypertarget{endnote-appendix}{\hyperlink{endnote-body}{}}}}}\makeatother Ra: -kamaṇe mahallakaṁ vihāraṁ kāreyya, bhikkhū vā anabhineyya vatthudesanāya, saṅghādiseso.



\pdfbookmark[2]{Saṅghādisesa 8}{sd8}
\subsection*{\hyperref[comm8]{Saṅghādisesa 8: Duṭṭhadosasikkhāpadaṁ}}
\label{sd8}

\linkdest{endnote-body}
\linkdest{endnote-body}
\linkdest{endnote-body}
Yo pana bhikkhu bhikkhuṁ duṭṭho doso appatīto amūlakena pārājikena dhammena anuddhaṁseyya: ``App'eva nāma naṁ imamhā brahmacariyā cāveyyan'ti,'' tato aparena samayena samanuggāhiyamāno\makeatletter\hyperlink{endnote-appendix}\Hy@raisedlink{\hypertarget{endnote-body}{}{\pagenote{%
		\hypertarget{endnote-appendix}{\hyperlink{endnote-body}{}}}}}\makeatother Dm: -ggahīya-.  vā asamanuggāhiyamāno\makeatletter\hyperlink{endnote-appendix}\Hy@raisedlink{\hypertarget{endnote-body}{}{\pagenote{%
		\hypertarget{endnote-appendix}{\hyperlink{endnote-body}{}}}}}\makeatother Dm: -ggahīya-.  vā, amūlakañ'c'eva\makeatletter\hyperlink{endnote-appendix}\Hy@raisedlink{\hypertarget{endnote-body}{}{\pagenote{%
		\hypertarget{endnote-appendix}{\hyperlink{endnote-body}{}}}}}\makeatother G: amūlakaṁ c'eva. taṁ adhikaraṇaṁ hoti, bhikkhu ca dosaṁ patiṭṭhāti, saṅghādiseso.



\pdfbookmark[2]{Saṅghādisesa 9}{sd9}
\subsection*{\hyperref[comm9]{Saṅghādisesa 9: Aññabhāgiyasikkhāpadaṁ}}
\label{sd9}

\linkdest{endnote-body}
\linkdest{endnote-body}
\linkdest{endnote-body}
\linkdest{endnote-body}
Yo pana bhikkhu bhikkhuṁ duṭṭho doso appatīto aññabhāgiyassa adhikaraṇassa kiñci desaṁ lesamattaṁ upādāya pārājikena dhammena anuddhaṁseyya: ``App'eva nāma naṁ imamhā brahmacariyā cāveyyan'ti,'' tato aparena samayena samanuggāhiyamāno\makeatletter\hyperlink{endnote-appendix}\Hy@raisedlink{\hypertarget{endnote-body}{}{\pagenote{%
		\hypertarget{endnote-appendix}{\hyperlink{endnote-body}{}}}}}\makeatother Dm: -ggahīya-.  vā asamanuggāhiyamāno\makeatletter\hyperlink{endnote-appendix}\Hy@raisedlink{\hypertarget{endnote-body}{}{\pagenote{%
		\hypertarget{endnote-appendix}{\hyperlink{endnote-body}{}}}}}\makeatother Dm: -ggahīya-.  vā, aññabhāgiyañ'c'eva\makeatletter\hyperlink{endnote-appendix}\Hy@raisedlink{\hypertarget{endnote-body}{}{\pagenote{%
		\hypertarget{endnote-appendix}{\hyperlink{endnote-body}{}}}}}\makeatother Ra: aññabhāgiyaṁ ceva. taṁ adhikaraṇaṁ hoti, koci deso lesamatto upādinno\makeatletter\hyperlink{endnote-appendix}\Hy@raisedlink{\hypertarget{endnote-body}{}{\pagenote{%
		\hypertarget{endnote-appendix}{\hyperlink{endnote-body}{}}}}}\makeatother Um, G, V: upādiṇṇo., bhikkhu ca dosaṁ patiṭṭhāti, saṅghādiseso.



\pdfbookmark[2]{Saṅghādisesa 10}{sd10}
\subsection*{\hyperref[comm10]{Saṅghādisesa 10: Saṅghabhedasikkhāpadaṁ}}
\label{sd10}

\linkdest{endnote-body}
\linkdest{endnote-body}
\linkdest{endnote-body}
\linkdest{endnote-body}
\linkdest{endnote-body}
Yo pana bhikkhu samaggassa saṅghassa bhedāya parakkameyya, bhedanasaṁvattanikaṁ vā adhikaraṇaṁ samādāya paggayha tiṭṭheyya, so bhikkhu bhikkhūhi evam'assa vacanīyo\makeatletter\hyperlink{endnote-appendix}\Hy@raisedlink{\hypertarget{endnote-body}{}{\pagenote{%
		\hypertarget{endnote-appendix}{\hyperlink{endnote-body}{}}}}}\makeatother V: vacaniyo.: ``Mā āyasmā\makeatletter\hyperlink{endnote-appendix}\Hy@raisedlink{\hypertarget{endnote-body}{}{\pagenote{%
		\hypertarget{endnote-appendix}{\hyperlink{endnote-body}{}}}}}\makeatother Dm, Um, UP: māyasmā. samaggassa saṅghassa bhedāya parakkami\makeatletter\hyperlink{endnote-appendix}\Hy@raisedlink{\hypertarget{endnote-body}{}{\pagenote{%
		\hypertarget{endnote-appendix}{\hyperlink{endnote-body}{}}}}}\makeatother Ra: parakkamī. bhedanasaṁvattanikaṁ vā adhikaraṇaṁ samādāya paggayha aṭṭhāsi. Samet'āyasmā saṅghena, samaggo hi saṅgho sammodamāno avivadamāno ek'uddeso phāsu viharatī'ti'', evañ'ca so bhikkhu bhikkhūhi vuccamāno tath'eva paggaṇheyya, so bhikkhu bhikkhūhi yāvatatiyaṁ samanubhāsitabbo tassa paṭinissaggāya, yāvatatiyañ'ce samanubhāsiyamāno taṁ paṭinissajeyya\makeatletter\hyperlink{endnote-appendix}\Hy@raisedlink{\hypertarget{endnote-body}{}{\pagenote{%
		\hypertarget{endnote-appendix}{\hyperlink{endnote-body}{}}}}}\makeatother D, W, Vibh Ce (but has -nissajjeyya in Pāc 68), Other eds.: -nissajjeyya. C reads -nissajjeyya here but -nissajeyya in Sd 12–13
and Pāc 68., icc'etaṁ kusalaṁ, no ce paṭinissajeyya\makeatletter\hyperlink{endnote-appendix}\Hy@raisedlink{\hypertarget{endnote-body}{}{\pagenote{%
		\hypertarget{endnote-appendix}{\hyperlink{endnote-body}{}}}}}\makeatother D, W, Vibh Ce (but has -nissajjeyya in Pāc 68), Other eds.: -nissajjeyya. C reads -nissajjeyya here but -nissajeyya in Sd 12–13
and Pāc 68., saṅghādiseso.



\pdfbookmark[2]{Saṅghādisesa 11}{sd11}
\subsection*{\hyperref[comm11]{Saṅghādisesa 11: Bhed'ānuvattakasikkhāpadaṁ}}
\label{sd11}

\linkdest{endnote-body}
\linkdest{endnote-body}
\linkdest{endnote-body}
\linkdest{endnote-body}
\linkdest{endnote-body}
\linkdest{endnote-body}
\linkdest{endnote-body}
\linkdest{endnote-body}
\linkdest{endnote-body}
\linkdest{endnote-body}
Tass'eva kho pana bhikkhussa bhikkhū honti anuvattakā vaggavādakā, eko vā dve vā tayo vā, te evaṁ vadeyyuṁ: ``Mā āyasmanto\makeatletter\hyperlink{endnote-appendix}\Hy@raisedlink{\hypertarget{endnote-body}{}{\pagenote{%
		\hypertarget{endnote-appendix}{\hyperlink{endnote-body}{}}}}}\makeatother Dm, Um, UP: māyasmanto. etaṁ bhikkhuṁ kiñci avacuttha, dhammavādī c'eso bhikkhu, vinayavādī c'eso bhikkhu, amhākañ'c'eso\makeatletter\hyperlink{endnote-appendix}\Hy@raisedlink{\hypertarget{endnote-body}{}{\pagenote{%
		\hypertarget{endnote-appendix}{\hyperlink{endnote-body}{}}}}}\makeatother ibh Ce, Um, W: amhākaṁ c'eso. bhikkhu, chandañ'ca ruciñ'ca ādāya voharati, jānāti no bhāsati, amhākam'p'etaṁ\makeatletter\hyperlink{endnote-appendix}\Hy@raisedlink{\hypertarget{endnote-body}{}{\pagenote{%
		\hypertarget{endnote-appendix}{\hyperlink{endnote-body}{}}}}}\makeatother C, D, W: amhākaṁ p'etaṁ. khamatī'ti,'' te bhikkhū bhikkhūhi evam'assu vacanīyā\makeatletter\hyperlink{endnote-appendix}\Hy@raisedlink{\hypertarget{endnote-body}{}{\pagenote{%
		\hypertarget{endnote-appendix}{\hyperlink{endnote-body}{}}}}}\makeatother V: vacaniyā.: ``Mā āyasmanto\makeatletter\hyperlink{endnote-appendix}\Hy@raisedlink{\hypertarget{endnote-body}{}{\pagenote{%
		\hypertarget{endnote-appendix}{\hyperlink{endnote-body}{}}}}}\makeatother Dm, Um, UP: māyasmanto. evaṁ avacuttha. Na c'eso bhikkhu dhammavādī, na c'eso bhikkhu vinayavādī. Mā āyasmantānam'pi\makeatletter\hyperlink{endnote-appendix}\Hy@raisedlink{\hypertarget{endnote-body}{}{\pagenote{%
		\hypertarget{endnote-appendix}{\hyperlink{endnote-body}{}}}}}\makeatother Dm, UP: māyasmantānam-pi, Um: māyasmantānaṁ pi. saṅghabhedo ruccittha\makeatletter\hyperlink{endnote-appendix}\Hy@raisedlink{\hypertarget{endnote-body}{}{\pagenote{%
		\hypertarget{endnote-appendix}{\hyperlink{endnote-body}{}}}}}\makeatother Mi & Mm Se, D, C, G, V, W: rucittha (= BhPm 1 & 2 v.l.). Samet'āyasmantānaṁ saṅghena, samaggo hi saṅgho sammodamāno avivadamāno\makeatletter\hyperlink{endnote-appendix}\Hy@raisedlink{\hypertarget{endnote-body}{}{\pagenote{%
		\hypertarget{endnote-appendix}{\hyperlink{endnote-body}{}}}}}\makeatother G: avivādamāno. ek'uddeso phāsu viharatī'ti,'' evañ'ca te bhikkhū bhikkhūhi vuccamānā tath'eva paggaṇheyyuṁ, te bhikkhū bhikkhūhi yāvatatiyaṁ samanubhāsitabbā tassa paṭinissaggāya, yāvatatiyañ'ce samanubhāsiyamānā taṁ paṭinissajeyyuṁ\makeatletter\hyperlink{endnote-appendix}\Hy@raisedlink{\hypertarget{endnote-body}{}{\pagenote{%
		\hypertarget{endnote-appendix}{\hyperlink{endnote-body}{}}}}}\makeatother Vibh Ce, C, D, W. Other eds: paṭinissajjeyyuṁ. Cf Sd 11. icc'etaṁ kusalaṁ, no ce paṭinissajeyyuṁ\makeatletter\hyperlink{endnote-appendix}\Hy@raisedlink{\hypertarget{endnote-body}{}{\pagenote{%
		\hypertarget{endnote-appendix}{\hyperlink{endnote-body}{}}}}}\makeatother Vibh Ce, C, D, W. Other eds: paṭinissajjeyyuṁ. Cf Sd 11., saṅghādiseso.



\pdfbookmark[2]{Saṅghādisesa 12}{sd12}
\subsection*{\hyperref[comm12]{Saṅghādisesa 12: Dubbacasikkhāpadaṁ}}
\label{sd12}

\linkdest{endnote-body}
\linkdest{endnote-body}
\linkdest{endnote-body}
\linkdest{endnote-body}
\linkdest{endnote-body}
\linkdest{endnote-body}
\linkdest{endnote-body}
\linkdest{endnote-body}
Bhikkhu pan'eva dubbacajātiko hoti, uddesapariyāpannesu sikkhāpadesu bhikkhūhi sahadhammikaṁ vuccamāno attānaṁ avacanīyaṁ karoti: ``Mā maṁ āyasmanto kiñci avacuttha, kalyāṇaṁ vā pāpakaṁ vā, aham'p'āyasmante na kiñci vakkhāmi, kalyāṇaṁ vā pāpakaṁ vā. Viramath'āyasmanto mama vacanāyā'ti'', so bhikkhu bhikkhūhi evam'assa vacanīyo\makeatletter\hyperlink{endnote-appendix}\Hy@raisedlink{\hypertarget{endnote-body}{}{\pagenote{%
		\hypertarget{endnote-appendix}{\hyperlink{endnote-body}{}}}}}\makeatother V: vacaniyo. (Not so avacanīyaṁ and vacanīyaṁ below.): ``Mā āyasmā\makeatletter\hyperlink{endnote-appendix}\Hy@raisedlink{\hypertarget{endnote-body}{}{\pagenote{%
		\hypertarget{endnote-appendix}{\hyperlink{endnote-body}{}}}}}\makeatother  attānaṁ avacanīyaṁ akāsi. Vacanīyam'ev'āyasmā\makeatletter\hyperlink{endnote-appendix}\Hy@raisedlink{\hypertarget{endnote-body}{}{\pagenote{%
		\hypertarget{endnote-appendix}{\hyperlink{endnote-body}{}}}}}\makeatother Vibh Ee, Mm Se, BhPm 2, D: “... vacanīyaṁ eva āyasmā.” V: “… vacanīyameva āyasmā.” attānaṁ karotu. Āyasmā'pi bhikkhū vadetu\makeatletter\hyperlink{endnote-appendix}\Hy@raisedlink{\hypertarget{endnote-body}{}{\pagenote{%
		\hypertarget{endnote-appendix}{\hyperlink{endnote-body}{}}}}}\makeatother Dm, Be Sp, Um: vadatu. saha dhammena\makeatletter\hyperlink{endnote-appendix}\Hy@raisedlink{\hypertarget{endnote-body}{}{\pagenote{%
		\hypertarget{endnote-appendix}{\hyperlink{endnote-body}{}}}}}\makeatother All printed editions, except Ra and BhPm 1 & 2: sahadhammena., bhikkhū'pi āyasmantaṁ vakkhanti saha dhammena. Evaṁ saṁvaddhā\makeatletter\hyperlink{endnote-appendix}\Hy@raisedlink{\hypertarget{endnote-body}{}{\pagenote{%
		\hypertarget{endnote-appendix}{\hyperlink{endnote-body}{}}}}}\makeatother Mi & Mm Se, G, Um: -vaḍḍhā. All printed editions: evaṁ saṁvaddhā. Mi & Mm Se, G, V, Um: -vaḍḍhā. hi tassa bhagavato parisā, yad'idaṁ aññam'aññavacanena aññam'aññavuṭṭhāpanenā'ti,'' evañ'ca so bhikkhu bhikkhūhi vuccamāno tath'eva paggaṇheyya, so bhikkhu bhikkhūhi yāvatatiyaṁ samanubhāsitabbo tassa paṭinissaggāya, yāvatatiyañ'ce samanubhāsiyamāno taṁ paṭinissajeyya\makeatletter\hyperlink{endnote-appendix}\Hy@raisedlink{\hypertarget{endnote-body}{}{\pagenote{%
		\hypertarget{endnote-appendix}{\hyperlink{endnote-body}{}}}}}\makeatother D, W, Vibh Ce (but has -nissajjeyya in Pāc 68), Other eds.: -nissajjeyya. C reads -nissajjeyya here but -nissajeyya in Sd 12–13
and Pāc 68. icc'etaṁ kusalaṁ, no ce paṭinissajeyya\makeatletter\hyperlink{endnote-appendix}\Hy@raisedlink{\hypertarget{endnote-body}{}{\pagenote{%
		\hypertarget{endnote-appendix}{\hyperlink{endnote-body}{}}}}}\makeatother D, W, Vibh Ce (but has -nissajjeyya in Pāc 68), Other eds.: -nissajjeyya. C reads -nissajjeyya here but -nissajeyya in Sd 12–13
and Pāc 68., saṅghādiseso.



\pdfbookmark[2]{Saṅghādisesa 13}{sd13}
\subsection*{\hyperref[comm13]{Saṅghādisesa 13: Kuladūsakasikkhāpadaṁ}}
\label{sd13}

\linkdest{endnote-body}
\linkdest{endnote-body}
\linkdest{endnote-body}
\linkdest{endnote-body}
\linkdest{endnote-body}
\linkdest{endnote-body}
\linkdest{endnote-body}
\linkdest{endnote-body}
\linkdest{endnote-body}
\linkdest{endnote-body}
\linkdest{endnote-body}
\linkdest{endnote-body}
\linkdest{endnote-body}
\linkdest{endnote-body}
Bhikkhu pan'eva aññataraṁ gāmaṁ vā nigamaṁ vā upanissāya viharati kuladūsako pāpasamācāro. Tassa kho pāpakā\makeatletter\hyperlink{endnote-appendix}\Hy@raisedlink{\hypertarget{endnote-body}{}{\pagenote{%
		\hypertarget{endnote-appendix}{\hyperlink{endnote-body}{}}}}}\makeatother BhPm 1 & 2, C, D, G, V, W, Ra: tassa pāpakā samācārā dissanti c'eva suyyanti\makeatletter\hyperlink{endnote-appendix}\Hy@raisedlink{\hypertarget{endnote-body}{}{\pagenote{%
		\hypertarget{endnote-appendix}{\hyperlink{endnote-body}{}}}}}\makeatother C, D, W: sūyanti. ca, kulāni ca tena duṭṭhāni dissanti c'eva suyyanti\makeatletter\hyperlink{endnote-appendix}\Hy@raisedlink{\hypertarget{endnote-body}{}{\pagenote{%
		\hypertarget{endnote-appendix}{\hyperlink{endnote-body}{}}}}}\makeatother C, D, W: sūyanti. ca, so bhikkhu bhikkhūhi evam'assa vacanīyo\makeatletter\hyperlink{endnote-appendix}\Hy@raisedlink{\hypertarget{endnote-body}{}{\pagenote{%
		\hypertarget{endnote-appendix}{\hyperlink{endnote-body}{}}}}}\makeatother V: vacaniyo.: ``Āyasmā kho kuladūsako pāpasamācāro. Āyasmato kho pāpakā samācārā dissanti c'eva suyyanti\makeatletter\hyperlink{endnote-appendix}\Hy@raisedlink{\hypertarget{endnote-body}{}{\pagenote{%
		\hypertarget{endnote-appendix}{\hyperlink{endnote-body}{}}}}}\makeatother C, D, W: sūyanti. ca, kulāni c'āyasmatā duṭṭhāni dissanti c'eva suyyanti\makeatletter\hyperlink{endnote-appendix}\Hy@raisedlink{\hypertarget{endnote-body}{}{\pagenote{%
		\hypertarget{endnote-appendix}{\hyperlink{endnote-body}{}}}}}\makeatother C, D, W: sūyanti. ca. Pakkamat'āyasmā imamhā āvāsā. Alaṁ te\makeatletter\hyperlink{endnote-appendix}\Hy@raisedlink{\hypertarget{endnote-body}{}{\pagenote{%
		\hypertarget{endnote-appendix}{\hyperlink{endnote-body}{}}}}}\makeatother Mi & Mm Se, BhPm 1 & 2, C, G, V, W, Um, Ra: alan-te. idha vāsenā'ti\makeatletter\hyperlink{endnote-appendix}\Hy@raisedlink{\hypertarget{endnote-body}{}{\pagenote{%
		\hypertarget{endnote-appendix}{\hyperlink{endnote-body}{}}}}}\makeatother BhPm 1 & 2, Um, Vibh Ee: idhavāsenā ti.,'' evañ'ca so bhikkhu bhikkhūhi vuccamāno te bhikkhū evaṁ vadeyya: ``Chandagāmino ca bhikkhū, dosagāmino ca bhikkhū, mohagāmino ca bhikkhū, bhayagāmino ca bhikkhū, tādisikāya āpattiyā ekaccaṁ pabbājenti, ekaccaṁ na pabbājentī'ti,'' so bhikkhu bhikkhūhi evam'assa vacanīyo: ``Mā āyasmā\makeatletter\hyperlink{endnote-appendix}\Hy@raisedlink{\hypertarget{endnote-body}{}{\pagenote{%
		\hypertarget{endnote-appendix}{\hyperlink{endnote-body}{}}}}}\makeatother  evaṁ avaca, na ca bhikkhū chandagāmino, na ca bhikkhū dosagāmino, na ca bhikkhū mohagāmino, na ca bhikkhū bhayagāmino. Āyasmā kho kuladūsako pāpasamācāro, āyasmato kho pāpakā samācārā dissanti c'eva suyyanti\makeatletter\hyperlink{endnote-appendix}\Hy@raisedlink{\hypertarget{endnote-body}{}{\pagenote{%
		\hypertarget{endnote-appendix}{\hyperlink{endnote-body}{}}}}}\makeatother C, D, W: sūyanti. ca, kulāni c'āyasmatā duṭṭhāni dissanti c'eva suyyanti\makeatletter\hyperlink{endnote-appendix}\Hy@raisedlink{\hypertarget{endnote-body}{}{\pagenote{%
		\hypertarget{endnote-appendix}{\hyperlink{endnote-body}{}}}}}\makeatother C, D, W: sūyanti. ca. Pakkamat'āyasmā imamhā āvāsā. Alaṁ te\makeatletter\hyperlink{endnote-appendix}\Hy@raisedlink{\hypertarget{endnote-body}{}{\pagenote{%
		\hypertarget{endnote-appendix}{\hyperlink{endnote-body}{}}}}}\makeatother Mi & Mm Se, BhPm 1 & 2, C, G, V, W, Um, Ra: alan-te. idha vāsenā'ti,'' evañ'ca so bhikkhu bhikkhūhi vuccamāno tath'eva paggaṇheyya, so bhikkhu bhikkhūhi yāvatatiyaṁ samanubhāsitabbo tassa paṭinissaggāya, yāvatatiyañ'ce samanubhāsiyamāno taṁ paṭinissajeyya\makeatletter\hyperlink{endnote-appendix}\Hy@raisedlink{\hypertarget{endnote-body}{}{\pagenote{%
		\hypertarget{endnote-appendix}{\hyperlink{endnote-body}{}}}}}\makeatother D, W, Vibh Ce (but has -nissajjeyya in Pāc 68), Other eds.: -nissajjeyya. C reads -nissajjeyya here but -nissajeyya in Sd 12–13
and Pāc 68. icc'etaṁ kusalaṁ, no ce paṭinissajeyya\makeatletter\hyperlink{endnote-appendix}\Hy@raisedlink{\hypertarget{endnote-body}{}{\pagenote{%
		\hypertarget{endnote-appendix}{\hyperlink{endnote-body}{}}}}}\makeatother D, W, Vibh Ce (but has -nissajjeyya in Pāc 68), Other eds.: -nissajjeyya. C reads -nissajjeyya here but -nissajeyya in Sd 12–13
and Pāc 68., saṅghādiseso.



\medskip

\linkdest{endnote-body}
\linkdest{endnote-body}
\linkdest{endnote-body}
\linkdest{endnote-body}
\linkdest{endnote-body}
\linkdest{endnote-body}
\linkdest{endnote-body}
\linkdest{endnote-body}
\linkdest{endnote-body}
\begin{center}
	Uddiṭṭhā kho āyasmanto terasa saṅghādisesā dhammā, nava paṭham'āpattikā\makeatletter\hyperlink{endnote-appendix}\Hy@raisedlink{\hypertarget{endnote-body}{}{\pagenote{%
		\hypertarget{endnote-appendix}{\hyperlink{endnote-body}{}}}}}\makeatother V: patham- cattāro yāvatatiyakā. Yesaṁ bhikkhu aññataraṁ vā aññataraṁ vā āpajjitvā, yāvat'ihaṁ\makeatletter\hyperlink{endnote-appendix}\Hy@raisedlink{\hypertarget{endnote-body}{}{\pagenote{%
		\hypertarget{endnote-appendix}{\hyperlink{endnote-body}{}}}}}\makeatother Be, UP, G, V: yāvatīhaṁ. jānaṁ paṭicchādeti, tāvat'ihaṁ\makeatletter\hyperlink{endnote-appendix}\Hy@raisedlink{\hypertarget{endnote-body}{}{\pagenote{%
		\hypertarget{endnote-appendix}{\hyperlink{endnote-body}{}}}}}\makeatother Be, UP, G, V: tāvatīhaṁ. tena bhikkhunā akāmā parivatthabbaṁ\makeatletter\hyperlink{endnote-appendix}\Hy@raisedlink{\hypertarget{endnote-body}{}{\pagenote{%
		\hypertarget{endnote-appendix}{\hyperlink{endnote-body}{}}}}}\makeatother V: parivaṭṭhabbaṁ.. Parivutthaparivāsena\makeatletter\hyperlink{endnote-appendix}\Hy@raisedlink{\hypertarget{endnote-body}{}{\pagenote{%
		\hypertarget{endnote-appendix}{\hyperlink{endnote-body}{}}}}}\makeatother V: parivuṭṭha-. bhikkhunā uttariṁ\makeatletter\hyperlink{endnote-appendix}\Hy@raisedlink{\hypertarget{endnote-body}{}{\pagenote{%
		\hypertarget{endnote-appendix}{\hyperlink{endnote-body}{}}}}}\makeatother Dm, Vibh Ce, Um: uttari. chārattaṁ bhikkhumānattāya paṭipajjitabbaṁ. Ciṇṇamānatto bhikkhu, yattha siyā vīsatigaṇo bhikkhusaṅgho\makeatletter\hyperlink{endnote-appendix}\Hy@raisedlink{\hypertarget{endnote-body}{}{\pagenote{%
		\hypertarget{endnote-appendix}{\hyperlink{endnote-body}{}}}}}\makeatother BhPm 1, C, V, W: -saṁgho, tattha so bhikkhu\makeatletter\hyperlink{endnote-appendix}\Hy@raisedlink{\hypertarget{endnote-body}{}{\pagenote{%
		\hypertarget{endnote-appendix}{\hyperlink{endnote-body}{}}}}}\makeatother Mi v.l.: bhikkhu bhikkhūhi. abbhetabbo. Ekena'pi ce ūno\makeatletter\hyperlink{endnote-appendix}\Hy@raisedlink{\hypertarget{endnote-body}{}{\pagenote{%
		\hypertarget{endnote-appendix}{\hyperlink{endnote-body}{}}}}}\makeatother V, Bh Pm 2 (syāma) v.l.: ono. Um, G: ūṇo. vīsatigaṇo bhikkhusaṅgho taṁ bhikkhuṁ abbheyya, so ca bhikkhu anabbhito, te ca bhikkhū gārayhā. Ayaṁ tattha sāmīci.

	\smallskip

	Tatth'āyasmante pucchāmi: Kacci'ttha parisuddhā?\\
	Dutiyam'pi pucchāmi: Kacci'ttha parisuddhā?\\
	Tatiyam'pi pucchāmi: Kacci'ttha parisuddhā?

	\smallskip

	Parisuddh'etth'āyasmanto, tasmā tuṇhī, evam'etaṁ dhārayāmi.
\end{center}

\linkdest{endnote11-body}
\linkdest{endnote-body}
\begin{outro}
	Terasa saṅghādisesā dhammā niṭṭhitā\makeatletter\hyperlink{endnote11-appendix}\Hy@raisedlink{\hypertarget{endnote11-body}{}{\pagenote{%
				\hypertarget{endnote11-appendix}{\hyperlink{endnote11-body}{Not in any edition or manuscript, but if a conclusion is to be recited then this one as given in the Parivāra would be the suitable one.\\
						When reciting in brief use: Saṅghādises'uddeso niṭṭhito.\makeatletter\hyperlink{endnote-appendix}\Hy@raisedlink{\hypertarget{endnote-body}{}{\pagenote{%
		\hypertarget{endnote-appendix}{\hyperlink{endnote-body}{}}}}}\makeatother Ñd Ce, UP, Mi Se: Saṅghādisesuddeso tatiyo. Dm: Saṅghādiseso niṭṭhito.}}}}}\makeatother
\end{outro}
% TODO change format to rule-count rule dhammā nitthita!
\clearpage

