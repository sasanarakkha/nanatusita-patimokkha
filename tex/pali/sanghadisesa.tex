
\setsecheadstyle{\sectionFmt}
\section{Saṅghādises'uddesa}
\label{sd}

\begin{intro}
	Ime kho pan'āyasmanto terasa saṅghādisesā dhammā uddesaṁ āgacchanti.
\end{intro}

\pdfbookmark[2]{Saṅghādisesa 1}{sd1}
\subsection*{\hyperref[comm1]{Saṅghādisesa 1: Sukkavissaṭṭhisikkhāpada}}

\label{sd1}

\linkdest{endnote38-body}
Sañcetanikā sukkavisaṭṭhi,\makeatletter\hyperlink{endnote38-appendix}\Hy@raisedlink{\hypertarget{endnote38-body}{}{\pagenote{%
		\hypertarget{endnote38-appendix}{\hyperlink{endnote38-body}{C, G, V, W, Dm, Um, UP, Bh Pm 1 \& 2, Pg, Ra, SVibh Ce, SVibh Ee: visaṭṭhi. Mm \& Mi Se: vissaṭṭhi.}}}}}\makeatother \thinspace aññatra supinantā, saṅghādiseso.



\pdfbookmark[2]{Saṅghādisesa 2}{sd2}
\subsection*{\hyperref[comm2]{Saṅghādisesa 2: Kāyasaṁsaggasikkhāpada}}
\label{sd2}

\linkdest{endnote39-body}
\linkdest{endnote40-body}
Yo pana bhikkhu otiṇṇo vipariṇatena cittena mātugāmena saddhiṁ kāyasaṁsaggaṁ samāpajjeyya, hatthagāhaṁ\makeatletter\hyperlink{endnote39-appendix}\Hy@raisedlink{\hypertarget{endnote39-body}{}{\pagenote{%
		\hypertarget{endnote39-appendix}{\hyperlink{endnote39-body}{Dm: \textit{hatthaggāhaṁ}.}}}}}\makeatother \thinspace vā veṇigāhaṁ\makeatletter\hyperlink{endnote40-appendix}\Hy@raisedlink{\hypertarget{endnote40-body}{}{\pagenote{%
		\hypertarget{endnote40-appendix}{\hyperlink{endnote40-body}{Dm: veṇiggāhaṁ. (Pg: \textit{venigāhaṁ}).}}}}}\makeatother \thinspace vā aññatarassa vā aññatarassa vā aṅgassa parāmasanaṁ, saṅghādiseso.



\pdfbookmark[2]{Saṅghādisesa 3}{sd3}
\subsection*{\hyperref[comm3]{Saṅghādisesa 3: Duṭṭhullavācāsikkhāpada}}
\label{sd3}

\linkdest{endnote41-body}
\linkdest{endnote42-body}
Yo pana bhikkhu otiṇṇo vipariṇatena cittena mātugāmaṁ duṭṭhullāhi vācāhi obhāseyya, yathā taṁ\makeatletter\hyperlink{endnote41-appendix}\Hy@raisedlink{\hypertarget{endnote41-body}{}{\pagenote{%
		\hypertarget{endnote41-appendix}{\hyperlink{endnote41-body}{All printed eds, except Mi Se, Um, Ra: \textit{yathā taṁ}.}}}}}\makeatother \thinspace yuvā yuvatiṁ, methun'ūpasaṁhitāhi,\makeatletter\hyperlink{endnote42-appendix}\Hy@raisedlink{\hypertarget{endnote42-body}{}{\pagenote{%
		\hypertarget{endnote42-appendix}{\hyperlink{endnote42-body}{Dm, SVibh Ee: \textit{-upa-}. Mi \& Mm Se, V: \textit{-sañhitāhi}, all other eds. \textit{-saṁhitāhi}.}}}}}\makeatother \thinspace saṅghādiseso.



\pdfbookmark[2]{Saṅghādisesa 4}{sd4}
\subsection*{\hyperref[comm4]{Saṅghādisesa 4: Attakāmapāricariyasikkhāpada}}
\label{sd4}

\linkdest{endnote43-body}
\linkdest{endnote44-body}
Yo pana bhikkhu otiṇṇo vipariṇatena cittena mātugāmassa santike attakāmapāricariyāya vaṇṇaṁ bhāseyya: ``Etad'aggaṁ bhagini pāricariyānaṁ yā mādisaṁ sīlavantaṁ kalyāṇadhammaṁ brahmacāriṁ etena dhammena paricareyyā'ti,''\makeatletter\hyperlink{endnote-appendix}\Hy@raisedlink{\hypertarget{endnote-body}{}{\pagenote{%
		\hypertarget{endnote-appendix}{\hyperlink{endnote-body}{Mm Se: \textit{pāri-}.}}}}}\makeatother \thinspace methun'ūpasaṁhitena,\makeatletter\hyperlink{endnote-appendix}\Hy@raisedlink{\hypertarget{endnote-body}{}{\pagenote{%
		\hypertarget{endnote-appendix}{\hyperlink{endnote-body}{Dm, SVibh Ee: \textit{methunupasaṁhitena}. Mi \& Mm Se, V: \textit{-ūpasañhitena}.}}}}}\makeatother \thinspace saṅghādiseso.



\pdfbookmark[2]{Saṅghādisesa 5}{sd5}
\subsection*{\hyperref[comm5]{Saṅghādisesa 5: Sañcarittasikkhāpada}}
\label{sd5}

\linkdest{endnote45-body}
\linkdest{endnote46-body}
Yo pana bhikkhu sañcarittaṁ samāpajjeyya, itthiyā vā purisamatiṁ purisassa vā itthimatiṁ,\makeatletter\hyperlink{endnote45-appendix}\Hy@raisedlink{\hypertarget{endnote45-body}{}{\pagenote{%
		\hypertarget{endnote45-appendix}{\hyperlink{endnote45-body}{Mi \& Mm Se: \textit{itthī-}.}}}}}\makeatother \thinspace jāyattane vā jārattane vā, antamaso taṁkhaṇikāya'pi,\makeatletter\hyperlink{endnote46-appendix}\Hy@raisedlink{\hypertarget{endnote46-body}{}{\pagenote{%
		\hypertarget{endnote46-appendix}{\hyperlink{endnote46-body}{Be \& Se SVibh: \textit{taṅkhaṇikāya}.}}}}}\makeatother \thinspace saṅghādiseso.



\pdfbookmark[2]{Saṅghādisesa 6}{sd6}
\subsection*{\hyperref[comm6]{Saṅghādisesa 6: Kuṭikārasikkhāpada}}
\label{sd6}

\linkdest{endnote47-body}
\linkdest{endnote48-body}
\linkdest{endnote49-body}
\linkdest{endnote50-body}
\linkdest{endnote51-body}
\linkdest{endnote52-body}
\linkdest{endnote53-body}
\linkdest{endnote54-body}
Saññācikāya\makeatletter\hyperlink{endnote47-appendix}\Hy@raisedlink{\hypertarget{endnote47-body}{}{\pagenote{%
		\hypertarget{endnote47-appendix}{\hyperlink{endnote47-body}{C, D, W: \textit{saṁyācikaya}.}}}}}\makeatother \thinspace pana bhikkhunā kuṭiṁ kārayamānena assāmikaṁ att'uddesaṁ, pamāṇikā kāretabbā. Tatr'idaṁ\makeatletter\hyperlink{endnote48-appendix}\Hy@raisedlink{\hypertarget{endnote48-body}{}{\pagenote{%
		\hypertarget{endnote48-appendix}{\hyperlink{endnote48-body}{V: \textit{tatrīdaṁ}.}}}}}\makeatother \thinspace pamāṇaṁ: dīghaso dvādasa vidatthiyo sugatavidatthiyā tiriyaṁ satt'antarā. Bhikkhū abhinetabbā vatthudesanāya. Tehi bhikkhūhi vatthuṁ\makeatletter\hyperlink{endnote49-appendix}\Hy@raisedlink{\hypertarget{endnote49-body}{}{\pagenote{%
		\hypertarget{endnote49-appendix}{\hyperlink{endnote49-body}{Dm, Um: \textit{vatthu} (So UP in Sd 7).}}}}}\makeatother \thinspace desetabbaṁ anārambhaṁ\makeatletter\hyperlink{endnote50-appendix}\Hy@raisedlink{\hypertarget{endnote50-body}{}{\pagenote{%
		\hypertarget{endnote50-appendix}{\hyperlink{endnote50-body}{SVibh Be v.l.: \textit{anārabbhaṁ}. UP (sīhala) v.l. \textit{anārabhaṁ}.}}}}}\makeatother \thinspace saparikkamanaṁ\makeatletter\hyperlink{endnote51-appendix}\Hy@raisedlink{\hypertarget{endnote51-body}{}{\pagenote{%
		\hypertarget{endnote51-appendix}{\hyperlink{endnote51-body}{Ra, Um, Pg: \textit{-kamaṇaṁ}.}}}}}\makeatother \thinspace. Sārambhe\makeatletter\hyperlink{endnote52-appendix}\Hy@raisedlink{\hypertarget{endnote52-body}{}{\pagenote{%
		\hypertarget{endnote52-appendix}{\hyperlink{endnote52-body}{SVibh Be v.l.: \textit{sārabbhe}.}}}}}\makeatother \thinspace ce bhikkhu vatthusmiṁ aparikkamane\makeatletter\hyperlink{endnote53-appendix}\Hy@raisedlink{\hypertarget{endnote53-body}{}{\pagenote{%
		\hypertarget{endnote53-appendix}{\hyperlink{endnote53-body}{Ra, Pg: \textit{-kamaṇe}.}}}}}\makeatother \thinspace saññācikāya\makeatletter\hyperlink{endnote54-appendix}\Hy@raisedlink{\hypertarget{endnote54-body}{}{\pagenote{%
		\hypertarget{endnote54-appendix}{\hyperlink{endnote54-body}{C, D, W: \textit{saṁyācikaya}.}}}}}\makeatother \thinspace kuṭiṁ kāreyya, bhikkhū vā anabhineyya vatthudesanāya, pamāṇaṁ vā atikkāmeyya, saṅghādiseso.



\pdfbookmark[2]{Saṅghādisesa 7}{sd7}
\subsection*{\hyperref[comm7]{Saṅghādisesa 7: Vihārakārasikkhāpada}}
\label{sd7}

\linkdest{endnote55-body}
\linkdest{endnote56-body}
\linkdest{endnote57-body}
\linkdest{endnote58-body}
Mahallakaṁ pana\makeatletter\hyperlink{endnote55-appendix}\Hy@raisedlink{\hypertarget{endnote55-body}{}{\pagenote{%
	  \hypertarget{endnote55-appendix}{\hyperlink{endnote55-body}{Mi Se, G, V: \textit{mahallakam-pana}.}}}}}\makeatother \thinspace bhikkhunā vihāraṁ kārayamānena sassāmikaṁ att'uddesaṁ bhikkhū abhinetabbā vatthudesanāya. Tehi bhikkhūhi vatthuṁ\makeatletter\hyperlink{endnote56-appendix}\Hy@raisedlink{\hypertarget{endnote56-body}{}{\pagenote{%
	  \hypertarget{endnote56-appendix}{\hyperlink{endnote56-body}{Dm, UP, Um: \textit{vatthu}.}}}}}\makeatother \thinspace desetabbaṁ anārambhaṁ saparikkamanaṁ.\makeatletter\hyperlink{endnote57-appendix}\Hy@raisedlink{\hypertarget{endnote57-body}{}{\pagenote{%
	  \hypertarget{endnote57-appendix}{\hyperlink{endnote57-body}{Ra: \textit{-kamaṇaṁ}.}}}}}\makeatother \thinspace Sārambhe ce bhikkhu vatthusmiṁ aparikkamane\makeatletter\hyperlink{endnote58-appendix}\Hy@raisedlink{\hypertarget{endnote58-body}{}{\pagenote{%
	  \hypertarget{endnote58-appendix}{\hyperlink{endnote58-body}{Ra: \textit{-kamaṇe}.}}}}}\makeatother \thinspace mahallakaṁ vihāraṁ kāreyya, bhikkhū vā anabhineyya vatthudesanāya, saṅghādiseso.



\pdfbookmark[2]{Saṅghādisesa 8}{sd8}
\subsection*{\hyperref[comm8]{Saṅghādisesa 8: Duṭṭhadosasikkhāpada}}
\label{sd8}

\linkdest{endnote59-body}
\linkdest{endnote60-body}
\linkdest{endnote61-body}
Yo pana bhikkhu bhikkhuṁ duṭṭho doso appatīto amūlakena pārājikena dhammena anuddhaṁseyya: ``App'eva nāma naṁ imamhā brahmacariyā cāveyyan'ti,'' tato aparena samayena samanuggāhiyamāno\makeatletter\hyperlink{endnote59-appendix}\Hy@raisedlink{\hypertarget{endnote59-body}{}{\pagenote{%
		\hypertarget{endnote59-appendix}{\hyperlink{endnote59-body}{Dm: \textit{-ggahīya-}.}}}}}\makeatother \thinspace vā asamanuggāhiyamāno\makeatletter\hyperlink{endnote60-appendix}\Hy@raisedlink{\hypertarget{endnote60-body}{}{\pagenote{%
		\hypertarget{endnote60-appendix}{\hyperlink{endnote60-body}{Dm: \textit{-ggahīya-}.}}}}}\makeatother \thinspace vā, amūlakañ'c'eva\makeatletter\hyperlink{endnote-appendix}\Hy@raisedlink{\hypertarget{endnote-body}{}{\pagenote{%
		\hypertarget{endnote-appendix}{\hyperlink{endnote-body}{G: \textit{amūlakaṁ c'eva}.}}}}}\makeatother \thinspace taṁ adhikaraṇaṁ hoti, bhikkhu ca dosaṁ patiṭṭhāti, saṅghādiseso.



\pdfbookmark[2]{Saṅghādisesa 9}{sd9}
\subsection*{\hyperref[comm9]{Saṅghādisesa 9: Aññabhāgiyasikkhāpada}}
\label{sd9}

\linkdest{endnote62-body}
\linkdest{endnote63-body}
\linkdest{endnote64-body}
\linkdest{endnote65-body}
Yo pana bhikkhu bhikkhuṁ duṭṭho doso appatīto aññabhāgiyassa adhikaraṇassa kiñci desaṁ lesamattaṁ upādāya pārājikena dhammena anuddhaṁseyya: ``App'eva nāma naṁ imamhā brahmacariyā cāveyyan'ti,'' tato aparena samayena samanuggāhiyamāno\makeatletter\hyperlink{endnote62-appendix}\Hy@raisedlink{\hypertarget{endnote62-body}{}{\pagenote{%
		\hypertarget{endnote62-appendix}{\hyperlink{endnote62-body}{Dm: \textit{-ggahīya-}.}}}}}\makeatother \thinspace vā asamanuggāhiyamāno\makeatletter\hyperlink{endnote63-appendix}\Hy@raisedlink{\hypertarget{endnote63-body}{}{\pagenote{%
		\hypertarget{endnote63-appendix}{\hyperlink{endnote63-body}{Dm: \textit{-ggahīya-}.}}}}}\makeatother \thinspace vā, aññabhāgiyañ'c'eva\makeatletter\hyperlink{endnote64-appendix}\Hy@raisedlink{\hypertarget{endnote64-body}{}{\pagenote{%
		\hypertarget{endnote64-appendix}{\hyperlink{endnote64-body}{Ra: \textit{aññabhāgiyaṁ ceva}.}}}}}\makeatother \thinspace taṁ adhikaraṇaṁ hoti, koci deso lesamatto upādinno,\makeatletter\hyperlink{endnote65-appendix}\Hy@raisedlink{\hypertarget{endnote65-body}{}{\pagenote{%
		\hypertarget{endnote65-appendix}{\hyperlink{endnote65-body}{Um, G, V: \textit{upādiṇṇo}.}}}}}\makeatother \thinspace bhikkhu ca dosaṁ patiṭṭhāti, saṅghādiseso.



\pdfbookmark[2]{Saṅghādisesa 10}{sd10}
\subsection*{\hyperref[comm10]{Saṅghādisesa 10: Saṅghabhedasikkhāpada}}
\label{sd10}

\linkdest{endnote66-body}
\linkdest{endnote67-body}
\linkdest{endnote68-body}
\linkdest{endnote69-body}
\linkdest{endnote70-body}
Yo pana bhikkhu samaggassa saṅghassa bhedāya parakkameyya, bhedanasaṁvattanikaṁ vā adhikaraṇaṁ samādāya paggayha tiṭṭheyya, so bhikkhu bhikkhūhi evam'assa vacanīyo:\makeatletter\hyperlink{endnote66-appendix}\Hy@raisedlink{\hypertarget{endnote66-body}{}{\pagenote{%
		\hypertarget{endnote66-appendix}{\hyperlink{endnote66-body}{V: \textit{vacaniyo}.}}}}}\makeatother \thinspace ``Mā āyasmā\makeatletter\hyperlink{endnote67-appendix}\Hy@raisedlink{\hypertarget{endnote67-body}{}{\pagenote{%
		\hypertarget{endnote67-appendix}{\hyperlink{endnote67-body}{Dm, Um, UP: \textit{māyasmā}.}}}}}\makeatother \thinspace samaggassa saṅghassa bhedāya parakkami\makeatletter\hyperlink{endnote68-appendix}\Hy@raisedlink{\hypertarget{endnote68-body}{}{\pagenote{%
		\hypertarget{endnote68-appendix}{\hyperlink{endnote68-body}{Ra: \textit{parakkamī}.}}}}}\makeatother \thinspace bhedanasaṁvattanikaṁ vā adhikaraṇaṁ samādāya paggayha aṭṭhāsi. Samet'āyasmā saṅghena, samaggo hi saṅgho sammodamāno avivadamāno ek'uddeso phāsu viharatī'ti'', evañ'ca so bhikkhu bhikkhūhi vuccamāno tath'eva paggaṇheyya, so bhikkhu bhikkhūhi yāvatatiyaṁ samanubhāsitabbo tassa paṭinissaggāya, yāvatatiyañ'ce samanubhāsiyamāno taṁ paṭinissajeyya,\makeatletter\hyperlink{endnote69-appendix}\Hy@raisedlink{\hypertarget{endnote69-body}{}{\pagenote{%
		\hypertarget{endnote69-appendix}{\hyperlink{endnote69-body}{D, W, SVibh Ce (but has \textit{-nissajjeyya} in Pāc 68), Other eds: \textit{-nissajjeyya}. C reads \textit{-nissajjeyya} here but \textit{-nissajeyya} in Sd 12–13
and Pāc 68.}}}}}\makeatother \thinspace icc'etaṁ kusalaṁ, no ce paṭinissajeyya,\makeatletter\hyperlink{endnote70-appendix}\Hy@raisedlink{\hypertarget{endnote70-body}{}{\pagenote{%
		\hypertarget{endnote70-appendix}{\hyperlink{endnote70-body}{D, W, SVibh Ce (but has \textit{-nissajjeyya} in Pāc 68), Other eds: \textit{-nissajjeyya}. C reads \textit{-nissajjeyya} here but \textit{-nissajeyya} in Sd 12–13
and Pāc 68.}}}}}\makeatother \thinspace saṅghādiseso.



\pdfbookmark[2]{Saṅghādisesa 11}{sd11}
\subsection*{\hyperref[comm11]{Saṅghādisesa 11: Bhed'ānuvattakasikkhāpada}}
\label{sd11}

\linkdest{endnote71-body}
\linkdest{endnote72-body}
\linkdest{endnote73-body}
\linkdest{endnote74-body}
\linkdest{endnote75-body}
\linkdest{endnote76-body}
\linkdest{endnote77-body}
\linkdest{endnote78-body}
\linkdest{endnote79-body}
\linkdest{endnote80-body}
Tass'eva kho pana bhikkhussa bhikkhū honti anuvattakā vaggavādakā, eko vā dve vā tayo vā, te evaṁ vadeyyuṁ: ``Mā āyasmanto\makeatletter\hyperlink{endnote71-appendix}\Hy@raisedlink{\hypertarget{endnote71-body}{}{\pagenote{%
		\hypertarget{endnote71-appendix}{\hyperlink{endnote71-body}{Dm, Um, UP: \textit{māyasmanto}.}}}}}\makeatother \thinspace etaṁ bhikkhuṁ kiñci avacuttha, dhammavādī c'eso bhikkhu, vinayavādī c'eso bhikkhu, amhākañ'c'eso\makeatletter\hyperlink{endnote72-appendix}\Hy@raisedlink{\hypertarget{endnote72-body}{}{\pagenote{%
		\hypertarget{endnote72-appendix}{\hyperlink{endnote72-body}{SVibh Ce, Um, W: \textit{amhākaṁ c'eso}.}}}}}\makeatother \thinspace bhikkhu, chandañ'ca ruciñ'ca ādāya voharati, jānāti no bhāsati, amhākam'p'etaṁ\makeatletter\hyperlink{endnote73-appendix}\Hy@raisedlink{\hypertarget{endnote73-body}{}{\pagenote{%
		\hypertarget{endnote73-appendix}{\hyperlink{endnote73-body}{C, D, W: \textit{amhākaṁ p'etaṁ}.}}}}}\makeatother \thinspace khamatī'ti,'' te bhikkhū bhikkhūhi evam'assu vacanīyā:\makeatletter\hyperlink{endnote74-appendix}\Hy@raisedlink{\hypertarget{endnote74-body}{}{\pagenote{%
		\hypertarget{endnote74-appendix}{\hyperlink{endnote74-body}{V: \textit{vacaniyā}.}}}}}\makeatother \thinspace ``Mā āyasmanto\makeatletter\hyperlink{endnote75-appendix}\Hy@raisedlink{\hypertarget{endnote75-body}{}{\pagenote{%
		\hypertarget{endnote75-appendix}{\hyperlink{endnote75-body}{Dm, Um, UP: \textit{māyasmanto}.}}}}}\makeatother \thinspace evaṁ avacuttha. Na c'eso bhikkhu dhammavādī, na c'eso bhikkhu vinayavādī. Mā āyasmantānam'pi\makeatletter\hyperlink{endnote76-appendix}\Hy@raisedlink{\hypertarget{endnote76-body}{}{\pagenote{%
		\hypertarget{endnote76-appendix}{\hyperlink{endnote76-body}{Dm, UP: \textit{māyasmantānam-pi}, Um: \textit{māyasmantānaṁ pi}.}}}}}\makeatother \thinspace saṅghabhedo ruccittha.\makeatletter\hyperlink{endnote77-appendix}\Hy@raisedlink{\hypertarget{endnote77-body}{}{\pagenote{%
		\hypertarget{endnote77-appendix}{\hyperlink{endnote77-body}{Mi \& Mm Se, D, C, G, V, W: \textit{rucittha} (= BhPm 1 \& 2 v.l.)}}}}}\makeatother \thinspace Samet'āyasmantānaṁ saṅghena, samaggo hi saṅgho sammodamāno avivadamāno\makeatletter\hyperlink{endnote78-appendix}\Hy@raisedlink{\hypertarget{endnote78-body}{}{\pagenote{%
		\hypertarget{endnote78-appendix}{\hyperlink{endnote78-body}{G: \textit{avivādamāno}.}}}}}\makeatother \thinspace ek'uddeso phāsu viharatī'ti,'' evañ'ca te bhikkhū bhikkhūhi vuccamānā tath'eva paggaṇheyyuṁ, te bhikkhū bhikkhūhi yāvatatiyaṁ samanubhāsitabbā tassa paṭinissaggāya, yāvatatiyañ'ce samanubhāsiyamānā taṁ paṭinissajeyyuṁ\makeatletter\hyperlink{endnote79-appendix}\Hy@raisedlink{\hypertarget{endnote79-body}{}{\pagenote{%
		\hypertarget{endnote79-appendix}{\hyperlink{endnote79-body}{SVibh Ce, C, D, W. Other eds: \textit{paṭinissajjeyyuṁ}. Cf Sd 11.}}}}}\makeatother \thinspace icc'etaṁ kusalaṁ, no ce paṭinissajeyyuṁ,\makeatletter\hyperlink{endnote80-appendix}\Hy@raisedlink{\hypertarget{endnote80-body}{}{\pagenote{%
		\hypertarget{endnote80-appendix}{\hyperlink{endnote80-body}{SVibh Ce, C, D, W. Other eds: \textit{paṭinissajjeyyuṁ}. Cf Sd 11.}}}}}\makeatother \thinspace saṅghādiseso.



\pdfbookmark[2]{Saṅghādisesa 12}{sd12}
\subsection*{\hyperref[comm12]{Saṅghādisesa 12: Dubbacasikkhāpada}}
\label{sd12}

\linkdest{endnote81-body}
\linkdest{endnote82-body}
\linkdest{endnote83-body}
\linkdest{endnote84-body}
\linkdest{endnote85-body}
\linkdest{endnote86-body}
\linkdest{endnote87-body}
\linkdest{endnote88-body}
Bhikkhu pan'eva dubbacajātiko hoti, uddesapariyāpannesu sikkhāpadesu bhikkhūhi sahadhammikaṁ vuccamāno attānaṁ avacanīyaṁ karoti: ``Mā maṁ āyasmanto kiñci avacuttha, kalyāṇaṁ vā pāpakaṁ vā, aham'p'āyasmante na kiñci vakkhāmi, kalyāṇaṁ vā pāpakaṁ vā. Viramath'āyasmanto mama vacanāyā'ti'', so bhikkhu bhikkhūhi evam'assa vacanīyo:\makeatletter\hyperlink{endnote81-appendix}\Hy@raisedlink{\hypertarget{endnote81-body}{}{\pagenote{%
		\hypertarget{endnote81-appendix}{\hyperlink{endnote81-body}{V: \textit{vacaniyo}. (Not so \textit{avacanīyaṁ} and \textit{vacanīyaṁ} below.)}}}}}\makeatother \thinspace ``Mā āyasmā\makeatletter\hyperlink{endnote82-appendix}\Hy@raisedlink{\hypertarget{endnote82-body}{}{\pagenote{%
		\hypertarget{endnote82-appendix}{\hyperlink{endnote82-body}{See Sd 10.}}}}}\makeatother \thinspace attānaṁ avacanīyaṁ akāsi. Vacanīyam'ev'āyasmā\makeatletter\hyperlink{endnote83-appendix}\Hy@raisedlink{\hypertarget{endnote83-body}{}{\pagenote{%
		\hypertarget{endnote83-appendix}{\hyperlink{endnote83-body}{SVibh Ee, Mm Se, BhPm 2, D: ...\textit{vacanīyaṁ eva āyasmā}.'' V: ...\textit{vacanīyameva āyasmā}.''}}}}}\makeatother \thinspace attānaṁ karotu. Āyasmā'pi bhikkhū vadetu\makeatletter\hyperlink{endnote84-appendix}\Hy@raisedlink{\hypertarget{endnote84-body}{}{\pagenote{%
		\hypertarget{endnote84-appendix}{\hyperlink{endnote84-body}{Dm, Be Sp, Um: vadatu.}}}}}\makeatother \thinspace saha dhammena,\makeatletter\hyperlink{endnote85-appendix}\Hy@raisedlink{\hypertarget{endnote85-body}{}{\pagenote{%
		\hypertarget{endnote85-appendix}{\hyperlink{endnote85-body}{All printed editions, except Ra and BhPm 1 \& 2: \textit{sahadhammena}.}}}}}\makeatother \thinspace bhikkhū'pi āyasmantaṁ vakkhanti saha dhammena. Evaṁ saṁvaddhā\makeatletter\hyperlink{endnote86-appendix}\Hy@raisedlink{\hypertarget{endnote86-body}{}{\pagenote{%
		\hypertarget{endnote86-appendix}{\hyperlink{endnote86-body}{Mi \& Mm Se, G, Um: \textit{-vaḍḍhā}. All printed editions: \textit{evaṁ saṁvaddhā}. Mi \& Mm Se, G, V, Um: \textit{-vaḍḍhā}.}}}}}\makeatother \thinspace hi tassa bhagavato parisā, yad'idaṁ aññam'aññavacanena aññam'aññavuṭṭhāpanenā'ti,'' evañ'ca so bhikkhu bhikkhūhi vuccamāno tath'eva paggaṇheyya, so bhikkhu bhikkhūhi yāvatatiyaṁ samanubhāsitabbo tassa paṭinissaggāya, yāvatatiyañ'ce samanubhāsiyamāno taṁ paṭinissajeyya\makeatletter\hyperlink{endnote87-appendix}\Hy@raisedlink{\hypertarget{endnote87-body}{}{\pagenote{%
		\hypertarget{endnote87-appendix}{\hyperlink{endnote87-body}{D, W, SVibh Ce (but has \textit{-nissajjeyya} in Pāc 68), Other eds.: \textit{-nissajjeyya}. C reads \textit{-nissajjeyya} here but \textit{-nissajeyya} in Sd 12–13
and Pāc 68.}}}}}\makeatother \thinspace icc'etaṁ kusalaṁ, no ce paṭinissajeyya,\makeatletter\hyperlink{endnote88-appendix}\Hy@raisedlink{\hypertarget{endnote88-body}{}{\pagenote{%
		\hypertarget{endnote88-appendix}{\hyperlink{endnote88-body}{D, W, SVibh Ce (but has \textit{-nissajjeyya} in Pāc 68), Other eds.: \textit{-nissajjeyya}. C reads \textit{-nissajjeyya} here but \textit{-nissajeyya} in Sd 12–13
and Pāc 68.}}}}}\makeatother \thinspace saṅghādiseso.



\pdfbookmark[2]{Saṅghādisesa 13}{sd13}
\subsection*{\hyperref[comm13]{Saṅghādisesa 13: Kuladūsakasikkhāpada}}
\label{sd13}

\linkdest{endnote89-body}
\linkdest{endnote90-body}
\linkdest{endnote91-body}
\linkdest{endnote92-body}
\linkdest{endnote93-body}
\linkdest{endnote94-body}
\linkdest{endnote95-body}
\linkdest{endnote96-body}
\linkdest{endnote97-body}
\linkdest{endnote98-body}
\linkdest{endnote99-body}
\linkdest{endnote100-body}
\linkdest{endnote101-body}
\linkdest{endnote102-body}
Bhikkhu pan'eva aññataraṁ gāmaṁ vā nigamaṁ vā upanissāya viharati kuladūsako pāpasamācāro. Tassa kho pāpakā\makeatletter\hyperlink{endnote89-appendix}\Hy@raisedlink{\hypertarget{endnote89-body}{}{\pagenote{%
		\hypertarget{endnote89-appendix}{\hyperlink{endnote89-body}{BhPm 1 \& 2, C, D, G, V, W, Ra: \textit{tassa pāpakā}...}}}}}\makeatother \thinspace samācārā dissanti c'eva suyyanti\makeatletter\hyperlink{endnote90-appendix}\Hy@raisedlink{\hypertarget{endnote90-body}{}{\pagenote{%
		\hypertarget{endnote90-appendix}{\hyperlink{endnote90-body}{C, D, W: \textit{sūyanti}.}}}}}\makeatother \thinspace ca, kulāni ca tena duṭṭhāni dissanti c'eva suyyanti\makeatletter\hyperlink{endnote91-appendix}\Hy@raisedlink{\hypertarget{endnote91-body}{}{\pagenote{%
		\hypertarget{endnote91-appendix}{\hyperlink{endnote91-body}{C, D, W: \textit{sūyanti}.}}}}}\makeatother \thinspace ca, so bhikkhu bhikkhūhi evam'assa vacanīyo:\makeatletter\hyperlink{endnote92-appendix}\Hy@raisedlink{\hypertarget{endnote92-body}{}{\pagenote{%
		\hypertarget{endnote92-appendix}{\hyperlink{endnote92-body}{ V: \textit{vacaniyo}.}}}}}\makeatother \thinspace ``Āyasmā kho kuladūsako pāpasamācāro. Āyasmato kho pāpakā samācārā dissanti c'eva suyyanti\makeatletter\hyperlink{endnote93-appendix}\Hy@raisedlink{\hypertarget{endnote93-body}{}{\pagenote{%
		\hypertarget{endnote93-appendix}{\hyperlink{endnote93-body}{C, D, W: \textit{sūyanti}.}}}}}\makeatother \thinspace ca, kulāni c'āyasmatā duṭṭhāni dissanti c'eva suyyanti\makeatletter\hyperlink{endnote94-appendix}\Hy@raisedlink{\hypertarget{endnote94-body}{}{\pagenote{%
		\hypertarget{endnote94-appendix}{\hyperlink{endnote94-body}{C, D, W: \textit{sūyanti}.}}}}}\makeatother \thinspace ca. Pakkamat'āyasmā imamhā āvāsā. Alaṁ te\makeatletter\hyperlink{endnote95-appendix}\Hy@raisedlink{\hypertarget{endnote95-body}{}{\pagenote{%
		\hypertarget{endnote95-appendix}{\hyperlink{endnote95-body}{Mi \& Mm Se, BhPm 1 \& 2, C, G, V, W, Um, Ra: \textit{alan-te}.}}}}}\makeatother \thinspace idha vāsenā'ti,''\makeatletter\hyperlink{endnote96-appendix}\Hy@raisedlink{\hypertarget{endnote96-body}{}{\pagenote{%
		\hypertarget{endnote96-appendix}{\hyperlink{endnote96-body}{BhPm 1 \& 2, Um, SVibh Ee: \textit{idhavāsenā ti}.}}}}}\makeatother \thinspace evañ'ca so bhikkhu bhikkhūhi vuccamāno te bhikkhū evaṁ vadeyya: ``Chandagāmino ca bhikkhū, dosagāmino ca bhikkhū, mohagāmino ca bhikkhū, bhayagāmino ca bhikkhū, tādisikāya āpattiyā ekaccaṁ pabbājenti, ekaccaṁ na pabbājentī'ti,'' so bhikkhu bhikkhūhi evam'assa vacanīyo: ``Mā āyasmā\makeatletter\hyperlink{endnote97-appendix}\Hy@raisedlink{\hypertarget{endnote97-body}{}{\pagenote{%
		\hypertarget{endnote97-appendix}{\hyperlink{endnote97-body}{See Sd 10.}}}}}\makeatother \thinspace evaṁ avaca, na ca bhikkhū chandagāmino, na ca bhikkhū dosagāmino, na ca bhikkhū mohagāmino, na ca bhikkhū bhayagāmino. Āyasmā kho kuladūsako pāpasamācāro, āyasmato kho pāpakā samācārā dissanti c'eva suyyanti\makeatletter\hyperlink{endnote98-appendix}\Hy@raisedlink{\hypertarget{endnote98-body}{}{\pagenote{%
		\hypertarget{endnote98-appendix}{\hyperlink{endnote98-body}{C, D, W: \textit{sūyanti}.}}}}}\makeatother \thinspace ca, kulāni c'āyasmatā duṭṭhāni dissanti c'eva suyyanti\makeatletter\hyperlink{endnote99-appendix}\Hy@raisedlink{\hypertarget{endnote99-body}{}{\pagenote{%
		\hypertarget{endnote99-appendix}{\hyperlink{endnote99-body}{C, D, W: \textit{sūyanti}.}}}}}\makeatother \thinspace ca. Pakkamat'āyasmā imamhā āvāsā. Alaṁ te\makeatletter\hyperlink{endnote100-appendix}\Hy@raisedlink{\hypertarget{endnote100-body}{}{\pagenote{%
		\hypertarget{endnote100-appendix}{\hyperlink{endnote100-body}{Mi \& Mm Se, BhPm 1 \& 2, C, G, V, W, Um, Ra: \textit{alan-te}.}}}}}\makeatother \thinspace idha vāsenā'ti,'' evañ'ca so bhikkhu bhikkhūhi vuccamāno tath'eva paggaṇheyya, so bhikkhu bhikkhūhi yāvatatiyaṁ samanubhāsitabbo tassa paṭinissaggāya, yāvatatiyañ'ce samanubhāsiyamāno taṁ paṭinissajeyya\makeatletter\hyperlink{endnote101-appendix}\Hy@raisedlink{\hypertarget{endnote101-body}{}{\pagenote{%
		\hypertarget{endnote101-appendix}{\hyperlink{endnote101-body}{D, W, SVibh Ce (but has \textit{-nissajjeyya} in Pāc 68), Other eds: \textit{-nissajjeyya}. C reads \textit{-nissajjeyya} here but \textit{-nissajeyya} in Sd 12–13 and Pāc 68.}}}}}\makeatother \thinspace icc'etaṁ kusalaṁ, no ce paṭinissajeyya,\makeatletter\hyperlink{endnote102-appendix}\Hy@raisedlink{\hypertarget{endnote102-body}{}{\pagenote{%
		\hypertarget{endnote102-appendix}{\hyperlink{endnote102-body}{D, W, SVibh Ce (but has \textit{-nissajjeyya} in Pāc 68), Other eds: \textit{-nissajjeyya}. C reads \textit{-nissajjeyya} here but \textit{-nissajeyya} in Sd 12–13 and Pāc 68.}}}}}\makeatother \thinspace saṅghādiseso.



\medskip

\linkdest{endnote103-body}
\linkdest{endnote104-body}
\linkdest{endnote105-body}
\linkdest{endnote106-body}
\linkdest{endnote107-body}
\linkdest{endnote108-body}
\linkdest{endnote109-body}
\linkdest{endnote110-body}
\linkdest{endnote111-body}
\begin{center}
	Uddiṭṭhā kho āyasmanto terasa saṅghādisesā dhammā, nava paṭham'āpattikā\makeatletter\hyperlink{endnote103-appendix}\Hy@raisedlink{\hypertarget{endnote103-body}{}{\pagenote{%
		\hypertarget{endnote103-appendix}{\hyperlink{endnote103-body}{V: \textit{patham-}.}}}}}\makeatother \thinspace cattāro yāvatatiyakā. Yesaṁ bhikkhu aññataraṁ vā aññataraṁ vā āpajjitvā, yāvat'ihaṁ\makeatletter\hyperlink{endnote104-appendix}\Hy@raisedlink{\hypertarget{endnote104-body}{}{\pagenote{%
		\hypertarget{endnote104-appendix}{\hyperlink{endnote104-body}{Be, UP, G, V: \textit{yāvatīhaṁ}.}}}}}\makeatother \thinspace jānaṁ paṭicchādeti, tāvat'ihaṁ\makeatletter\hyperlink{endnote105-appendix}\Hy@raisedlink{\hypertarget{endnote105-body}{}{\pagenote{%
		\hypertarget{endnote105-appendix}{\hyperlink{endnote105-body}{Be, UP, G, V: \textit{tāvatīhaṁ}.}}}}}\makeatother \thinspace tena bhikkhunā akāmā parivatthabbaṁ.\makeatletter\hyperlink{endnote106-appendix}\Hy@raisedlink{\hypertarget{endnote106-body}{}{\pagenote{%
		\hypertarget{endnote106-appendix}{\hyperlink{endnote106-body}{V: \textit{parivaṭṭhabbaṁ}.}}}}}\makeatother \thinspace Parivutthaparivāsena\makeatletter\hyperlink{endnote107-appendix}\Hy@raisedlink{\hypertarget{endnote107-body}{}{\pagenote{%
		\hypertarget{endnote107-appendix}{\hyperlink{endnote107-body}{V: \textit{parivuṭṭha-}.}}}}}\makeatother \thinspace bhikkhunā uttariṁ\makeatletter\hyperlink{endnote108-appendix}\Hy@raisedlink{\hypertarget{endnote108-body}{}{\pagenote{%
		\hypertarget{endnote108-appendix}{\hyperlink{endnote108-body}{Dm, SVibh Ce, Um: \textit{uttari}.}}}}}\makeatother \thinspace chārattaṁ bhikkhumānattāya paṭipajjitabbaṁ. Ciṇṇamānatto bhikkhu, yattha siyā vīsatigaṇo bhikkhusaṅgho,\makeatletter\hyperlink{endnote109-appendix}\Hy@raisedlink{\hypertarget{endnote109-body}{}{\pagenote{%
		\hypertarget{endnote109-appendix}{\hyperlink{endnote109-body}{BhPm 1, C, V, W: \textit{-saṁgho}.}}}}}\makeatother \thinspace tattha so bhikkhu\makeatletter\hyperlink{endnote110-appendix}\Hy@raisedlink{\hypertarget{endnote110-body}{}{\pagenote{%
		\hypertarget{endnote110-appendix}{\hyperlink{endnote110-body}{Mi v.l.: bhikkhu \textit{bhikkhūhi}.}}}}}\makeatother \thinspace abbhetabbo. Ekena'pi ce ūno\makeatletter\hyperlink{endnote111-appendix}\Hy@raisedlink{\hypertarget{endnote111-body}{}{\pagenote{%
		\hypertarget{endnote111-appendix}{\hyperlink{endnote111-body}{V, Bh Pm 2 (syāma) v.l: \textit{ono}. Um, G: \textit{ūṇo}.}}}}}\makeatother \thinspace vīsatigaṇo bhikkhusaṅgho taṁ bhikkhuṁ abbheyya, so ca bhikkhu anabbhito, te ca bhikkhū gārayhā. Ayaṁ tattha sāmīci.

	\smallskip

	Tatth'āyasmante pucchāmi: Kacci'ttha parisuddhā?\\
	Dutiyam'pi pucchāmi: Kacci'ttha parisuddhā?\\
	Tatiyam'pi pucchāmi: Kacci'ttha parisuddhā?

	\smallskip

	Parisuddh'etth'āyasmanto, tasmā tuṇhī, evam'etaṁ dhārayāmi.
\end{center}

\linkdest{endnote11-body}
\linkdest{endnote113-body}
\begin{outro}
	Terasa saṅghādisesā dhammā niṭṭhitā\makeatletter\hyperlink{endnote11-appendix}\Hy@raisedlink{\hypertarget{endnote11-body}{}{\pagenote{%
				\hypertarget{endnote11-appendix}{\hyperlink{endnote11-body}{Not in any edition or manuscript, but if a conclusion is to be recited then this one as given in the Parivāra would be the suitable one. When reciting in brief use: \textit{Saṅghādises'uddeso niṭṭhito}.}}}}}\makeatother
        \end{outro}

\clearpage

