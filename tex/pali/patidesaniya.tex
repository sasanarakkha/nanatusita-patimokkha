
\section{Pāṭidesanīyā}
\label{pd}

\linkdest{endnote-body}
\begin{intro}
	Ime kho pan'āyasmanto cattāro pāṭidesanīyā\makeatletter\hyperlink{endnote-appendix}\Hy@raisedlink{\hypertarget{endnote-body}{}{\pagenote{%
		\hypertarget{endnote-appendix}{\hyperlink{endnote-body}{}}}}}\makeatother C, D, V, W, G, Vibh Ee: pāṭidesaniyā. dhammā uddesaṁ āgacchanti.
\end{intro}

\setsubsecheadstyle{\subsubsectionFmt}
\pdfbookmark[3]{Pāṭidesanīyā 1}{pd1}
\subsection*{\hyperref[ack1]{Pāṭidesanīyā 1: Paṭhamapāṭidesanīyasikkhāpadaṁ}}
\label{pd1}

\linkdest{endnote-body}
\linkdest{endnote-body}
\linkdest{endnote-body}
Yo pana bhikkhu aññātikāya bhikkhuniyā antaragharaṁ paviṭṭhāya hatthato khādanīyaṁ vā bhojanīyaṁ\makeatletter\hyperlink{endnote-appendix}\Hy@raisedlink{\hypertarget{endnote-body}{}{\pagenote{%
		\hypertarget{endnote-appendix}{\hyperlink{endnote-body}{}}}}}\makeatother C, D, G, V, W, Vibh Ee, Um: khādaniyaṁ & bhojaniyaṁ. vā sahatthā paṭiggahetvā khādeyya vā bhuñjeyya vā, paṭidesetabbaṁ tena bhikkhunā, ``Gārayhaṁ āvuso dhammaṁ āpajjiṁ, asappāyaṁ, pāṭidesanīyaṁ,\makeatletter\hyperlink{endnote-appendix}\Hy@raisedlink{\hypertarget{endnote-body}{}{\pagenote{%
		\hypertarget{endnote-appendix}{\hyperlink{endnote-body}{}}}}}\makeatother C, D, G, V, W, Vibh Ee: pāṭidesaniyaṁ. taṁ paṭidesemī'ti.''\makeatletter\hyperlink{endnote-appendix}\Hy@raisedlink{\hypertarget{endnote-body}{}{\pagenote{%
		\hypertarget{endnote-appendix}{\hyperlink{endnote-body}{}}}}}\makeatother Bh Pm 1: iti. (Not so in Pd 2, but again in Pd 3–4.)



\pdfbookmark[3]{Pāṭidesanīyā 2}{pd2}
\subsection*{\hyperref[ack2]{Pāṭidesanīyā 2: Dutiyapāṭidesanīyasikkhāpadaṁ}}
\label{pd2}

\linkdest{endnote-body}
\linkdest{endnote-body}
\linkdest{endnote-body}
\linkdest{endnote-body}
\linkdest{endnote-body}
\linkdest{endnote-body}
\linkdest{endnote-body}
\linkdest{endnote-body}
\linkdest{endnote-body}
\linkdest{endnote-body}
Bhikkhū pan'eva kulesu nimantitā bhuñjanti, tatra ce\makeatletter\hyperlink{endnote-appendix}\Hy@raisedlink{\hypertarget{endnote-body}{}{\pagenote{%
		\hypertarget{endnote-appendix}{\hyperlink{endnote-body}{}}}}}\makeatother Bh Pm 1 & 2, C, D, G, V, W, Dm, Um, UP, Vibh Ce, Vibh Ee, Mi Se, Ra, Pg: tatra ce sā bhikkhunī. Mi Se v.l., Mm Se: tatra ce
bhikkhunī. bhikkhunī\makeatletter\hyperlink{endnote-appendix}\Hy@raisedlink{\hypertarget{endnote-body}{}{\pagenote{%
		\hypertarget{endnote-appendix}{\hyperlink{endnote-body}{}}}}}\makeatother W: bhikkhuni. vosāsamānarūpā ṭhitā hoti, ``Idha sūpaṁ detha, idha odanaṁ dethā'ti,'' tehi bhikkhūhi sā bhikkhunī\makeatletter\hyperlink{endnote-appendix}\Hy@raisedlink{\hypertarget{endnote-body}{}{\pagenote{%
		\hypertarget{endnote-appendix}{\hyperlink{endnote-body}{}}}}}\makeatother W: bhikkhuni. apasādetabbā, ``Apasakka tāva bhagini,\makeatletter\hyperlink{endnote-appendix}\Hy@raisedlink{\hypertarget{endnote-body}{}{\pagenote{%
		\hypertarget{endnote-appendix}{\hyperlink{endnote-body}{}}}}}\makeatother C, Um: bhaginī. yāva bhikkhū bhuñjantī'ti,'' ekassa'pi ce\makeatletter\hyperlink{endnote-appendix}\Hy@raisedlink{\hypertarget{endnote-body}{}{\pagenote{%
		\hypertarget{endnote-appendix}{\hyperlink{endnote-body}{}}}}}\makeatother D, W, Um, UP, Vibh Ee, Sannē: ce pi. bhikkhuno nappaṭibhāseyya\makeatletter\hyperlink{endnote-appendix}\Hy@raisedlink{\hypertarget{endnote-body}{}{\pagenote{%
		\hypertarget{endnote-appendix}{\hyperlink{endnote-body}{}}}}}\makeatother Dm, UP, Vibh Ee: na paṭibhāseyya. (Pg: nappaṭibhāseyya.) taṁ bhikkhuniṁ apasādetuṁ,\makeatletter\hyperlink{endnote-appendix}\Hy@raisedlink{\hypertarget{endnote-body}{}{\pagenote{%
		\hypertarget{endnote-appendix}{\hyperlink{endnote-body}{}}}}}\makeatother G: appasādetuṁ ? immediately (inked) corrected to appa-. ``Apasakka tāva bhagini,\makeatletter\hyperlink{endnote-appendix}\Hy@raisedlink{\hypertarget{endnote-body}{}{\pagenote{%
		\hypertarget{endnote-appendix}{\hyperlink{endnote-body}{}}}}}\makeatother C, Um: bhaginī. yāva bhikkhū bhuñjantī'ti,'' paṭidesetabbaṁ tehi bhikkhūhi, ``Gārayhaṁ āvuso dhammaṁ āpajjimhā,\makeatletter\hyperlink{endnote-appendix}\Hy@raisedlink{\hypertarget{endnote-body}{}{\pagenote{%
		\hypertarget{endnote-appendix}{\hyperlink{endnote-body}{}}}}}\makeatother C, D, W, Um, Sannē: āpajjimha (= also a legitimate 1 pl. a-aorist.) In G the ā character stroke in -imhā has been scribbled
through making it -imha. asappāyaṁ, pāṭidesanīyaṁ,\makeatletter\hyperlink{endnote-appendix}\Hy@raisedlink{\hypertarget{endnote-body}{}{\pagenote{%
		\hypertarget{endnote-appendix}{\hyperlink{endnote-body}{}}}}}\makeatother C, D, G, V, W, Vibh Ee: pāṭidesaniyaṁ. taṁ paṭidesemā'ti.''



\pdfbookmark[3]{Pāṭidesanīyā 3}{pd3}
\subsection*{\hyperref[ack3]{Pāṭidesanīyā 3: Tatiyapāṭidesanīyasikkhāpadaṁ}}
\label{pd3}

\linkdest{endnote-body}
\linkdest{endnote-body}
\linkdest{endnote-body}
\linkdest{endnote-body}
\linkdest{endnote-body}
\linkdest{endnote-body}
\linkdest{endnote-body}
Yāni kho pana tāni sekhasammatāni\makeatletter\hyperlink{endnote-appendix}\Hy@raisedlink{\hypertarget{endnote-body}{}{\pagenote{%
		\hypertarget{endnote-appendix}{\hyperlink{endnote-body}{}}}}}\makeatother Dm, Mi & Mm Se, G, V: sekkha-. (Pg: sekha-) kulāni, yo pana bhikkhu tathārūpesu sekhasammatesu\makeatletter\hyperlink{endnote-appendix}\Hy@raisedlink{\hypertarget{endnote-body}{}{\pagenote{%
		\hypertarget{endnote-appendix}{\hyperlink{endnote-body}{}}}}}\makeatother Dm, Mi & Mm Se, G, V: sekkha-. kulesu pubbe animantito\makeatletter\hyperlink{endnote-appendix}\Hy@raisedlink{\hypertarget{endnote-body}{}{\pagenote{%
		\hypertarget{endnote-appendix}{\hyperlink{endnote-body}{}}}}}\makeatother C, W: apanimantito. agilāno\makeatletter\hyperlink{endnote-appendix}\Hy@raisedlink{\hypertarget{endnote-body}{}{\pagenote{%
		\hypertarget{endnote-appendix}{\hyperlink{endnote-body}{}}}}}\makeatother V: agīlāno. khādanīyaṁ vā bhojanīyaṁ\makeatletter\hyperlink{endnote-appendix}\Hy@raisedlink{\hypertarget{endnote-body}{}{\pagenote{%
		\hypertarget{endnote-appendix}{\hyperlink{endnote-body}{}}}}}\makeatother C, D, G, V, W, Vibh Ee, Um: khādaniyaṁ & bhojaniyaṁ. vā sahatthā paṭiggahetvā khādeyya vā bhuñjeyya vā, paṭidesetabbaṁ tena bhikkhunā, ``Gārayhaṁ āvuso dhammaṁ āpajjiṁ, asappāyaṁ, pāṭidesanīyaṁ,\makeatletter\hyperlink{endnote-appendix}\Hy@raisedlink{\hypertarget{endnote-body}{}{\pagenote{%
		\hypertarget{endnote-appendix}{\hyperlink{endnote-body}{}}}}}\makeatother C, D, G, V, W, Vibh Ee: pāṭidesaniyaṁ. taṁ paṭidesemī'ti.''\makeatletter\hyperlink{endnote-appendix}\Hy@raisedlink{\hypertarget{endnote-body}{}{\pagenote{%
		\hypertarget{endnote-appendix}{\hyperlink{endnote-body}{}}}}}\makeatother Bh Pm 1: iti.



\pdfbookmark[3]{Pāṭidesanīyā 4}{pd4}
\subsection*{\hyperref[ack4]{Pāṭidesanīyā 4: Catutthapāṭidesanīyasikkhāpadaṁ}}
\label{pd4}

\linkdest{endnote-body}
\linkdest{endnote-body}
\linkdest{endnote-body}
\linkdest{endnote-body}
\linkdest{endnote-body}
\linkdest{endnote-body}
Yāni kho pana tāni āraññakāni sen'āsanāni sāsaṅkasammatāni\makeatletter\hyperlink{endnote-appendix}\Hy@raisedlink{\hypertarget{endnote-body}{}{\pagenote{%
		\hypertarget{endnote-appendix}{\hyperlink{endnote-body}{}}}}}\makeatother  C, W: -saṁka-. sappaṭibhayāni, yo pana bhikkhu tathārūpesu sen'āsanesu\makeatletter\hyperlink{endnote-appendix}\Hy@raisedlink{\hypertarget{endnote-body}{}{\pagenote{%
		\hypertarget{endnote-appendix}{\hyperlink{endnote-body}{}}}}}\makeatother Bh Pm 1 & 2, C, D, G, V, W, Mi & Mm Se, Vibh Ce, Ra, Sannē: ”… senāsanesu viharanto ....” Dm, Vibh Ee, Um, and UP omit
viharanto. (Pg 71 appears not to have it: “... tathārūpesu senāsanesu na pubbe appaṭisaṁviditaṁ anārocitaṁ khādanīyaṁ vā ... pe ...
agilāno yo pana bhikkhu khādeyya vā ...”) pubbe appaṭisaṁviditaṁ khādanīyaṁ vā bhojanīyaṁ\makeatletter\hyperlink{endnote-appendix}\Hy@raisedlink{\hypertarget{endnote-body}{}{\pagenote{%
		\hypertarget{endnote-appendix}{\hyperlink{endnote-body}{}}}}}\makeatother C, D, G, V, W, Vibh Ee, Um: khādaniyaṁ & bhojaniyaṁ. vā ajjhārāme sahatthā paṭiggahetvā agilāno\makeatletter\hyperlink{endnote-appendix}\Hy@raisedlink{\hypertarget{endnote-body}{}{\pagenote{%
		\hypertarget{endnote-appendix}{\hyperlink{endnote-body}{}}}}}\makeatother V: agīlāno. khādeyya vā bhuñjeyya vā, paṭidesetabbaṁ tena bhikkhunā: ``Gārayhaṁ āvuso dhammaṁ āpajjiṁ, asappāyaṁ, pāṭidesanīyaṁ,\makeatletter\hyperlink{endnote-appendix}\Hy@raisedlink{\hypertarget{endnote-body}{}{\pagenote{%
		\hypertarget{endnote-appendix}{\hyperlink{endnote-body}{}}}}}\makeatother C, D, W, Vibh Ee: pāṭidesaniyaṁ. taṁ paṭidesemī'ti.''\makeatletter\hyperlink{endnote-appendix}\Hy@raisedlink{\hypertarget{endnote-body}{}{\pagenote{%
		\hypertarget{endnote-appendix}{\hyperlink{endnote-body}{}}}}}\makeatother Bh Pm 1: iti.



\medskip

\linkdest{endnote-body}
\linkdest{endnote-body}
\begin{center}
	Uddiṭṭhā kho āyasmanto cattāro pāṭidesanīyā\makeatletter\hyperlink{endnote-appendix}\Hy@raisedlink{\hypertarget{endnote-body}{}{\pagenote{%
		\hypertarget{endnote-appendix}{\hyperlink{endnote-body}{}}}}}\makeatother C, D, V, W, G, Vibh Ee: pāṭidesaniyā. dhammā.

	\smallskip

	Tatth'āyasmante pucchāmi: Kacci'ttha parisuddhā?\\
	Dutiyam'pi pucchāmi: Kacci'ttha parisuddhā?\\
	Tatiyam'pi pucchāmi: Kacci'ttha parisuddhā?

	\smallskip

	Parisuddh'etth'āyasmanto, tasmā tuṇhī, evam'etaṁ dhārayāmi.\makeatletter\hyperlink{endnote-appendix}\Hy@raisedlink{\hypertarget{endnote-body}{}{\pagenote{%
		\hypertarget{endnote-appendix}{\hyperlink{endnote-body}{}}}}}\makeatother Dm, UP, Ra, Um: dhārayāmī ti.
\end{center}

\linkdest{endnote-body}
\linkdest{endnote-body}
\begin{outro}
	Cattāro pāṭidesanīyā\makeatletter\hyperlink{endnote-appendix}\Hy@raisedlink{\hypertarget{endnote-body}{}{\pagenote{%
		\hypertarget{endnote-appendix}{\hyperlink{endnote-body}{}}}}}\makeatother C, D, V, W, G, Vibh Ee: pāṭidesaniyā. dhammā niṭṭhitā\makeatletter\hyperlink{endnote-appendix}\Hy@raisedlink{\hypertarget{endnote-body}{}{\pagenote{%
		\hypertarget{endnote-appendix}{\hyperlink{endnote-body}{}}}}}\makeatother C, G, W, Dm, UP, Ñd Ce, Mm Se, Bh Pm 1 & 2, Um. Mi Se: Cattāro pāṭidesanīyā niṭṭhitā. D: Pāṭidesaniyaṁ niṭṭhitaṁ.
\end{outro}

\clearpage

