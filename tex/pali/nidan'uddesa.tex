
\section{Nidān'uddesa}
\label{nidan'uddesa}

\ifninebythirteenversion\vspace{0.2em}\fi
\ifafiveversion\vspace{0.2em}\fi

\linkdest{endnote14-body}
\linkdest{endnote15-body}
Suṇātu me bhante/āvuso saṅgho, ajj'uposatho paṇṇaraso/cātuddaso/sāmaggo,\makeatletter\hyperlink{endnote15-appendix}\Hy@raisedlink{\hypertarget{endnote15-body}{}{\pagenote{%
		\hypertarget{endnote15-appendix}{\hyperlink{endnote15-body}{In brackets in Mi Se. Dm, Mv Ee, W: \textit{pannaraso}. Not in SVibh Ce: ...\textit{me saṅgho, yadi saṅghassa}}}}}}\makeatother \thinspace yadi saṅghassa pattakallaṁ, saṅgho uposathaṁ kareyya pātimokkhaṁ\makeatletter\hyperlink{endnote14-appendix}\Hy@raisedlink{\hypertarget{endnote14-body}{}{\pagenote{%

		\hypertarget{endnote14-appendix}{\hyperlink{endnote14-body}{G, Mi \& Mm Se, V: \textit{pāṭi-}.}}}}}\makeatother \thinspace uddiseyya.

Kiṁ saṅghassa pubbakiccaṁ?

\linkdest{endnote16-body}
Pārisuddhiṁ āyasmanto ārocetha. Pātimokkhaṁ uddisissāmi. Taṁ sabb'eva santā sādhukaṁ suṇoma manasikaroma. Yassa siyā āpatti, so āvikareyya\makeatletter\hyperlink{endnote16-appendix}\Hy@raisedlink{\hypertarget{endnote16-body}{}{\pagenote{%
		\hypertarget{endnote16-appendix}{\hyperlink{endnote16-body}{V, Ce Mv, Ra: \textit{āvīkareyya}.}}}}}\makeatother \thinspace. Asantiyā āpattiyā, tuṇhī bhavitabbaṁ. Tuṇhībhāvena kho pan'āyasmante parisuddhā'ti vedissāmi.

\linkdest{endnote17-body}
\linkdest{endnote18-body}
\linkdest{endnote19-body}
\linkdest{endnote20-body}
\linkdest{endnote21-body}
\linkdest{endnote22-body}
\linkdest{endnote23-body}
\linkdest{endnote24-body}
Yathā kho pana paccekapuṭṭhassa veyyākaraṇaṁ hoti, evam'evaṁ\makeatletter\hyperlink{endnote17-appendix}\Hy@raisedlink{\hypertarget{endnote17-body}{}{\pagenote{%
		\hypertarget{endnote17-appendix}{\hyperlink{endnote17-body}{D, G, V, W, Dm, Ce Mv, Ra, Mi Se, BhPm 1 \& 2, Pg, Ee Kkh: \textit{evam-evaṁ}, Mv Ee: \textit{evaṁ eva}. Mm Se: \textit{evaṁ evaṁ}. UP, Um, Be Mv v.l \& Mi Se v.l.: \textit{evam-eva}.}}}}}\makeatother \thinspace evarūpāya parisāya yāvatatiyaṁ anussāvitaṁ\makeatletter\hyperlink{endnote18-appendix}\Hy@raisedlink{\hypertarget{endnote18-body}{}{\pagenote{%
		\hypertarget{endnote18-appendix}{\hyperlink{endnote18-body}{C, D, G, V, W, Dm, Ce Mv, Ra, BhPm 1 \& 2, Um, UP, Pg: \textit{anusāvitaṁ}.}}}}}\makeatother \thinspace hoti. Yo pana bhikkhu yāvatatiyaṁ anussāviyamāne\makeatletter\hyperlink{endnote19-appendix}\Hy@raisedlink{\hypertarget{endnote19-body}{}{\pagenote{%
		\hypertarget{endnote19-appendix}{\hyperlink{endnote19-body}{C, D, G, V, W, Dm, Ce Mv, Ra, BhPm 1 \& 2, Um, UP, Pg: \textit{anusāviyamāne}.}}}}}\makeatother \thinspace saramāno santiṁ āpattiṁ n'āvikareyya,\makeatletter\hyperlink{endnote20-appendix}\Hy@raisedlink{\hypertarget{endnote20-body}{}{\pagenote{%
		\hypertarget{endnote20-appendix}{\hyperlink{endnote20-body}{V, Ce Mv, G, Ra: \textit{nāvīkareyya}.}}}}}\makeatother \thinspace  sampajānamusāvād'assa hoti. Sampajānamusāvādo kho pan'āyasmanto antarāyiko dhammo vutto bhagavatā. Tasmā saramānena bhikkhunā āpannena visuddh'āpekkhena santī āpatti\makeatletter\hyperlink{endnote21-appendix}\Hy@raisedlink{\hypertarget{endnote21-body}{}{\pagenote{%
		\hypertarget{endnote21-appendix}{\hyperlink{endnote21-body}{C, G, V, W, BhPm 2, UP, Um: \textit{santi āpatti}. Ra: \textit{santī āpattī}.}}}}}\makeatother \thinspace  āvikātabbā,\makeatletter\hyperlink{endnote22-appendix}\Hy@raisedlink{\hypertarget{endnote22-body}{}{\pagenote{%
		\hypertarget{endnote22-appendix}{\hyperlink{endnote22-body}{V, Ce Mv, Ra: \textit{āvīkātabbā}.}}}}}\makeatother \thinspace āvikatā\makeatletter\hyperlink{endnote23-appendix}\Hy@raisedlink{\hypertarget{endnote23-body}{}{\pagenote{%
		\hypertarget{endnote23-appendix}{\hyperlink{endnote23-body}{V, Ce Mv, Ra: \textit{āvīkatā}.}}}}}\makeatother \thinspace hi'ssa phāsu hoti.\makeatletter\hyperlink{endnote24-appendix}\Hy@raisedlink{\hypertarget{endnote24-body}{}{\pagenote{%
		\hypertarget{endnote24-appendix}{\hyperlink{endnote24-body}{C, D, G, V, W, Mi \& Mm Se, BhPm 1 \& 2. Other eds: \textit{hotī ti}.}}}}}\makeatother \thinspace

\linkdest{endnote8-body}
\begin{center}
  Uddiṭṭhaṁ kho āyasmanto nidānaṁ.\makeatletter\hyperlink{endnote8-appendix}\Hy@raisedlink{\hypertarget{endnote8-body}{}{\pagenote{%
		\hypertarget{endnote8-appendix}{\hyperlink{endnote8-body}{This can be skipped since it doesn't occur in the Canon. The Nidāna can instead be concluded with \textit{Nidānaṁ niṭṭhitaṁ}.''}}}}}\makeatother \thinspace

  \ifninebythirteenversion\clearpage\fi

  Tatth'āyasmante pucchāmi: Kacci'ttha parisuddhā?\\
  Dutiyam'pi pucchāmi: Kacci'ttha parisuddhā?\\
  Tatiyam'pi pucchāmi: Kacci'ttha parisuddhā?

  \smallskip

\linkdest{endnote25-body}
  Parisuddh'etth'āyasmanto, tasmā tuṇhī, evam'etaṁ dhārayāmi.\makeatletter\hyperlink{endnote25-appendix}\Hy@raisedlink{\hypertarget{endnote25-body}{}{\pagenote{%
		\hypertarget{endnote25-appendix}{\hyperlink{endnote25-body}{C, D, G, V, W, Mi \& Mm Se. Dm, UP, Ra, Um: \textit{dhārayāmī ti}. (So in the conclusions of the offence sections of SVibh Ce \& SVibh Ee, but this can not be regarded as a v.l. It is the normal way the SVibh presents its material as there is no Nidāna in the SVibh and therefore no conclusion. In the Nidāna conclusion C reads \textit{dhārayāmi}, but in the other sections \textit{dhārayāmī ti}, however, in the other sections the latter reading is clearly a later correction as the ti has been written over the \textit{kuṇḍaliya} [serpent-like] paragraph markers [¢] and the i stroke has been changed to ī.) BhPm 1 \& 2: \textit{dhārayāmi iti}. The whole Nidāna :w
             conclusion (from \textit{uddiṭṭhaṁ} to \textit{dhārayāmi}) is not found in Mm Se.}}}}}\makeatother \thinspace
\end{center}

\linkdest{endnote9-body}
\begin{outro}
  Nidānaṁ niṭṭhitaṁ\makeatletter\hyperlink{endnote9-appendix}\Hy@raisedlink{\hypertarget{endnote9-body}{}{\pagenote{%
		\hypertarget{endnote9-appendix}{\hyperlink{endnote9-body}{Not in any edition or manuscript, but if a conclusion is to be recited then this one as given in the Parivāra would be the suitable one.\\
			When reciting in brief use: Nidān'uddeso niṭṭhito.\\
  Mm Se, D, V, W, P. UP, Um, Ñd \& Mi Se: \textit{Nidānuddeso paṭhamo}. BhPm 1
\& 2, C, G, Ra: \textit{Nidānuddeso}. Dm: \textit{Nidānaṃ niṭṭhitaṃ}. }}}}}\makeatother \thinspace
\end{outro}

\clearpage

