
\section{Pātimokkhaniṭṭhāna}
\label{patimokkhanitthana}

Uddiṭṭhaṁ kho āyasmanto nidānaṁ.\\
Uddiṭṭhā cattāro pārājikā dhammā.\\
Uddiṭṭhā terasa saṅghādisesā dhammā.\\
Uddiṭṭhā dve aniyatā dhammā.\\
Uddiṭṭhā tiṁsa nissaggiyā pācittiyā dhammā.\\
Uddiṭṭhā dvenavuti pācittiyā dhammā.\\
\linkdest{endnote-body}
Uddiṭṭhā cattāro pāṭidesanīyā\makeatletter\hyperlink{endnote-appendix}\Hy@raisedlink{\hypertarget{endnote-body}{}{\pagenote{%
		\hypertarget{endnote-appendix}{\hyperlink{endnote-body}{}}}}}\makeatother C, D, G, V, W, Vibh Ee: pāṭidesaniyā. dhammā.\\
\linkdest{endnote-body}
Uddiṭṭhā sekhiyā\makeatletter\hyperlink{endnote-appendix}\Hy@raisedlink{\hypertarget{endnote-body}{}{\pagenote{%
		\hypertarget{endnote-appendix}{\hyperlink{endnote-body}{}}}}}\makeatother Mi Se, V: pañcasattati sekhiyā dhammā.\\
\linkdest{endnote-body}
Uddiṭṭhā satta adhikaraṇasamathā\makeatletter\hyperlink{endnote-appendix}\Hy@raisedlink{\hypertarget{endnote-body}{}{\pagenote{%
		\hypertarget{endnote-appendix}{\hyperlink{endnote-body}{}}}}}\makeatother Mi & Mm Se, C, G, V, W: sattādhikaraṇasamathā. dhammā.

\linkdest{endnote-body}
\linkdest{endnote-body}
\linkdest{endnote-body}
\linkdest{endnote-body}
Ettakaṁ tassa\makeatletter\hyperlink{endnote-appendix}\Hy@raisedlink{\hypertarget{endnote-body}{}{\pagenote{%
		\hypertarget{endnote-appendix}{\hyperlink{endnote-body}{}}}}}\makeatother Mi & Mm Se, G, V: ettakan-tassa. bhagavato sutt'āgataṁ suttapariyāpannaṁ anvaḍḍhamāsaṁ\makeatletter\hyperlink{endnote-appendix}\Hy@raisedlink{\hypertarget{endnote-body}{}{\pagenote{%
		\hypertarget{endnote-appendix}{\hyperlink{endnote-body}{}}}}}\makeatother Bh Pm 1 & 2, C, D, Dm, Um, UP, Vibh Ce, Vibh Ee, Ra, Pg: anvaddha-. uddesaṁ āgacchati.\makeatletter\hyperlink{endnote-appendix}\Hy@raisedlink{\hypertarget{endnote-body}{}{\pagenote{%
		\hypertarget{endnote-appendix}{\hyperlink{endnote-body}{}}}}}\makeatother BhPm 1, P: āgacchanti. Tattha sabbeh'eva samaggehi sammodamānehi avivadamānehi sikkhitabban''ti.\makeatletter\hyperlink{endnote-appendix}\Hy@raisedlink{\hypertarget{endnote-body}{}{\pagenote{%
		\hypertarget{endnote-appendix}{\hyperlink{endnote-body}{}}}}}\makeatother Bh Pm 1 & 2: sikkhitabbaṁ iti.

\linkdest{endnote-body}
Vitthār'uddeso pañcamo.\makeatletter\hyperlink{endnote-appendix}\Hy@raisedlink{\hypertarget{endnote-body}{}{\pagenote{%
		\hypertarget{endnote-appendix}{\hyperlink{endnote-body}{}}}}}\makeatother Dm, Ñd Ce, and Mi Se. Also in  Sannē; see Suguṇasāra 111. Not in other eds. Also found in the Burmese script
Bhikkhupātimokkha MS, BNF Pali 8, at the Bibliothèque Nationale de France; see EFEO DATA 101. The Burmese script
Bhikkhunīpātimokkha MS BNF 844:3 has vitthāruddeso catuttho

\linkdest{endnote-body}
\linkdest{endnote-body}
\begin{outro}
	Bhikkhupātimokkhaṁ\makeatletter\hyperlink{endnote-appendix}\Hy@raisedlink{\hypertarget{endnote-body}{}{\pagenote{%
		\hypertarget{endnote-appendix}{\hyperlink{endnote-body}{}}}}}\makeatother Mm Se, G, V: pāṭi-. niṭṭhitaṁ\makeatletter\hyperlink{endnote-appendix}\Hy@raisedlink{\hypertarget{endnote-body}{}{\pagenote{%
		\hypertarget{endnote-appendix}{\hyperlink{endnote-body}{}}}}}\makeatother Mi Se: bhikkhupātimokkhapāli niṭṭhitā.
\end{outro}

\clearpage

