
\section{Nidān'uddeso}
\label{nidan'uddeso}

\ifninebythirteenversion\vspace{0.2em}\fi
\ifafiveversion\vspace{0.2em}\fi

\linkdest{endnote14-body}
\linkdest{endnote15-body}
Suṇātu me bhante/āvuso saṅgho, ajj'uposatho paṇṇaraso/cātuddaso/sāmaggo\makeatletter\hyperlink{endnote15-appendix}\Hy@raisedlink{\hypertarget{endnote15-body}{}{\pagenote{%
		\hypertarget{endnote15-appendix}{\hyperlink{endnote15-body}{}}}}}\makeatother In brackets in Mi Se. Dm, Mv Ee, W: pannaraso. Not in Vibh Ce: .”… me saṅgho, yadi saṅghassa ...”, yadi saṅghassa pattakallaṁ, saṅgho uposathaṁ kareyya pātimokkhaṁ\makeatletter\hyperlink{endnote14-appendix}\Hy@raisedlink{\hypertarget{endnote14-body}{}{\pagenote{%
		\hypertarget{endnote14-appendix}{\hyperlink{endnote14-body}{**EN**G, Mi, Mm Se, V: pāṭi-.}}}}}\makeatother uddiseyya.

\makeatletter\hyperlink{endnote-appendix}\Hy@raisedlink{\hypertarget{endnote-body}{}{\pagenote{%
		\hypertarget{endnote-appendix}{\hyperlink{endnote-body}{}}}}}\makeatother

Kiṁ saṅghassa pubbakiccaṁ?

\linkdest{endnote-body}
Pārisuddhiṁ āyasmanto ārocetha. Pātimokkhaṁ uddisissāmi. Taṁ sabb'eva santā sādhukaṁ suṇoma manasikaroma. Yassa siyā āpatti, so āvikareyya\makeatletter\hyperlink{endnote-appendix}\Hy@raisedlink{\hypertarget{endnote-body}{}{\pagenote{%
		\hypertarget{endnote-appendix}{\hyperlink{endnote-body}{}}}}}\makeatother V, Ce Mv, Ra: āvīkareyya.. Asantiyā āpattiyā, tuṇhī bhavitabbaṁ. Tuṇhībhāvena kho pan'āyasmante parisuddhā'ti vedissāmi.

\linkdest{endnote-body}
\linkdest{endnote-body}
\linkdest{endnote-body}
\linkdest{endnote-body}
\linkdest{endnote-body}
\linkdest{endnote-body}
\linkdest{endnote-body}
Yathā kho pana paccekapuṭṭhassa veyyākaraṇaṁ hoti, evam'evaṁ\makeatletter\hyperlink{endnote-appendix}\Hy@raisedlink{\hypertarget{endnote-body}{}{\pagenote{%
		\hypertarget{endnote-appendix}{\hyperlink{endnote-body}{}}}}}\makeatother  D, G, V, W, Dm, Ce Mv, Ra, Mi Se, BhPm 1 & 2, Pg, Ee Kkh: evam-evaṁ, Mv Ee: evaṁ eva. Mm Se: evaṁ evaṁ. UP, Um, Be
Mv v.l & Mi Se v.l.: evam-eva. evarūpāya parisāya yāvatatiyaṁ anussāvitaṁ\makeatletter\hyperlink{endnote-appendix}\Hy@raisedlink{\hypertarget{endnote-body}{}{\pagenote{%
		\hypertarget{endnote-appendix}{\hyperlink{endnote-body}{}}}}}\makeatother C, D, G, V, W, Dm, Ce Mv, Ra, BhPm 1 & 2, Um, UP, Pg: anusāv-. hoti. Yo pana bhikkhu yāvatatiyaṁ anussāviyamāne\makeatletter\hyperlink{endnote-appendix}\Hy@raisedlink{\hypertarget{endnote-body}{}{\pagenote{%
		\hypertarget{endnote-appendix}{\hyperlink{endnote-body}{}}}}}\makeatother C, D, G, V, W, Dm, Ce Mv, Ra, BhPm 1 & 2, Um, UP, Pg: anusāv-. saramāno santiṁ āpattiṁ n'āvikareyya\makeatletter\hyperlink{endnote-appendix}\Hy@raisedlink{\hypertarget{endnote-body}{}{\pagenote{%
		\hypertarget{endnote-appendix}{\hyperlink{endnote-body}{}}}}}\makeatother V, Ce Mv, G, Ra: nāvīkareyya., sampajānamusāvād'assa hoti. Sampajānamusāvādo kho pan'āyasmanto antarāyiko dhammo vutto bhagavatā. Tasmā saramānena bhikkhunā āpannena visuddh'āpekkhena santī āpatti\makeatletter\hyperlink{endnote-appendix}\Hy@raisedlink{\hypertarget{endnote-body}{}{\pagenote{%
		\hypertarget{endnote-appendix}{\hyperlink{endnote-body}{}}}}}\makeatother C, G, V, W, BhPm 2, UP, Um: santi āpatti. Ra: santī āpattī. āvikātabbā\makeatletter\hyperlink{endnote-appendix}\Hy@raisedlink{\hypertarget{endnote-body}{}{\pagenote{%
		\hypertarget{endnote-appendix}{\hyperlink{endnote-body}{}}}}}\makeatother V, Ce Mv, Ra: āvīkātabbā., āvikatā\makeatletter\hyperlink{endnote-appendix}\Hy@raisedlink{\hypertarget{endnote-body}{}{\pagenote{%
		\hypertarget{endnote-appendix}{\hyperlink{endnote-body}{}}}}}\makeatother V, Ce Mv, Ra: āvīkatā. hi'ssa phāsu hoti.\makeatletter\hyperlink{endnote-appendix}\Hy@raisedlink{\hypertarget{endnote-body}{}{\pagenote{%
		\hypertarget{endnote-appendix}{\hyperlink{endnote-body}{}}}}}\makeatother C, D, G, V, W, Mi & Mm Se, BhPm 1 & 2. Other eds.: hotī ti.

\linkdest{endnote8-body}
\begin{center}
  Uddiṭṭhaṁ kho āyasmanto nidānaṁ.\makeatletter\hyperlink{endnote8-appendix}\Hy@raisedlink{\hypertarget{endnote8-body}{}{\pagenote{%
		\hypertarget{endnote8-appendix}{\hyperlink{endnote8-body}{This can be skipped since it doesn't occur in the Canon. The Nidāna can instead be concluded with ``Nidānaṁ niṭṭhitaṁ.''}}}}}\makeatother

  \ifninebythirteenversion\clearpage\fi

  Tatth'āyasmante pucchāmi: Kacci'ttha parisuddhā?\\
  Dutiyam'pi pucchāmi: Kacci'ttha parisuddhā?\\
  Tatiyam'pi pucchāmi: Kacci'ttha parisuddhā?

  \smallskip

\linkdest{endnote-body}
  Parisuddh'etth'āyasmanto, tasmā tuṇhī, evam'etaṁ dhārayāmi.\makeatletter\hyperlink{endnote-appendix}\Hy@raisedlink{\hypertarget{endnote-body}{}{\pagenote{%
		\hypertarget{endnote-appendix}{\hyperlink{endnote-body}{}}}}}\makeatother C, D, G, V, W, Mi & Mm Se. Dm, UP, Ra, Um: dhārayāmī ti. (So in the conclusions of the offence sections of Vibh Ce &
Vibh Ee, but this can not be regarded as a v.l. It is the normal way the Vibh presents its material as there is no Nidāna in
the Vibh and therefore no conclusion. In the Nidāna conclusion C reads dhārayāmi, but in the other sections dhārayāmī ti,
however, in the other sections the latter reading is clearly a later correction as the ti has been written over the kuṇḍaliya
[serpent-like] paragraph markers [¢] and the i stroke has been changed to ī.)
BhPm 1 & 2: dhārayāmi iti. The whole Nidāna conclusion (from uddiṭṭhaṁ to dhārayāmi) is not found in Mm Se.
\end{center}

\linkdest{endnote9-body}
\begin{outro}
  Nidānaṁ niṭṭhitaṁ\makeatletter\hyperlink{endnote9-appendix}\Hy@raisedlink{\hypertarget{endnote9-body}{}{\pagenote{%
		\hypertarget{endnote9-appendix}{\hyperlink{endnote9-body}{Not in any edition or manuscript, but if a conclusion is to be recited then this one as given in the Parivāra would be the suitable one.\\
			When reciting in brief use: Nidān'uddeso niṭṭhito.}}}}}\makeatother
\end{outro}

\clearpage

