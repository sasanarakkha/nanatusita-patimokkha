
\section{Aniyat'uddeso}
\label{aniy}

\begin{intro}
	Ime kho pan'āyasmanto dve aniyatā dhammā uddesaṁ āgacchanti.
\end{intro}

\pdfbookmark[2]{Aniyata 1}{aniy1}
\subsection*{\hyperref[unc1]{Aniyata 1: Paṭhama-aniyatasikkhāpadaṁ}}
\label{aniy1}

\linkdest{endnote-body}
Yo pana bhikkhu mātugāmena saddhiṁ eko ekāya raho paṭicchanne āsane alaṅ'kammaniye\makeatletter\hyperlink{endnote-appendix}\Hy@raisedlink{\hypertarget{endnote-body}{}{\pagenote{%
		\hypertarget{endnote-appendix}{\hyperlink{endnote-body}{}}}}}\makeatother C, Vibh Ee: alaṁkammaṇiye. Dm, Um, Mm Se, V, W: alaṁkammaniye. UP, G,
BhPm 1 & 2, Ra: alaṁ kammaniye. nisajjaṁ kappeyya, tam'enaṁ saddheyyavacasā upāsikā disvā tiṇṇaṁ dhammānaṁ aññatarena vadeyya: pārājikena vā saṅghādisesena vā pācittiyena vā, nisajjaṁ bhikkhu paṭijānamāno tiṇṇaṁ dhammānaṁ aññatarena kāretabbo: pārājikena vā saṅghādisesena vā pācittiyena vā, yena vā sā saddheyyavacasā upāsikā vadeyya, tena so bhikkhu kāretabbo, ayaṁ dhammo aniyato.



\pdfbookmark[2]{Aniyata 2}{aniy2}
\subsection*{\hyperref[unc2]{Aniyata 2: Dutiya-aniyatasikkhāpadaṁ}}
\label{aniy2}

\linkdest{endnote-body}
Na h'eva kho pana paṭicchannaṁ āsanaṁ hoti n'ālaṅ'kammaniyaṁ\makeatletter\hyperlink{endnote-appendix}\Hy@raisedlink{\hypertarget{endnote-body}{}{\pagenote{%
		\hypertarget{endnote-appendix}{\hyperlink{endnote-body}{}}}}}\makeatother Vibh Ee: -kammaṇiyaṁ. Mm Se, BhPm 1–2, C, D, G, V, W, Um, Ra, Vibh Ee: nālaṁ kammaniyaṁ. UP, BhPm 1 & 2, Ra: nālaṁ
kammanīyaṁ.), alañ'ca kho hoti mātugāmaṁ duṭṭhullāhi vācāhi obhāsituṁ. Yo pana bhikkhu tathārūpe āsane mātugāmena saddhiṁ eko ekāya raho nisajjaṁ kappeyya, tam'enaṁ saddheyyavacasā upāsikā disvā dvinnaṁ dhammānaṁ aññatarena vadeyya saṅghādisesena vā pācittiyena vā, nisajjaṁ bhikkhu paṭijānamāno dvinnaṁ dhammānaṁ aññatarena kāretabbo saṅghādisesena vā pācittiyena vā, yena vā sā saddheyyavacasā upāsikā vadeyya, tena so bhikkhu kāretabbo, ayam'pi dhammo aniyato.



\medskip

\begin{center}
	Uddiṭṭhā kho āyasmanto dve aniyatā dhammā.

	\smallskip

	Tatth'āyasmante pucchāmi: Kacci'ttha parisuddhā?\\
	Dutiyam'pi pucchāmi: Kacci'ttha parisuddhā?\\
	Tatiyam'pi pucchāmi: Kacci'ttha parisuddhā?

	\smallskip

\linkdest{endnote-body}
	Parisuddh'etth'āyasmanto, tasmā tuṇhī, evam'etaṁ dhārayāmi.\makeatletter\hyperlink{endnote-appendix}\Hy@raisedlink{\hypertarget{endnote-body}{}{\pagenote{%
		\hypertarget{endnote-appendix}{\hyperlink{endnote-body}{}}}}}\makeatother Dm, UP, Ra, Um: dhārayāmī ti.
\end{center}

\linkdest{endnote12-body}
\linkdest{endnote-body}
\begin{outro}
	Dve aniyatā dhammā niṭṭhitā\makeatletter\hyperlink{endnote12-appendix}\Hy@raisedlink{\hypertarget{endnote12-body}{}{\pagenote{%
				\hypertarget{endnote12-appendix}{\hyperlink{endnote12-body}{Not in any edition or manuscript, but if a conclusion is to be recited then this one as given in the Parivāra would be the suitable one.\\
						When reciting in brief use: Aniyat'uddeso niṭṭhito.\makeatletter\hyperlink{endnote-appendix}\Hy@raisedlink{\hypertarget{endnote-body}{}{\pagenote{%
		\hypertarget{endnote-appendix}{\hyperlink{endnote-body}{}}}}}\makeatother Ñd Ce, Um, UP, Mi Se: Aniyatuddeso catuttho. Dm: Aniyato niṭṭhito.}}}}}\makeatother
\end{outro}

\clearpage

