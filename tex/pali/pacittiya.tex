
\section{Pācittiyā}
\label{pc}

\begin{intro}
	Ime kho pan'āyasmanto dvenavuti pācittiyā dhammā uddesaṁ āgacchanti.
\end{intro}

\subsection{Musāvādavaggo}

\pdfbookmark[3]{Pācittiya 1}{pac1}
\subsubsection*{\hyperref[exp1]{Pācittiya 1: Musāvādasikkhāpadaṁ}}
\label{pac1}

Sampajānamusāvāde, pācittiyaṁ.



\pdfbookmark[3]{Pācittiya 2}{pac2}
\subsubsection*{\hyperref[exp2]{Pācittiya 2: Omasavādasikkhāpadaṁ}}
\label{pac2}

Omasavāde, pācittiyaṁ.



\pdfbookmark[3]{Pācittiya 3}{pac3}
\subsubsection*{\hyperref[exp3]{Pācittiya 3: Pesuññasikkhāpadaṁ}}
\label{pac3}

Bhikkhupesuññe, pācittiyaṁ.



\pdfbookmark[3]{Pācittiya 4}{exp4}
\subsubsection*{\hyperref[exp4]{Pācittiya 4: Padasodhammasikkhāpadaṁ}}
\label{pac4}

\linkdest{endnote226-body}
Yo pana bhikkhu anupasampannaṁ\makeatletter\hyperlink{endnote226-appendix}\Hy@raisedlink{\hypertarget{endnote226-body}{}{\pagenote{%
		\hypertarget{endnote226-appendix}{\hyperlink{endnote226-body}{V: \textit{anūpasampannaṁ}.}}}}}\makeatother \thinspace padaso dhammaṁ vāceyya, pācittiyaṁ.



\pdfbookmark[3]{Pācittiya 5}{pac5}
\subsubsection*{\hyperref[exp5]{Pācittiya 5: Paṭhamasahaseyyasikkhāpadaṁ}}
\label{pac5}

\linkdest{endnote227-body}
\linkdest{endnote228-body}
\linkdest{endnote229-body}
Yo pana bhikkhu anupasampannena\makeatletter\hyperlink{endnote227-appendix}\Hy@raisedlink{\hypertarget{endnote227-body}{}{\pagenote{%
		\hypertarget{endnote227-appendix}{\hyperlink{endnote227-body}{V: \textit{anūpasampannena}.}}}}}\makeatother \thinspace uttariṁ dirattatirattaṁ\makeatletter\hyperlink{endnote228-appendix}\Hy@raisedlink{\hypertarget{endnote228-body}{}{\pagenote{%
		\hypertarget{endnote228-appendix}{\hyperlink{endnote228-body}{Mi \& Mm Se, SVibh Ee: \textit{dvi-}. Dm, Um, UP, Mi \& Mm Se, SVibh Ee: \textit{uttaridirattatirattaṁ}.}}}}}\makeatother \thinspace saha seyyaṁ\makeatletter\hyperlink{endnote229-appendix}\Hy@raisedlink{\hypertarget{endnote229-body}{}{\pagenote{%
		\hypertarget{endnote229-appendix}{\hyperlink{endnote229-body}{Mi Se, Bh Pm 1 \& 2: \textit{saha seyyaṁ}. Other printed editions (SVibh Ee, SVibh Ce, UP, Mm Se): \textit{sahaseyyaṁ}.}}}}}\makeatother \thinspace kappeyya, pācittiyaṁ.



\pdfbookmark[3]{Pācittiya 6}{pac6}
\subsubsection*{\hyperref[exp6]{Pācittiya 6: Dutiyasahaseyyasikkhāpadaṁ}}
\label{pac6}

\linkdest{endnote230-body}
Yo pana bhikkhu mātugāmena saha seyyaṁ\makeatletter\hyperlink{endnote230-appendix}\Hy@raisedlink{\hypertarget{endnote230-body}{}{\pagenote{%
		\hypertarget{endnote230-appendix}{\hyperlink{endnote230-body}{In G the correction \textit{saddhiṁ} has been inserted before \textit{sahaseyyaṁ}.}}}}}\makeatother \thinspace kappeyya, pācittiyaṁ.



\pdfbookmark[3]{Pācittiya 7}{pac7}
\subsubsection*{\hyperref[exp7]{Pācittiya 7: Dhammadesanāsikkhāpadaṁ}}
\label{pac7}

\linkdest{endnote231-body}
Yo pana bhikkhu mātugāmassa uttariṁ chappañcavācāhi\makeatletter\hyperlink{endnote231-appendix}\Hy@raisedlink{\hypertarget{endnote231-body}{}{\pagenote{%
		\hypertarget{endnote231-appendix}{\hyperlink{endnote231-body}{Dm, Um, UP, Mi \& Mm Se, SVibh Ee: \textit{uttarichappañcavācāhi}. Cf Pāc 5.}}}}}\makeatother \thinspace dhammaṁ deseyya, aññatra viññunā purisaviggahena, pācittiyaṁ.



\pdfbookmark[3]{Pācittiya 8}{pac8}
\subsubsection*{\hyperref[exp8]{Pācittiya 8: Bhūtārocanasikkhāpadaṁ}}
\label{pac8}

Yo pana bhikkhu anupasampannassa uttarimanussadhammaṁ āroceyya bhūtasmiṁ, pācittiyaṁ.



\pdfbookmark[3]{Pācittiya 9}{pac9}
\subsubsection*{\hyperref[exp9]{Pācittiya 9: Duṭṭhullārocanasikkhāpadaṁ}}
\label{pac9}

\linkdest{endnote232-body}
\linkdest{endnote233-body}
Yo pana bhikkhu bhikkhussa duṭṭhullaṁ āpattiṁ anupasampannassa\makeatletter\hyperlink{endnote232-appendix}\Hy@raisedlink{\hypertarget{endnote232-body}{}{\pagenote{%
		\hypertarget{endnote232-appendix}{\hyperlink{endnote232-body}{V: \textit{anūpasampannassa}. (No long \textit{ū} in Pāc 8.)}}}}}\makeatother \thinspace āroceyya, aññatra bhikkhusammutiyā,\makeatletter\hyperlink{endnote233-appendix}\Hy@raisedlink{\hypertarget{endnote233-body}{}{\pagenote{%
		\hypertarget{endnote233-appendix}{\hyperlink{endnote233-body}{Mi \& Mm Se, BhPm 1 v.l.: \textit{sammatiyā}.}}}}}\makeatother \thinspace pācittiyaṁ.
% v.l: --> v.l.:



\pdfbookmark[3]{Pācittiya 10}{pac10}
\subsubsection*{\hyperref[exp10]{Pācittiya 10: Paṭhavīkhaṇanasikkhāpadaṁ}}
\label{pac10}

\linkdest{endnote234-body}
Yo pana bhikkhu paṭhaviṁ\makeatletter\hyperlink{endnote234-appendix}\Hy@raisedlink{\hypertarget{endnote234-body}{}{\pagenote{%
		\hypertarget{endnote234-appendix}{\hyperlink{endnote234-body}{Dm, V: \textit{pathaviṁ}.}}}}}\makeatother \thinspace khaṇeyya vā khaṇāpeyya vā, pācittiyaṁ.

\linkdest{endnote235-body}
\linkdest{endnote236-body}
\begin{center}
	Musāvādavaggo\makeatletter\hyperlink{endnote235-appendix}\Hy@raisedlink{\hypertarget{endnote235-body}{}{\pagenote{%
		\hypertarget{endnote235-appendix}{\hyperlink{endnote235-body}{Mm Se: \textit{musāvādāvagga}. (Probably a misprint or a corruption as initial members of compounds normally aren't inflected.)}}}}}\makeatother \thinspace paṭhamo\makeatletter\hyperlink{endnote236-appendix}\Hy@raisedlink{\hypertarget{endnote236-body}{}{\pagenote{%
		\hypertarget{endnote236-appendix}{\hyperlink{endnote236-body}{V: \textit{pathamo}.}}}}}\makeatother \thinspace
\end{center}



\subsection{Bhūtagāmavaggo}
% \vspace{0.2cm}

\pdfbookmark[3]{Pācittiya 11}{pac11}
\subsubsection*{\hyperref[exp11]{Pācittiya 11: Bhūtagāmasikkhāpadaṁ}}
\label{pac11}

\linkdest{endnote237-body}
Bhūtagāmapātabyatāya,\makeatletter\hyperlink{endnote237-appendix}\Hy@raisedlink{\hypertarget{endnote237-body}{}{\pagenote{%
		\hypertarget{endnote237-appendix}{\hyperlink{endnote237-body}{SVibh Ce, C, W, Ra: \textit{-pātavyatāya}.}}}}}\makeatother \thinspace pācittiyaṁ.



\pdfbookmark[3]{Pācittiya 12}{pac12}
\subsubsection*{\hyperref[exp12]{Pācittiya 12: Aññavādakasikkhāpadaṁ}}
\label{pac12}

Aññavādake vihesake, pācittiyaṁ.



\pdfbookmark[3]{Pācittiya 13}{pac13}
\subsubsection*{\hyperref[exp13]{Pācittiya 13: Ujjhāpanakasikkhāpadaṁ}}
\label{pac13}

\linkdest{endnote238-body}
Ujjhāpanake khiyyanake,\makeatletter\hyperlink{endnote238-appendix}\Hy@raisedlink{\hypertarget{endnote238-body}{}{\pagenote{%
		\hypertarget{endnote238-appendix}{\hyperlink{endnote238-body}{Bh Pm 1 \& 2, Dm, UP, Mi \& Mm Se, V: \textit{khiyyanake}. C, D, G, W, Um, SVibh Ce, SVibh Ee, Ra, Pg: \textit{khīyanake}.}}}}}\makeatother \thinspace pācittiyaṁ.



\pdfbookmark[3]{Pācittiya 14}{pac14}
\subsubsection*{\hyperref[exp14]{Pācittiya 14: Paṭhamasen'āsanasikkhāpadaṁ}}
\label{pac14}

\linkdest{endnote239-body}
\linkdest{endnote240-body}
\linkdest{endnote241-body}
\linkdest{endnote242-body}
\linkdest{endnote243-body}
\linkdest{endnote244-body}
Yo pana bhikkhu saṅghikaṁ\makeatletter\hyperlink{endnote239-appendix}\Hy@raisedlink{\hypertarget{endnote239-body}{}{\pagenote{%
		\hypertarget{endnote239-appendix}{\hyperlink{endnote239-body}{BhPm 1, C, V, W: \textit{saṁghikaṁ}.}}}}}\makeatother \thinspace mañcaṁ vā pīṭhaṁ\makeatletter\hyperlink{endnote240-appendix}\Hy@raisedlink{\hypertarget{endnote240-body}{}{\pagenote{%
		\hypertarget{endnote240-appendix}{\hyperlink{endnote240-body}{V: \textit{pithaṁ}.}}}}}\makeatother \thinspace vā bhisiṁ vā kocchaṁ vā ajjhokāse santharitvā\makeatletter\hyperlink{endnote241-appendix}\Hy@raisedlink{\hypertarget{endnote241-body}{}{\pagenote{%
		\hypertarget{endnote241-appendix}{\hyperlink{endnote241-body}{V: \textit{saṇthar-}.}}}}}\makeatother \thinspace vā santharāpetvā\makeatletter\hyperlink{endnote242-appendix}\Hy@raisedlink{\hypertarget{endnote242-body}{}{\pagenote{%
		\hypertarget{endnote242-appendix}{\hyperlink{endnote242-body}{V: \textit{saṇthar-}.}}}}}\makeatother \thinspace vā, taṁ pakkamanto n'eva uddhareyya na uddharāpeyya,\makeatletter\hyperlink{endnote243-appendix}\Hy@raisedlink{\hypertarget{endnote243-body}{}{\pagenote{%
		\hypertarget{endnote243-appendix}{\hyperlink{endnote243-body}{D: \textit{n'uddharāpeyya}.}}}}}\makeatother \thinspace anāpucchaṁ\makeatletter\hyperlink{endnote244-appendix}\Hy@raisedlink{\hypertarget{endnote244-body}{}{\pagenote{%
		\hypertarget{endnote244-appendix}{\hyperlink{endnote244-body}{Ra, SK, Pg, Sannē: \textit{anāpucchā}.}}}}}\makeatother \thinspace vā gaccheyya, pācittiyaṁ.



\pdfbookmark[3]{Pācittiya 15}{pac15}
\subsubsection*{\hyperref[exp15]{Pācittiya 15: Dutiyasen'āsanasikkhāpadaṁ}}
\label{pac15}

\linkdest{endnote245-body}
\linkdest{endnote246-body}
\linkdest{endnote247-body}
\linkdest{endnote248-body}
Yo pana bhikkhu saṅghike vihāre seyyaṁ santharitvā\makeatletter\hyperlink{endnote245-appendix}\Hy@raisedlink{\hypertarget{endnote245-body}{}{\pagenote{%
		\hypertarget{endnote245-appendix}{\hyperlink{endnote245-body}{V: \textit{saṇthar-}.}}}}}\makeatother \thinspace vā santharāpetvā\makeatletter\hyperlink{endnote246-appendix}\Hy@raisedlink{\hypertarget{endnote246-body}{}{\pagenote{%
		\hypertarget{endnote246-appendix}{\hyperlink{endnote246-body}{V: \textit{saṇthar-}.}}}}}\makeatother \thinspace vā, taṁ pakkamanto n'eva uddhareyya na uddharāpeyya,\makeatletter\hyperlink{endnote247-appendix}\Hy@raisedlink{\hypertarget{endnote247-body}{}{\pagenote{%
		\hypertarget{endnote247-appendix}{\hyperlink{endnote247-body}{D: \textit{n'uddharāpeyya}.}}}}}\makeatother \thinspace anāpucchaṁ\makeatletter\hyperlink{endnote248-appendix}\Hy@raisedlink{\hypertarget{endnote248-body}{}{\pagenote{%
		\hypertarget{endnote248-appendix}{\hyperlink{endnote248-body}{Ra, SK, Pg, Sannē: \textit{anāpucchā}.}}}}}\makeatother \thinspace vā gaccheyya, pācittiyaṁ.



\pdfbookmark[3]{Pācittiya 16}{pac16}
\subsubsection*{\hyperref[exp16]{Pācittiya 16: Anupakhajjasikkhāpadaṁ}}
\label{pac16}

\linkdest{endnote249-body}
\linkdest{endnote250-body}
\linkdest{endnote251-body}
Yo pana bhikkhu saṅghike\makeatletter\hyperlink{endnote249-appendix}\Hy@raisedlink{\hypertarget{endnote249-body}{}{\pagenote{%
		\hypertarget{endnote249-appendix}{\hyperlink{endnote249-body}{BhPm 1, C, V, W: \textit{saṁghikaṁ}.}}}}}\makeatother \thinspace vihāre jānaṁ pubb'upagataṁ\makeatletter\hyperlink{endnote250-appendix}\Hy@raisedlink{\hypertarget{endnote250-body}{}{\pagenote{%
		\hypertarget{endnote250-appendix}{\hyperlink{endnote250-body}{Mi \& Mm Se, Bh Pm 1 \& 2, D, W, Um, Pg: \textit{pubbūpagataṁ}. (C unclear.)}}}}}\makeatother \thinspace bhikkhuṁ anupakhajja\makeatletter\hyperlink{endnote251-appendix}\Hy@raisedlink{\hypertarget{endnote251-body}{}{\pagenote{%
		\hypertarget{endnote251-appendix}{\hyperlink{endnote251-body}{Mi \& Mm Se, V: \textit{anūpakhajja}.}}}}}\makeatother \thinspace seyyaṁ kappeyya: ``Yassa sambādho bhavissati, so pakkamissatī'ti'', etad'eva paccayaṁ karitvā anaññaṁ, pācittiyaṁ.



\pdfbookmark[3]{Pācittiya 17}{pac17}
\subsubsection*{\hyperref[exp17]{Pācittiya 17: Nikkaḍḍhanasikkhāpadaṁ}}
\label{pac17}

\linkdest{endnote252-body}
\linkdest{endnote253-body}
\linkdest{endnote254-body}
Yo pana bhikkhu bhikkhuṁ kupito\makeatletter\hyperlink{endnote252-appendix}\Hy@raisedlink{\hypertarget{endnote252-body}{}{\pagenote{%
		\hypertarget{endnote252-appendix}{\hyperlink{endnote252-body}{V: \textit{kuppito}. Cf NP 25 \& Pāc 74.}}}}}\makeatother \thinspace anattamano saṅghikā vihārā nikkaḍḍheyya\makeatletter\hyperlink{endnote253-appendix}\Hy@raisedlink{\hypertarget{endnote253-body}{}{\pagenote{%
		\hypertarget{endnote253-appendix}{\hyperlink{endnote253-body}{V: \textit{nikaḍheyya}. Cf \textit{aḍhamāso} at NP 24 and Pāc 57 in V.}}}}}\makeatother \thinspace vā nikkaḍḍhāpeyya\makeatletter\hyperlink{endnote254-appendix}\Hy@raisedlink{\hypertarget{endnote254-body}{}{\pagenote{%
		\hypertarget{endnote254-appendix}{\hyperlink{endnote254-body}{V: \textit{nikaḍhāpeyya}.}}}}}\makeatother \thinspace vā, pācittiyaṁ.



\pdfbookmark[3]{Pācittiya 18}{pac18}
\subsubsection*{\hyperref[exp18]{Pācittiya 18: Vehāsakuṭisikkhāpadaṁ}}
\label{pac18}

\linkdest{endnote255-body}
\linkdest{endnote256-body}
Yo pana bhikkhu saṅghike vihāre uparivehāsakuṭiyā āhaccapādakaṁ mañcaṁ vā pīṭhaṁ\makeatletter\hyperlink{endnote255-appendix}\Hy@raisedlink{\hypertarget{endnote255-body}{}{\pagenote{%
		\hypertarget{endnote255-appendix}{\hyperlink{endnote255-body}{V: \textit{pithaṁ}.}}}}}\makeatother \thinspace vā abhinisīdeyya\makeatletter\hyperlink{endnote256-appendix}\Hy@raisedlink{\hypertarget{endnote256-body}{}{\pagenote{%
		\hypertarget{endnote256-appendix}{\hyperlink{endnote256-body}{Bh Pm 1 \& 2, C, D, W, Ra, UP sīhala v.l.: \textit{sahasā abhinisīdeyya}. In G the correction \textit{sahasā} has been inserted later. It is not mentioned in the Sannē or Pg.}}}}}\makeatother \thinspace vā abhinipajjeyya vā, pācittiyaṁ.



\pdfbookmark[3]{Pācittiya 19}{pac19}
\subsubsection*{\hyperref[exp19]{Pācittiya 19: Mahallakavihārasikkhāpadaṁ}}
\label{pac19}

\linkdest{endnote257-body}
\linkdest{endnote258-body}
\linkdest{endnote259-body}
\linkdest{endnote260-body}
\linkdest{endnote261-body}
\linkdest{endnote262-body}
Mahallakaṁ pana\makeatletter\hyperlink{endnote257-appendix}\Hy@raisedlink{\hypertarget{endnote257-body}{}{\pagenote{%
		\hypertarget{endnote257-appendix}{\hyperlink{endnote257-body}{Mi Se, G, V, W: \textit{mahallakam-pana}.}}}}}\makeatother \thinspace bhikkhunā vihāraṁ kārayamānena, yāva dvārakosā\makeatletter\hyperlink{endnote258-appendix}\Hy@raisedlink{\hypertarget{endnote258-body}{}{\pagenote{%
		\hypertarget{endnote258-appendix}{\hyperlink{endnote258-body}{Bh Pm 1 \& 2, C, W, Ra, Pg, SVibh Ce, UP, Mi \& Mm Se: \textit{aggala-}. V: \textit{aggaḷaṭṭhappanāya}.}}}}}\makeatother \thinspace aggaḷaṭṭhapanāya ālokasandhiparikammāya dvatticchadanassa\makeatletter\hyperlink{endnote259-appendix}\Hy@raisedlink{\hypertarget{endnote259-body}{}{\pagenote{%
		\hypertarget{endnote259-appendix}{\hyperlink{endnote259-body}{SVibh Ee, Mi \& Mm Se: \textit{dvi-}; see NP 10.}}}}}\makeatother \thinspace pariyāyaṁ appaharite ṭhitena\makeatletter\hyperlink{endnote260-appendix}\Hy@raisedlink{\hypertarget{endnote260-body}{}{\pagenote{%
		\hypertarget{endnote260-appendix}{\hyperlink{endnote260-body}{V: \textit{thitena}.}}}}}\makeatother \thinspace adhiṭṭhātabbaṁ; tato ce uttariṁ,\makeatletter\hyperlink{endnote261-appendix}\Hy@raisedlink{\hypertarget{endnote261-body}{}{\pagenote{%
		\hypertarget{endnote261-appendix}{\hyperlink{endnote261-body}{Dm, Um, SVibh Ee: \textit{uttari} (but Be Sp \& Ee Sp read \textit{uttariṁ}.)}}}}}\makeatother \thinspace appaharite'pi ṭhito,\makeatletter\hyperlink{endnote262-appendix}\Hy@raisedlink{\hypertarget{endnote262-body}{}{\pagenote{%
		\hypertarget{endnote262-appendix}{\hyperlink{endnote262-body}{V: \textit{thito}. (D: \textit{appaharite ṭhito pi}.)}}}}}\makeatother \thinspace adhiṭṭhaheyya, pācittiyaṁ.



\pdfbookmark[3]{Pācittiya 20}{pac20}
\subsubsection*{\hyperref[exp20]{Pācittiya 20: Sappāṇakasikkhāpadaṁ}}
\label{pac20}

Yo pana bhikkhu jānaṁ sappāṇakaṁ udakaṁ tiṇaṁ vā mattikaṁ vā siñceyya vā siñcāpeyya vā, pācittiyaṁ.

\linkdest{endnote263-body}
\begin{center}
	Bhūtagāmavaggo\makeatletter\hyperlink{endnote263-appendix}\Hy@raisedlink{\hypertarget{endnote263-body}{}{\pagenote{%
		\hypertarget{endnote263-appendix}{\hyperlink{endnote263-body}{SVibh Ce v.l.: \textit{senāsanavaggob}.}}}}}\makeatother \thinspace dutiyo
\end{center}



\subsection{Bhikkhunovādavaggo}
% \vspace{0.2cm}

\pdfbookmark[3]{Pācittiya 21}{pac21}
\subsubsection*{\hyperref[exp]{Pācittiya 21: Ovādasikkhāpadaṁ}}
\label{pac21}

Yo pana bhikkhu asammato bhikkhuniyo ovadeyya, pācittiyaṁ.



\pdfbookmark[3]{Pācittiya 22}{pac22}
\subsubsection*{\hyperref[exp22]{Pācittiya 22: Atthaṅgatasikkhāpadaṁ}}
\label{pac22}

\linkdest{endnote264-body}
\linkdest{endnote265-body}
Sammato'pi\makeatletter\hyperlink{endnote264-appendix}\Hy@raisedlink{\hypertarget{endnote264-body}{}{\pagenote{%
		\hypertarget{endnote264-appendix}{\hyperlink{endnote264-body}{SVibh Ee: \textit{ce pi}.}}}}}\makeatother \thinspace ce bhikkhu atthaṅ'gate suriye\makeatletter\hyperlink{endnote265-appendix}\Hy@raisedlink{\hypertarget{endnote265-body}{}{\pagenote{%
		\hypertarget{endnote265-appendix}{\hyperlink{endnote265-body}{Dm: \textit{sūriye}. (= Sanskritisation; see Pecenko, Ee A-ṭ introduction p.liii.}}}}}\makeatother \thinspace bhikkhuniyo ovadeyya, pācittiyaṁ.



\pdfbookmark[3]{Pācittiya 23}{pac23}
\subsubsection*{\hyperref[exp23]{Pācittiya 23: Bhikkhunupassayasikkhāpadaṁ}}
\label{pac23}

\linkdest{endnote266-body}
\linkdest{endnote267-body}
Yo pana bhikkhu bhikkhun'ūpassayaṁ\makeatletter\hyperlink{endnote266-appendix}\Hy@raisedlink{\hypertarget{endnote266-body}{}{\pagenote{%
		\hypertarget{endnote266-appendix}{\hyperlink{endnote266-body}{C, G, W, Dm: \textit{bhikkhunupassayaṁ}. Um: \textit{bhikkhūnūpa-}.}}}}}\makeatother \thinspace upasaṅkamitvā bhikkhuniyo ovadeyya, aññatra samayā, pācittiyaṁ. Tatth'āyaṁ samayo: gilānā\makeatletter\hyperlink{endnote267-appendix}\Hy@raisedlink{\hypertarget{endnote267-body}{}{\pagenote{%
		\hypertarget{endnote267-appendix}{\hyperlink{endnote267-body}{V: \textit{gīlānā}.}}}}}\makeatother \thinspace hoti bhikkhunī; ayaṁ tattha samayo.



\pdfbookmark[3]{Pācittiya 24}{pac24}
\subsubsection*{\hyperref[exp24]{Pācittiya 24: Āmisasikkhāpadaṁ}}
\label{pac24}

\linkdest{endnote268-body}
\linkdest{endnote269-body}
Yo pana bhikkhu evaṁ vadeyya: ``Āmisahetu\makeatletter\hyperlink{endnote268-appendix}\Hy@raisedlink{\hypertarget{endnote268-body}{}{\pagenote{%
		\hypertarget{endnote268-appendix}{\hyperlink{endnote268-body}{V: \textit{āmissahetu}.}}}}}\makeatother \thinspace  bhikkhū\makeatletter\hyperlink{endnote269-appendix}\Hy@raisedlink{\hypertarget{endnote269-body}{}{\pagenote{%
		\hypertarget{endnote269-appendix}{\hyperlink{endnote269-body}{Dm, Um, SVibh Ee: \textit{āmisahetu therā bhikkhū}.}}}}}\makeatother \thinspace bhikkhuniyo ovadantī''ti, pācittiyaṁ.



\pdfbookmark[3]{Pācittiya 25}{pac25}
\subsubsection*{\hyperref[exp25]{Pācittiya 25: Cīvaradānasikkhāpadaṁ}}
\label{pac25}

\linkdest{endnote270-body}
Yo pana bhikkhu aññātikāya bhikkhuniyā cīvaraṁ dadeyya, aññatra pārivattakā,\makeatletter\hyperlink{endnote270-appendix}\Hy@raisedlink{\hypertarget{endnote270-body}{}{\pagenote{%
		\hypertarget{endnote270-appendix}{\hyperlink{endnote270-body}{Mi \& Mm Se, SVibh Ce, UP, Ra, BhPm 1 \& 2, C, D, G, V, W, Um, Pg: \textit{-vaṭṭakā}.}}}}}\makeatother \thinspace pācittiyaṁ.



\pdfbookmark[3]{Pācittiya 26}{pac26}
\subsubsection*{\hyperref[exp26]{Pācittiya 26: Cīvarasibbanasikkhāpadaṁ}}
\label{pac26}

Yo pana bhikkhu aññātikāya bhikkhuniyā cīvaraṁ sibbeyya vā sibbāpeyya vā, pācittiyaṁ.



\pdfbookmark[3]{Pācittiya 27}{pac27}
\subsubsection*{\hyperref[exp27]{Pācittiya 27: Saṁvidhānasikkhāpadaṁ}}
\label{pac27}

\linkdest{endnote271-body}
\linkdest{endnote272-body}
Yo pana bhikkhu bhikkhuniyā saddhiṁ saṁvidhāya ek'addhānamaggaṁ paṭipajjeyya antamaso gām'antaram'pi, aññatra samayā, pācittiyaṁ. Tatth'āyaṁ samayo: satthagamanīyo\makeatletter\hyperlink{endnote271-appendix}\Hy@raisedlink{\hypertarget{endnote271-body}{}{\pagenote{%
		\hypertarget{endnote271-appendix}{\hyperlink{endnote271-body}{V: \textit{-gamaniyo}.}}}}}\makeatother \thinspace hoti maggo sāsaṅkasammato;\makeatletter\hyperlink{endnote272-appendix}\Hy@raisedlink{\hypertarget{endnote272-body}{}{\pagenote{%
		\hypertarget{endnote272-appendix}{\hyperlink{endnote272-body}{C, W: \textit{saṁka-}. }}}}}\makeatother \thinspace sappaṭibhayo; ayaṁ tattha samayo.



\pdfbookmark[3]{Pācittiya 28}{pac28}
\subsubsection*{\hyperref[exp28]{Pācittiya 28: Nāvābhiruhanasikkhāpadaṁ}}
\label{pac28}

\linkdest{endnote273-body}
\linkdest{endnote274-body}
\linkdest{endnote275-body}
\linkdest{endnote276-body}
Yo pana bhikkhu bhikkhuniyā saddhiṁ saṁvidhāya ekaṁ nāvaṁ\makeatletter\hyperlink{endnote273-appendix}\Hy@raisedlink{\hypertarget{endnote273-body}{}{\pagenote{%
		\hypertarget{endnote273-appendix}{\hyperlink{endnote273-body}{Mi Se, G, V, Pg, Burmese ms. v.l. in SVibh Ee, Bh Pm 2 v.l.: \textit{ekanāvaṁ}. (Mm Se: \textit{ekaṁnāvaṁ}.)}}}}}\makeatother \thinspace abhirūheyya\makeatletter\hyperlink{endnote274-appendix}\Hy@raisedlink{\hypertarget{endnote274-body}{}{\pagenote{%
		\hypertarget{endnote274-appendix}{\hyperlink{endnote274-body}{BhPm 1 \& 2, C, V, W, Dm, UP: \textit{-ruheyya}.}}}}}\makeatother \thinspace uddhaṁgāminiṁ\makeatletter\hyperlink{endnote275-appendix}\Hy@raisedlink{\hypertarget{endnote275-body}{}{\pagenote{%
		\hypertarget{endnote275-appendix}{\hyperlink{endnote275-body}{UP: uddhaṁ gāmaniṁ adho gāmaniṁ. Mi \& Mm Se, Bh Pm 1 \& 2, C, D, Ra, Pg, SVibh Ce: \textit{uddhagāmaniṁ}.}}}}}\makeatother \thinspace vā adhogāminiṁ vā, aññatra tiriyaṁtaraṇāya,\makeatletter\hyperlink{endnote276-appendix}\Hy@raisedlink{\hypertarget{endnote276-body}{}{\pagenote{%
		\hypertarget{endnote276-appendix}{\hyperlink{endnote276-body}{Dm, SVibh Ce, UP, Bh Pm 1 \& 2, D, Ra: \textit{tiriyaṁ taraṇāya}. C, W, SVibh Ee: \textit{tiriyaṁtaraṇāya}, Mi \& Mm Se, G, Um, V: \textit{tiriyan-}.}}}}}\makeatother \thinspace taraṇāya. pācittiyaṁ.



\pdfbookmark[3]{Pācittiya 29}{pac29}
\subsubsection*{\hyperref[exp29]{Pācittiya 29: Paripācitasikkhāpadaṁ}}
\label{pac29}

\linkdest{endnote277-body}
\linkdest{endnote278-body}
Yo pana bhikkhu jānaṁ bhikkhunīparipācitaṁ\makeatletter\hyperlink{endnote277-appendix}\Hy@raisedlink{\hypertarget{endnote277-body}{}{\pagenote{%
		\hypertarget{endnote277-appendix}{\hyperlink{endnote277-body}{D, Dm, UP, V: \textit{bhikkhuni-}.}}}}}\makeatother \thinspace piṇḍapātaṁ bhuñjeyya, aññatra pubbe gihīsamārambhā,\makeatletter\hyperlink{endnote278-appendix}\Hy@raisedlink{\hypertarget{endnote278-body}{}{\pagenote{%
		\hypertarget{endnote278-appendix}{\hyperlink{endnote278-body}{D, Dm, Bh Pm 1, SVibh Ee, UP, Mi \& Mm Se: gihi. C, W, Um, Pg, Ra, SVibh Ce, Ee Sp: gihī. V: \textit{gīhi-}.}}}}}\makeatother \thinspace pācittiyaṁ.



\pdfbookmark[3]{Pācittiya 30}{pac30}
\subsubsection*{\hyperref[exp30]{Pācittiya 30: Rahonisajjasikkhāpadaṁ}}
\label{pac30}

Yo pana bhikkhu bhikkhuniyā saddhiṁ eko ekāya raho nisajjaṁ kappeyya, pācittiyaṁ.

\linkdest{endnote279-body}
\begin{center}
	Ovādavaggo\makeatletter\hyperlink{endnote279-appendix}\Hy@raisedlink{\hypertarget{endnote279-body}{}{\pagenote{%
		\hypertarget{endnote279-appendix}{\hyperlink{endnote279-body}{Dm, Mm Se, UP, SVibh Ee: \textit{ovādavaggo}. Bh Pm 1 \& 2, C, D, G, V, W, Um, Mi Se, SVibh Ce, Ra: \textit{bhikkhunovādavaggo}.}}}}}\makeatother \thinspace tatiyo
\end{center}



\subsection{Bhojanavaggo}
% \vspace{0.2cm}

\pdfbookmark[3]{Pācittiya 31}{pac31}
\subsubsection*{\hyperref[exp31]{Pācittiya 31: Āvasathapiṇḍasikkhāpadaṁ}}
\label{pac31}

\linkdest{endnote280-body}
\linkdest{endnote281-body}
Agilānena\makeatletter\hyperlink{endnote280-appendix}\Hy@raisedlink{\hypertarget{endnote280-body}{}{\pagenote{%
		\hypertarget{endnote280-appendix}{\hyperlink{endnote280-body}{V: \textit{agīlānena}.}}}}}\makeatother \thinspace bhikkhunā eko āvasathapiṇḍo bhuñjitabbo; tato ce uttariṁ\makeatletter\hyperlink{endnote281-appendix}\Hy@raisedlink{\hypertarget{endnote281-body}{}{\pagenote{%
		\hypertarget{endnote281-appendix}{\hyperlink{endnote281-body}{Be \& UP, Um, SVibh Ee: \textit{uttari}.}}}}}\makeatother \thinspace bhuñjeyya, pācittiyaṁ.



\pdfbookmark[3]{Pācittiya 32}{pac31}
\subsubsection*{\hyperref[exp32]{Pācittiya 32: Gaṇabhojanasikkhāpadaṁ}}
\label{pac32}

\linkdest{endnote282-body}
\linkdest{endnote283-body}
Gaṇabhojane, aññatra samayā, pācittiyaṁ. Tatth'āyaṁ samayo: gilānasamayo,\makeatletter\hyperlink{endnote282-appendix}\Hy@raisedlink{\hypertarget{endnote282-body}{}{\pagenote{%
		\hypertarget{endnote282-appendix}{\hyperlink{endnote282-body}{V: \textit{gīlāna-}.}}}}}\makeatother \thinspace cīvaradānasamayo, cīvarakārasamayo, addhānagamanasamayo, nāv'ābhirūhanasamayo,\makeatletter\hyperlink{endnote283-appendix}\Hy@raisedlink{\hypertarget{endnote283-body}{}{\pagenote{%
		\hypertarget{endnote283-appendix}{\hyperlink{endnote283-body}{Dm, Um, V: \textit{-ruhana-}.}}}}}\makeatother \thinspace mahāsamayo, samaṇabhattasamayo; ayaṁ tattha samayo.



\pdfbookmark[3]{Pācittiya 33}{pac33}
\subsubsection*{\hyperref[exp33]{Pācittiya 33: Paramparabhojanasikkhāpadaṁ}}
\label{pac33}

\linkdest{endnote284-body}
\linkdest{endnote285-body}
Paramparabhojane,\makeatletter\hyperlink{endnote284-appendix}\Hy@raisedlink{\hypertarget{endnote284-body}{}{\pagenote{%
		\hypertarget{endnote284-appendix}{\hyperlink{endnote284-body}{V: parappara-. SVibh Ee: \textit{paraṁpara-}.}}}}}\makeatother \thinspace aññatra samayā, pācittiyaṁ. Tatth'āyaṁ samayo: gilānasamayo,\makeatletter\hyperlink{endnote285-appendix}\Hy@raisedlink{\hypertarget{endnote285-body}{}{\pagenote{%
		\hypertarget{endnote285-appendix}{\hyperlink{endnote285-body}{V: \textit{gīlāna-}.}}}}}\makeatother \thinspace cīvaradānasamayo, cīvarakārasamayo; ayaṁ tattha samayo.



\pdfbookmark[3]{Pācittiya 34}{pac34}
\subsubsection*{\hyperref[exp34]{Pācittiya 34: Kāṇamātusikkhāpadaṁ}}
\label{pac34}

\linkdest{endnote286-body}
\linkdest{endnote287-body}
\linkdest{endnote288-body}
\linkdest{endnote289-body}
\linkdest{endnote290-body}
\linkdest{endnote291-body}
\linkdest{endnote292-body}
\linkdest{endnote293-body}
Bhikkhuṁ pan'eva kulaṁ upagataṁ pūvehi\makeatletter\hyperlink{endnote286-appendix}\Hy@raisedlink{\hypertarget{endnote286-body}{}{\pagenote{%
		\hypertarget{endnote286-appendix}{\hyperlink{endnote286-body}{V, Bh Pm 2 v.l.: \textit{puvehi}.}}}}}\makeatother \thinspace vā manthehi\makeatletter\hyperlink{endnote287-appendix}\Hy@raisedlink{\hypertarget{endnote287-body}{}{\pagenote{%
		\hypertarget{endnote287-appendix}{\hyperlink{endnote287-body}{V: \textit{maṇḥehi}.}}}}}\makeatother \thinspace vā abhihaṭṭhuṁ pavāreyya,\makeatletter\hyperlink{endnote288-appendix}\Hy@raisedlink{\hypertarget{endnote288-body}{}{\pagenote{%
		\hypertarget{endnote288-appendix}{\hyperlink{endnote288-body}{Mi Se, G: \textit{abhihaṭṭhum-pavāreyya}. V: \textit{abhihaṭṭham-pavāreyya}. Cf NP 7.}}}}}\makeatother \thinspace ākaṅkhamānena bhikkhunā dvattipattapūrā\makeatletter\hyperlink{endnote289-appendix}\Hy@raisedlink{\hypertarget{endnote289-body}{}{\pagenote{%
		\hypertarget{endnote289-appendix}{\hyperlink{endnote289-body}{SVibh Ee, Mi \& Mm Se: \textit{dvi-}; see NP 10. V: \textit{-purā}.}}}}}\makeatother \thinspace paṭiggahetabbā; tato ce uttariṁ\makeatletter\hyperlink{endnote290-appendix}\Hy@raisedlink{\hypertarget{endnote290-body}{}{\pagenote{%
		\hypertarget{endnote290-appendix}{\hyperlink{endnote290-body}{Be \& UP, Um, SVibh Ee: \textit{uttari}.}}}}}\makeatother \thinspace See NP 3. paṭiggaṇheyya,\makeatletter\hyperlink{endnote291-appendix}\Hy@raisedlink{\hypertarget{endnote291-body}{}{\pagenote{%
		\hypertarget{endnote291-appendix}{\hyperlink{endnote291-body}{C, D, W: \textit{patigaṇheyya}. (Cf. NP 5, NP 10.)}}}}}\makeatother \thinspace pācittiyaṁ. Dvattipattapūre\makeatletter\hyperlink{endnote292-appendix}\Hy@raisedlink{\hypertarget{endnote292-body}{}{\pagenote{%
		\hypertarget{endnote292-appendix}{\hyperlink{endnote292-body}{SVibh Ee, Mi \& Mm Se: \textit{dvi-}; see NP 10. V: \textit{-pure}.}}}}}\makeatother \thinspace paṭiggahetvā, tato nīharitvā, bhikkhūhi saddhiṁ saṁvibhajitabbaṁ.\makeatletter\hyperlink{endnote293-appendix}\Hy@raisedlink{\hypertarget{endnote293-body}{}{\pagenote{%
		\hypertarget{endnote293-appendix}{\hyperlink{endnote293-body}{V, Bh Pm 2 v.l.: \textit{saṁvibhajjitabbaṁ}.}}}}}\makeatother \thinspace Ayaṁ tattha sāmīci.



\pdfbookmark[3]{Pācittiya 35}{pac35}
\subsubsection*{\hyperref[exp35]{Pācittiya 35: Paṭhamapavāraṇāsikkhāpadaṁ}}
\label{pac35}

\linkdest{endnote294-body}
Yo pana bhikkhu bhuttāvī pavārito anatirittaṁ khādanīyaṁ vā bhojanīyaṁ\makeatletter\hyperlink{endnote294-appendix}\Hy@raisedlink{\hypertarget{endnote294-body}{}{\pagenote{%
		\hypertarget{endnote294-appendix}{\hyperlink{endnote294-body}{C, D, G, V, W, SVibh Ee, Um: \textit{khādaniyaṁ \& bhojaniyaṁ} throughout the text.}}}}}\makeatother \thinspace vā khādeyya vā bhuñjeyya vā, pācittiyaṁ.



\pdfbookmark[3]{Pācittiya 36}{pac36}
\subsubsection*{\hyperref[exp36]{Pācittiya 36: Dutiyapavāraṇāsikkhāpadaṁ}}
\label{pac36}

\linkdest{endnote295-body}
\linkdest{endnote296-body}
\linkdest{endnote297-body}
\linkdest{endnote298-body}
Yo pana bhikkhu bhikkhuṁ bhuttāviṁ pavāritaṁ anatirittena khādanīyena vā bhojanīyena\makeatletter\hyperlink{endnote295-appendix}\Hy@raisedlink{\hypertarget{endnote295-body}{}{\pagenote{%
		\hypertarget{endnote295-appendix}{\hyperlink{endnote295-body}{C, D, G, V, W, SVibh Ee, Um: \textit{khādaniyena \& bhojaniyena}.}}}}}\makeatother \thinspace vā abhihaṭṭhuṁ pavāreyya,\makeatletter\hyperlink{endnote296-appendix}\Hy@raisedlink{\hypertarget{endnote296-body}{}{\pagenote{%
		\hypertarget{endnote296-appendix}{\hyperlink{endnote296-body}{Mi Se, G: \textit{abhihaṭṭhum-pavāreyya}. V: \textit{abhihaṭṭham-pavāreyya}. Cf NP 7 and Pāc 34.}}}}}\makeatother \thinspace ``Handa bhikkhu khāda vā bhuñja vā'ti,'' jānaṁ\makeatletter\hyperlink{endnote297-appendix}\Hy@raisedlink{\hypertarget{endnote297-body}{}{\pagenote{%
		\hypertarget{endnote297-appendix}{\hyperlink{endnote297-body}{Um omits \textit{jānaṁ}.}}}}}\makeatother \thinspace āsādan'āpekkho,\makeatletter\hyperlink{endnote298-appendix}\Hy@raisedlink{\hypertarget{endnote298-body}{}{\pagenote{%
		\hypertarget{endnote298-appendix}{\hyperlink{endnote298-body}{Bh Pm 1 \& 2, C, D, W, Ra, Ce Kkh: \textit{-āpekho}. (Cf \textit{-āpekho} v.l. at Nid and Pāc 56, 60.)}}}}}\makeatother \thinspace bhuttasmiṁ, pācittiyaṁ.



\pdfbookmark[3]{Pācittiya 37}{pac37}
\subsubsection*{\hyperref[exp37]{Pācittiya 37: Vikālabhojanasikkhāpadaṁ}}
\label{pac37}

\linkdest{endnote299-body}
Yo pana bhikkhu vikāle khādanīyaṁ vā bhojanīyaṁ\makeatletter\hyperlink{endnote299-appendix}\Hy@raisedlink{\hypertarget{endnote299-body}{}{\pagenote{%
		\hypertarget{endnote299-appendix}{\hyperlink{endnote299-body}{C, D, G, V, W, SVibh Ee, Um: \textit{khādaniyaṁ \& bhojaniyaṁ}.}}}}}\makeatother \thinspace vā khādeyya vā bhuñjeyya vā, pācittiyaṁ.



\pdfbookmark[3]{Pācittiya 38}{pac38}
\subsubsection*{\hyperref[exp38]{Pācittiya 38: Sannidhikārakasikkhāpadaṁ}}
\label{pac38}

\linkdest{endnote300-body}
Yo pana bhikkhu sannidhikārakaṁ khādanīyaṁ vā bhojanīyaṁ\makeatletter\hyperlink{endnote300-appendix}\Hy@raisedlink{\hypertarget{endnote300-body}{}{\pagenote{%
		\hypertarget{endnote300-appendix}{\hyperlink{endnote300-body}{C, D, G, V, W, SVibh Ee, Um: \textit{khādaniyaṁ \& bhojaniyaṁ}.}}}}}\makeatother \thinspace vā khādeyya vā bhuñjeyya vā, pācittiyaṁ.



\pdfbookmark[3]{Pācittiya 39}{pac39}
\subsubsection*{\hyperref[exp39]{Pācittiya 39: Paṇītabhojanasikkhāpadaṁ}}
\label{pac39}

\linkdest{endnote301-body}
\linkdest{endnote302-body}
\linkdest{endnote303-body}
\linkdest{endnote304-body}
\linkdest{endnote305-body}
\linkdest{endnote306-body}
Yāni kho pana tāni paṇītabhojanāni, seyyath'īdaṁ:\makeatletter\hyperlink{endnote301-appendix}\Hy@raisedlink{\hypertarget{endnote301-body}{}{\pagenote{%
		\hypertarget{endnote301-appendix}{\hyperlink{endnote301-body}{Dm, UP: \textit{seyyathidaṁ}. Cf NP 23.}}}}}\makeatother \thinspace sappi, navanītaṁ,\makeatletter\hyperlink{endnote302-appendix}\Hy@raisedlink{\hypertarget{endnote302-body}{}{\pagenote{%
		\hypertarget{endnote302-appendix}{\hyperlink{endnote302-body}{V: \textit{navanitaṁ}. Cf NP 23.}}}}}\makeatother \thinspace telaṁ, madhuphāṇitaṁ,\makeatletter\hyperlink{endnote303-appendix}\Hy@raisedlink{\hypertarget{endnote303-body}{}{\pagenote{%
		\hypertarget{endnote303-appendix}{\hyperlink{endnote303-body}{C, D, W: \textit{madhupphāṇitaṁ}.}}}}}\makeatother \thinspace maccho, maṁsaṁ, khīraṁ, dadhi;\makeatletter\hyperlink{endnote304-appendix}\Hy@raisedlink{\hypertarget{endnote304-body}{}{\pagenote{%
		\hypertarget{endnote304-appendix}{\hyperlink{endnote304-body}{C, P, (Wae Uda Pm, Sirimalwatta Pm): \textit{dadhiṃ}. (This reading has later been
scribbled through in C.) Both \textit{dadhi} and \textit{dadhiṃ} are neuter nominative according
to CPED, although normally \textit{dadhiṃ} is accusative. Cf J-a IV 140: \textit{khīraṃ viya
dadhiṃ viya obhāsantaṃ}.}}}}}\makeatother \thinspace yo pana bhikkhu evarūpāni paṇītabhojanāni agilāno\makeatletter\hyperlink{endnote305-appendix}\Hy@raisedlink{\hypertarget{endnote305-body}{}{\pagenote{%
		\hypertarget{endnote305-appendix}{\hyperlink{endnote305-body}{V: \textit{gīlāno}.}}}}}\makeatother \thinspace attano atthāya viññāpetvā bhuñjeyya,\makeatletter\hyperlink{endnote306-appendix}\Hy@raisedlink{\hypertarget{endnote306-body}{}{\pagenote{%
		\hypertarget{endnote306-appendix}{\hyperlink{endnote306-body}{C, D, W: \textit{paribhuñjeyya}.}}}}}\makeatother \thinspace pācittiyaṁ.



\pdfbookmark[3]{Pācittiya 40}{pac40}
\subsubsection*{\hyperref[exp40]{Pācittiya 40: Dantaponasikkhāpadaṁ}}
\label{pac40}

\linkdest{endnote307-body}
\linkdest{endnote308-body}
Yo pana bhikkhu adinnaṁ mukhadvāraṁ āhāraṁ\makeatletter\hyperlink{endnote307-appendix}\Hy@raisedlink{\hypertarget{endnote307-body}{}{\pagenote{%
		\hypertarget{endnote307-appendix}{\hyperlink{endnote307-body}{V: \textit{adinnaṁ mukhadvāraṁ āhareyya}.}}}}}\makeatother \thinspace āhareyya, aññatra udakadantapoṇā,\makeatletter\hyperlink{endnote308-appendix}\Hy@raisedlink{\hypertarget{endnote308-body}{}{\pagenote{%
		\hypertarget{endnote308-appendix}{\hyperlink{endnote308-body}{Bh Pm 1 \& 2, Ra, Dm, SVibh Ce, Pg: \textit{-dantaponā}.}}}}}\makeatother \thinspace pācittiyaṁ.

\begin{center}
	Bhojanavaggo catuttho
\end{center}



\subsection{Acelakavaggo}
% \vspace{0.2cm}

\pdfbookmark[3]{Pācittiya 41}{pac41}
\subsubsection*{\hyperref[exp41]{Pācittiya 41: Acelakasikkhāpadaṁ}}
\label{pac41}

\linkdest{endnote309-body}
\linkdest{endnote310-body}
Yo pana bhikkhu acelakassa\makeatletter\hyperlink{endnote309-appendix}\Hy@raisedlink{\hypertarget{endnote309-body}{}{\pagenote{%
		\hypertarget{endnote309-appendix}{\hyperlink{endnote309-body}{C: \textit{aceḷak-}.}}}}}\makeatother \thinspace vā paribbājakassa vā paribbājikāya vā sahatthā khādanīyaṁ vā bhojanīyaṁ\makeatletter\hyperlink{endnote310-appendix}\Hy@raisedlink{\hypertarget{endnote310-body}{}{\pagenote{%
		\hypertarget{endnote310-appendix}{\hyperlink{endnote310-body}{C, D, G, V, W, SVibh Ee, Um: \textit{khādaniyaṁ \& bhojaniyaṁ}.}}}}}\makeatother \thinspace vā dadeyya, pācittiyaṁ.



\pdfbookmark[3]{Pācittiya 42}{pac42}
\subsubsection*{\hyperref[exp42]{Pācittiya 42: Uyyojanasikkhāpadaṁ}}
\label{pac42}

\linkdest{endnote311-body}
\linkdest{endnote312-body}
\linkdest{endnote313-body}
Yo pana bhikkhu bhikkhuṁ evaṁ vadeyya,\makeatletter\hyperlink{endnote311-appendix}\Hy@raisedlink{\hypertarget{endnote311-body}{}{\pagenote{%
		\hypertarget{endnote311-appendix}{\hyperlink{endnote311-body}{(= Mi \& Mm Se, G, D, Bh Pm 1 \& 2, V, Ra.) C, W, Dm, Um, UP, SVibh Ee, SVibh Ce: ...\textit{bhikkhuṁ ehāvuso}..., i.e., no \textit{evaṁ
vadeyya}. (Pg and Sannē also do not to have it.)}}}}}\makeatother \thinspace ``Eh'āvuso, gāmaṁ vā nigamaṁ vā piṇḍāya pavisissāmā'ti,''\makeatletter\hyperlink{endnote312-appendix}\Hy@raisedlink{\hypertarget{endnote312-body}{}{\pagenote{%
		\hypertarget{endnote312-appendix}{\hyperlink{endnote312-body}{V: \textit{pavīsissāmā}.}}}}}\makeatother \thinspace tassa dāpetvā vā adāpetvā vā uyyojeyya, ``Gacch'āvuso! Na me tayā saddhiṁ kathā vā nisajjā vā phāsu hoti; ekakassa\makeatletter\hyperlink{endnote313-appendix}\Hy@raisedlink{\hypertarget{endnote313-body}{}{\pagenote{%
		\hypertarget{endnote313-appendix}{\hyperlink{endnote313-body}{V: \textit{ekatassa}.}}}}}\makeatother \thinspace me kathā vā nisajjā vā phāsu hotī'ti;'' etad'eva paccayaṁ karitvā anaññaṁ, pācittiyaṁ.



\pdfbookmark[3]{Pācittiya 43}{pac43}
\subsubsection*{\hyperref[exp43]{Pācittiya 43: Sabhojanasikkhāpadaṁ}}
\label{pac43}

\linkdest{endnote314-body}
Yo pana bhikkhu sabhojane kule anupakhajja\makeatletter\hyperlink{endnote314-appendix}\Hy@raisedlink{\hypertarget{endnote314-body}{}{\pagenote{%
		\hypertarget{endnote314-appendix}{\hyperlink{endnote314-body}{Dm, Mi \& Mm Se: \textit{anūpakhajja}.}}}}}\makeatother \thinspace nisajjaṁ kappeyya, pācittiyaṁ.



\pdfbookmark[3]{Pācittiya 44}{pac44}
\subsubsection*{\hyperref[exp44]{Pācittiya 44: Rahopaṭicchannasikkhāpadaṁ}}
\label{pac44}

Yo pana bhikkhu mātugāmena saddhiṁ raho paṭicchanne āsane nisajjaṁ kappeyya, pācittiyaṁ.



\pdfbookmark[3]{Pācittiya 45}{pac45}
\subsubsection*{\hyperref[exp45]{Pācittiya 45: Rahonisajjasikkhāpadaṁ}}
\label{pac45}

Yo pana bhikkhu mātugāmena saddhiṁ eko ekāya raho nisajjaṁ kappeyya, pācittiyaṁ.



\pdfbookmark[3]{Pācittiya 46}{pac46}
\subsubsection*{\hyperref[exp46]{Pācittiya 46: Cārittasikkhāpadaṁ}}
\label{pac46}

\linkdest{endnote315-body}
Yo pana bhikkhu nimantito sabhatto samāno santaṁ bhikkhuṁ anāpucchā purebhattaṁ vā pacchābhattaṁ vā kulesu cārittaṁ\makeatletter\hyperlink{endnote315-appendix}\Hy@raisedlink{\hypertarget{endnote315-body}{}{\pagenote{%
		\hypertarget{endnote315-appendix}{\hyperlink{endnote315-body}{V: \textit{carittaṁ}.}}}}}\makeatother \thinspace āpajjeyya aññatra samayā, pācittiyaṁ. Tatth'āyaṁ samayo: cīvaradānasamayo, cīvarakārasamayo; ayaṁ tattha samayo.



\pdfbookmark[3]{Pācittiya 47}{pac47}
\subsubsection*{\hyperref[exp47]{Pācittiya 47: Mahānāmasikkhāpadaṁ}}
\label{pac47}

\linkdest{endnote316-body}
\linkdest{endnote317-body}
\linkdest{endnote318-body}
Agilānena\makeatletter\hyperlink{endnote316-appendix}\Hy@raisedlink{\hypertarget{endnote316-body}{}{\pagenote{%
		\hypertarget{endnote316-appendix}{\hyperlink{endnote316-body}{V: \textit{agīlānena}.}}}}}\makeatother \thinspace bhikkhunā cātumāsapaccayapavāraṇā\makeatletter\hyperlink{endnote317-appendix}\Hy@raisedlink{\hypertarget{endnote317-body}{}{\pagenote{%
		\hypertarget{endnote317-appendix}{\hyperlink{endnote317-body}{C, G, W, UP, Dm, SVibh Ce: cātumāsappaccaya-. D, Mi \& Mm Se, Bh Pm 1 \& 2, Um, V, SVibh Ee, Pg: \textit{cātumāsapaccaya-}.}}}}}\makeatother \thinspace sāditabbā; aññatra punapavāraṇāya, aññatra niccapavāraṇāya; tato ce uttariṁ\makeatletter\hyperlink{endnote318-appendix}\Hy@raisedlink{\hypertarget{endnote318-body}{}{\pagenote{%
		\hypertarget{endnote318-appendix}{\hyperlink{endnote318-body}{Be \& UP, Um, SVibh Ee: \textit{uttari}. See NP 3.}}}}}\makeatother \thinspace sādiyeyya, pācittiyaṁ.



\pdfbookmark[3]{Pācittiya 48}{pac48}
\subsubsection*{\hyperref[exp48]{Pācittiya 48: Uyyuttasenāsikkhāpadaṁ}}
\label{pac48}

\linkdest{endnote319-body}
\linkdest{endnote320-body}
Yo pana bhikkhu uyyuttaṁ\makeatletter\hyperlink{endnote319-appendix}\Hy@raisedlink{\hypertarget{endnote319-body}{}{\pagenote{%
		\hypertarget{endnote319-appendix}{\hyperlink{endnote319-body}{G: \textit{uyyutaṁ}.}}}}}\makeatother \thinspace senaṁ dassanāya gaccheyya; aññatra tathārūpapaccayā,\makeatletter\hyperlink{endnote320-appendix}\Hy@raisedlink{\hypertarget{endnote320-body}{}{\pagenote{%
		\hypertarget{endnote320-appendix}{\hyperlink{endnote320-body}{C, D, V, W, Dm, SVibh Ce, UP, Bh Pm 1 \& 2, Ra, Pg: -rūpappaccaya. (In G later corrected from -p- to -pp-.) See note to
\textit{-magga(p)paṭipannassa} at NP 16 and \textit{cātumāsapaccaya-} at Pāc 47.}}}}}\makeatother \thinspace pācittiyaṁ.




\pdfbookmark[3]{Pācittiya 49}{pac49}
\subsubsection*{\hyperref[exp49]{Pācittiya 49: Senāvāsasikkhāpadaṁ}}
\label{pac49}

\linkdest{endnote321-body}
\linkdest{endnote322-body}
\linkdest{endnote323-body}
Siyā ca tassa bhikkhuno koci'd'eva paccayo senaṁ gamanāya,\makeatletter\hyperlink{endnote321-appendix}\Hy@raisedlink{\hypertarget{endnote321-body}{}{\pagenote{%
		\hypertarget{endnote321-appendix}{\hyperlink{endnote321-body}{C, W: \textit{senaṅgamanāya}.}}}}}\makeatother \thinspace dirattatirattaṁ\makeatletter\hyperlink{endnote322-appendix}\Hy@raisedlink{\hypertarget{endnote322-body}{}{\pagenote{%
		\hypertarget{endnote322-appendix}{\hyperlink{endnote322-body}{Mi \& Mm Se,SVibh Ee: \textit{dvi-}.}}}}}\makeatother \thinspace tena bhikkhunā senāya vasitabbaṁ; tato ce uttariṁ\makeatletter\hyperlink{endnote323-appendix}\Hy@raisedlink{\hypertarget{endnote323-body}{}{\pagenote{%
		\hypertarget{endnote323-appendix}{\hyperlink{endnote323-body}{Be \& UP, Um, SVibh Ee: \textit{uttari}. See NP 3.}}}}}\makeatother \thinspace vaseyya, pācittiyaṁ.



\pdfbookmark[3]{Pācittiya 50}{pac50}
\subsubsection*{\hyperref[exp50]{Pācittiya 50: Uyyodhikasikkhāpadaṁ}}
\label{pac50}

\linkdest{endnote324-body}
\linkdest{endnote325-body}
Dirattatirattañ'ce\makeatletter\hyperlink{endnote324-appendix}\Hy@raisedlink{\hypertarget{endnote324-body}{}{\pagenote{%
		\hypertarget{endnote324-appendix}{\hyperlink{endnote324-body}{Mi \& Mm Se,SVibh Ee: \textit{dvi-}.}}}}}\makeatother \thinspace bhikkhu senāya vasamāno, uyyodhikaṁ vā bal'aggaṁ vā senābyūhaṁ\makeatletter\hyperlink{endnote325-appendix}\Hy@raisedlink{\hypertarget{endnote325-body}{}{\pagenote{%
		\hypertarget{endnote325-appendix}{\hyperlink{endnote325-body}{Ce Kkh: \textit{-vyūhaṁ}. G, Um, UP, V, Ra, Pg: \textit{-byuhaṁ}.}}}}}\makeatother \thinspace vā anīkadassanaṁ vā gaccheyya, pācittiyaṁ.

\linkdest{endnote326-body}
\begin{center}
	Acelakavaggo\makeatletter\hyperlink{endnote326-appendix}\Hy@raisedlink{\hypertarget{endnote326-body}{}{\pagenote{%
		\hypertarget{endnote326-appendix}{\hyperlink{endnote326-body}{C: \textit{aceḷaka-}.}}}}}\makeatother \thinspace pañcamo
\end{center}



\subsection{Surāpānavaggo}
% \vspace{0.2cm}

\pdfbookmark[3]{Pācittiya 51}{pac51}
\subsubsection*{\hyperref[exp51]{Pācittiya 51: Surāpānasikkhāpadaṁ}}
\label{pac51}

Surāmerayapāne pācittiyaṁ.



\pdfbookmark[3]{Pācittiya 52}{pac52}
\subsubsection*{\hyperref[exp52]{Pācittiya 52: Aṅgulipatodakasikkhāpadaṁ}}
\label{pac52}

Aṅgulipatodake pācittiyaṁ.



\pdfbookmark[3]{Pācittiya 53}{pac53}
\subsubsection*{\hyperref[exp53]{Pācittiya 53: Hassadhammasikkhāpadaṁ}}
\label{pac53}

\linkdest{endnote327-body}
Udake hassadhamme\makeatletter\hyperlink{endnote327-appendix}\Hy@raisedlink{\hypertarget{endnote327-body}{}{\pagenote{%
		\hypertarget{endnote327-appendix}{\hyperlink{endnote327-body}{Dm, Um: \textit{hasa-}. Mi Se, G, V, SVibh Ee: \textit{hāsa-}. C, D, W, UP, Ee Sp, Mm Se, Mi Se, SVibh Ce v.l.: \textit{hassa-}. SVibh Ee gives all three
readings as Burmese ms. v.l.l.}}}}}\makeatother \thinspace pācittiyaṁ.



\pdfbookmark[3]{Pācittiya 54}{pac54}
\subsubsection*{\hyperref[exp54]{Pācittiya 54: Anādariyasikkhāpadaṁ}}
\label{pac54}

Anādariye pācittiyaṁ.



\pdfbookmark[3]{Pācittiya 55}{pac55}
\subsubsection*{\hyperref[exp55]{Pācittiya 55: Bhiṁsāpanasikkhāpadaṁ}}
\label{pac55}

Yo pana bhikkhu bhikkhuṁ bhiṁsāpeyya, pācittiyaṁ.



\pdfbookmark[3]{Pācittiya 56}{pac56}
\subsubsection*{\hyperref[exp56]{Pācittiya 56: Jotikasikkhāpadaṁ}}
\label{pac56}

\linkdest{endnote328-body}
\linkdest{endnote329-body}
\linkdest{endnote330-body}
Yo pana bhikkhu agilāno\makeatletter\hyperlink{endnote328-appendix}\Hy@raisedlink{\hypertarget{endnote328-body}{}{\pagenote{%
		\hypertarget{endnote328-appendix}{\hyperlink{endnote328-body}{V: \textit{agīlāno}.}}}}}\makeatother \thinspace visibban'āpekkho\makeatletter\hyperlink{endnote329-appendix}\Hy@raisedlink{\hypertarget{endnote329-body}{}{\pagenote{%
		\hypertarget{endnote329-appendix}{\hyperlink{endnote329-body}{Mm Se, Pg: \textit{visīvanāpekkho}. Bh Pm 1 \& 2, C, D, W, Sannē: \textit{visīvanāpekho}. SVibh Ce, Um, Ra: \textit{visibbanāpekho}. (Cf v.l. at Pāc 36).}}}}}\makeatother \thinspace jotiṁ samādaheyya vā samādahāpeyya vā, aññatra tathārūpapaccayā,\makeatletter\hyperlink{endnote330-appendix}\Hy@raisedlink{\hypertarget{endnote330-body}{}{\pagenote{%
		\hypertarget{endnote330-appendix}{\hyperlink{endnote330-body}{Bh Pm 1 \& 2, C, D, Dm, Um, UP, V, SVibh Ce: \textit{-rūpappaccayā}; see Pāc 48.}}}}}\makeatother \thinspace pācittiyaṁ.



\pdfbookmark[3]{Pācittiya 57}{pac57}
\subsubsection*{\hyperref[exp57]{Pācittiya 57: Nahānasikkhāpadaṁ}}
\label{pac57}

\linkdest{endnote331-body}
\linkdest{endnote332-body}
\linkdest{endnote333-body}
\linkdest{endnote334-body}
\linkdest{endnote335-body}
\linkdest{endnote336-body}
\linkdest{endnote337-body}
\linkdest{endnote338-body}
Yo pana bhikkhu oren'aḍḍhamāsaṁ\makeatletter\hyperlink{endnote331-appendix}\Hy@raisedlink{\hypertarget{endnote331-body}{}{\pagenote{%
		\hypertarget{endnote331-appendix}{\hyperlink{endnote331-body}{Mi \& Mm Se, G, V: \textit{aḍḍha-}.}}}}}\makeatother \thinspace nahāyeyya,\makeatletter\hyperlink{endnote332-appendix}\Hy@raisedlink{\hypertarget{endnote332-body}{}{\pagenote{%
		\hypertarget{endnote332-appendix}{\hyperlink{endnote332-body}{SVibh Ee, Mi \& Mm Se: \textit{nhāyeyya}. V: \textit{ṇhāyeyya}.}}}}}\makeatother \thinspace aññatra samayā, pācittiyaṁ. Tatth'āyaṁ samayo: diyaḍḍho\makeatletter\hyperlink{endnote333-appendix}\Hy@raisedlink{\hypertarget{endnote333-body}{}{\pagenote{%
		\hypertarget{endnote333-appendix}{\hyperlink{endnote333-body}{V: \textit{diyaḍho}.}}}}}\makeatother \thinspace māso seso gimhānan'ti,\makeatletter\hyperlink{endnote334-appendix}\Hy@raisedlink{\hypertarget{endnote334-body}{}{\pagenote{%
		\hypertarget{endnote334-appendix}{\hyperlink{endnote334-body}{SVibh Be v.l.: \textit{gimhānaṁ}.}}}}}\makeatother \thinspace vassānassa paṭhamo\makeatletter\hyperlink{endnote335-appendix}\Hy@raisedlink{\hypertarget{endnote335-body}{}{\pagenote{%
		\hypertarget{endnote335-appendix}{\hyperlink{endnote335-body}{V: \textit{pathamo}.}}}}}\makeatother \thinspace māso, icc'ete aḍḍhateyyamāsā,\makeatletter\hyperlink{endnote336-appendix}\Hy@raisedlink{\hypertarget{endnote336-body}{}{\pagenote{%
		\hypertarget{endnote336-appendix}{\hyperlink{endnote336-body}{V: \textit{aḍha-}.}}}}}\makeatother \thinspace uṇhasamayo, pariḷāhasamayo,\makeatletter\hyperlink{endnote337-appendix}\Hy@raisedlink{\hypertarget{endnote337-body}{}{\pagenote{%
		\hypertarget{endnote337-appendix}{\hyperlink{endnote337-body}{C: \textit{parilāha-}.}}}}}\makeatother \thinspace gilānasamayo,\makeatletter\hyperlink{endnote338-appendix}\Hy@raisedlink{\hypertarget{endnote338-body}{}{\pagenote{%
		\hypertarget{endnote338-appendix}{\hyperlink{endnote338-body}{V: \textit{gīlāna-}.}}}}}\makeatother \thinspace kammasamayo, addhānagamanasamayo, vātavuṭṭhisamayo; ayaṁ tattha samayo.



\pdfbookmark[3]{Pācittiya 58}{pac58}
\subsubsection*{\hyperref[exp58]{Pācittiya 58: Dubbaṇṇakaraṇasikkhāpadaṁ}}
\label{pac58}

\linkdest{endnote338-body}
\linkdest{endnote339-body}
Navaṁ pana\makeatletter\hyperlink{endnote339-appendix}\Hy@raisedlink{\hypertarget{endnote339-body}{}{\pagenote{%
		\hypertarget{endnote339-appendix}{\hyperlink{endnote339-body}{Mi Se, G, V, P: \textit{navam-pana}.}}}}}\makeatother \thinspace bhikkhunā cīvaralābhena tiṇṇaṁ dubbaṇṇakaraṇānaṁ aññataraṁ dubbaṇṇakaraṇaṁ ādātabbaṁ, nīlaṁ vā kaddamaṁ\makeatletter\hyperlink{endnote339-appendix}\Hy@raisedlink{\hypertarget{endnote339-body}{}{\pagenote{%
		\hypertarget{endnote330-appendix}{\hyperlink{endnote339-body}{V: \textit{kaddumaṁ}.}}}}}\makeatother \thinspace vā kāḷasāmaṁ vā. Anādā ce bhikkhu tiṇṇaṁ dubbaṇṇakaraṇānaṁ aññataraṁ dubbaṇṇakaraṇaṁ navaṁ cīvaraṁ paribhuñjeyya, pācittiyaṁ.



\pdfbookmark[3]{Pācittiya 59}{pac59}
\subsubsection*{\hyperref[exp59]{Pācittiya 59: Vikappanasikkhāpadaṁ}}
\label{pac59}
% FIXME I pulled end notes from the draft but there is a different to the final release. some endnotes will be different.
\linkdest{endnote340-body}
\linkdest{endnote341-body}
\linkdest{endnote342-body}
Yo pana bhikkhu bhikkhussa vā bhikkhuniyā\makeatletter\hyperlink{endnote340-appendix}\Hy@raisedlink{\hypertarget{endnote340-body}{}{\pagenote{%
		\hypertarget{endnote340-appendix}{\hyperlink{endnote340-body}{V: \textit{bhikkhunīyā}.}}}}}\makeatother \thinspace vā sikkhamānāya vā sāmaṇerassa vā sāmaṇeriyā\makeatletter\hyperlink{endnote341-appendix}\Hy@raisedlink{\hypertarget{endnote341-body}{}{\pagenote{%
		\hypertarget{endnote341-appendix}{\hyperlink{endnote341-body}{Bh Pm 1 \& 2: \textit{sāmaṇerāya}. Mm Se: \textit{samaṇerassa}. \textit{samaṇeriyā}.}}}}}\makeatother \thinspace vā sāmaṁ cīvaraṁ vikappetvā apaccuddhārakaṁ\makeatletter\hyperlink{endnote342-appendix}\Hy@raisedlink{\hypertarget{endnote342-body}{}{\pagenote{%
		\hypertarget{endnote342-appendix}{\hyperlink{endnote342-body}{Dm, Um, UP: \textit{appaccuddhāraṇaṁ.} Pg (p. 57): \textit{apaccuddhārakaṁ}.}}}}}\makeatother \thinspace paribhuñjeyya, pācittiyaṁ.



\pdfbookmark[3]{Pācittiya 60}{pac60}
\subsubsection*{\hyperref[exp60]{Pācittiya 60: Apanidhānasikkhāpadaṁ}}
\label{pac60}

\linkdest{endnote343-body}
\linkdest{endnote344-body}
Yo pana bhikkhu bhikkhussa pattaṁ vā cīvaraṁ vā nisīdanaṁ vā sūcigharaṁ\makeatletter\hyperlink{endnote343-appendix}\Hy@raisedlink{\hypertarget{endnote343-body}{}{\pagenote{%
		\hypertarget{endnote343-appendix}{\hyperlink{endnote343-body}{D, V, Ra: \textit{suci-}. Cf. Pāc 86.}}}}}\makeatother \thinspace vā kāyabandhanaṁ vā apanidheyya vā apanidhāpeyya vā antamaso hass'āpekkho'pi\makeatletter\hyperlink{endnote344-appendix}\Hy@raisedlink{\hypertarget{endnote344-body}{}{\pagenote{%
		\hypertarget{endnote344-appendix}{\hyperlink{endnote344-body}{Dm, Um, V:  \textit{hasāpekkho}; SVibh Ce, Mi \& Mm Se, G, Pg:  \textit{hassāpekkho}. SVibh Ee:  \textit{hāsāpekkho}. Bh Pm 1 \& 2, C, D, W, Ra:  \textit{hassāpekho}. Bh Pm 2 v.l. \textit{hasāpekho}. (Cf Nid: \textit{visuddhāpekho}.) SVibh Ee gives all three as Burmese MS. v.l.l.}}}}}\makeatother \thinspace pācittiyaṁ.

\linkdest{endnote345-body}
\begin{center}
	Surāpānavaggo\makeatletter\hyperlink{endnote345-appendix}\Hy@raisedlink{\hypertarget{endnote345-body}{}{\pagenote{%
		\hypertarget{endnote345-appendix}{\hyperlink{endnote345-body}{V: \textit{-pāṇa-}.}}}}}\makeatother \thinspace chaṭṭho
\end{center}



\subsection{Sappāṇakavaggo}
% \vspace{0.2cm}

\pdfbookmark[3]{Pācittiya 61}{pac61}
\subsubsection*{\hyperref[exp61]{Pācittiya 61: Sañciccasikkhāpadaṁ}}
\label{pac61}

\linkdest{endnote346-body}
Yo pana bhikkhu sañcicca pāṇaṁ\makeatletter\hyperlink{endnote346-appendix}\Hy@raisedlink{\hypertarget{endnote346-body}{}{\pagenote{%
		\hypertarget{endnote346-appendix}{\hyperlink{endnote346-body}{C, W: \textit{pānaṁ}.}}}}}\makeatother \thinspace jīvitā voropeyya, pācittiyaṁ.



\pdfbookmark[3]{Pācittiya 62}{pac62}
\subsubsection*{\hyperref[exp62]{Pācittiya 62: Sappāṇakasikkhāpadaṁ}}
\label{pac62}

\linkdest{endnote347-body}
Yo pana bhikkhu jānaṁ sappāṇakaṁ\makeatletter\hyperlink{endnote347-appendix}\Hy@raisedlink{\hypertarget{endnote347-body}{}{\pagenote{%
		\hypertarget{endnote347-appendix}{\hyperlink{endnote347-body}{C: \textit{-pānakaṁ}.}}}}}\makeatother \thinspace udakaṁ paribhuñjeyya, pācittiyaṁ.



\pdfbookmark[3]{Pācittiya 63}{pac63}
\subsubsection*{\hyperref[exp63]{Pācittiya 63: Ukkoṭanasikkhāpadaṁ}}
\label{pac63}

\linkdest{endnote348-body}
Yo pana bhikkhu jānaṁ yathādhammaṁ nihat'ādhikaraṇaṁ\makeatletter\hyperlink{endnote348-appendix}\Hy@raisedlink{\hypertarget{endnote348-body}{}{\pagenote{%
		\hypertarget{endnote348-appendix}{\hyperlink{endnote348-body}{Mi \& Mm Se: \textit{nīhat-}.}}}}}\makeatother \thinspace punakammāya ukkoṭeyya, pācittiyaṁ.



\pdfbookmark[3]{Pācittiya 64}{pac64}
\subsubsection*{\hyperref[exp64]{Pācittiya 64: Duṭṭhullasikkhāpadaṁ}}
\label{pac64}

Yo pana bhikkhu bhikkhussa jānaṁ duṭṭhullaṁ āpattiṁ paṭicchādeyya, pācittiyaṁ.



\pdfbookmark[3]{Pācittiya 65}{pac65}
\subsubsection*{\hyperref[exp65]{Pācittiya 65: Ūnavīsativassasikkhāpadaṁ}}
\label{pac65}

\linkdest{endnote349-body}
\linkdest{endnote350-body}
\linkdest{endnote351-body}
Yo pana bhikkhu jānaṁ ūnavīsativassaṁ\makeatletter\hyperlink{endnote349-appendix}\Hy@raisedlink{\hypertarget{endnote349-body}{}{\pagenote{%
		\hypertarget{endnote349-appendix}{\hyperlink{endnote349-body}{G: ūṇa-. V: \textit{ona-}.}}}}}\makeatother \thinspace puggalaṁ upasampādeyya, so ca puggalo anupasampanno,\makeatletter\hyperlink{endnote350-appendix}\Hy@raisedlink{\hypertarget{endnote350-body}{}{\pagenote{%
		\hypertarget{endnote350-appendix}{\hyperlink{endnote350-body}{V: \textit{anūpasampanno}.}}}}}\makeatother \thinspace te ca\makeatletter\hyperlink{endnote351-appendix}\Hy@raisedlink{\hypertarget{endnote351-body}{}{\pagenote{%
		\hypertarget{endnote351-appendix}{\hyperlink{endnote351-body}{No \textit{ca} in V.}}}}}\makeatother \thinspace bhikkhū gārayhā. Idaṁ tasmiṁ pācittiyaṁ.



\pdfbookmark[3]{Pācittiya 66}{pac66}
\subsubsection*{\hyperref[exp66]{Pācittiya 66: Theyyasatthasikkhāpadaṁ}}
\label{pac66}

Yo pana bhikkhu jānaṁ theyyasatthena saddhiṁ saṁvidhāya ek'addhānamaggaṁ paṭipajjeyya antamaso gām'antaram'pi, pācittiyaṁ.



\pdfbookmark[3]{Pācittiya 67}{pac67}
\subsubsection*{\hyperref[exp67]{Pācittiya 67: Saṁvidhānasikkhāpadaṁ}}
\label{pac67}

Yo pana bhikkhu mātugāmena saddhiṁ saṁvidhāya ek'addhānamaggaṁ paṭipajjeyya antamaso gām'antaram'pi, pācittiyaṁ.



\pdfbookmark[3]{Pācittiya 68}{pac68}
\subsubsection*{\hyperref[exp68]{Pācittiya 68: Ariṭṭhasikkhāpadaṁ}}
\label{pac68}

\linkdest{endnote352-body}
\linkdest{endnote353-body}
\linkdest{endnote354-body}
\linkdest{endnote355-body}
\linkdest{endnote356-body}
\linkdest{endnote357-body}
\linkdest{endnote358-body}
\linkdest{endnote359-body}
\linkdest{endnote360-body}
\linkdest{endnote361-body}
\linkdest{endnote362-body}
Yo pana bhikkhu evaṁ vadeyya, ``Tath'āhaṁ bhagavatā dhammaṁ desitaṁ ājānāmi,\makeatletter\hyperlink{endnote352-appendix}\Hy@raisedlink{\hypertarget{endnote352-body}{}{\pagenote{%
		\hypertarget{endnote352-appendix}{\hyperlink{endnote352-body}{G, P: \textit{ajānāmi}.}}}}}\makeatother \thinspace yathā ye'me antarāyikā dhammā vuttā bhagavatā, te paṭisevato n'ālaṁ antarāyāyā'ti,'' so bhikkhu bhikkhūhi evam'assa vacanīyo,\makeatletter\hyperlink{endnote353-appendix}\Hy@raisedlink{\hypertarget{endnote353-body}{}{\pagenote{%
		\hypertarget{endnote353-appendix}{\hyperlink{endnote353-body}{V: \textit{vacaniyo}.}}}}}\makeatother \thinspace ``Mā āyasmā\makeatletter\hyperlink{endnote354-appendix}\Hy@raisedlink{\hypertarget{endnote354-body}{}{\pagenote{%
		\hypertarget{endnote354-appendix}{\hyperlink{endnote354-body}{Dm, UP, SVibh Ee, Um: \textit{māyasmā}.}}}}}\makeatother \thinspace evaṁ avaca, mā bhagavantaṁ abbhācikkhi,\makeatletter\hyperlink{endnote355-appendix}\Hy@raisedlink{\hypertarget{endnote355-body}{}{\pagenote{%
		\hypertarget{endnote355-appendix}{\hyperlink{endnote355-body}{Mi Se, G, V: \textit{abbhācikkha}.}}}}}\makeatother \thinspace na hi sādhu bhagavato abbhakkhānaṁ,\makeatletter\hyperlink{endnote356-appendix}\Hy@raisedlink{\hypertarget{endnote356-body}{}{\pagenote{%
		\hypertarget{endnote356-appendix}{\hyperlink{endnote356-body}{Um, SVibh Be v.l., Mi Se v.l.: \textit{abbhācikkhanaṁ}.}}}}}\makeatother \thinspace na hi bhagavā evaṁ vadeyya; anekapariyāyena āvuso\makeatletter\hyperlink{endnote357-appendix}\Hy@raisedlink{\hypertarget{endnote357-body}{}{\pagenote{%
		\hypertarget{endnote357-appendix}{\hyperlink{endnote357-body}{Dm, Um, UP: \textit{-pariyāyenāvuso}.}}}}}\makeatother \thinspace antarāyikā dhammā antarāyikā\makeatletter\hyperlink{endnote358-appendix}\Hy@raisedlink{\hypertarget{endnote358-body}{}{\pagenote{%
		\hypertarget{endnote358-appendix}{\hyperlink{endnote358-body}{Mi \& Mm Se, G, V, Ra: \textit{āvuso antarāyikā dhammā vuttā bhagavatā} D: \textit{anekapariyāyena āvuso antarāyikā vuttā bhagavatā} (Probably a misprint as not found in Malwatta mss.) (Pg unclear.)}}}}}\makeatother \thinspace vuttā bhagavatā, alañ'ca pana te paṭisevato antarāyāyā'ti,'' evañ'ca\makeatletter\hyperlink{endnote359-appendix}\Hy@raisedlink{\hypertarget{endnote359-body}{}{\pagenote{%
		\hypertarget{endnote359-appendix}{\hyperlink{endnote359-body}{SVibh Ce, SVibh Ee, Um, SVibh Be v.l., Mi v.l.: `\textit{evañ-ca pana so}. (Pg: ...\textit{evaṁ so bhikkhu bhikkhūhi}...)}}}}}\makeatother \thinspace so bhikkhu bhikkhūhi vuccamāno tath'eva paggaṇheyya, so bhikkhu bhikkhūhi yāvatatiyaṁ samanubhāsitabbo tassa paṭinissaggāya, yāvatatiyañ'ce\makeatletter\hyperlink{endnote360-appendix}\Hy@raisedlink{\hypertarget{endnote360-body}{}{\pagenote{%
		\hypertarget{endnote360-appendix}{\hyperlink{endnote360-body}{C, W, Bh Pm 1 \& 2, SVibh Ce: \textit{yāvatatiyaṁ ce}.}}}}}\makeatother \thinspace samanubhāsiyamāno taṁ paṭinissajeyya,\makeatletter\hyperlink{endnote361-appendix}\Hy@raisedlink{\hypertarget{endnote361-body}{}{\pagenote{%
		\hypertarget{endnote361-appendix}{\hyperlink{endnote361-body}{C, D, W. Other editions: \textit{paṭinissajjeyya}. See Sd 10.}}}}}\makeatother \thinspace icc'etaṁ kusalaṁ, no ce paṭinissajeyya,\makeatletter\hyperlink{endnote362-appendix}\Hy@raisedlink{\hypertarget{endnote362-body}{}{\pagenote{%
		\hypertarget{endnote362-appendix}{\hyperlink{endnote362-body}{C, D, W. Other editions: \textit{paṭinissajjeyya}. See Sd 10.}}}}}\makeatother \thinspace pācittiyaṁ.



\pdfbookmark[3]{Pācittiya 69}{pac69}
\subsubsection*{\hyperref[exp69]{Pācittiya 69: Ukkhittasambhogasikkhāpadaṁ}}
\label{pac69}

\linkdest{endnote363-body}
\linkdest{endnote364-body}
\linkdest{endnote365-body}
Yo pana bhikkhu jānaṁ tathāvādinā bhikkhunā akaṭ'ānudhammena\makeatletter\hyperlink{endnote363-appendix}\Hy@raisedlink{\hypertarget{endnote363-body}{}{\pagenote{%
		\hypertarget{endnote363-appendix}{\hyperlink{endnote363-body}{Bh Pm 1 \& 2, G, Um, UP, V: \textit{akatānudhammena}.}}}}}\makeatother \thinspace taṁ diṭṭhiṁ appaṭinissaṭṭhena saddhiṁ sambhuñjeyya\makeatletter\hyperlink{endnote364-appendix}\Hy@raisedlink{\hypertarget{endnote364-body}{}{\pagenote{%
		\hypertarget{endnote364-appendix}{\hyperlink{endnote364-body}{G, SVibh Ee: \textit{saṁbhuñjeyya}.}}}}}\makeatother \thinspace vā saṁvaseyya\makeatletter\hyperlink{endnote365-appendix}\Hy@raisedlink{\hypertarget{endnote365-body}{}{\pagenote{%
		\hypertarget{endnote365-appendix}{\hyperlink{endnote365-body}{D, G, V, SVibh Ee: \textit{saṁvāseyya}.}}}}}\makeatother \thinspace vā saha vā seyyaṁ kappeyya, pācittiyaṁ.



\pdfbookmark[3]{Pācittiya 70}{pac70}
\subsubsection*{\hyperref[exp70]{Pācittiya 70: Kaṇṭakasikkhāpadaṁ}}
\label{pac70}

\linkdest{endnote366-body}
\linkdest{endnote367-body}
\linkdest{endnote368-body}
\linkdest{endnote369-body}
\linkdest{endnote370-body}
\linkdest{endnote371-body}
\linkdest{endnote372-body}
\linkdest{endnote373-body}
\linkdest{endnote374-body}
\linkdest{endnote375-body}
\linkdest{endnote376-body}
\linkdest{endnote377-body}
\linkdest{endnote378-body}
Samaṇ'uddeso'pi ce evaṁ vadeyya, ``Tath'āhaṁ bhagavatā dhammaṁ desitaṁ ājānāmi,\makeatletter\hyperlink{endnote366-appendix}\Hy@raisedlink{\hypertarget{endnote366-body}{}{\pagenote{%
		\hypertarget{endnote366-appendix}{\hyperlink{endnote366-body}{G, P: \textit{ajānāmi}.}}}}}\makeatother \thinspace yathā ye'me antarāyikā dhammā vuttā bhagavatā, te paṭisevato n'ālaṁ antarāyāyā'ti,'' so samaṇ'uddeso bhikkhūhi evam'assa vacanīyo,\makeatletter\hyperlink{endnote367-appendix}\Hy@raisedlink{\hypertarget{endnote367-body}{}{\pagenote{%
		\hypertarget{endnote367-appendix}{\hyperlink{endnote367-body}{V: \textit{vacaniyo}.}}}}}\makeatother \thinspace ``Mā āvuso\makeatletter\hyperlink{endnote368-appendix}\Hy@raisedlink{\hypertarget{endnote368-body}{}{\pagenote{%
		\hypertarget{endnote368-appendix}{\hyperlink{endnote368-body}{Dm, UP, SVibh Ee: \textit{māvuso}.}}}}}\makeatother \thinspace samaṇ'uddesa evaṁ avaca, mā bhagavantaṁ abbhācikkhi,\makeatletter\hyperlink{endnote369-appendix}\Hy@raisedlink{\hypertarget{endnote369-body}{}{\pagenote{%
		\hypertarget{endnote369-appendix}{\hyperlink{endnote369-body}{Mi Se, G, V: \textit{abbhācikkha}.}}}}}\makeatother \thinspace na hi sādhu bhagavato abbhakkhānaṁ,\makeatletter\hyperlink{endnote370-appendix}\Hy@raisedlink{\hypertarget{endnote370-body}{}{\pagenote{%
		\hypertarget{endnote370-appendix}{\hyperlink{endnote370-body}{Um, SVibh Be v.l., Mi Se v.l.: \textit{abbhācikkhanaṁ}.}}}}}\makeatother \thinspace na hi bhagavā evaṁ vadeyya; anekapariyāyena āvuso\makeatletter\hyperlink{endnote371-appendix}\Hy@raisedlink{\hypertarget{endnote371-body}{}{\pagenote{%
		\hypertarget{endnote371-appendix}{\hyperlink{endnote371-body}{Dm, Um, UP: \textit{-pariyāyenāvuso}.}}}}}\makeatother \thinspace samaṇ'uddesa antarāyikā dhammā antarāyikā\makeatletter\hyperlink{endnote372-appendix}\Hy@raisedlink{\hypertarget{endnote372-body}{}{\pagenote{%
		\hypertarget{endnote372-appendix}{\hyperlink{endnote372-body}{Mi \& Mm Se, G, V, Ra: \textit{āvuso antarāyikā dhammā vuttā bhagavatā} D: \textit{anekapariyāyena āvuso antarāyikā vuttā bhagavatā} (Probably a misprint as not found in Malwatta mss.) (Pg unclear.)}}}}}\makeatother \thinspace vuttā bhagavatā, alañ'ca pana te paṭisevato antarāyāyā'ti,'' evañ'ca\makeatletter\hyperlink{endnote373-appendix}\Hy@raisedlink{\hypertarget{endnote373-body}{}{\pagenote{%
		\hypertarget{endnote373-appendix}{\hyperlink{endnote373-body}{SVibh Ce, SVibh Ee, Um, SVibh Be v.l., Mi v.l.: \textit{evañ-ca pana so}. (Pg: ...\textit{evaṁ so bhikkhu bhikkhūhi}...)}}}}}\makeatother \thinspace so samaṇ'uddeso bhikkhūhi vuccamāno tath'eva paggaṇheyya, so samaṇ'uddeso bhikkhūhi evam'assa vacanīyo,\makeatletter\hyperlink{endnote374-appendix}\Hy@raisedlink{\hypertarget{endnote374-body}{}{\pagenote{%
		\hypertarget{endnote374-appendix}{\hyperlink{endnote374-body}{V: \textit{vacaniyo}.}}}}}\makeatother \thinspace ``Ajja't'agge te āvuso samaṇ'uddesa na c'eva so bhagavā satthā apadisitabbo, yam'pi c'aññe samaṇ'uddesā labhanti bhikkhūhi saddhiṁ dirattatirattaṁ\makeatletter\hyperlink{endnote375-appendix}\Hy@raisedlink{\hypertarget{endnote375-body}{}{\pagenote{%
		\hypertarget{endnote375-appendix}{\hyperlink{endnote375-body}{Mm Se, SVibh Ee: \textit{dvi-}.}}}}}\makeatother \thinspace saha seyyaṁ,\makeatletter\hyperlink{endnote376-appendix}\Hy@raisedlink{\hypertarget{endnote376-body}{}{\pagenote{%
		\hypertarget{endnote376-appendix}{\hyperlink{endnote376-body}{Dm, SVibh Ce, UP, Mm \& Mi Se, V, SVibh Ee,: \textit{sahaseyyaṁ}. See Pāc 5.}}}}}\makeatother \thinspace sā'pi te n'atthi, carapire\makeatletter\hyperlink{endnote377-appendix}\Hy@raisedlink{\hypertarget{endnote377-body}{}{\pagenote{%
		\hypertarget{endnote377-appendix}{\hyperlink{endnote377-body}{Dm, Um, UP, SVibh Ee, Mi \& Mm Se, V, W: pire. Bh Pm 1 \& 2, C, D, SVibh Ce, Ra, Pg, Ce Kkh: \textit{pare}. G: \textit{cara pi pare}.}}}}}\makeatother \thinspace vinassā'ti.'' Yo pana bhikkhu jānaṁ tathānāsitaṁ samaṇ'uddesaṁ upalāpeyya vā upaṭṭhāpeyya vā sambhuñjeyya\makeatletter\hyperlink{endnote378-appendix}\Hy@raisedlink{\hypertarget{endnote378-body}{}{\pagenote{%
		\hypertarget{endnote378-appendix}{\hyperlink{endnote378-body}{G, SVibh Ee: \textit{saṁbhuñjeyya}.}}}}}\makeatother \thinspace vā saha vā seyyaṁ kappeyya, pācittiyaṁ.

\linkdest{endnote379-body}
\begin{center}
	Sappāṇakavaggo\makeatletter\hyperlink{endnote379-appendix}\Hy@raisedlink{\hypertarget{endnote379-body}{}{\pagenote{%
		\hypertarget{endnote379-appendix}{\hyperlink{endnote379-body}{Mi \& Mm Se, G, V: \textit{sappāṇavaggo}.}}}}}\makeatother \thinspace sattamo
\end{center}



\subsection{Sahadhammikavaggo}
% \vspace{0.2cm}

\pdfbookmark[3]{Pācittiya 71}{pac71}
\subsubsection*{\hyperref[exp71]{Pācittiya 71: Sahadhammikasikkhāpadaṁ}}
\label{pac71}

\linkdest{endnote380-body}
\linkdest{endnote381-body}
\linkdest{endnote382-body}
Yo pana bhikkhu bhikkhūhi sahadhammikaṁ vuccamāno evaṁ vadeyya, ``Na tāv'āhaṁ āvuso etasmiṁ sikkhāpade sikkhissāmi, yāva na aññaṁ\makeatletter\hyperlink{endnote380-appendix}\Hy@raisedlink{\hypertarget{endnote380-body}{}{\pagenote{%
		\hypertarget{endnote380-appendix}{\hyperlink{endnote380-body}{Mi \& Mm Se: \textit{naññaṁ}. G: \textit{na aṁñaṁ}.}}}}}\makeatother \thinspace bhikkhuṁ byattaṁ\makeatletter\hyperlink{endnote381-appendix}\Hy@raisedlink{\hypertarget{endnote381-body}{}{\pagenote{%
		\hypertarget{endnote381-appendix}{\hyperlink{endnote381-body}{Bh Pm 1 \& 2, C, D, W, UP, Ra, SVibh Ce, Pg: \textit{vyattaṁ}.}}}}}\makeatother \thinspace vinayadharaṁ paripucchāmī''ti, pācittiyaṁ. Sikkhamānena, bhikkhave, bhikkhunā aññātabbaṁ paripucchitabbaṁ paripañhitabbaṁ.\makeatletter\hyperlink{endnote382-appendix}\Hy@raisedlink{\hypertarget{endnote382-body}{}{\pagenote{%
		\hypertarget{endnote382-appendix}{\hyperlink{endnote382-body}{D, G, V: - \textit{paṇhi-}.}}}}}\makeatother \thinspace Ayaṁ tattha sāmīci.



\pdfbookmark[3]{Pācittiya 72}{pac72}
\subsubsection*{\hyperref[exp72]{Pācittiya 72: Vilekhanasikkhāpadaṁ}}
\label{pac72}

\linkdest{endnote383-body}
\linkdest{endnote384-body}
\linkdest{endnote385-body}
Yo pana bhikkhu pātimokkhe\makeatletter\hyperlink{endnote383-appendix}\Hy@raisedlink{\hypertarget{endnote383-body}{}{\pagenote{%
		\hypertarget{endnote383-appendix}{\hyperlink{endnote383-body}{Mm Se, G, V: \textit{pāṭimokkhe}.}}}}}\makeatother \thinspace uddissamāne evaṁ vadeyya, ``Kiṁ pan'imehi\makeatletter\hyperlink{endnote384-appendix}\Hy@raisedlink{\hypertarget{endnote384-body}{}{\pagenote{%
		\hypertarget{endnote384-appendix}{\hyperlink{endnote384-body}{Mi \& Mm Se, G, V: \textit{kim-pan'imehi}.}}}}}\makeatother \thinspace khudd'ānukhuddakehi sikkhāpadehi uddiṭṭhehi; yāva'd'eva kukkuccāya, vihesāya, vilekhāya saṁvattantī'ti,'' sikkhāpadavivaṇṇake,\makeatletter\hyperlink{endnote385-appendix}\Hy@raisedlink{\hypertarget{endnote385-body}{}{\pagenote{%
		\hypertarget{endnote385-appendix}{\hyperlink{endnote385-body}{Dm, UP, G, V, SVibh Ce, SVibh Ee: \textit{vivaṇṇake}.}}}}}\makeatother \thinspace BhPm 1 \& 2, C, D, W, Mi \& Mm Se, Um, Ra, Pg, Ce Kkh: vivaṇṇanake. pācittiyaṁ.



\pdfbookmark[3]{Pācittiya 73}{pac73}
\subsubsection*{\hyperref[exp73]{Pācittiya 73: Mohanasikkhāpadaṁ}}
\label{pac73}

\linkdest{endnote386-body}
\linkdest{endnote387-body}
\linkdest{endnote388-body}
\linkdest{endnote389-body}
\linkdest{endnote390-body}
\linkdest{endnote391-body}
\linkdest{endnote392-body}
\linkdest{endnote393-body}
\linkdest{endnote394-body}
\linkdest{endnote395-body}
\linkdest{endnote396-body}
\linkdest{endnote397-body}
\linkdest{endnote398-body}
\linkdest{endnote399-body}
Yo pana bhikkhu anvaḍḍhamāsaṁ\makeatletter\hyperlink{endnote386-appendix}\Hy@raisedlink{\hypertarget{endnote386-body}{}{\pagenote{%
		\hypertarget{endnote386-appendix}{\hyperlink{endnote386-body}{As in Pāc 57, only Mi \& Mm Se, \& V read \textit{anvaḍḍha-}. The rest read \textit{anvaddha-}.}}}}}\makeatother \thinspace  pātimokkhe\makeatletter\hyperlink{endnote387-appendix}\Hy@raisedlink{\hypertarget{endnote387-body}{}{\pagenote{%
		\hypertarget{endnote387-appendix}{\hyperlink{endnote387-body}{Mm Se, G, V: \textit{pāṭimokkhe}.}}}}}\makeatother \thinspace uddissamāne evaṁ vadeyya, ``Idān'eva kho\makeatletter\hyperlink{endnote388-appendix}\Hy@raisedlink{\hypertarget{endnote388-body}{}{\pagenote{%
		\hypertarget{endnote388-appendix}{\hyperlink{endnote388-body}{Bh Pm 1 \& 2, C, W, UP, Ra: \textit{kho āvuso}.}}}}}\makeatother \thinspace ahaṁ jānāmi,\makeatletter\hyperlink{endnote389-appendix}\Hy@raisedlink{\hypertarget{endnote389-body}{}{\pagenote{%
		\hypertarget{endnote389-appendix}{\hyperlink{endnote389-body}{Bh Pm 1 \& 2, Mi \& Mm Se, V, Ra, Pg: \textit{ājānāmi}.}}}}}\makeatother \thinspace ayam'pi\makeatletter\hyperlink{endnote390-appendix}\Hy@raisedlink{\hypertarget{endnote390-body}{}{\pagenote{%
		\hypertarget{endnote390-appendix}{\hyperlink{endnote390-body}{Um: \textit{ayaṁ pi}.}}}}}\makeatother \thinspace kira dhammo sutt'āgato suttapariyāpanno anvaḍḍhamāsaṁ\makeatletter\hyperlink{endnote391-appendix}\Hy@raisedlink{\hypertarget{endnote391-body}{}{\pagenote{%
		\hypertarget{endnote391-appendix}{\hyperlink{endnote391-body}{Mi \& Mm Se, \& V: \textit{anvaḍḍha-}. In the second occurrence of this word in this rule G read \textit{-ḍḍh-}, but was corrected to \textit{-ddh-}.}}}}}\makeatother \thinspace uddesaṁ āgacchatī'ti,'' tañ'ce\makeatletter\hyperlink{endnote392-appendix}\Hy@raisedlink{\hypertarget{endnote392-body}{}{\pagenote{%
		\hypertarget{endnote392-appendix}{\hyperlink{endnote392-body}{C: \textit{taṁ ce}.}}}}}\makeatother \thinspace bhikkhuṁ aññe bhikkhū jāneyyuṁ, ``Nisinnapubbaṁ iminā bhikkhunā dvattikkhattuṁ\makeatletter\hyperlink{endnote393-appendix}\Hy@raisedlink{\hypertarget{endnote393-body}{}{\pagenote{%
		\hypertarget{endnote393-appendix}{\hyperlink{endnote393-body}{SVibh Ee, Mm Se: \textit{dvi-}. (Mi Se reads \textit{dva-} here, instead of \textit{dvi-} elsewhere. See NP 10.)}}}}}\makeatother \thinspace pātimokkhe\makeatletter\hyperlink{endnote394-appendix}\Hy@raisedlink{\hypertarget{endnote394-body}{}{\pagenote{%
		\hypertarget{endnote394-appendix}{\hyperlink{endnote394-body}{Mm Se, G, V: \textit{pāṭimokkhe}.}}}}}\makeatother \thinspace uddissamāne. Ko pana vādo bhiyyo'ti,\makeatletter\hyperlink{endnote395-appendix}\Hy@raisedlink{\hypertarget{endnote395-body}{}{\pagenote{%
		\hypertarget{endnote395-appendix}{\hyperlink{endnote395-body}{Mi \& Mm Se, C, D, V: \textit{bhiyyo ti}. Bh Pm 1 \& 2, G, Um: \textit{bhīyyo ti}. Others MS and texts have \textit{bhiyyo na ca} without \textit{ti}. (Pg unclear.)}}}}}\makeatother \thinspace na ca tassa bhikkhuno aññāṇakena mutti atthi, yañ'ca tattha āpattiṁ āpanno, tañ'ca yathā dhammo\makeatletter\hyperlink{endnote396-appendix}\Hy@raisedlink{\hypertarget{endnote396-body}{}{\pagenote{%
		\hypertarget{endnote396-appendix}{\hyperlink{endnote396-body}{Bh Pm 1 \& 2, Ra: \textit{yathā dhammo}. Other printed eds: \textit{yathādhammo}.}}}}}\makeatother \thinspace kāretabbo, uttariñ'c'assa\makeatletter\hyperlink{endnote397-appendix}\Hy@raisedlink{\hypertarget{endnote397-body}{}{\pagenote{%
		\hypertarget{endnote397-appendix}{\hyperlink{endnote397-body}{Dm, SVibh Ee, Um: \textit{uttari cassa}. C, G, W, Bh Pm 1 \& 2, SVibh Ce, Ra: \textit{uttariṁ cassa}.}}}}}\makeatother \thinspace moho āropetabbo, ``Tassa te āvuso alābhā, tassa te dulladdhaṁ. Yaṁ tvaṁ pātimokkhe\makeatletter\hyperlink{endnote398-appendix}\Hy@raisedlink{\hypertarget{endnote398-body}{}{\pagenote{%
		\hypertarget{endnote398-appendix}{\hyperlink{endnote398-body}{Mm Se, G, V: \textit{pāṭimokkhe}.}}}}}\makeatother \thinspace uddissamāne, na sādhukaṁ aṭṭhikatvā\makeatletter\hyperlink{endnote399-appendix}\Hy@raisedlink{\hypertarget{endnote399-body}{}{\pagenote{%
		\hypertarget{endnote399-appendix}{\hyperlink{endnote399-body}{Dm, Um, UP: \textit{aṭṭhiṁ katvā}.}}}}}\makeatother \thinspace manasikarosī'ti.'' Idaṁ tasmiṁ mohanake, pācittiyaṁ.



\pdfbookmark[3]{Pācittiya 74}{pac74}
\subsubsection*{\hyperref[exp74]{Pācittiya 74: Pahārasikkhāpadaṁ}}
\label{pac74}

\linkdest{endnote400-body}
Yo pana bhikkhu bhikkhussa kupito\makeatletter\hyperlink{endnote400-appendix}\Hy@raisedlink{\hypertarget{endnote400-body}{}{\pagenote{%
		\hypertarget{endnote400-appendix}{\hyperlink{endnote400-body}{V: \textit{kuppito}. (Cf NP 25 \& Pāc 17.)}}}}}\makeatother \thinspace anattamano pahāraṁ dadeyya, pācittiyaṁ.



\pdfbookmark[3]{Pācittiya 75}{pac75}
\subsubsection*{\hyperref[exp75]{Pācittiya 75: Talasattikasikkhāpadaṁ}}
\label{pac75}

\linkdest{endnote401-body}
Yo pana bhikkhu bhikkhussa kupito\makeatletter\hyperlink{endnote401-appendix}\Hy@raisedlink{\hypertarget{endnote401-body}{}{\pagenote{%
		\hypertarget{endnote401-appendix}{\hyperlink{endnote401-body}{V: \textit{kuppito}.}}}}}\makeatother \thinspace anattamano talasattikaṁ uggireyya, pācittiyaṁ.



\pdfbookmark[3]{Pācittiya 76}{pac76}
\subsubsection*{\hyperref[exp76]{Pācittiya 76: Amūlakasikkhāpadaṁ}}
\label{pac76}

Yo pana bhikkhu bhikkhuṁ amūlakena saṅghādisesena anuddhaṁseyya, pācittiyaṁ.



\pdfbookmark[3]{Pācittiya 77}{pac77}
\subsubsection*{\hyperref[exp77]{Pācittiya 77: Sañciccasikkhāpadaṁ}}
\label{pac77}

\linkdest{endnote402-body}
\linkdest{endnote403-body}
Yo pana bhikkhu bhikkhussa sañcicca\makeatletter\hyperlink{endnote402-appendix}\Hy@raisedlink{\hypertarget{endnote402-body}{}{\pagenote{%
		\hypertarget{endnote402-appendix}{\hyperlink{endnote402-body}{W: \textit{saṁcicca} (but not so at Pār 3 and Pāc 61.)}}}}}\makeatother \thinspace kukkuccaṁ upadaheyya,\makeatletter\hyperlink{endnote403-appendix}\Hy@raisedlink{\hypertarget{endnote403-body}{}{\pagenote{%
		\hypertarget{endnote403-appendix}{\hyperlink{endnote403-body}{Ra, Pg, UP v.l.: \textit{uppādeyya}. G: \textit{uppādaheyya}. V: \textit{upādaheyya}.}}}}}\makeatother \thinspace ``Iti'ssa muhuttam'pi aphāsu bhavissatī'ti,'' etad'eva paccayaṁ karitvā anaññaṁ, pācittiyaṁ.



\pdfbookmark[3]{Pācittiya 78}{pac78}
\subsubsection*{\hyperref[exp78]{Pācittiya 78: Upassutisikkhāpadaṁ}}
\label{pac78}

\linkdest{endnote404-body}
Yo pana bhikkhu bhikkhūnaṁ bhaṇḍanajātānaṁ kalahajātānaṁ vivād'āpannānaṁ upassutiṁ\makeatletter\hyperlink{endnote404-appendix}\Hy@raisedlink{\hypertarget{endnote404-body}{}{\pagenote{%
		\hypertarget{endnote404-appendix}{\hyperlink{endnote404-body}{Mi Se, Bh Pm 2, Pg: \textit{upassuti}. V: \textit{upassūti}.}}}}}\makeatother \thinspace tiṭṭheyya: ``Yaṁ ime bhaṇissanti, taṁ sossāmī'ti,'' etad'eva paccayaṁ karitvā anaññaṁ, pācittiyaṁ.



\pdfbookmark[3]{Pācittiya 79}{pac79}
\subsubsection*{\hyperref[exp79]{Pācittiya 79: Kammappaṭibāhanasikkhāpadaṁ}}
\label{pac79}

\linkdest{endnote405-body}
Yo pana bhikkhu dhammikānaṁ kammānaṁ chandaṁ datvā pacchā khiyyanadhammaṁ\makeatletter\hyperlink{endnote405-appendix}\Hy@raisedlink{\hypertarget{endnote405-body}{}{\pagenote{%
		\hypertarget{endnote405-appendix}{\hyperlink{endnote405-body}{Bh Pm 1 \& 2, C, D, G, W, Dm, Um, Ra, SVibh Ce, \textit{Parivāra} Be: \textit{khīyana-}. Mi \& Mm Se: \textit{khiyyana-}. (Also at Pāc 81.) Parivāra Ce: \textit{khiyana-}. V: \textit{khiyya-}. SVibh Ee, \textit{Parivāra} Ee: \textit{khīya-}. (This reading is also at A III 269, IV 374.) Cf \textit{khiyyanaka} at Pāc 13.}}}}}\makeatother \thinspace āpajjeyya, pācittiyaṁ.



\pdfbookmark[3]{Pācittiya 80}{pac80}
\subsubsection*{\hyperref[exp80]{Pācittiya 80: Chandaṁ-adatvā-gamanasikkhāpadaṁ}}
\label{pac80}

Yo pana bhikkhu saṅghe vinicchayakathāya vattamānāya chandaṁ adatvā uṭṭhāy'āsanā pakkameyya, pācittiyaṁ.



\pdfbookmark[3]{Pācittiya 81}{pac81}
\subsubsection*{\hyperref[exp81]{Pācittiya 81: Dubbalasikkhāpadaṁ}}
\label{pac81}

\linkdest{endnote406-body}
\linkdest{endnote407-body}
Yo pana bhikkhu samaggena saṅghena cīvaraṁ datvā pacchā khiyyanadhammaṁ\makeatletter\hyperlink{endnote406-appendix}\Hy@raisedlink{\hypertarget{endnote406-body}{}{\pagenote{%
		\hypertarget{endnote406-appendix}{\hyperlink{endnote406-body}{Bh Pm 1 \& 2, C, D, G, W, Dm, Um, Ra, SVibh Ce, \textit{Parivāra} Be: \textit{khīyana-}. Mi \& Mm Se: \textit{khiyyana-}. (Also at Pāc 81.) \textit{Parivāra} Ce: \textit{khiyana-}. V: \textit{khiyya-}. SVibh Ee, \textit{Parivāra} Ee: \textit{khīya-}. (This reading is also at A III 269, IV 374.) Cf \textit{khiyyanaka} at Pāc 13.}}}}}\makeatother \thinspace āpajjeyya, ``Yathāsanthutaṁ\makeatletter\hyperlink{endnote407-appendix}\Hy@raisedlink{\hypertarget{endnote407-body}{}{\pagenote{%
		\hypertarget{endnote407-appendix}{\hyperlink{endnote407-body}{D: \textit{-santhavaṁ}. SVibh Ee: \textit{-santataṁ}. Pg, G: \textit{-santhataṁ}. V: \textit{-saṇṭhataṁ}.}}}}}\makeatother \thinspace bhikkhū saṅghikaṁ lābhaṁ pariṇāmentī'ti'', pācittiyaṁ.



\pdfbookmark[3]{Pācittiya 82}{pac82}
\subsubsection*{\hyperref[exp82]{Pācittiya 82: Pariṇāmanasikkhāpadaṁ}}
\label{pac82}

Yo pana bhikkhu jānaṁ saṅghikaṁ lābhaṁ pariṇataṁ puggalassa pariṇāmeyya, pācittiyaṁ.

\begin{center}
	Sahadhammikavaggo aṭṭhamo
\end{center}



\subsection{Rājavaggo}
% \vspace{0.2cm}

\pdfbookmark[3]{Pācittiya 83}{pac83}
\subsubsection*{\hyperref[exp83]{Pācittiya 83: Antepurasikkhāpadaṁ}}
\label{pac83}

\linkdest{endnote408-body}
\linkdest{endnote409-body}
\linkdest{endnote410-body}
Yo pana bhikkhu rañño khattiyassa muddh'ābhisittassa\makeatletter\hyperlink{endnote408-appendix}\Hy@raisedlink{\hypertarget{endnote408-body}{}{\pagenote{%
		\hypertarget{endnote408-appendix}{\hyperlink{endnote408-body}{Bh Pm 1 \& 2, D, Ra, SVibh Ce, SVibh Ee, Pg: \textit{muddhāvasitassa}. (Pg: ...\textit{muddhāni abhisitassa rañño}... \textit{muddhāni avasitto}.)}}}}}\makeatother \thinspace anikkhantarājake aniggataratanake\makeatletter\hyperlink{endnote409-appendix}\Hy@raisedlink{\hypertarget{endnote409-body}{}{\pagenote{%
		\hypertarget{endnote409-appendix}{\hyperlink{endnote409-body}{Bh Pm 1 \& 2, C, G, W, Mi Se, SVibh Ce, Ee Sp, Ce Kkh, Pg: \textit{anībhata-}. V: \textit{anibhata-}. D, Ra, UP sīhala v.l.: \textit{anīhata}. (The \textit{bh} and \textit{h} characters are very similar in Sinhala script.)}}}}}\makeatother \thinspace pubbe appaṭisaṁvidito indakhīlaṁ atikkameyya,\makeatletter\hyperlink{endnote410-appendix}\Hy@raisedlink{\hypertarget{endnote410-body}{}{\pagenote{%
		\hypertarget{endnote410-appendix}{\hyperlink{endnote410-body}{Mi \& Mm Se, G, Bh Pm 1 \& 2, C, V, W, Ra: \textit{atikkameyya}. Other eds. \textit{atikkāmeyya}.}}}}}\makeatother \thinspace pācittiyaṁ.



\pdfbookmark[3]{Pācittiya 84}{pac84}
\subsubsection*{\hyperref[exp84]{Pācittiya 84: Ratanasikkhāpadaṁ}}
\label{pac84}

\linkdest{endnote411-body}
Yo pana bhikkhu ratanaṁ vā ratanasammataṁ vā, aññatra ajjhārāmā vā ajjhāvasathā vā uggaṇheyya vā uggaṇhāpeyya vā, pācittiyaṁ. Ratanaṁ vā pana bhikkhunā ratanasammataṁ vā ajjhārāme vā ajjhāvasathe vā uggahetvā vā uggahāpetvā\makeatletter\hyperlink{endnote411-appendix}\Hy@raisedlink{\hypertarget{endnote411-body}{}{\pagenote{%
		\hypertarget{endnote411-appendix}{\hyperlink{endnote411-body}{Bh Pm 1 \& 2, Mi \& Mm Se, G, V, Ra, Pg: \textit{uggaṇhāpetvā}.}}}}}\makeatother \thinspace vā nikkhipitabbaṁ, ``Yassa bhavissati, so harissatī'ti.'' Ayaṁ tattha sāmīci.



\pdfbookmark[3]{Pācittiya 85}{pac85}
\subsubsection*{\hyperref[exp85]{Pācittiya 85: Vikālagāmappavesanasikkhāpadaṁ}}
\label{pac85}

\linkdest{endnote412-body}
\linkdest{endnote413-body}
Yo pana bhikkhu santaṁ bhikkhuṁ anāpucchā vikāle gāmaṁ paviseyya,\makeatletter\hyperlink{endnote412-appendix}\Hy@raisedlink{\hypertarget{endnote412-body}{}{\pagenote{%
		\hypertarget{endnote412-appendix}{\hyperlink{endnote412-body}{V: \textit{pavīseyya}.}}}}}\makeatother \thinspace aññatra tathārūpā accāyikā karaṇīyā,\makeatletter\hyperlink{endnote413-appendix}\Hy@raisedlink{\hypertarget{endnote413-body}{}{\pagenote{%
		\hypertarget{endnote413-appendix}{\hyperlink{endnote413-body}{V: \textit{karaṇiyā}.}}}}}\makeatother \thinspace pācittiyaṁ.



\pdfbookmark[3]{Pācittiya 86}{pac86}
\subsubsection*{\hyperref[exp86]{Pācittiya 86: Sūcigharasikkhāpadaṁ}}
\label{pac86}

\linkdest{endnote414-body}
Yo pana bhikkhu aṭṭhimayaṁ vā dantamayaṁ vā visāṇamayaṁ vā sūcigharaṁ\makeatletter\hyperlink{endnote414-appendix}\Hy@raisedlink{\hypertarget{endnote414-body}{}{\pagenote{%
		\hypertarget{endnote414-appendix}{\hyperlink{endnote414-body}{V: \textit{suci-}. Cf. Pāc. 60.}}}}}\makeatother \thinspace kārāpeyya, bhedanakaṁ pācittiyaṁ.



\pdfbookmark[3]{Pācittiya 87}{pac87}
\subsubsection*{\hyperref[exp87]{Pācittiya 87: Mañcapīṭhasikkhāpadaṁ}}
\label{pac87}

\linkdest{endnote415-body}
\linkdest{endnote416-body}
\linkdest{endnote417-body}
\linkdest{endnote418-body}
\linkdest{endnote419-body}
Navaṁ pana\makeatletter\hyperlink{endnote415-appendix}\Hy@raisedlink{\hypertarget{endnote415-body}{}{\pagenote{%
		\hypertarget{endnote415-appendix}{\hyperlink{endnote415-body}{Bh Pm 1 \& 2, Mi Se, G, V: \textit{navampana}.}}}}}\makeatother \thinspace bhikkhunā mañcaṁ vā pīṭhaṁ\makeatletter\hyperlink{endnote416-appendix}\Hy@raisedlink{\hypertarget{endnote416-body}{}{\pagenote{%
		\hypertarget{endnote416-appendix}{\hyperlink{endnote416-body}{V: \textit{pithaṁ}.}}}}}\makeatother \thinspace vā kārayamānena aṭṭh'aṅgulapādakaṁ kāretabbaṁ sugataṅgulena,\makeatletter\hyperlink{endnote417-appendix}\Hy@raisedlink{\hypertarget{endnote417-body}{}{\pagenote{%
		\hypertarget{endnote417-appendix}{\hyperlink{endnote417-body}{V: \textit{sutaṅgulena-}.}}}}}\makeatother \thinspace aññatra heṭṭhimāya\makeatletter\hyperlink{endnote418-appendix}\Hy@raisedlink{\hypertarget{endnote418-body}{}{\pagenote{%
		\hypertarget{endnote418-appendix}{\hyperlink{endnote418-body}{Mm Se: \textit{hetthimāya}.}}}}}\makeatother \thinspace aṭaniyā.\makeatletter\hyperlink{endnote419-appendix}\Hy@raisedlink{\hypertarget{endnote419-body}{}{\pagenote{%
		\hypertarget{endnote419-appendix}{\hyperlink{endnote419-body}{V: \textit{aṭṭhaniyā}.}}}}}\makeatother \thinspace Taṁ atikkāmayato, chedanakaṁ pācittiyaṁ.



\pdfbookmark[3]{Pācittiya 88}{pac88}
\subsubsection*{\hyperref[exp88]{Pācittiya 88: Tūlonaddhasikkhāpadaṁ}}
\label{pac88}

\linkdest{endnote420-body}
\linkdest{endnote421-body}
\linkdest{endnote422-body}
Yo pana bhikkhu mañcaṁ vā pīṭhaṁ\makeatletter\hyperlink{endnote420-appendix}\Hy@raisedlink{\hypertarget{endnote420-body}{}{\pagenote{%
		\hypertarget{endnote420-appendix}{\hyperlink{endnote420-body}{V: \textit{pithaṁ}.}}}}}\makeatother \thinspace vā tūl'onaddhaṁ\makeatletter\hyperlink{endnote421-appendix}\Hy@raisedlink{\hypertarget{endnote421-body}{}{\pagenote{%
		\hypertarget{endnote421-appendix}{\hyperlink{endnote421-body}{C, UP, V, Ra: \textit{tul-}.}}}}}\makeatother \thinspace kārāpeyya, uddālanakaṁ\makeatletter\hyperlink{endnote422-appendix}\Hy@raisedlink{\hypertarget{endnote422-body}{}{\pagenote{%
		\hypertarget{endnote422-appendix}{\hyperlink{endnote422-body}{Bh Pm 1 \& 2, Ra, Pg: \textit{uddāḷanakaṁ}.}}}}}\makeatother \thinspace pācittiyaṁ.



\pdfbookmark[3]{Pācittiya 89}{pac89}
\subsubsection*{\hyperref[exp89]{Pācittiya 89: Nisīdanasikkhāpadaṁ}}
\label{pac89}

\linkdest{endnote423-body}
\linkdest{endnote424-body}
\linkdest{endnote425-body}
Nisīdanaṁ pana\makeatletter\hyperlink{endnote423-appendix}\Hy@raisedlink{\hypertarget{endnote423-body}{}{\pagenote{%
		\hypertarget{endnote423-appendix}{\hyperlink{endnote423-body}{Bh Pm 1 \& 2, C, G, V, W, Mi Se, Sannē: \textit{nisīdanam-pana}.}}}}}\makeatother \thinspace bhikkhunā kārayamānena pamāṇikaṁ kāretabbaṁ. Tatr'idaṁ pamāṇaṁ,\makeatletter\hyperlink{endnote424-appendix}\Hy@raisedlink{\hypertarget{endnote424-body}{}{\pagenote{%
		\hypertarget{endnote424-appendix}{\hyperlink{endnote424-body}{V: \textit{tatrīdaṁ}.}}}}}\makeatother \thinspace dīghaso dve vidatthiyo sugatavidatthiyā, tiriyaṁ diyaḍḍhaṁ,\makeatletter\hyperlink{endnote425-appendix}\Hy@raisedlink{\hypertarget{endnote425-body}{}{\pagenote{%
		\hypertarget{endnote425-appendix}{\hyperlink{endnote425-body}{V: \textit{diyaḍhaṁ}.}}}}}\makeatother \thinspace dasā vidatthi. Taṁ atikkāmayato, chedanakaṁ pācittiyaṁ.



\pdfbookmark[3]{Pācittiya 90}{pac90}
\subsubsection*{\hyperref[exp90]{Pācittiya 90: Kaṇḍuppaṭicchādisikkhāpadaṁ}}
\label{pac90}

\linkdest{endnote426-body}
\linkdest{endnote427-body}
Kaṇḍupaṭicchādiṁ\makeatletter\hyperlink{endnote426-appendix}\Hy@raisedlink{\hypertarget{endnote426-body}{}{\pagenote{%
		\hypertarget{endnote426-appendix}{\hyperlink{endnote426-body}{Dm: kaṇḍuppaṭicchādiṁ. Bh Pm 1 \& 2, C, G: \textit{-cchādim-pana}.}}}}}\makeatother \thinspace pana bhikkhunā kārayamānena pamāṇikā kāretabbā. Tatr'idaṁ\makeatletter\hyperlink{endnote427-appendix}\Hy@raisedlink{\hypertarget{endnote427-body}{}{\pagenote{%
		\hypertarget{endnote427-appendix}{\hyperlink{endnote427-body}{V: \textit{tatrīdaṁ}.}}}}}\makeatother \thinspace pamāṇaṁ, dīghaso catasso vidatthiyo sugatavidatthiyā, tiriyaṁ dve vidatthiyo. Taṁ atikkāmayato, chedanakaṁ pācittiyaṁ.



\pdfbookmark[3]{Pācittiya 91}{pac91}
\subsubsection*{\hyperref[exp91]{Pācittiya 91: Vassikasāṭikāsikkhāpadaṁ}}
\label{pac91}

\linkdest{endnote428-body}
\linkdest{endnote429-body}
\linkdest{endnote430-body}
\linkdest{endnote431-body}
Vassikasāṭikaṁ\makeatletter\hyperlink{endnote428-appendix}\Hy@raisedlink{\hypertarget{endnote428-body}{}{\pagenote{%
		\hypertarget{endnote428-appendix}{\hyperlink{endnote428-body}{G, Mi Se v.l. (\textit{porānapotthake, marammapotthake}): \textit{-sāṭikā}. It is possible that originally this rule and the previous one read \textit{-cchādī/-cchādi} and \textit{-sāṭikā}, i.e., nominative feminines (as found in the \textit{padabhājana}). The sentence is passive and the patient is therefore in the nominative.}}}}}\makeatother \thinspace pana\makeatletter\hyperlink{endnote429-appendix}\Hy@raisedlink{\hypertarget{endnote429-body}{}{\pagenote{%
		\hypertarget{endnote429-appendix}{\hyperlink{endnote429-body}{C, Sannē: \textit{-sāṭikam-pana}.}}}}}\makeatother \thinspace bhikkhunā kārayamānena pamāṇikā kāretabbā. Tatr'idaṁ\makeatletter\hyperlink{endnote430-appendix}\Hy@raisedlink{\hypertarget{endnote430-body}{}{\pagenote{%
		\hypertarget{endnote430-appendix}{\hyperlink{endnote430-body}{V: \textit{tatrīdaṁ}.}}}}}\makeatother \thinspace pamāṇaṁ, dīghaso cha vidatthiyo sugatavidatthiyā, tiriyaṁ aḍḍhateyyā.\makeatletter\hyperlink{endnote431-appendix}\Hy@raisedlink{\hypertarget{endnote431-body}{}{\pagenote{%
		\hypertarget{endnote431-appendix}{\hyperlink{endnote431-body}{V: \textit{aḍhateyya}.}}}}}\makeatother \thinspace Taṁ atikkāmayato, chedanakaṁ pācittiyaṁ.



\pdfbookmark[3]{Pācittiya 92}{pac92}
\subsubsection*{\hyperref[exp92]{Pācittiya 92: Nandasikkhāpadaṁ}}
\label{pac92}

\linkdest{endnote432-body}
\linkdest{endnote433-body}
\linkdest{endnote434-body}
\linkdest{endnote435-body}
Yo pana bhikkhu sugatacīvarappamāṇaṁ\makeatletter\hyperlink{endnote432-appendix}\Hy@raisedlink{\hypertarget{endnote432-body}{}{\pagenote{%
		\hypertarget{endnote432-appendix}{\hyperlink{endnote432-body}{G: \textit{sugatacīvaram-pamāṇaṁ}.}}}}}\makeatother \thinspace cīvaraṁ kārāpeyya atirekaṁ vā, chedanakaṁ pācittiyaṁ. Tatr'idaṁ\makeatletter\hyperlink{endnote433-appendix}\Hy@raisedlink{\hypertarget{endnote433-body}{}{\pagenote{%
		\hypertarget{endnote433-appendix}{\hyperlink{endnote433-body}{V: \textit{tatrīdaṁ}.}}}}}\makeatother \thinspace sugatassa sugatacīvarappamāṇaṁ,\makeatletter\hyperlink{endnote434-appendix}\Hy@raisedlink{\hypertarget{endnote434-body}{}{\pagenote{%
		\hypertarget{endnote434-appendix}{\hyperlink{endnote434-body}{G: \textit{sugatacīvaram-pamāṇaṁ}.}}}}}\makeatother \thinspace dīghaso nava vidatthiyo sugatavidatthiyā, tiriyaṁ cha vidatthiyo. Idaṁ sugatassa sugatacīvarappamāṇaṁ.\makeatletter\hyperlink{endnote435-appendix}\Hy@raisedlink{\hypertarget{endnote435-body}{}{\pagenote{%
		\hypertarget{endnote435-appendix}{\hyperlink{endnote435-body}{C, W, Dm, Um, Mi Se v.l.:  pamāṇan-ti. D:  pamāṇaṁ ti. G (In a later faint correction.): sugatacīvaram-pamāṇan-ti. This
quotation mark ti here seems to be a remnant from the quotation of the rule in the Suttavibhaṅga. In the Suttavibhaṅga all rules end in ti as they were spoken by the Buddha, while the {Pātimokkha} is recited by other monks.}}}}}\makeatother \thinspace
% TODO replace endotes with final version
\linkdest{endnote436-body}
\begin{center}
	Rājavaggo\makeatletter\hyperlink{endnote436-appendix}\Hy@raisedlink{\hypertarget{endnote436-body}{}{\pagenote{%
		\hypertarget{endnote436-appendix}{\hyperlink{endnote436-body}{All editions, except SVibh Ce, have: \textit{ratanavagga}. The SVibh Ce reading has been chosen here as it is found in the \textit{Parivāra},
Vin V 27; see the section on chapter-division in the Introduction. The \textit{Sikkhāpada-uddāna} at the end of Bh Pm 1 and 2 (see below) also has \textit{rājavagga} in its summary of the Pācittiya section-titles.}}}}}\makeatother \thinspace navamo
\end{center}



\medskip

\begin{center}
	Uddiṭṭhā kho āyasmanto dvenavuti pācittiyā dhammā.

	\smallskip

	Tatth'āyasmante pucchāmi: Kacci'ttha parisuddhā?\\
	Dutiyam'pi pucchāmi: Kacci'ttha parisuddhā?\\
	Tatiyam'pi pucchāmi: Kacci'ttha parisuddhā?

	\smallskip

\linkdest{endnote437-body}
	Parisuddh'etth'āyasmanto, tasmā tuṇhī, evam'etaṁ dhārayāmi.\makeatletter\hyperlink{endnote437-appendix}\Hy@raisedlink{\hypertarget{endnote437-body}{}{\pagenote{%
		\hypertarget{endnote437-appendix}{\hyperlink{endnote437-body}{Dm, UP, Ra, Um: \textit{dhārayāmī ti}.}}}}}\makeatother \thinspace
\end{center}

\begin{outro}
	Dvenavuti pācittiyā dhammā niṭṭhitā
\end{outro}

\clearpage

