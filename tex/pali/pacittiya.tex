
\section{Pācittiyā}
\label{pc}

\begin{intro}
	Ime kho pan'āyasmanto dvenavuti pācittiyā dhammā uddesaṁ āgacchanti.
\end{intro}

\subsection{Musāvādavaggo}
% \vspace{0.2cm}

\pdfbookmark[3]{Pācittiya 1}{pac1}
\subsubsection*{\hyperref[exp1]{Pācittiya 1: Musāvādasikkhāpadaṁ}}
\label{pac1}

Sampajānamusāvāde, pācittiyaṁ.



\pdfbookmark[3]{Pācittiya 2}{pac2}
\subsubsection*{\hyperref[exp2]{Pācittiya 2: Omasavādasikkhāpadaṁ}}
\label{pac2}

Omasavāde, pācittiyaṁ.



\pdfbookmark[3]{Pācittiya 3}{pac3}
\subsubsection*{\hyperref[exp3]{Pācittiya 3: Pesuññasikkhāpadaṁ}}
\label{pac3}

Bhikkhupesuññe, pācittiyaṁ.



\pdfbookmark[3]{Pācittiya 4}{exp4}
\subsubsection*{\hyperref[exp4]{Pācittiya 4: Padasodhammasikkhāpadaṁ}}
\label{pac4}

\linkdest{endnote-body}
Yo pana bhikkhu anupasampannaṁ\makeatletter\hyperlink{endnote-appendix}\Hy@raisedlink{\hypertarget{endnote-body}{}{\pagenote{%
		\hypertarget{endnote-appendix}{\hyperlink{endnote-body}{}}}}}\makeatother V: anūpasampannaṁ. padaso dhammaṁ vāceyya, pācittiyaṁ.



\pdfbookmark[3]{Pācittiya 5}{pac5}
\subsubsection*{\hyperref[exp5]{Pācittiya 5: Paṭhamasahaseyyasikkhāpadaṁ}}
\label{pac5}

\linkdest{endnote-body}
\linkdest{endnote-body}
\linkdest{endnote-body}
Yo pana bhikkhu anupasampannena\makeatletter\hyperlink{endnote-appendix}\Hy@raisedlink{\hypertarget{endnote-body}{}{\pagenote{%
		\hypertarget{endnote-appendix}{\hyperlink{endnote-body}{}}}}}\makeatother V: anūpasampannena. uttariṁ dirattatirattaṁ\makeatletter\hyperlink{endnote-appendix}\Hy@raisedlink{\hypertarget{endnote-body}{}{\pagenote{%
		\hypertarget{endnote-appendix}{\hyperlink{endnote-body}{}}}}}\makeatother Mi & Mm Se,Vibh Ee: dvi-. Dm, Um, UP, Mi & Mm Se, Vibh Ee: uttaridirattatirattaṁ. saha seyyaṁ\makeatletter\hyperlink{endnote-appendix}\Hy@raisedlink{\hypertarget{endnote-body}{}{\pagenote{%
		\hypertarget{endnote-appendix}{\hyperlink{endnote-body}{}}}}}\makeatother Mi Se, Bh Pm 1 & 2: saha seyyaṁ. Other printed editions (Vibh Ee, Vibh Ce, UP, Mm Se): sahaseyyaṁ. kappeyya, pācittiyaṁ.



\pdfbookmark[3]{Pācittiya 6}{pac6}
\subsubsection*{\hyperref[exp6]{Pācittiya 6: Dutiyasahaseyyasikkhāpadaṁ}}
\label{pac6}

\linkdest{endnote-body}
Yo pana bhikkhu mātugāmena saha seyyaṁ\makeatletter\hyperlink{endnote-appendix}\Hy@raisedlink{\hypertarget{endnote-body}{}{\pagenote{%
		\hypertarget{endnote-appendix}{\hyperlink{endnote-body}{}}}}}\makeatother In G the correction saddhiṁ has been inserted before sahaseyyaṁ. kappeyya, pācittiyaṁ.



\pdfbookmark[3]{Pācittiya 7}{pac7}
\subsubsection*{\hyperref[exp7]{Pācittiya 7: Dhammadesanāsikkhāpadaṁ}}
\label{pac7}

\linkdest{endnote-body}
Yo pana bhikkhu mātugāmassa uttariṁ chappañcavācāhi\makeatletter\hyperlink{endnote-appendix}\Hy@raisedlink{\hypertarget{endnote-body}{}{\pagenote{%
		\hypertarget{endnote-appendix}{\hyperlink{endnote-body}{}}}}}\makeatother Dm, Um, UP, Mi & Mm Se, Vibh Ee: uttarichappañcavācāhi. Cf Pāc 5. dhammaṁ deseyya, aññatra viññunā purisaviggahena, pācittiyaṁ.



\pdfbookmark[3]{Pācittiya 8}{pac8}
\subsubsection*{\hyperref[exp8]{Pācittiya 8: Bhūtārocanasikkhāpadaṁ}}
\label{pac8}

Yo pana bhikkhu anupasampannassa uttarimanussadhammaṁ āroceyya bhūtasmiṁ, pācittiyaṁ.



\pdfbookmark[3]{Pācittiya 9}{pac9}
\subsubsection*{\hyperref[exp9]{Pācittiya 9: Duṭṭhullārocanasikkhāpadaṁ}}
\label{pac9}

\linkdest{endnote-body}
\linkdest{endnote-body}
Yo pana bhikkhu bhikkhussa duṭṭhullaṁ āpattiṁ anupasampannassa\makeatletter\hyperlink{endnote-appendix}\Hy@raisedlink{\hypertarget{endnote-body}{}{\pagenote{%
		\hypertarget{endnote-appendix}{\hyperlink{endnote-body}{}}}}}\makeatother V: anūpasampannassa. (No long ū in Pāc 8.) āroceyya, aññatra bhikkhusammutiyā,\makeatletter\hyperlink{endnote-appendix}\Hy@raisedlink{\hypertarget{endnote-body}{}{\pagenote{%
		\hypertarget{endnote-appendix}{\hyperlink{endnote-body}{}}}}}\makeatother Mi & Mm Se, BhPm 1 v.l.: sammatiyā. pācittiyaṁ.



\pdfbookmark[3]{Pācittiya 10}{pac10}
\subsubsection*{\hyperref[exp10]{Pācittiya 10: Paṭhavīkhaṇanasikkhāpadaṁ}}
\label{pac10}

\linkdest{endnote-body}
Yo pana bhikkhu paṭhaviṁ\makeatletter\hyperlink{endnote-appendix}\Hy@raisedlink{\hypertarget{endnote-body}{}{\pagenote{%
		\hypertarget{endnote-appendix}{\hyperlink{endnote-body}{}}}}}\makeatother Dm, V: pathaviṁ. khaṇeyya vā khaṇāpeyya vā, pācittiyaṁ.

\linkdest{endnote-body}
\linkdest{endnote-body}
\begin{center}
	Musāvādavaggo\makeatletter\hyperlink{endnote-appendix}\Hy@raisedlink{\hypertarget{endnote-body}{}{\pagenote{%
		\hypertarget{endnote-appendix}{\hyperlink{endnote-body}{}}}}}\makeatother Mm Se: musāvādāvagga. (Probably a misprint or a corruption as initial members of compounds normally aren't inflected.) paṭhamo\makeatletter\hyperlink{endnote-appendix}\Hy@raisedlink{\hypertarget{endnote-body}{}{\pagenote{%
		\hypertarget{endnote-appendix}{\hyperlink{endnote-body}{}}}}}\makeatother V: pathamo.
\end{center}



\subsection{Bhūtagāmavaggo}
% \vspace{0.2cm}

\pdfbookmark[3]{Pācittiya 11}{pac11}
\subsubsection*{\hyperref[exp11]{Pācittiya 11: Bhūtagāmasikkhāpadaṁ}}
\label{pac11}

\linkdest{endnote-body}
Bhūtagāmapātabyatāya,\makeatletter\hyperlink{endnote-appendix}\Hy@raisedlink{\hypertarget{endnote-body}{}{\pagenote{%
		\hypertarget{endnote-appendix}{\hyperlink{endnote-body}{}}}}}\makeatother Vibh Ce, C, W, Ra: -pātavyatāya. pācittiyaṁ.



\pdfbookmark[3]{Pācittiya 12}{pac12}
\subsubsection*{\hyperref[exp12]{Pācittiya 12: Aññavādakasikkhāpadaṁ}}
\label{pac12}

Aññavādake vihesake, pācittiyaṁ.



\pdfbookmark[3]{Pācittiya 13}{pac13}
\subsubsection*{\hyperref[exp13]{Pācittiya 13: Ujjhāpanakasikkhāpadaṁ}}
\label{pac13}

\linkdest{endnote-body}
Ujjhāpanake khiyyanake,\makeatletter\hyperlink{endnote-appendix}\Hy@raisedlink{\hypertarget{endnote-body}{}{\pagenote{%
		\hypertarget{endnote-appendix}{\hyperlink{endnote-body}{}}}}}\makeatother Bh Pm 1 & 2, Dm, UP, Mi & Mm Se, V: khiyyanake. C, D, G, W, Um, Vibh Ce, Vibh Ee, Ra, Pg: khīyanake. pācittiyaṁ.



\pdfbookmark[3]{Pācittiya 14}{pac14}
\subsubsection*{\hyperref[exp14]{Pācittiya 14: Paṭhamasen'āsanasikkhāpadaṁ}}
\label{pac14}

\linkdest{endnote-body}
\linkdest{endnote-body}
\linkdest{endnote-body}
\linkdest{endnote-body}
\linkdest{endnote-body}
\linkdest{endnote-body}
Yo pana bhikkhu saṅghikaṁ\makeatletter\hyperlink{endnote-appendix}\Hy@raisedlink{\hypertarget{endnote-body}{}{\pagenote{%
		\hypertarget{endnote-appendix}{\hyperlink{endnote-body}{}}}}}\makeatother BhPm 1, C, V, W: saṁghikaṁ. mañcaṁ vā pīṭhaṁ\makeatletter\hyperlink{endnote-appendix}\Hy@raisedlink{\hypertarget{endnote-body}{}{\pagenote{%
		\hypertarget{endnote-appendix}{\hyperlink{endnote-body}{}}}}}\makeatother V: pithaṁ. vā bhisiṁ vā kocchaṁ vā ajjhokāse santharitvā\makeatletter\hyperlink{endnote-appendix}\Hy@raisedlink{\hypertarget{endnote-body}{}{\pagenote{%
		\hypertarget{endnote-appendix}{\hyperlink{endnote-body}{}}}}}\makeatother V: saṇthar- vā santharāpetvā\makeatletter\hyperlink{endnote-appendix}\Hy@raisedlink{\hypertarget{endnote-body}{}{\pagenote{%
		\hypertarget{endnote-appendix}{\hyperlink{endnote-body}{}}}}}\makeatother V: saṇthar- vā, taṁ pakkamanto n'eva uddhareyya na uddharāpeyya,\makeatletter\hyperlink{endnote-appendix}\Hy@raisedlink{\hypertarget{endnote-body}{}{\pagenote{%
		\hypertarget{endnote-appendix}{\hyperlink{endnote-body}{}}}}}\makeatother D: n'uddharāpeyya. anāpucchaṁ\makeatletter\hyperlink{endnote-appendix}\Hy@raisedlink{\hypertarget{endnote-body}{}{\pagenote{%
		\hypertarget{endnote-appendix}{\hyperlink{endnote-body}{}}}}}\makeatother Ra, Pg, Sannē: anāpucchā. Also in the 1981 Øri Kaḷyāṇi Yogāørama Samåṭhāva Bhikkhupātimokkhapāḷi edition. vā gaccheyya, pācittiyaṁ.



\pdfbookmark[3]{Pācittiya 15}{pac15}
\subsubsection*{\hyperref[exp15]{Pācittiya 15: Dutiyasen'āsanasikkhāpadaṁ}}
\label{pac15}

\linkdest{endnote-body}
\linkdest{endnote-body}
\linkdest{endnote-body}
\linkdest{endnote-body}
Yo pana bhikkhu saṅghike vihāre seyyaṁ santharitvā\makeatletter\hyperlink{endnote-appendix}\Hy@raisedlink{\hypertarget{endnote-body}{}{\pagenote{%
		\hypertarget{endnote-appendix}{\hyperlink{endnote-body}{}}}}}\makeatother V: saṇthar- vā santharāpetvā\makeatletter\hyperlink{endnote-appendix}\Hy@raisedlink{\hypertarget{endnote-body}{}{\pagenote{%
		\hypertarget{endnote-appendix}{\hyperlink{endnote-body}{}}}}}\makeatother V: saṇthar- vā, taṁ pakkamanto n'eva uddhareyya na uddharāpeyya,\makeatletter\hyperlink{endnote-appendix}\Hy@raisedlink{\hypertarget{endnote-body}{}{\pagenote{%
		\hypertarget{endnote-appendix}{\hyperlink{endnote-body}{}}}}}\makeatother D: n'uddharāpeyya. anāpucchaṁ\makeatletter\hyperlink{endnote-appendix}\Hy@raisedlink{\hypertarget{endnote-body}{}{\pagenote{%
		\hypertarget{endnote-appendix}{\hyperlink{endnote-body}{}}}}}\makeatother Ra, Pg, Sannē: anāpucchā. Also in the 1981 Øri Kaḷyāṇi Yogāørama Samåṭhāva Bhikkhupātimokkhapāḷi edition. vā gaccheyya, pācittiyaṁ.



\pdfbookmark[3]{Pācittiya 16}{pac16}
\subsubsection*{\hyperref[exp16]{Pācittiya 16: Anupakhajjasikkhāpadaṁ}}
\label{pac16}

\linkdest{endnote-body}
\linkdest{endnote-body}
\linkdest{endnote-body}
Yo pana bhikkhu saṅghike\makeatletter\hyperlink{endnote-appendix}\Hy@raisedlink{\hypertarget{endnote-body}{}{\pagenote{%
		\hypertarget{endnote-appendix}{\hyperlink{endnote-body}{}}}}}\makeatother BhPm 1, C, V, W: saṁghikaṁ. vihāre jānaṁ pubb'upagataṁ\makeatletter\hyperlink{endnote-appendix}\Hy@raisedlink{\hypertarget{endnote-body}{}{\pagenote{%
		\hypertarget{endnote-appendix}{\hyperlink{endnote-body}{}}}}}\makeatother Mi & Mm Se, Bh Pm 1 & 2, D, W, Um, Pg: pubbūpagataṁ. (C unclear.) bhikkhuṁ anupakhajja\makeatletter\hyperlink{endnote-appendix}\Hy@raisedlink{\hypertarget{endnote-body}{}{\pagenote{%
		\hypertarget{endnote-appendix}{\hyperlink{endnote-body}{}}}}}\makeatother Mi & Mm Se, V: anūpakhajja. seyyaṁ kappeyya: ``Yassa sambādho bhavissati, so pakkamissatī'ti'', etad'eva paccayaṁ karitvā anaññaṁ, pācittiyaṁ.



\pdfbookmark[3]{Pācittiya 17}{pac17}
\subsubsection*{\hyperref[exp17]{Pācittiya 17: Nikkaḍḍhanasikkhāpadaṁ}}
\label{pac17}

\linkdest{endnote-body}
\linkdest{endnote-body}
\linkdest{endnote-body}
Yo pana bhikkhu bhikkhuṁ kupito\makeatletter\hyperlink{endnote-appendix}\Hy@raisedlink{\hypertarget{endnote-body}{}{\pagenote{%
		\hypertarget{endnote-appendix}{\hyperlink{endnote-body}{}}}}}\makeatother V: kuppito. Cf NP 25 & Pāc 74. anattamano saṅghikā vihārā nikkaḍḍheyya\makeatletter\hyperlink{endnote-appendix}\Hy@raisedlink{\hypertarget{endnote-body}{}{\pagenote{%
		\hypertarget{endnote-appendix}{\hyperlink{endnote-body}{}}}}}\makeatother V: nikaḍheyya. Cf aḍhamāso at NP 24 and Pāc 57 in V. vā nikkaḍḍhāpeyya\makeatletter\hyperlink{endnote-appendix}\Hy@raisedlink{\hypertarget{endnote-body}{}{\pagenote{%
		\hypertarget{endnote-appendix}{\hyperlink{endnote-body}{}}}}}\makeatother V: nikaḍhāpeyya. vā, pācittiyaṁ.



\pdfbookmark[3]{Pācittiya 18}{pac18}
\subsubsection*{\hyperref[exp18]{Pācittiya 18: Vehāsakuṭisikkhāpadaṁ}}
\label{pac18}

\linkdest{endnote-body}
\linkdest{endnote-body}
Yo pana bhikkhu saṅghike vihāre uparivehāsakuṭiyā āhaccapādakaṁ mañcaṁ vā pīṭhaṁ\makeatletter\hyperlink{endnote-appendix}\Hy@raisedlink{\hypertarget{endnote-body}{}{\pagenote{%
		\hypertarget{endnote-appendix}{\hyperlink{endnote-body}{}}}}}\makeatother V: pithaṁ. vā abhinisīdeyya\makeatletter\hyperlink{endnote-appendix}\Hy@raisedlink{\hypertarget{endnote-body}{}{\pagenote{%
		\hypertarget{endnote-appendix}{\hyperlink{endnote-body}{}}}}}\makeatother Bh Pm 1 & 2, C, D, W, Ra, UP sīhala v.l.: sahasā abhinisīdeyya. In G the correction sahasā has been inserted later. It is not
mentioned in the Sannē or Pg. vā abhinipajjeyya vā, pācittiyaṁ.



\pdfbookmark[3]{Pācittiya 19}{pac19}
\subsubsection*{\hyperref[exp19]{Pācittiya 19: Mahallakavihārasikkhāpadaṁ}}
\label{pac19}

\linkdest{endnote-body}
\linkdest{endnote-body}
\linkdest{endnote-body}
\linkdest{endnote-body}
\linkdest{endnote-body}
\linkdest{endnote-body}
Mahallakaṁ pana\makeatletter\hyperlink{endnote-appendix}\Hy@raisedlink{\hypertarget{endnote-body}{}{\pagenote{%
		\hypertarget{endnote-appendix}{\hyperlink{endnote-body}{}}}}}\makeatother Mi Se, G, V, W: mahallakam-pana. bhikkhunā vihāraṁ kārayamānena, yāva dvārakosā\makeatletter\hyperlink{endnote-appendix}\Hy@raisedlink{\hypertarget{endnote-body}{}{\pagenote{%
		\hypertarget{endnote-appendix}{\hyperlink{endnote-body}{}}}}}\makeatother Bh Pm 1 & 2, C, W, Ra, Pg, Vibh Ce, UP, Mi & Mm Se: aggala-. V: aggaḷaṭṭhappanāya. aggaḷaṭṭhapanāya ālokasandhiparikammāya dvatticchadanassa\makeatletter\hyperlink{endnote-appendix}\Hy@raisedlink{\hypertarget{endnote-body}{}{\pagenote{%
		\hypertarget{endnote-appendix}{\hyperlink{endnote-body}{}}}}}\makeatother Vibh Ee, Mi & Mm Se: dvi-; see NP 10. pariyāyaṁ appaharite ṭhitena\makeatletter\hyperlink{endnote-appendix}\Hy@raisedlink{\hypertarget{endnote-body}{}{\pagenote{%
		\hypertarget{endnote-appendix}{\hyperlink{endnote-body}{}}}}}\makeatother V: thitena. adhiṭṭhātabbaṁ; tato ce uttariṁ,\makeatletter\hyperlink{endnote-appendix}\Hy@raisedlink{\hypertarget{endnote-body}{}{\pagenote{%
		\hypertarget{endnote-appendix}{\hyperlink{endnote-body}{}}}}}\makeatother Dm, Um, Vibh Ee: uttari (but Be Sp & Ee Sp read uttariṁ.) appaharite'pi ṭhito,\makeatletter\hyperlink{endnote-appendix}\Hy@raisedlink{\hypertarget{endnote-body}{}{\pagenote{%
		\hypertarget{endnote-appendix}{\hyperlink{endnote-body}{}}}}}\makeatother V: thito. (D: appaharite ṭhito pi.) adhiṭṭhaheyya, pācittiyaṁ.



\pdfbookmark[3]{Pācittiya 20}{pac20}
\subsubsection*{\hyperref[exp20]{Pācittiya 20: Sappāṇakasikkhāpadaṁ}}
\label{pac20}

Yo pana bhikkhu jānaṁ sappāṇakaṁ udakaṁ tiṇaṁ vā mattikaṁ vā siñceyya vā siñcāpeyya vā, pācittiyaṁ.

\linkdest{endnote-body}
\begin{center}
	Bhūtagāmavaggo\makeatletter\hyperlink{endnote-appendix}\Hy@raisedlink{\hypertarget{endnote-body}{}{\pagenote{%
		\hypertarget{endnote-appendix}{\hyperlink{endnote-body}{}}}}}\makeatother Vibh Ce v.l.: senāsanavaggo. dutiyo
\end{center}



\subsection{Bhikkhunovādavaggo}
% \vspace{0.2cm}

\pdfbookmark[3]{Pācittiya 21}{pac21}
\subsubsection*{\hyperref[exp]{Pācittiya 21: Ovādasikkhāpadaṁ}}
\label{pac21}

Yo pana bhikkhu asammato bhikkhuniyo ovadeyya, pācittiyaṁ.



\pdfbookmark[3]{Pācittiya 22}{pac22}
\subsubsection*{\hyperref[exp22]{Pācittiya 22: Atthaṅgatasikkhāpadaṁ}}
\label{pac22}

\linkdest{endnote-body}
\linkdest{endnote-body}
Sammato'pi\makeatletter\hyperlink{endnote-appendix}\Hy@raisedlink{\hypertarget{endnote-body}{}{\pagenote{%
		\hypertarget{endnote-appendix}{\hyperlink{endnote-body}{}}}}}\makeatother Vibh Ee: ce pi. ce bhikkhu atthaṅ'gate suriye\makeatletter\hyperlink{endnote-appendix}\Hy@raisedlink{\hypertarget{endnote-body}{}{\pagenote{%
		\hypertarget{endnote-appendix}{\hyperlink{endnote-body}{}}}}}\makeatother Dm: sūriye. (= Sanskritisation; see Pecenko, Ee A-ṭ introduction p.liii. bhikkhuniyo ovadeyya, pācittiyaṁ.



\pdfbookmark[3]{Pācittiya 23}{pac23}
\subsubsection*{\hyperref[exp23]{Pācittiya 23: Bhikkhunupassayasikkhāpadaṁ}}
\label{pac23}

\linkdest{endnote-body}
\linkdest{endnote-body}
Yo pana bhikkhu bhikkhun'ūpassayaṁ\makeatletter\hyperlink{endnote-appendix}\Hy@raisedlink{\hypertarget{endnote-body}{}{\pagenote{%
		\hypertarget{endnote-appendix}{\hyperlink{endnote-body}{}}}}}\makeatother C, G, W,   Dm: bhikkhunupassayaṁ. Um: bhikkhūnūpa- upasaṅkamitvā bhikkhuniyo ovadeyya, aññatra samayā, pācittiyaṁ. Tatth'āyaṁ samayo: gilānā\makeatletter\hyperlink{endnote-appendix}\Hy@raisedlink{\hypertarget{endnote-body}{}{\pagenote{%
		\hypertarget{endnote-appendix}{\hyperlink{endnote-body}{}}}}}\makeatother V: gīlānā. hoti bhikkhunī; ayaṁ tattha samayo.



\pdfbookmark[3]{Pācittiya 24}{pac24}
\subsubsection*{\hyperref[exp24]{Pācittiya 24: Āmisasikkhāpadaṁ}}
\label{pac24}

\linkdest{endnote-body}
\linkdest{endnote-body}
Yo pana bhikkhu evaṁ vadeyya: ``Āmisahetu\makeatletter\hyperlink{endnote-appendix}\Hy@raisedlink{\hypertarget{endnote-body}{}{\pagenote{%
		\hypertarget{endnote-appendix}{\hyperlink{endnote-body}{}}}}}\makeatother V: āmissahetu. bhikkhū\makeatletter\hyperlink{endnote-appendix}\Hy@raisedlink{\hypertarget{endnote-body}{}{\pagenote{%
		\hypertarget{endnote-appendix}{\hyperlink{endnote-body}{}}}}}\makeatother Dm, Um, Vibh Ee: .”.. āmisahetu therā bhikkhū ....” bhikkhuniyo ovadantī''ti, pācittiyaṁ.



\pdfbookmark[3]{Pācittiya 25}{pac25}
\subsubsection*{\hyperref[exp25]{Pācittiya 25: Cīvaradānasikkhāpadaṁ}}
\label{pac25}

\linkdest{endnote-body}
Yo pana bhikkhu aññātikāya bhikkhuniyā cīvaraṁ dadeyya, aññatra pārivattakā,\makeatletter\hyperlink{endnote-appendix}\Hy@raisedlink{\hypertarget{endnote-body}{}{\pagenote{%
		\hypertarget{endnote-appendix}{\hyperlink{endnote-body}{}}}}}\makeatother Mi & Mm Se, Vibh Ce, UP, Ra, BhPm 1 & 2, C, D, G, V, W, Um, Pg: -vaṭṭakā. pācittiyaṁ.



\pdfbookmark[3]{Pācittiya 26}{pac26}
\subsubsection*{\hyperref[exp26]{Pācittiya 26: Cīvarasibbanasikkhāpadaṁ}}
\label{pac26}

Yo pana bhikkhu aññātikāya bhikkhuniyā cīvaraṁ sibbeyya vā sibbāpeyya vā, pācittiyaṁ.



\pdfbookmark[3]{Pācittiya 27}{pac27}
\subsubsection*{\hyperref[exp27]{Pācittiya 27: Saṁvidhānasikkhāpadaṁ}}
\label{pac27}

\linkdest{endnote-body}
\linkdest{endnote-body}
Yo pana bhikkhu bhikkhuniyā saddhiṁ saṁvidhāya ek'addhānamaggaṁ paṭipajjeyya antamaso gām'antaram'pi, aññatra samayā, pācittiyaṁ. Tatth'āyaṁ samayo: satthagamanīyo\makeatletter\hyperlink{endnote-appendix}\Hy@raisedlink{\hypertarget{endnote-body}{}{\pagenote{%
		\hypertarget{endnote-appendix}{\hyperlink{endnote-body}{}}}}}\makeatother V: -gamaniyo. hoti maggo sāsaṅkasammato\makeatletter\hyperlink{endnote-appendix}\Hy@raisedlink{\hypertarget{endnote-body}{}{\pagenote{%
		\hypertarget{endnote-appendix}{\hyperlink{endnote-body}{}}}}}\makeatother C, W: saṁka-. sappaṭibhayo; ayaṁ tattha samayo.



\pdfbookmark[3]{Pācittiya 28}{pac28}
\subsubsection*{\hyperref[exp28]{Pācittiya 28: Nāvābhiruhanasikkhāpadaṁ}}
\label{pac28}

\linkdest{endnote-body}
\linkdest{endnote-body}
\linkdest{endnote-body}
\linkdest{endnote-body}
Yo pana bhikkhu bhikkhuniyā saddhiṁ saṁvidhāya ekaṁ nāvaṁ\makeatletter\hyperlink{endnote-appendix}\Hy@raisedlink{\hypertarget{endnote-body}{}{\pagenote{%
		\hypertarget{endnote-appendix}{\hyperlink{endnote-body}{}}}}}\makeatother Mi Se, G, V, Pg, Burmese ms. v.l. in Vibh Ee, Bh Pm 2 v.l.: ekanāvaṁ. (Mm Se: ekaṁnāvaṁ.) abhirūheyya\makeatletter\hyperlink{endnote-appendix}\Hy@raisedlink{\hypertarget{endnote-body}{}{\pagenote{%
		\hypertarget{endnote-appendix}{\hyperlink{endnote-body}{}}}}}\makeatother BhPm 1 & 2, C, V, W, Dm, UP: -ruheyya. uddhaṁgāminiṁ\makeatletter\hyperlink{endnote-appendix}\Hy@raisedlink{\hypertarget{endnote-body}{}{\pagenote{%
		\hypertarget{endnote-appendix}{\hyperlink{endnote-body}{}}}}}\makeatother UP: uddhaṁ gāmaniṁ adho gāmaniṁ. Mi & Mm Se, Bh Pm 1 & 2, C, D, Ra, Pg, Vibh Ce: uddhagāmaniṁ. vā adhogāminiṁ vā, aññatra tiriyaṁtaraṇāya,\makeatletter\hyperlink{endnote-appendix}\Hy@raisedlink{\hypertarget{endnote-body}{}{\pagenote{%
		\hypertarget{endnote-appendix}{\hyperlink{endnote-body}{}}}}}\makeatother Dm, Vibh Ce, UP, Bh Pm 1 & 2, D, Ra: tiriyaṁ taraṇāya. C, W, Vibh Ee: tiriyaṁtaraṇāya, Mi & Mm Se, G, Um, V: tiriyan-
taraṇāya. pācittiyaṁ.



\pdfbookmark[3]{Pācittiya 29}{pac29}
\subsubsection*{\hyperref[exp29]{Pācittiya 29: Paripācitasikkhāpadaṁ}}
\label{pac29}

\linkdest{endnote-body}
\linkdest{endnote-body}
Yo pana bhikkhu jānaṁ bhikkhunīparipācitaṁ\makeatletter\hyperlink{endnote-appendix}\Hy@raisedlink{\hypertarget{endnote-body}{}{\pagenote{%
		\hypertarget{endnote-appendix}{\hyperlink{endnote-body}{}}}}}\makeatother  D, Dm, UP, V: bhikkhuni-. piṇḍapātaṁ bhuñjeyya, aññatra pubbe gihīsamārambhā,\makeatletter\hyperlink{endnote-appendix}\Hy@raisedlink{\hypertarget{endnote-body}{}{\pagenote{%
		\hypertarget{endnote-appendix}{\hyperlink{endnote-body}{}}}}}\makeatother D, Dm, Bh Pm 1, Vibh Ee, UP, Mi & Mm Se: gihi. C, W, Um, Pg, Ra, Vibh Ce, Ee Sp: gihī. V: gīhi- pācittiyaṁ.



\pdfbookmark[3]{Pācittiya 30}{pac30}
\subsubsection*{\hyperref[exp30]{Pācittiya 30: Rahonisajjasikkhāpadaṁ}}
\label{pac30}

Yo pana bhikkhu bhikkhuniyā saddhiṁ eko ekāya raho nisajjaṁ kappeyya, pācittiyaṁ.

\linkdest{endnote-body}
\begin{center}
	Ovādavaggo\makeatletter\hyperlink{endnote-appendix}\Hy@raisedlink{\hypertarget{endnote-body}{}{\pagenote{%
		\hypertarget{endnote-appendix}{\hyperlink{endnote-body}{}}}}}\makeatother Dm, Mm Se, UP, Vibh Ee: ovādavaggo. Bh Pm 1 & 2, C, D, G, V, W, Um, Mi Se, Vibh Ce, Ra: bhikkhunovādavaggo. tatiyo
\end{center}



\subsection{Bhojanavaggo}
% \vspace{0.2cm}

\pdfbookmark[3]{Pācittiya 31}{pac31}
\subsubsection*{\hyperref[exp31]{Pācittiya 31: Āvasathapiṇḍasikkhāpadaṁ}}
\label{pac31}

\linkdest{endnote-body}
\linkdest{endnote-body}
Agilānena\makeatletter\hyperlink{endnote-appendix}\Hy@raisedlink{\hypertarget{endnote-body}{}{\pagenote{%
		\hypertarget{endnote-appendix}{\hyperlink{endnote-body}{}}}}}\makeatother V: agīlānena. bhikkhunā eko āvasathapiṇḍo bhuñjitabbo; tato ce uttariṁ\makeatletter\hyperlink{endnote-appendix}\Hy@raisedlink{\hypertarget{endnote-body}{}{\pagenote{%
		\hypertarget{endnote-appendix}{\hyperlink{endnote-body}{}}}}}\makeatother Be & UP, Um, Vibh Ee: uttari. bhuñjeyya, pācittiyaṁ.



\pdfbookmark[3]{Pācittiya 32}{pac31}
\subsubsection*{\hyperref[exp32]{Pācittiya 32: Gaṇabhojanasikkhāpadaṁ}}
\label{pac32}

\linkdest{endnote-body}
\linkdest{endnote-body}
Gaṇabhojane, aññatra samayā, pācittiyaṁ. Tatth'āyaṁ samayo: gilānasamayo,\makeatletter\hyperlink{endnote-appendix}\Hy@raisedlink{\hypertarget{endnote-body}{}{\pagenote{%
		\hypertarget{endnote-appendix}{\hyperlink{endnote-body}{}}}}}\makeatother V: gīlāna-. cīvaradānasamayo, cīvarakārasamayo, addhānagamanasamayo, nāv'ābhirūhanasamayo,\makeatletter\hyperlink{endnote-appendix}\Hy@raisedlink{\hypertarget{endnote-body}{}{\pagenote{%
		\hypertarget{endnote-appendix}{\hyperlink{endnote-body}{}}}}}\makeatother Dm, Um, V: -ruhana-. mahāsamayo, samaṇabhattasamayo; ayaṁ tattha samayo.



\pdfbookmark[3]{Pācittiya 33}{pac33}
\subsubsection*{\hyperref[exp33]{Pācittiya 33: Paramparabhojanasikkhāpadaṁ}}
\label{pac33}

\linkdest{endnote-body}
\linkdest{endnote-body}
Paramparabhojane,\makeatletter\hyperlink{endnote-appendix}\Hy@raisedlink{\hypertarget{endnote-body}{}{\pagenote{%
		\hypertarget{endnote-appendix}{\hyperlink{endnote-body}{}}}}}\makeatother V: parappara-. Vibh Ee: paraṁpara-. aññatra samayā, pācittiyaṁ. Tatth'āyaṁ samayo: gilānasamayo,\makeatletter\hyperlink{endnote-appendix}\Hy@raisedlink{\hypertarget{endnote-body}{}{\pagenote{%
		\hypertarget{endnote-appendix}{\hyperlink{endnote-body}{}}}}}\makeatother V: gīlāna-. cīvaradānasamayo, cīvarakārasamayo; ayaṁ tattha samayo.



\pdfbookmark[3]{Pācittiya 34}{pac34}
\subsubsection*{\hyperref[exp34]{Pācittiya 34: Kāṇamātusikkhāpadaṁ}}
\label{pac34}

\linkdest{endnote-body}
\linkdest{endnote-body}
\linkdest{endnote-body}
\linkdest{endnote-body}
\linkdest{endnote-body}
\linkdest{endnote-body}
\linkdest{endnote-body}
\linkdest{endnote-body}
Bhikkhuṁ pan'eva kulaṁ upagataṁ pūvehi\makeatletter\hyperlink{endnote-appendix}\Hy@raisedlink{\hypertarget{endnote-body}{}{\pagenote{%
		\hypertarget{endnote-appendix}{\hyperlink{endnote-body}{}}}}}\makeatother V, Bh Pm 2 v.l.: puvehi. vā manthehi\makeatletter\hyperlink{endnote-appendix}\Hy@raisedlink{\hypertarget{endnote-body}{}{\pagenote{%
		\hypertarget{endnote-appendix}{\hyperlink{endnote-body}{}}}}}\makeatother V: maṇýehi. vā abhihaṭṭhuṁ pavāreyya,\makeatletter\hyperlink{endnote-appendix}\Hy@raisedlink{\hypertarget{endnote-body}{}{\pagenote{%
		\hypertarget{endnote-appendix}{\hyperlink{endnote-body}{}}}}}\makeatother Mi Se, G: abhihaṭṭhum-pavāreyya. V: abhihaṭṭham-pavāreyya. Cf NP 7. ākaṅkhamānena bhikkhunā dvattipattapūrā\makeatletter\hyperlink{endnote-appendix}\Hy@raisedlink{\hypertarget{endnote-body}{}{\pagenote{%
		\hypertarget{endnote-appendix}{\hyperlink{endnote-body}{}}}}}\makeatother Vibh Ee, Mi & Mm Se: dvi-; see NP 10. V: -purā. paṭiggahetabbā; tato ce uttariṁ\makeatletter\hyperlink{endnote-appendix}\Hy@raisedlink{\hypertarget{endnote-body}{}{\pagenote{%
		\hypertarget{endnote-appendix}{\hyperlink{endnote-body}{}}}}}\makeatother Be & UP, Um, Vibh Ee: uttari. See NP 3. paṭiggaṇheyya,\makeatletter\hyperlink{endnote-appendix}\Hy@raisedlink{\hypertarget{endnote-body}{}{\pagenote{%
		\hypertarget{endnote-appendix}{\hyperlink{endnote-body}{}}}}}\makeatother  C, D, W: patigaṇheyya. (Cf. NP 5, NP 10.) pācittiyaṁ. Dvattipattapūre\makeatletter\hyperlink{endnote-appendix}\Hy@raisedlink{\hypertarget{endnote-body}{}{\pagenote{%
		\hypertarget{endnote-appendix}{\hyperlink{endnote-body}{}}}}}\makeatother Vibh Ee, Mi & Mm Se: dvi-; see NP 10. V: -pure. paṭiggahetvā, tato nīharitvā, bhikkhūhi saddhiṁ saṁvibhajitabbaṁ.\makeatletter\hyperlink{endnote-appendix}\Hy@raisedlink{\hypertarget{endnote-body}{}{\pagenote{%
		\hypertarget{endnote-appendix}{\hyperlink{endnote-body}{}}}}}\makeatother V, Bh Pm 2 v.l.: saṁvibhajjitabbaṁ. Ayaṁ tattha sāmīci.



\pdfbookmark[3]{Pācittiya 35}{pac35}
\subsubsection*{\hyperref[exp35]{Pācittiya 35: Paṭhamapavāraṇāsikkhāpadaṁ}}
\label{pac35}

\linkdest{endnote-body}
Yo pana bhikkhu bhuttāvī pavārito anatirittaṁ khādanīyaṁ vā bhojanīyaṁ\makeatletter\hyperlink{endnote-appendix}\Hy@raisedlink{\hypertarget{endnote-body}{}{\pagenote{%
		\hypertarget{endnote-appendix}{\hyperlink{endnote-body}{}}}}}\makeatother C, D, G, V, W, Vibh Ee, Um: khādaniyaṁ & bhojaniyaṁ throughout the text. vā khādeyya vā bhuñjeyya vā, pācittiyaṁ.



\pdfbookmark[3]{Pācittiya 36}{pac36}
\subsubsection*{\hyperref[exp36]{Pācittiya 36: Dutiyapavāraṇāsikkhāpadaṁ}}
\label{pac36}

\linkdest{endnote-body}
\linkdest{endnote-body}
\linkdest{endnote-body}
\linkdest{endnote-body}
Yo pana bhikkhu bhikkhuṁ bhuttāviṁ pavāritaṁ anatirittena khādanīyena vā bhojanīyena\makeatletter\hyperlink{endnote-appendix}\Hy@raisedlink{\hypertarget{endnote-body}{}{\pagenote{%
		\hypertarget{endnote-appendix}{\hyperlink{endnote-body}{}}}}}\makeatother C, D, G, V, W, Vibh Ee, Um: khādaniyena & bhojaniyena. vā abhihaṭṭhuṁ pavāreyya,\makeatletter\hyperlink{endnote-appendix}\Hy@raisedlink{\hypertarget{endnote-body}{}{\pagenote{%
		\hypertarget{endnote-appendix}{\hyperlink{endnote-body}{}}}}}\makeatother Mi Se, G: abhihaṭṭhum-pavāreyya. V: abhihaṭṭham-pavāreyya. Cf NP 7 and Pāc 34. ``Handa bhikkhu khāda vā bhuñja vā'ti,'' jānaṁ\makeatletter\hyperlink{endnote-appendix}\Hy@raisedlink{\hypertarget{endnote-body}{}{\pagenote{%
		\hypertarget{endnote-appendix}{\hyperlink{endnote-body}{}}}}}\makeatother Um omits jānaṁ.. āsādan'āpekkho,\makeatletter\hyperlink{endnote-appendix}\Hy@raisedlink{\hypertarget{endnote-body}{}{\pagenote{%
		\hypertarget{endnote-appendix}{\hyperlink{endnote-body}{}}}}}\makeatother Bh Pm 1 & 2, C, D, W, Ra, Ce Kkh: -āpekho. (Cf -āpekho v.l. at Nid and Pāc 56, 60.) bhuttasmiṁ, pācittiyaṁ.



\pdfbookmark[3]{Pācittiya 37}{pac37}
\subsubsection*{\hyperref[exp37]{Pācittiya 37: Vikālabhojanasikkhāpadaṁ}}
\label{pac37}

\linkdest{endnote-body}
Yo pana bhikkhu vikāle khādanīyaṁ vā bhojanīyaṁ\makeatletter\hyperlink{endnote-appendix}\Hy@raisedlink{\hypertarget{endnote-body}{}{\pagenote{%
		\hypertarget{endnote-appendix}{\hyperlink{endnote-body}{}}}}}\makeatother C, D, G, V, W, Vibh Ee, Um: khādaniyaṁ & bhojaniyaṁ. vā khādeyya vā bhuñjeyya vā, pācittiyaṁ.



\pdfbookmark[3]{Pācittiya 38}{pac38}
\subsubsection*{\hyperref[exp38]{Pācittiya 38: Sannidhikārakasikkhāpadaṁ}}
\label{pac38}

\linkdest{endnote-body}
Yo pana bhikkhu sannidhikārakaṁ khādanīyaṁ vā bhojanīyaṁ\makeatletter\hyperlink{endnote-appendix}\Hy@raisedlink{\hypertarget{endnote-body}{}{\pagenote{%
		\hypertarget{endnote-appendix}{\hyperlink{endnote-body}{}}}}}\makeatother C, D, G, V, W, Vibh Ee, Um: khādaniyaṁ & bhojaniyaṁ. vā khādeyya vā bhuñjeyya vā, pācittiyaṁ.



\pdfbookmark[3]{Pācittiya 39}{pac39}
\subsubsection*{\hyperref[exp39]{Pācittiya 39: Paṇītabhojanasikkhāpadaṁ}}
\label{pac39}

\linkdest{endnote-body}
\linkdest{endnote-body}
\linkdest{endnote-body}
\linkdest{endnote-body}
\linkdest{endnote-body}
\linkdest{endnote-body}
Yāni kho pana tāni paṇītabhojanāni, seyyath'īdaṁ:\makeatletter\hyperlink{endnote-appendix}\Hy@raisedlink{\hypertarget{endnote-body}{}{\pagenote{%
		\hypertarget{endnote-appendix}{\hyperlink{endnote-body}{}}}}}\makeatother Dm, UP: seyyathidaṁ. Cf NP 23. sappi, navanītaṁ,\makeatletter\hyperlink{endnote-appendix}\Hy@raisedlink{\hypertarget{endnote-body}{}{\pagenote{%
		\hypertarget{endnote-appendix}{\hyperlink{endnote-body}{}}}}}\makeatother V: navanitaṁ. Cf NP 23. telaṁ, madhuphāṇitaṁ,\makeatletter\hyperlink{endnote-appendix}\Hy@raisedlink{\hypertarget{endnote-body}{}{\pagenote{%
		\hypertarget{endnote-appendix}{\hyperlink{endnote-body}{}}}}}\makeatother C, D, W: madhupphāṇitaṁ. maccho, maṁsaṁ, khīraṁ, dadhi;\makeatletter\hyperlink{endnote-appendix}\Hy@raisedlink{\hypertarget{endnote-body}{}{\pagenote{%
		\hypertarget{endnote-appendix}{\hyperlink{endnote-body}{}}}}}\makeatother C, P (, Wae Uda Pm, Sirimalwatta Pm): dadhiṁ. (This reading has later been scribbled through in C.) Both dadhi and dadhiṁ
are neuter nominative according to CPED, although normally dadhiṁ is accusative. Cf J-a IV 140: ”khīraṁ viya dadhiṁ viya
obhāsantaṁ.” yo pana bhikkhu evarūpāni paṇītabhojanāni agilāno\makeatletter\hyperlink{endnote-appendix}\Hy@raisedlink{\hypertarget{endnote-body}{}{\pagenote{%
		\hypertarget{endnote-appendix}{\hyperlink{endnote-body}{}}}}}\makeatother V: gīlāno. attano atthāya viññāpetvā bhuñjeyya,\makeatletter\hyperlink{endnote-appendix}\Hy@raisedlink{\hypertarget{endnote-body}{}{\pagenote{%
		\hypertarget{endnote-appendix}{\hyperlink{endnote-body}{}}}}}\makeatother C, D, W: paribhuñjeyya. pācittiyaṁ.



\pdfbookmark[3]{Pācittiya 40}{pac40}
\subsubsection*{\hyperref[exp40]{Pācittiya 40: Dantaponasikkhāpadaṁ}}
\label{pac40}

\linkdest{endnote-body}
\linkdest{endnote-body}
Yo pana bhikkhu adinnaṁ mukhadvāraṁ āhāraṁ\makeatletter\hyperlink{endnote-appendix}\Hy@raisedlink{\hypertarget{endnote-body}{}{\pagenote{%
		\hypertarget{endnote-appendix}{\hyperlink{endnote-body}{}}}}}\makeatother V: adinnaṁ mukhadvāraṁ āhareyya.. āhareyya, aññatra udakadantapoṇā,\makeatletter\hyperlink{endnote-appendix}\Hy@raisedlink{\hypertarget{endnote-body}{}{\pagenote{%
		\hypertarget{endnote-appendix}{\hyperlink{endnote-body}{}}}}}\makeatother Bh Pm 1 & 2, Ra, Dm, Vibh Ce, Pg: -dantaponā. pācittiyaṁ.

\begin{center}
	Bhojanavaggo catuttho
\end{center}



\subsection{Acelakavaggo}
% \vspace{0.2cm}

\pdfbookmark[3]{Pācittiya 41}{pac41}
\subsubsection*{\hyperref[exp41]{Pācittiya 41: Acelakasikkhāpadaṁ}}
\label{pac41}

\linkdest{endnote-body}
\linkdest{endnote-body}
Yo pana bhikkhu acelakassa\makeatletter\hyperlink{endnote-appendix}\Hy@raisedlink{\hypertarget{endnote-body}{}{\pagenote{%
		\hypertarget{endnote-appendix}{\hyperlink{endnote-body}{}}}}}\makeatother C: aceḷak-. vā paribbājakassa vā paribbājikāya vā sahatthā khādanīyaṁ vā bhojanīyaṁ\makeatletter\hyperlink{endnote-appendix}\Hy@raisedlink{\hypertarget{endnote-body}{}{\pagenote{%
		\hypertarget{endnote-appendix}{\hyperlink{endnote-body}{}}}}}\makeatother C, D, G, V, W, Vibh Ee, Um: khādaniyaṁ & bhojaniyaṁ. vā dadeyya, pācittiyaṁ.



\pdfbookmark[3]{Pācittiya 42}{pac42}
\subsubsection*{\hyperref[exp42]{Pācittiya 42: Uyyojanasikkhāpadaṁ}}
\label{pac42}

\linkdest{endnote-body}
\linkdest{endnote-body}
\linkdest{endnote-body}
Yo pana bhikkhu bhikkhuṁ evaṁ vadeyya,\makeatletter\hyperlink{endnote-appendix}\Hy@raisedlink{\hypertarget{endnote-body}{}{\pagenote{%
		\hypertarget{endnote-appendix}{\hyperlink{endnote-body}{}}}}}\makeatother (= Mi & Mm Se, G, D, Bh Pm 1 & 2, V, Ra.) C, W, Dm, Um, UP, Vibh Ee, Vibh Ce: ... bhikkhuṁ ehāvuso ..., i.e., no “evaṁ
vadeyya.” (Pg and Sannē also do not to have it.) ``Eh'āvuso, gāmaṁ vā nigamaṁ vā piṇḍāya pavisissāmā'ti,''\makeatletter\hyperlink{endnote-appendix}\Hy@raisedlink{\hypertarget{endnote-body}{}{\pagenote{%
		\hypertarget{endnote-appendix}{\hyperlink{endnote-body}{}}}}}\makeatother V: pavīsissāmā. tassa dāpetvā vā adāpetvā vā uyyojeyya, ``Gacch'āvuso! Na me tayā saddhiṁ kathā vā nisajjā vā phāsu hoti; ekakassa\makeatletter\hyperlink{endnote-appendix}\Hy@raisedlink{\hypertarget{endnote-body}{}{\pagenote{%
		\hypertarget{endnote-appendix}{\hyperlink{endnote-body}{}}}}}\makeatother V: ekatassa. me kathā vā nisajjā vā phāsu hotī'ti;'' etad'eva paccayaṁ karitvā anaññaṁ, pācittiyaṁ.



\pdfbookmark[3]{Pācittiya 43}{pac43}
\subsubsection*{\hyperref[exp43]{Pācittiya 43: Sabhojanasikkhāpadaṁ}}
\label{pac43}

\linkdest{endnote-body}
Yo pana bhikkhu sabhojane kule anupakhajja\makeatletter\hyperlink{endnote-appendix}\Hy@raisedlink{\hypertarget{endnote-body}{}{\pagenote{%
		\hypertarget{endnote-appendix}{\hyperlink{endnote-body}{}}}}}\makeatother Dm, Mi & Mm Se: anūpakhajja. nisajjaṁ kappeyya, pācittiyaṁ.



\pdfbookmark[3]{Pācittiya 44}{pac44}
\subsubsection*{\hyperref[exp44]{Pācittiya 44: Rahopaṭicchannasikkhāpadaṁ}}
\label{pac44}

Yo pana bhikkhu mātugāmena saddhiṁ raho paṭicchanne āsane nisajjaṁ kappeyya, pācittiyaṁ.



\pdfbookmark[3]{Pācittiya 45}{pac45}
\subsubsection*{\hyperref[exp45]{Pācittiya 45: Rahonisajjasikkhāpadaṁ}}
\label{pac45}

Yo pana bhikkhu mātugāmena saddhiṁ eko ekāya raho nisajjaṁ kappeyya, pācittiyaṁ.



\pdfbookmark[3]{Pācittiya 46}{pac46}
\subsubsection*{\hyperref[exp46]{Pācittiya 46: Cārittasikkhāpadaṁ}}
\label{pac46}

\linkdest{endnote-body}
Yo pana bhikkhu nimantito sabhatto samāno santaṁ bhikkhuṁ anāpucchā purebhattaṁ vā pacchābhattaṁ vā kulesu cārittaṁ\makeatletter\hyperlink{endnote-appendix}\Hy@raisedlink{\hypertarget{endnote-body}{}{\pagenote{%
		\hypertarget{endnote-appendix}{\hyperlink{endnote-body}{}}}}}\makeatother V: carittaṁ. āpajjeyya aññatra samayā, pācittiyaṁ. Tatth'āyaṁ samayo: cīvaradānasamayo, cīvarakārasamayo; ayaṁ tattha samayo.



\pdfbookmark[3]{Pācittiya 47}{pac47}
\subsubsection*{\hyperref[exp47]{Pācittiya 47: Mahānāmasikkhāpadaṁ}}
\label{pac47}

\linkdest{endnote-body}
\linkdest{endnote-body}
\linkdest{endnote-body}
Agilānena\makeatletter\hyperlink{endnote-appendix}\Hy@raisedlink{\hypertarget{endnote-body}{}{\pagenote{%
		\hypertarget{endnote-appendix}{\hyperlink{endnote-body}{}}}}}\makeatother  V: agīlānena. bhikkhunā cātumāsapaccayapavāraṇā\makeatletter\hyperlink{endnote-appendix}\Hy@raisedlink{\hypertarget{endnote-body}{}{\pagenote{%
		\hypertarget{endnote-appendix}{\hyperlink{endnote-body}{}}}}}\makeatother C, G, W, UP, Dm, Vibh Ce: cātumāsappaccaya-. D, Mi & Mm Se, Bh Pm 1 & 2, Um, V, Vibh Ee, Pg: cātumāsapaccaya-. sāditabbā; aññatra punapavāraṇāya, aññatra niccapavāraṇāya; tato ce uttariṁ\makeatletter\hyperlink{endnote-appendix}\Hy@raisedlink{\hypertarget{endnote-body}{}{\pagenote{%
		\hypertarget{endnote-appendix}{\hyperlink{endnote-body}{}}}}}\makeatother Be & UP, Um, Vibh Ee: uttari. See NP 3. sādiyeyya, pācittiyaṁ.



\pdfbookmark[3]{Pācittiya 48}{pac48}
\subsubsection*{\hyperref[exp48]{Pācittiya 48: Uyyuttasenāsikkhāpadaṁ}}
\label{pac48}

\linkdest{endnote-body}
\linkdest{endnote-body}
Yo pana bhikkhu uyyuttaṁ\makeatletter\hyperlink{endnote-appendix}\Hy@raisedlink{\hypertarget{endnote-body}{}{\pagenote{%
		\hypertarget{endnote-appendix}{\hyperlink{endnote-body}{}}}}}\makeatother G: uyyutaṁ.  senaṁ dassanāya gaccheyya; aññatra tathārūpapaccayā,\makeatletter\hyperlink{endnote-appendix}\Hy@raisedlink{\hypertarget{endnote-body}{}{\pagenote{%
		\hypertarget{endnote-appendix}{\hyperlink{endnote-body}{}}}}}\makeatother C, D, V, W, Dm, Vibh Ce, UP, Bh Pm 1 & 2, Ra, Pg: -rūpappaccaya. (In G later corrected from -p- to -pp-.) See note to
-magga(p)paṭipannassa at NP 16 and cātumāsapaccaya- at Pāc 47. pācittiyaṁ.




\pdfbookmark[3]{Pācittiya 49}{pac49}
\subsubsection*{\hyperref[exp49]{Pācittiya 49: Senāvāsasikkhāpadaṁ}}
\label{pac49}

\linkdest{endnote-body}
\linkdest{endnote-body}
\linkdest{endnote-body}
Siyā ca tassa bhikkhuno koci'd'eva paccayo senaṁ gamanāya,\makeatletter\hyperlink{endnote-appendix}\Hy@raisedlink{\hypertarget{endnote-body}{}{\pagenote{%
		\hypertarget{endnote-appendix}{\hyperlink{endnote-body}{}}}}}\makeatother C, W: senaṅgamanāya. dirattatirattaṁ\makeatletter\hyperlink{endnote-appendix}\Hy@raisedlink{\hypertarget{endnote-body}{}{\pagenote{%
		\hypertarget{endnote-appendix}{\hyperlink{endnote-body}{}}}}}\makeatother Mi & Mm Se,Vibh Ee: dvi-. tena bhikkhunā senāya vasitabbaṁ; tato ce uttariṁ\makeatletter\hyperlink{endnote-appendix}\Hy@raisedlink{\hypertarget{endnote-body}{}{\pagenote{%
		\hypertarget{endnote-appendix}{\hyperlink{endnote-body}{}}}}}\makeatother Be & UP, Um, Vibh Ee: uttari. See NP 3. vaseyya, pācittiyaṁ.



\pdfbookmark[3]{Pācittiya 50}{pac50}
\subsubsection*{\hyperref[exp50]{Pācittiya 50: Uyyodhikasikkhāpadaṁ}}
\label{pac50}

\linkdest{endnote-body}
\linkdest{endnote-body}
Dirattatirattañ'ce\makeatletter\hyperlink{endnote-appendix}\Hy@raisedlink{\hypertarget{endnote-body}{}{\pagenote{%
		\hypertarget{endnote-appendix}{\hyperlink{endnote-body}{}}}}}\makeatother Mi & Mm Se,Vibh Ee: dvi-. bhikkhu senāya vasamāno, uyyodhikaṁ vā bal'aggaṁ vā senābyūhaṁ\makeatletter\hyperlink{endnote-appendix}\Hy@raisedlink{\hypertarget{endnote-body}{}{\pagenote{%
		\hypertarget{endnote-appendix}{\hyperlink{endnote-body}{}}}}}\makeatother Ce Kkh: -vyūhaṁ. G, Um, UP, V, Ra, Pg: -byuhaṁ. vā anīkadassanaṁ vā gaccheyya, pācittiyaṁ.

\linkdest{endnote-body}
\begin{center}
	Acelakavaggo\makeatletter\hyperlink{endnote-appendix}\Hy@raisedlink{\hypertarget{endnote-body}{}{\pagenote{%
		\hypertarget{endnote-appendix}{\hyperlink{endnote-body}{}}}}}\makeatother C: aceḷaka-. pañcamo
\end{center}



\subsection{Surāpānavaggo}
% \vspace{0.2cm}

\pdfbookmark[3]{Pācittiya 51}{pac51}
\subsubsection*{\hyperref[exp51]{Pācittiya 51: Surāpānasikkhāpadaṁ}}
\label{pac51}

Surāmerayapāne pācittiyaṁ.



\pdfbookmark[3]{Pācittiya 52}{pac52}
\subsubsection*{\hyperref[exp52]{Pācittiya 52: Aṅgulipatodakasikkhāpadaṁ}}
\label{pac52}

Aṅgulipatodake pācittiyaṁ.



\pdfbookmark[3]{Pācittiya 53}{pac53}
\subsubsection*{\hyperref[exp53]{Pācittiya 53: Hassadhammasikkhāpadaṁ}}
\label{pac53}

\linkdest{endnote-body}
Udake hassadhamme\makeatletter\hyperlink{endnote-appendix}\Hy@raisedlink{\hypertarget{endnote-body}{}{\pagenote{%
		\hypertarget{endnote-appendix}{\hyperlink{endnote-body}{}}}}}\makeatother Dm, Um: hasa-. Mi Se, G, V, Vibh Ee: hāsa-. C, D, W, UP, Ee Sp, Mm Se, Mi Se, Vibh Ce v.l.: hassa-. Vibh Ee gives all three
readings as Burmese ms. v.l.l. pācittiyaṁ.



\pdfbookmark[3]{Pācittiya 54}{pac54}
\subsubsection*{\hyperref[exp54]{Pācittiya 54: Anādariyasikkhāpadaṁ}}
\label{pac54}

Anādariye pācittiyaṁ.



\pdfbookmark[3]{Pācittiya 55}{pac55}
\subsubsection*{\hyperref[exp55]{Pācittiya 55: Bhiṁsāpanasikkhāpadaṁ}}
\label{pac55}

Yo pana bhikkhu bhikkhuṁ bhiṁsāpeyya, pācittiyaṁ.



\pdfbookmark[3]{Pācittiya 56}{pac56}
\subsubsection*{\hyperref[exp56]{Pācittiya 56: Jotikasikkhāpadaṁ}}
\label{pac56}

\linkdest{endnote-body}
\linkdest{endnote-body}
\linkdest{endnote-body}
Yo pana bhikkhu agilāno\makeatletter\hyperlink{endnote-appendix}\Hy@raisedlink{\hypertarget{endnote-body}{}{\pagenote{%
		\hypertarget{endnote-appendix}{\hyperlink{endnote-body}{}}}}}\makeatother V: agīlāno. visibban'āpekkho\makeatletter\hyperlink{endnote-appendix}\Hy@raisedlink{\hypertarget{endnote-body}{}{\pagenote{%
		\hypertarget{endnote-appendix}{\hyperlink{endnote-body}{}}}}}\makeatother Mm Se, Pg: visīvanāpekkho. Bh Pm 1 & 2, C, D, W, Sannē: visīvanāpekho. Vibh Ce, Um, Ra: visibbanāpekho. (Cf v.l. at Pāc 36). jotiṁ samādaheyya vā samādahāpeyya vā, aññatra tathārūpapaccayā,\makeatletter\hyperlink{endnote-appendix}\Hy@raisedlink{\hypertarget{endnote-body}{}{\pagenote{%
		\hypertarget{endnote-appendix}{\hyperlink{endnote-body}{}}}}}\makeatother Bh Pm 1 & 2, C, D, Dm, Um, UP, V, Vibh Ce: -rūpappaccayā; see Pāc 48. pācittiyaṁ.



\pdfbookmark[3]{Pācittiya 57}{pac57}
\subsubsection*{\hyperref[exp57]{Pācittiya 57: Nahānasikkhāpadaṁ}}
\label{pac57}

\linkdest{endnote-body}
\linkdest{endnote-body}
\linkdest{endnote-body}
\linkdest{endnote-body}
\linkdest{endnote-body}
\linkdest{endnote-body}
\linkdest{endnote-body}
\linkdest{endnote-body}
Yo pana bhikkhu oren'aḍḍhamāsaṁ\makeatletter\hyperlink{endnote-appendix}\Hy@raisedlink{\hypertarget{endnote-body}{}{\pagenote{%
		\hypertarget{endnote-appendix}{\hyperlink{endnote-body}{}}}}}\makeatother Mi & Mm Se, G, V: aḍḍha-. nahāyeyya,\makeatletter\hyperlink{endnote-appendix}\Hy@raisedlink{\hypertarget{endnote-body}{}{\pagenote{%
		\hypertarget{endnote-appendix}{\hyperlink{endnote-body}{}}}}}\makeatother Vibh Ee, Mi & Mm Se: nhāyeyya. V: ṇhāyeyya. aññatra samayā, pācittiyaṁ. Tatth'āyaṁ samayo: diyaḍḍho\makeatletter\hyperlink{endnote-appendix}\Hy@raisedlink{\hypertarget{endnote-body}{}{\pagenote{%
		\hypertarget{endnote-appendix}{\hyperlink{endnote-body}{}}}}}\makeatother V: diyaḍho. māso seso gimhānan'ti,\makeatletter\hyperlink{endnote-appendix}\Hy@raisedlink{\hypertarget{endnote-body}{}{\pagenote{%
		\hypertarget{endnote-appendix}{\hyperlink{endnote-body}{}}}}}\makeatother Vibh Be v.l.: gimhānaṁ. vassānassa paṭhamo\makeatletter\hyperlink{endnote-appendix}\Hy@raisedlink{\hypertarget{endnote-body}{}{\pagenote{%
		\hypertarget{endnote-appendix}{\hyperlink{endnote-body}{}}}}}\makeatother V: pathamo. māso, icc'ete aḍḍhateyyamāsā,\makeatletter\hyperlink{endnote-appendix}\Hy@raisedlink{\hypertarget{endnote-body}{}{\pagenote{%
		\hypertarget{endnote-appendix}{\hyperlink{endnote-body}{}}}}}\makeatother V: aḍha- uṇhasamayo, pariḷāhasamayo,\makeatletter\hyperlink{endnote-appendix}\Hy@raisedlink{\hypertarget{endnote-body}{}{\pagenote{%
		\hypertarget{endnote-appendix}{\hyperlink{endnote-body}{}}}}}\makeatother C: parilāha-. gilānasamayo,\makeatletter\hyperlink{endnote-appendix}\Hy@raisedlink{\hypertarget{endnote-body}{}{\pagenote{%
		\hypertarget{endnote-appendix}{\hyperlink{endnote-body}{}}}}}\makeatother V: gīlāna- kammasamayo, addhānagamanasamayo, vātavuṭṭhisamayo; ayaṁ tattha samayo.



\pdfbookmark[3]{Pācittiya 58}{pac58}
\subsubsection*{\hyperref[exp58]{Pācittiya 58: Dubbaṇṇakaraṇasikkhāpadaṁ}}
\label{pac58}

\linkdest{endnote-body}
\linkdest{endnote-body}
Navaṁ pana\makeatletter\hyperlink{endnote-appendix}\Hy@raisedlink{\hypertarget{endnote-body}{}{\pagenote{%
		\hypertarget{endnote-appendix}{\hyperlink{endnote-body}{}}}}}\makeatother Mi Se, G, V, P: navam-pana. bhikkhunā cīvaralābhena tiṇṇaṁ dubbaṇṇakaraṇānaṁ aññataraṁ dubbaṇṇakaraṇaṁ ādātabbaṁ, nīlaṁ vā kaddamaṁ\makeatletter\hyperlink{endnote-appendix}\Hy@raisedlink{\hypertarget{endnote-body}{}{\pagenote{%
		\hypertarget{endnote-appendix}{\hyperlink{endnote-body}{}}}}}\makeatother V: kaddumaṁ. vā kāḷasāmaṁ vā. Anādā ce bhikkhu tiṇṇaṁ dubbaṇṇakaraṇānaṁ aññataraṁ dubbaṇṇakaraṇaṁ navaṁ cīvaraṁ paribhuñjeyya, pācittiyaṁ.



\pdfbookmark[3]{Pācittiya 59}{pac59}
\subsubsection*{\hyperref[exp59]{Pācittiya 59: Vikappanasikkhāpadaṁ}}
\label{pac59}

\linkdest{endnote-body}
\linkdest{endnote-body}
\linkdest{endnote-body}
Yo pana bhikkhu bhikkhussa vā bhikkhuniyā\makeatletter\hyperlink{endnote-appendix}\Hy@raisedlink{\hypertarget{endnote-body}{}{\pagenote{%
		\hypertarget{endnote-appendix}{\hyperlink{endnote-body}{}}}}}\makeatother V: bhikkhunīyā. vā sikkhamānāya vā sāmaṇerassa vā sāmaṇeriyā\makeatletter\hyperlink{endnote-appendix}\Hy@raisedlink{\hypertarget{endnote-body}{}{\pagenote{%
		\hypertarget{endnote-appendix}{\hyperlink{endnote-body}{}}}}}\makeatother Bh Pm 1 & 2: sāmaṇerāya. The editor of the Sannē gives this as a v.l. of “some (MSS).” Mm Se: samaṇerassa ... samaṇeriyā. vā sāmaṁ cīvaraṁ vikappetvā apaccuddhārakaṁ\makeatletter\hyperlink{endnote-appendix}\Hy@raisedlink{\hypertarget{endnote-body}{}{\pagenote{%
		\hypertarget{endnote-appendix}{\hyperlink{endnote-body}{}}}}}\makeatother Dm, Um, UP: appaccuddhāraṇaṁ. Pg (p. 57): apaccuddhārakaṁ. paribhuñjeyya, pācittiyaṁ.



\pdfbookmark[3]{Pācittiya 60}{pac60}
\subsubsection*{\hyperref[exp60]{Pācittiya 60: Apanidhānasikkhāpadaṁ}}
\label{pac60}

\linkdest{endnote-body}
\linkdest{endnote-body}
Yo pana bhikkhu bhikkhussa pattaṁ vā cīvaraṁ vā nisīdanaṁ vā sūcigharaṁ\makeatletter\hyperlink{endnote-appendix}\Hy@raisedlink{\hypertarget{endnote-body}{}{\pagenote{%
		\hypertarget{endnote-appendix}{\hyperlink{endnote-body}{}}}}}\makeatother D, V, Ra: suci-. Cf. Pāc 86. vā kāyabandhanaṁ vā apanidheyya vā apanidhāpeyya vā antamaso hass'āpekkho'pi\makeatletter\hyperlink{endnote-appendix}\Hy@raisedlink{\hypertarget{endnote-body}{}{\pagenote{%
		\hypertarget{endnote-appendix}{\hyperlink{endnote-body}{}}}}}\makeatother Dm, Um, V:  hasāpekkho; Vibh Ce, Mi & Mm Se, G, Pg:  hassāpekkho. Vibh Ee:  hāsāpekkho. Bh Pm 1 & 2, C, D, W, Ra:  hassāpekho. Bh Pm 2 v.l. hasāpekho. (Cf Nid: visuddhāpekho.) Vibh Ee gives all three as Burmese MS. v.l.l. pācittiyaṁ.

\linkdest{endnote-body}
\begin{center}
	Surāpānavaggo\makeatletter\hyperlink{endnote-appendix}\Hy@raisedlink{\hypertarget{endnote-body}{}{\pagenote{%
		\hypertarget{endnote-appendix}{\hyperlink{endnote-body}{}}}}}\makeatother V: -pāṇa-. chaṭṭho
\end{center}



\subsection{Sappāṇavaggo}
% \vspace{0.2cm}

\pdfbookmark[3]{Pācittiya 61}{pac61}
\subsubsection*{\hyperref[exp61]{Pācittiya 61: Sañciccasikkhāpadaṁ}}
\label{pac61}

\linkdest{endnote-body}
Yo pana bhikkhu sañcicca pāṇaṁ\makeatletter\hyperlink{endnote-appendix}\Hy@raisedlink{\hypertarget{endnote-body}{}{\pagenote{%
		\hypertarget{endnote-appendix}{\hyperlink{endnote-body}{}}}}}\makeatother C, W: pānaṁ. jīvitā voropeyya, pācittiyaṁ.



\pdfbookmark[3]{Pācittiya 62}{pac62}
\subsubsection*{\hyperref[exp62]{Pācittiya 62: Sappāṇakasikkhāpadaṁ}}
\label{pac62}

\linkdest{endnote-body}
Yo pana bhikkhu jānaṁ sappāṇakaṁ\makeatletter\hyperlink{endnote-appendix}\Hy@raisedlink{\hypertarget{endnote-body}{}{\pagenote{%
		\hypertarget{endnote-appendix}{\hyperlink{endnote-body}{}}}}}\makeatother C: -pānakaṁ. udakaṁ paribhuñjeyya, pācittiyaṁ.



\pdfbookmark[3]{Pācittiya 63}{pac63}
\subsubsection*{\hyperref[exp63]{Pācittiya 63: Ukkoṭanasikkhāpadaṁ}}
\label{pac63}

\linkdest{endnote-body}
Yo pana bhikkhu jānaṁ yathādhammaṁ nihat'ādhikaraṇaṁ\makeatletter\hyperlink{endnote-appendix}\Hy@raisedlink{\hypertarget{endnote-body}{}{\pagenote{%
		\hypertarget{endnote-appendix}{\hyperlink{endnote-body}{}}}}}\makeatother Mi & Mm Se: nīhat-. punakammāya ukkoṭeyya, pācittiyaṁ.



\pdfbookmark[3]{Pācittiya 64}{pac64}
\subsubsection*{\hyperref[exp64]{Pācittiya 64: Duṭṭhullasikkhāpadaṁ}}
\label{pac64}

Yo pana bhikkhu bhikkhussa jānaṁ duṭṭhullaṁ āpattiṁ paṭicchādeyya, pācittiyaṁ.



\pdfbookmark[3]{Pācittiya 65}{pac65}
\subsubsection*{\hyperref[exp65]{Pācittiya 65: Ūnavīsativassasikkhāpadaṁ}}
\label{pac65}

\linkdest{endnote-body}
\linkdest{endnote-body}
\linkdest{endnote-body}
Yo pana bhikkhu jānaṁ ūnavīsativassaṁ\makeatletter\hyperlink{endnote-appendix}\Hy@raisedlink{\hypertarget{endnote-body}{}{\pagenote{%
		\hypertarget{endnote-appendix}{\hyperlink{endnote-body}{}}}}}\makeatother G: ūṇa-. V: ona- puggalaṁ upasampādeyya, so ca puggalo anupasampanno,\makeatletter\hyperlink{endnote-appendix}\Hy@raisedlink{\hypertarget{endnote-body}{}{\pagenote{%
		\hypertarget{endnote-appendix}{\hyperlink{endnote-body}{}}}}}\makeatother V: anūpasampanno. te ca\makeatletter\hyperlink{endnote-appendix}\Hy@raisedlink{\hypertarget{endnote-body}{}{\pagenote{%
		\hypertarget{endnote-appendix}{\hyperlink{endnote-body}{}}}}}\makeatother No ca in V. bhikkhū gārayhā. Idaṁ tasmiṁ pācittiyaṁ.



\pdfbookmark[3]{Pācittiya 66}{pac66}
\subsubsection*{\hyperref[exp66]{Pācittiya 66: Theyyasatthasikkhāpadaṁ}}
\label{pac66}

Yo pana bhikkhu jānaṁ theyyasatthena saddhiṁ saṁvidhāya ek'addhānamaggaṁ paṭipajjeyya antamaso gām'antaram'pi, pācittiyaṁ.



\pdfbookmark[3]{Pācittiya 67}{pac67}
\subsubsection*{\hyperref[exp67]{Pācittiya 67: Saṁvidhānasikkhāpadaṁ}}
\label{pac67}

Yo pana bhikkhu mātugāmena saddhiṁ saṁvidhāya ek'addhānamaggaṁ paṭipajjeyya antamaso gām'antaram'pi, pācittiyaṁ.



\pdfbookmark[3]{Pācittiya 68}{pac68}
\subsubsection*{\hyperref[exp68]{Pācittiya 68: Ariṭṭhasikkhāpadaṁ}}
\label{pac68}

\linkdest{endnote-body}
\linkdest{endnote-body}
\linkdest{endnote-body}
\linkdest{endnote-body}
\linkdest{endnote-body}
\linkdest{endnote-body}
\linkdest{endnote-body}
\linkdest{endnote-body}
\linkdest{endnote-body}
\linkdest{endnote-body}
\linkdest{endnote-body}
Yo pana bhikkhu evaṁ vadeyya, ``Tath'āhaṁ bhagavatā dhammaṁ desitaṁ ājānāmi,\makeatletter\hyperlink{endnote-appendix}\Hy@raisedlink{\hypertarget{endnote-body}{}{\pagenote{%
		\hypertarget{endnote-appendix}{\hyperlink{endnote-body}{}}}}}\makeatother G, P: ajānāmi. yathā ye'me antarāyikā dhammā vuttā bhagavatā, te paṭisevato n'ālaṁ antarāyāyā'ti,'' so bhikkhu bhikkhūhi evam'assa vacanīyo,\makeatletter\hyperlink{endnote-appendix}\Hy@raisedlink{\hypertarget{endnote-body}{}{\pagenote{%
		\hypertarget{endnote-appendix}{\hyperlink{endnote-body}{}}}}}\makeatother V: vacaniyo. ``Mā āyasmā\makeatletter\hyperlink{endnote-appendix}\Hy@raisedlink{\hypertarget{endnote-body}{}{\pagenote{%
		\hypertarget{endnote-appendix}{\hyperlink{endnote-body}{}}}}}\makeatother Dm, UP, Vibh Ee, Um: māyasmā. evaṁ avaca, mā bhagavantaṁ abbhācikkhi,\makeatletter\hyperlink{endnote-appendix}\Hy@raisedlink{\hypertarget{endnote-body}{}{\pagenote{%
		\hypertarget{endnote-appendix}{\hyperlink{endnote-body}{}}}}}\makeatother Mi Se, G, V: abbhācikkha. na hi sādhu bhagavato abbhakkhānaṁ,\makeatletter\hyperlink{endnote-appendix}\Hy@raisedlink{\hypertarget{endnote-body}{}{\pagenote{%
		\hypertarget{endnote-appendix}{\hyperlink{endnote-body}{}}}}}\makeatother Um, Vibh Be v.l., Mi Se v.l.: abbhācikkhanaṁ. na hi bhagavā evaṁ vadeyya; anekapariyāyena āvuso\makeatletter\hyperlink{endnote-appendix}\Hy@raisedlink{\hypertarget{endnote-body}{}{\pagenote{%
		\hypertarget{endnote-appendix}{\hyperlink{endnote-body}{}}}}}\makeatother Dm, Um, UP: -pariyāyenāvuso. antarāyikā dhammā antarāyikā\makeatletter\hyperlink{endnote-appendix}\Hy@raisedlink{\hypertarget{endnote-body}{}{\pagenote{%
		\hypertarget{endnote-appendix}{\hyperlink{endnote-body}{}}}}}\makeatother Mi & Mm Se, G, V, Ra: .”..  āvuso antarāyikā dhammā vuttā bhagavatā  ....” D: ”`...  anekapariyāyena āvuso antarāyikā vuttā
bhagavatā ...” (Probably a misprint as not found in Malwatta mss.) (Pg unclear.) vuttā bhagavatā, alañ'ca pana te paṭisevato antarāyāyā'ti,'' evañ'ca\makeatletter\hyperlink{endnote-appendix}\Hy@raisedlink{\hypertarget{endnote-body}{}{\pagenote{%
		\hypertarget{endnote-appendix}{\hyperlink{endnote-body}{}}}}}\makeatother Vibh Ce, Vibh Ee, Um, Vibh Be v.l., Mi v.l.: ”evañ-ca pana so”. (Pg: ... evaṁ so bhikkhu bhikkhūhi ...) so bhikkhu bhikkhūhi vuccamāno tath'eva paggaṇheyya, so bhikkhu bhikkhūhi yāvatatiyaṁ samanubhāsitabbo tassa paṭinissaggāya, yāvatatiyañ'ce\makeatletter\hyperlink{endnote-appendix}\Hy@raisedlink{\hypertarget{endnote-body}{}{\pagenote{%
		\hypertarget{endnote-appendix}{\hyperlink{endnote-body}{}}}}}\makeatother C, W, Bh Pm 1 & 2, Vibh Ce: ”yāvatatiyaṁ ce”. samanubhāsiyamāno taṁ paṭinissajeyya,\makeatletter\hyperlink{endnote-appendix}\Hy@raisedlink{\hypertarget{endnote-body}{}{\pagenote{%
		\hypertarget{endnote-appendix}{\hyperlink{endnote-body}{}}}}}\makeatother C, D, W. Other editions: paṭinissajjeyya. See Sd 10. icc'etaṁ kusalaṁ, no ce paṭinissajeyya,\makeatletter\hyperlink{endnote-appendix}\Hy@raisedlink{\hypertarget{endnote-body}{}{\pagenote{%
		\hypertarget{endnote-appendix}{\hyperlink{endnote-body}{}}}}}\makeatother C, D, W. Other editions: paṭinissajjeyya. See Sd 10. pācittiyaṁ.



\pdfbookmark[3]{Pācittiya 69}{pac69}
\subsubsection*{\hyperref[exp69]{Pācittiya 69: Ukkhittasambhogasikkhāpadaṁ}}
\label{pac69}

\linkdest{endnote-body}
\linkdest{endnote-body}
\linkdest{endnote-body}
Yo pana bhikkhu jānaṁ tathāvādinā bhikkhunā akaṭ'ānudhammena\makeatletter\hyperlink{endnote-appendix}\Hy@raisedlink{\hypertarget{endnote-body}{}{\pagenote{%
		\hypertarget{endnote-appendix}{\hyperlink{endnote-body}{}}}}}\makeatother Bh Pm 1 & 2, G, Um, UP, V: akatānudhammena. taṁ diṭṭhiṁ appaṭinissaṭṭhena saddhiṁ sambhuñjeyya\makeatletter\hyperlink{endnote-appendix}\Hy@raisedlink{\hypertarget{endnote-body}{}{\pagenote{%
		\hypertarget{endnote-appendix}{\hyperlink{endnote-body}{}}}}}\makeatother  G, Vibh Ee: saṁbhuñjeyya. vā saṁvaseyya\makeatletter\hyperlink{endnote-appendix}\Hy@raisedlink{\hypertarget{endnote-body}{}{\pagenote{%
		\hypertarget{endnote-appendix}{\hyperlink{endnote-body}{}}}}}\makeatother D, G, V, Vibh Ee: saṁvāseyya. vā saha vā seyyaṁ kappeyya, pācittiyaṁ.



\pdfbookmark[3]{Pācittiya 70}{pac70}
\subsubsection*{\hyperref[exp70]{Pācittiya 70: Kaṇṭakasikkhāpadaṁ}}
\label{pac70}

\linkdest{endnote-body}
\linkdest{endnote-body}
\linkdest{endnote-body}
\linkdest{endnote-body}
\linkdest{endnote-body}
\linkdest{endnote-body}
\linkdest{endnote-body}
\linkdest{endnote-body}
\linkdest{endnote-body}
\linkdest{endnote-body}
\linkdest{endnote-body}
\linkdest{endnote-body}
\linkdest{endnote-body}
Samaṇ'uddeso'pi ce evaṁ vadeyya, ``Tath'āhaṁ bhagavatā dhammaṁ desitaṁ ājānāmi,\makeatletter\hyperlink{endnote-appendix}\Hy@raisedlink{\hypertarget{endnote-body}{}{\pagenote{%
		\hypertarget{endnote-appendix}{\hyperlink{endnote-body}{}}}}}\makeatother G, P: ajānāmi. yathā ye'me antarāyikā dhammā vuttā bhagavatā, te paṭisevato n'ālaṁ antarāyāyā'ti,'' so samaṇ'uddeso bhikkhūhi evam'assa vacanīyo,\makeatletter\hyperlink{endnote-appendix}\Hy@raisedlink{\hypertarget{endnote-body}{}{\pagenote{%
		\hypertarget{endnote-appendix}{\hyperlink{endnote-body}{}}}}}\makeatother V: vacaniyo. ``Mā āvuso\makeatletter\hyperlink{endnote-appendix}\Hy@raisedlink{\hypertarget{endnote-body}{}{\pagenote{%
		\hypertarget{endnote-appendix}{\hyperlink{endnote-body}{}}}}}\makeatother Dm, UP, Vibh Ee: māvuso. samaṇ'uddesa evaṁ avaca, mā bhagavantaṁ abbhācikkhi,\makeatletter\hyperlink{endnote-appendix}\Hy@raisedlink{\hypertarget{endnote-body}{}{\pagenote{%
		\hypertarget{endnote-appendix}{\hyperlink{endnote-body}{}}}}}\makeatother Mi Se, G, V: abbhācikkha. na hi sādhu bhagavato abbhakkhānaṁ,\makeatletter\hyperlink{endnote-appendix}\Hy@raisedlink{\hypertarget{endnote-body}{}{\pagenote{%
		\hypertarget{endnote-appendix}{\hyperlink{endnote-body}{}}}}}\makeatother Um, Vibh Be v.l., Mi Se v.l.: abbhācikkhanaṁ. na hi bhagavā evaṁ vadeyya; anekapariyāyena āvuso\makeatletter\hyperlink{endnote-appendix}\Hy@raisedlink{\hypertarget{endnote-body}{}{\pagenote{%
		\hypertarget{endnote-appendix}{\hyperlink{endnote-body}{}}}}}\makeatother Dm, Um, UP: -pariyāyenāvuso. samaṇ'uddesa antarāyikā dhammā antarāyikā\makeatletter\hyperlink{endnote-appendix}\Hy@raisedlink{\hypertarget{endnote-body}{}{\pagenote{%
		\hypertarget{endnote-appendix}{\hyperlink{endnote-body}{}}}}}\makeatother Mi & Mm Se, G, V, Ra: .”..  āvuso antarāyikā dhammā vuttā bhagavatā  ....” D: “`...  anekapariyāyena āvuso antarāyikā vuttā
bhagavatā ...” (Probably a misprint as not found in Malwatta mss.) (Pg unclear.) vuttā bhagavatā, alañ'ca pana te paṭisevato antarāyāyā'ti,'' evañ'ca\makeatletter\hyperlink{endnote-appendix}\Hy@raisedlink{\hypertarget{endnote-body}{}{\pagenote{%
		\hypertarget{endnote-appendix}{\hyperlink{endnote-body}{}}}}}\makeatother Vibh Ce, Vibh Ee, Um, Vibh Be v.l., Mi v.l.: “evañ-ca pana so”. (Pg: ... evaṁ so bhikkhu bhikkhūhi ...) so samaṇ'uddeso bhikkhūhi vuccamāno tath'eva paggaṇheyya, so samaṇ'uddeso bhikkhūhi evam'assa vacanīyo,\makeatletter\hyperlink{endnote-appendix}\Hy@raisedlink{\hypertarget{endnote-body}{}{\pagenote{%
		\hypertarget{endnote-appendix}{\hyperlink{endnote-body}{}}}}}\makeatother V: vacaniyo. ``Ajja't'agge te āvuso samaṇ'uddesa na c'eva so bhagavā satthā apadisitabbo, yam'pi c'aññe samaṇ'uddesā labhanti bhikkhūhi saddhiṁ dirattatirattaṁ\makeatletter\hyperlink{endnote-appendix}\Hy@raisedlink{\hypertarget{endnote-body}{}{\pagenote{%
		\hypertarget{endnote-appendix}{\hyperlink{endnote-body}{}}}}}\makeatother Mm Se, Vibh Ee: dvi-. saha seyyaṁ,\makeatletter\hyperlink{endnote-appendix}\Hy@raisedlink{\hypertarget{endnote-body}{}{\pagenote{%
		\hypertarget{endnote-appendix}{\hyperlink{endnote-body}{}}}}}\makeatother Dm, Vibh Ce, UP, Mm & Mi Se, V, Vibh Ee,: sahaseyyaṁ. See Pāc 5. sā'pi te n'atthi, carapire\makeatletter\hyperlink{endnote-appendix}\Hy@raisedlink{\hypertarget{endnote-body}{}{\pagenote{%
		\hypertarget{endnote-appendix}{\hyperlink{endnote-body}{}}}}}\makeatother Dm, Um, UP, Vibh Ee, Mi & Mm Se, V, W: pire. Bh Pm 1 & 2, C, D, Vibh Ce, Ra, Pg, Ce Kkh: pare. G: cara pi pare. vinassā'ti.'' Yo pana bhikkhu jānaṁ tathānāsitaṁ samaṇ'uddesaṁ upalāpeyya vā upaṭṭhāpeyya vā sambhuñjeyya\makeatletter\hyperlink{endnote-appendix}\Hy@raisedlink{\hypertarget{endnote-body}{}{\pagenote{%
		\hypertarget{endnote-appendix}{\hyperlink{endnote-body}{}}}}}\makeatother G, Vibh Ee: saṁbhuñjeyya. vā saha vā seyyaṁ kappeyya, pācittiyaṁ.

\linkdest{endnote-body}
\begin{center}
	Sappāṇakavaggo\makeatletter\hyperlink{endnote-appendix}\Hy@raisedlink{\hypertarget{endnote-body}{}{\pagenote{%
		\hypertarget{endnote-appendix}{\hyperlink{endnote-body}{}}}}}\makeatother Mi & Mm Se, G, V: sappāṇavaggo. sattamo
\end{center}



\subsection{Sahadhammikavaggo}
% \vspace{0.2cm}

\pdfbookmark[3]{Pācittiya 71}{pac71}
\subsubsection*{\hyperref[exp71]{Pācittiya 71: Sahadhammikasikkhāpadaṁ}}
\label{pac71}

\linkdest{endnote-body}
\linkdest{endnote-body}
\linkdest{endnote-body}
Yo pana bhikkhu bhikkhūhi sahadhammikaṁ vuccamāno evaṁ vadeyya, ``Na tāv'āhaṁ āvuso etasmiṁ sikkhāpade sikkhissāmi, yāva na aññaṁ\makeatletter\hyperlink{endnote-appendix}\Hy@raisedlink{\hypertarget{endnote-body}{}{\pagenote{%
		\hypertarget{endnote-appendix}{\hyperlink{endnote-body}{}}}}}\makeatother Mi & Mm Se: naññaṁ. G: na aṁñaṁ.  bhikkhuṁ byattaṁ\makeatletter\hyperlink{endnote-appendix}\Hy@raisedlink{\hypertarget{endnote-body}{}{\pagenote{%
		\hypertarget{endnote-appendix}{\hyperlink{endnote-body}{}}}}}\makeatother Bh Pm 1 & 2, C, D, W, UP, Ra, Vibh Ce, Pg: vyattaṁ. vinayadharaṁ paripucchāmī''ti, pācittiyaṁ. Sikkhamānena, bhikkhave, bhikkhunā aññātabbaṁ paripucchitabbaṁ paripañhitabbaṁ.\makeatletter\hyperlink{endnote-appendix}\Hy@raisedlink{\hypertarget{endnote-body}{}{\pagenote{%
		\hypertarget{endnote-appendix}{\hyperlink{endnote-body}{}}}}}\makeatother D, G, V: - paṇhi-. Ayaṁ tattha sāmīci.



\pdfbookmark[3]{Pācittiya 72}{pac72}
\subsubsection*{\hyperref[exp72]{Pācittiya 72: Vilekhanasikkhāpadaṁ}}
\label{pac72}

\linkdest{endnote-body}
\linkdest{endnote-body}
\linkdest{endnote-body}
Yo pana bhikkhu pātimokkhe\makeatletter\hyperlink{endnote-appendix}\Hy@raisedlink{\hypertarget{endnote-body}{}{\pagenote{%
		\hypertarget{endnote-appendix}{\hyperlink{endnote-body}{}}}}}\makeatother Mm Se, G, V: pāṭimokkhe. uddissamāne evaṁ vadeyya, ``Kiṁ pan'imehi\makeatletter\hyperlink{endnote-appendix}\Hy@raisedlink{\hypertarget{endnote-body}{}{\pagenote{%
		\hypertarget{endnote-appendix}{\hyperlink{endnote-body}{}}}}}\makeatother Mi & Mm Se, G, V: kim-pan'imehi. khudd'ānukhuddakehi sikkhāpadehi uddiṭṭhehi; yāva'd'eva kukkuccāya, vihesāya, vilekhāya saṁvattantī'ti,'' sikkhāpadavivaṇṇake,\makeatletter\hyperlink{endnote-appendix}\Hy@raisedlink{\hypertarget{endnote-body}{}{\pagenote{%
		\hypertarget{endnote-appendix}{\hyperlink{endnote-body}{}}}}}\makeatother Dm, UP, G, V, Vibh Ce, Vibh Ee: vivaṇṇake. BhPm 1 & 2, C, D, W, Mi & Mm Se, Um, Ra, Pg, Ce Kkh: vivaṇṇanake. pācittiyaṁ.



\pdfbookmark[3]{Pācittiya 73}{pac73}
\subsubsection*{\hyperref[exp73]{Pācittiya 73: Mohanasikkhāpadaṁ}}
\label{pac73}

\linkdest{endnote-body}
\linkdest{endnote-body}
\linkdest{endnote-body}
\linkdest{endnote-body}
\linkdest{endnote-body}
\linkdest{endnote-body}
\linkdest{endnote-body}
\linkdest{endnote-body}
\linkdest{endnote-body}
\linkdest{endnote-body}
\linkdest{endnote-body}
\linkdest{endnote-body}
\linkdest{endnote-body}
\linkdest{endnote-body}
Yo pana bhikkhu anvaḍḍhamāsaṁ\makeatletter\hyperlink{endnote-appendix}\Hy@raisedlink{\hypertarget{endnote-body}{}{\pagenote{%
		\hypertarget{endnote-appendix}{\hyperlink{endnote-body}{}}}}}\makeatother As in Pāc 57, only Mi, Mm Se, & V read anvaḍḍha-. The rest read anvaddha-.  pātimokkhe\makeatletter\hyperlink{endnote-appendix}\Hy@raisedlink{\hypertarget{endnote-body}{}{\pagenote{%
		\hypertarget{endnote-appendix}{\hyperlink{endnote-body}{}}}}}\makeatother Mm Se, G, V: pāṭimokkhe. uddissamāne evaṁ vadeyya, ``Idān'eva kho\makeatletter\hyperlink{endnote-appendix}\Hy@raisedlink{\hypertarget{endnote-body}{}{\pagenote{%
		\hypertarget{endnote-appendix}{\hyperlink{endnote-body}{}}}}}\makeatother Bh Pm 1 & 2, C, W, UP, Ra: kho āvuso. ahaṁ jānāmi,\makeatletter\hyperlink{endnote-appendix}\Hy@raisedlink{\hypertarget{endnote-body}{}{\pagenote{%
		\hypertarget{endnote-appendix}{\hyperlink{endnote-body}{}}}}}\makeatother Bh Pm 1 & 2, Mi & Mm Se, V, Ra, Pg: ājānāmi. ayam'pi\makeatletter\hyperlink{endnote-appendix}\Hy@raisedlink{\hypertarget{endnote-body}{}{\pagenote{%
		\hypertarget{endnote-appendix}{\hyperlink{endnote-body}{}}}}}\makeatother Um: ayaṁ pi. kira dhammo sutt'āgato suttapariyāpanno anvaḍḍhamāsaṁ\makeatletter\hyperlink{endnote-appendix}\Hy@raisedlink{\hypertarget{endnote-body}{}{\pagenote{%
		\hypertarget{endnote-appendix}{\hyperlink{endnote-body}{}}}}}\makeatother Mi, Mm Se, & V: anvaḍḍha-. In the second occurrence of this word in this rule G read -ḍḍh-, but was corrected to -ddh-. uddesaṁ āgacchatī'ti,'' tañ'ce\makeatletter\hyperlink{endnote-appendix}\Hy@raisedlink{\hypertarget{endnote-body}{}{\pagenote{%
		\hypertarget{endnote-appendix}{\hyperlink{endnote-body}{}}}}}\makeatother C: taṁ ce. bhikkhuṁ aññe bhikkhū jāneyyuṁ, ``Nisinnapubbaṁ iminā bhikkhunā dvattikkhattuṁ\makeatletter\hyperlink{endnote-appendix}\Hy@raisedlink{\hypertarget{endnote-body}{}{\pagenote{%
		\hypertarget{endnote-appendix}{\hyperlink{endnote-body}{}}}}}\makeatother Vibh Ee, Mm Se: dvi-. (Mi Se reads dva- here, instead of dvi- elsewhere. See NP 10.) pātimokkhe\makeatletter\hyperlink{endnote-appendix}\Hy@raisedlink{\hypertarget{endnote-body}{}{\pagenote{%
		\hypertarget{endnote-appendix}{\hyperlink{endnote-body}{}}}}}\makeatother Mm Se, G, V: pāṭimokkhe. uddissamāne. Ko pana vādo bhiyyo'ti,\makeatletter\hyperlink{endnote-appendix}\Hy@raisedlink{\hypertarget{endnote-body}{}{\pagenote{%
		\hypertarget{endnote-appendix}{\hyperlink{endnote-body}{}}}}}\makeatother Mi & Mm Se, C, D, V: bhiyyo ti. Bh Pm 1 & 2, G, Um: bhīyyo ti. Others MS and texts have ..”. bhiyyo na ca ...”` without ti. (Pg
unclear.) na ca tassa bhikkhuno aññāṇakena mutti atthi, yañ'ca tattha āpattiṁ āpanno, tañ'ca yathā dhammo\makeatletter\hyperlink{endnote-appendix}\Hy@raisedlink{\hypertarget{endnote-body}{}{\pagenote{%
		\hypertarget{endnote-appendix}{\hyperlink{endnote-body}{}}}}}\makeatother Bh Pm 1 & 2, Ra: yathā dhammo. Other printed eds: yathādhammo. kāretabbo, uttariñ'c'assa\makeatletter\hyperlink{endnote-appendix}\Hy@raisedlink{\hypertarget{endnote-body}{}{\pagenote{%
		\hypertarget{endnote-appendix}{\hyperlink{endnote-body}{}}}}}\makeatother Dm, Vibh Ee, Um: uttari cassa. C, G, W, Bh Pm 1 & 2, Vibh Ce, Ra: uttariṁ cassa. moho āropetabbo, ``Tassa te āvuso alābhā, tassa te dulladdhaṁ. Yaṁ tvaṁ pātimokkhe\makeatletter\hyperlink{endnote-appendix}\Hy@raisedlink{\hypertarget{endnote-body}{}{\pagenote{%
		\hypertarget{endnote-appendix}{\hyperlink{endnote-body}{}}}}}\makeatother Mm Se, G, V: pāṭimokkhe. uddissamāne, na sādhukaṁ aṭṭhikatvā\makeatletter\hyperlink{endnote-appendix}\Hy@raisedlink{\hypertarget{endnote-body}{}{\pagenote{%
		\hypertarget{endnote-appendix}{\hyperlink{endnote-body}{}}}}}\makeatother Dm, Um, UP: aṭṭhiṁ katvā. manasikarosī'ti.'' Idaṁ tasmiṁ mohanake, pācittiyaṁ.



\pdfbookmark[3]{Pācittiya 74}{pac74}
\subsubsection*{\hyperref[exp74]{Pācittiya 74: Pahārasikkhāpadaṁ}}
\label{pac74}

\linkdest{endnote-body}
Yo pana bhikkhu bhikkhussa kupito\makeatletter\hyperlink{endnote-appendix}\Hy@raisedlink{\hypertarget{endnote-body}{}{\pagenote{%
		\hypertarget{endnote-appendix}{\hyperlink{endnote-body}{}}}}}\makeatother V: kuppito. (Cf NP 25 & Pāc 17.) anattamano pahāraṁ dadeyya, pācittiyaṁ.



\pdfbookmark[3]{Pācittiya 75}{pac75}
\subsubsection*{\hyperref[exp75]{Pācittiya 75: Talasattikasikkhāpadaṁ}}
\label{pac75}

\linkdest{endnote-body}
Yo pana bhikkhu bhikkhussa kupito\makeatletter\hyperlink{endnote-appendix}\Hy@raisedlink{\hypertarget{endnote-body}{}{\pagenote{%
		\hypertarget{endnote-appendix}{\hyperlink{endnote-body}{}}}}}\makeatother V: kuppito. anattamano talasattikaṁ uggireyya, pācittiyaṁ.



\pdfbookmark[3]{Pācittiya 76}{pac76}
\subsubsection*{\hyperref[exp76]{Pācittiya 76: Amūlakasikkhāpadaṁ}}
\label{pac76}

Yo pana bhikkhu bhikkhuṁ amūlakena saṅghādisesena anuddhaṁseyya, pācittiyaṁ.



\pdfbookmark[3]{Pācittiya 77}{pac77}
\subsubsection*{\hyperref[exp77]{Pācittiya 77: Sañciccasikkhāpadaṁ}}
\label{pac77}

\linkdest{endnote-body}
\linkdest{endnote-body}
Yo pana bhikkhu bhikkhussa sañcicca\makeatletter\hyperlink{endnote-appendix}\Hy@raisedlink{\hypertarget{endnote-body}{}{\pagenote{%
		\hypertarget{endnote-appendix}{\hyperlink{endnote-body}{}}}}}\makeatother W: saṁcicca (but not so at Pār 3 and Pāc 61.) kukkuccaṁ upadaheyya,\makeatletter\hyperlink{endnote-appendix}\Hy@raisedlink{\hypertarget{endnote-body}{}{\pagenote{%
		\hypertarget{endnote-appendix}{\hyperlink{endnote-body}{}}}}}\makeatother Ra, Pg, UP v.l.: uppādeyya. G: uppādaheyya. V: upādaheyya. ``Iti'ssa muhuttam'pi aphāsu bhavissatī'ti,'' etad'eva paccayaṁ karitvā anaññaṁ, pācittiyaṁ.



\pdfbookmark[3]{Pācittiya 78}{pac78}
\subsubsection*{\hyperref[exp78]{Pācittiya 78: Upassutisikkhāpadaṁ}}
\label{pac78}

\linkdest{endnote-body}
Yo pana bhikkhu bhikkhūnaṁ bhaṇḍanajātānaṁ kalahajātānaṁ vivād'āpannānaṁ upassutiṁ\makeatletter\hyperlink{endnote-appendix}\Hy@raisedlink{\hypertarget{endnote-body}{}{\pagenote{%
		\hypertarget{endnote-appendix}{\hyperlink{endnote-body}{}}}}}\makeatother Mi Se, Bh Pm 2, Pg: upassuti. V: upassūti. tiṭṭheyya: ``Yaṁ ime bhaṇissanti, taṁ sossāmī'ti,'' etad'eva paccayaṁ karitvā anaññaṁ, pācittiyaṁ.



\pdfbookmark[3]{Pācittiya 79}{pac79}
\subsubsection*{\hyperref[exp79]{Pācittiya 79: Kammappaṭibāhanasikkhāpadaṁ}}
\label{pac79}

\linkdest{endnote-body}
Yo pana bhikkhu dhammikānaṁ kammānaṁ chandaṁ datvā pacchā khiyyanadhammaṁ\makeatletter\hyperlink{endnote-appendix}\Hy@raisedlink{\hypertarget{endnote-body}{}{\pagenote{%
		\hypertarget{endnote-appendix}{\hyperlink{endnote-body}{}}}}}\makeatother Bh Pm 1 & 2, C, D, G, W, Dm, Um, Ra, Vibh Ce, Parivāra Be: khīyana-. Mi & Mm Se: khiyyana-. (Also at Pāc 81.) Parivāra Ce:
khiyana-. V: khiyya-. Vibh Ee, Parivāra Ee: khīya-. (This reading is also at A III 269, IV 374.) Cf khiyyanaka at Pāc 13. āpajjeyya, pācittiyaṁ.



\pdfbookmark[3]{Pācittiya 80}{pac80}
\subsubsection*{\hyperref[exp80]{Pācittiya 80: Chandaṁ-adatvā-gamanasikkhāpadaṁ}}
\label{pac80}

Yo pana bhikkhu saṅghe vinicchayakathāya vattamānāya chandaṁ adatvā uṭṭhāy'āsanā pakkameyya, pācittiyaṁ.



\pdfbookmark[3]{Pācittiya 81}{pac81}
\subsubsection*{\hyperref[exp81]{Pācittiya 81: Dubbalasikkhāpadaṁ}}
\label{pac81}

\linkdest{endnote-body}
\linkdest{endnote-body}
Yo pana bhikkhu samaggena saṅghena cīvaraṁ datvā pacchā khiyyanadhammaṁ\makeatletter\hyperlink{endnote-appendix}\Hy@raisedlink{\hypertarget{endnote-body}{}{\pagenote{%
		\hypertarget{endnote-appendix}{\hyperlink{endnote-body}{}}}}}\makeatother Bh Pm 1 & 2, C, D, G, W, Dm, Um, Ra, Vibh Ce, Parivāra Be: khīyana-. Mi & Mm Se: khiyyana-. (Also at Pāc 81.) Parivāra Ce:
khiyana-. V: khiyya-. Vibh Ee, Parivāra Ee: khīya-. (This reading is also at A III 269, IV 374.) Cf khiyyanaka at Pāc 13. āpajjeyya, ``Yathāsanthutaṁ\makeatletter\hyperlink{endnote-appendix}\Hy@raisedlink{\hypertarget{endnote-body}{}{\pagenote{%
		\hypertarget{endnote-appendix}{\hyperlink{endnote-body}{}}}}}\makeatother D: -santhavaṁ. Vibh Ee: -santataṁ. Pg, G: -santhataṁ. V: -saṇṭhataṁ. bhikkhū saṅghikaṁ lābhaṁ pariṇāmentī'ti'', pācittiyaṁ.



\pdfbookmark[3]{Pācittiya 82}{pac82}
\subsubsection*{\hyperref[exp82]{Pācittiya 82: Pariṇāmanasikkhāpadaṁ}}
\label{pac82}

Yo pana bhikkhu jānaṁ saṅghikaṁ lābhaṁ pariṇataṁ puggalassa pariṇāmeyya, pācittiyaṁ.

\begin{center}
	Sahadhammikavaggo aṭṭhamo
\end{center}



\subsection{Rājavaggo}
% \vspace{0.2cm}

\pdfbookmark[3]{Pācittiya 83}{pac83}
\subsubsection*{\hyperref[exp83]{Pācittiya 83: Antepurasikkhāpadaṁ}}
\label{pac83}

\linkdest{endnote-body}
\linkdest{endnote-body}
\linkdest{endnote-body}
Yo pana bhikkhu rañño khattiyassa muddh'ābhisittassa\makeatletter\hyperlink{endnote-appendix}\Hy@raisedlink{\hypertarget{endnote-body}{}{\pagenote{%
		\hypertarget{endnote-appendix}{\hyperlink{endnote-body}{}}}}}\makeatother Bh Pm 1 & 2, D, Ra, Vibh Ce, Vibh Ee, Pg: muddhāvasitassa. (Pg: ... muddhāni abhisitassa rañño ... muddhāni avasitto.”) anikkhantarājake aniggataratanake\makeatletter\hyperlink{endnote-appendix}\Hy@raisedlink{\hypertarget{endnote-body}{}{\pagenote{%
		\hypertarget{endnote-appendix}{\hyperlink{endnote-body}{}}}}}\makeatother Bh Pm 1 & 2, C, G, W, Mi Se, Vibh Ce, Ee Sp, Ce Kkh, Pg: anībhata-. V: anibhata-. D, Ra, UP sīhala v.l.: anīhata. (The bh and h
characters are very similar in Sinhala script.) pubbe appaṭisaṁvidito indakhīlaṁ atikkameyya,\makeatletter\hyperlink{endnote-appendix}\Hy@raisedlink{\hypertarget{endnote-body}{}{\pagenote{%
		\hypertarget{endnote-appendix}{\hyperlink{endnote-body}{}}}}}\makeatother Mi & Mm Se, G, Bh Pm 1 & 2, C, V, W, Ra: atikkameyya. Other eds. atikkāmeyya. pācittiyaṁ.



\pdfbookmark[3]{Pācittiya 84}{pac84}
\subsubsection*{\hyperref[exp84]{Pācittiya 84: Ratanasikkhāpadaṁ}}
\label{pac84}

\linkdest{endnote-body}
Yo pana bhikkhu ratanaṁ vā ratanasammataṁ vā, aññatra ajjhārāmā vā ajjhāvasathā vā uggaṇheyya vā uggaṇhāpeyya vā, pācittiyaṁ. Ratanaṁ vā pana bhikkhunā ratanasammataṁ vā ajjhārāme vā ajjhāvasathe vā uggahetvā vā uggahāpetvā\makeatletter\hyperlink{endnote-appendix}\Hy@raisedlink{\hypertarget{endnote-body}{}{\pagenote{%
		\hypertarget{endnote-appendix}{\hyperlink{endnote-body}{}}}}}\makeatother Bh Pm 1 & 2, Mi & Mm Se, G, V, Ra, Pg: uggaṇhāpetvā. vā nikkhipitabbaṁ, ``Yassa bhavissati, so harissatī'ti.'' Ayaṁ tattha sāmīci.



\pdfbookmark[3]{Pācittiya 85}{pac85}
\subsubsection*{\hyperref[exp85]{Pācittiya 85: Vikālagāmappavesanasikkhāpadaṁ}}
\label{pac85}

\linkdest{endnote-body}
\linkdest{endnote-body}
Yo pana bhikkhu santaṁ bhikkhuṁ anāpucchā vikāle gāmaṁ paviseyya,\makeatletter\hyperlink{endnote-appendix}\Hy@raisedlink{\hypertarget{endnote-body}{}{\pagenote{%
		\hypertarget{endnote-appendix}{\hyperlink{endnote-body}{}}}}}\makeatother  V: pavīseyya. aññatra tathārūpā accāyikā karaṇīyā,\makeatletter\hyperlink{endnote-appendix}\Hy@raisedlink{\hypertarget{endnote-body}{}{\pagenote{%
		\hypertarget{endnote-appendix}{\hyperlink{endnote-body}{}}}}}\makeatother V: karaṇiyā. pācittiyaṁ.



\pdfbookmark[3]{Pācittiya 86}{pac86}
\subsubsection*{\hyperref[exp86]{Pācittiya 86: Sūcigharasikkhāpadaṁ}}
\label{pac86}

\linkdest{endnote-body}
Yo pana bhikkhu aṭṭhimayaṁ vā dantamayaṁ vā visāṇamayaṁ vā sūcigharaṁ\makeatletter\hyperlink{endnote-appendix}\Hy@raisedlink{\hypertarget{endnote-body}{}{\pagenote{%
		\hypertarget{endnote-appendix}{\hyperlink{endnote-body}{}}}}}\makeatother V: suci-. Cf. Pāc. 60. kārāpeyya, bhedanakaṁ pācittiyaṁ.



\pdfbookmark[3]{Pācittiya 87}{pac87}
\subsubsection*{\hyperref[exp87]{Pācittiya 87: Mañcapīṭhasikkhāpadaṁ}}
\label{pac87}

\linkdest{endnote-body}
\linkdest{endnote-body}
\linkdest{endnote-body}
\linkdest{endnote-body}
\linkdest{endnote-body}
Navaṁ pana\makeatletter\hyperlink{endnote-appendix}\Hy@raisedlink{\hypertarget{endnote-body}{}{\pagenote{%
		\hypertarget{endnote-appendix}{\hyperlink{endnote-body}{}}}}}\makeatother Bh Pm 1 & 2, Mi Se, G, V: navampana. bhikkhunā mañcaṁ vā pīṭhaṁ\makeatletter\hyperlink{endnote-appendix}\Hy@raisedlink{\hypertarget{endnote-body}{}{\pagenote{%
		\hypertarget{endnote-appendix}{\hyperlink{endnote-body}{}}}}}\makeatother V: pithaṁ. vā kārayamānena aṭṭh'aṅgulapādakaṁ kāretabbaṁ sugataṅgulena,\makeatletter\hyperlink{endnote-appendix}\Hy@raisedlink{\hypertarget{endnote-body}{}{\pagenote{%
		\hypertarget{endnote-appendix}{\hyperlink{endnote-body}{}}}}}\makeatother V: sutaṅgulena-. aññatra heṭṭhimāya\makeatletter\hyperlink{endnote-appendix}\Hy@raisedlink{\hypertarget{endnote-body}{}{\pagenote{%
		\hypertarget{endnote-appendix}{\hyperlink{endnote-body}{}}}}}\makeatother Mm Se: hetthimāya. aṭaniyā.\makeatletter\hyperlink{endnote-appendix}\Hy@raisedlink{\hypertarget{endnote-body}{}{\pagenote{%
		\hypertarget{endnote-appendix}{\hyperlink{endnote-body}{}}}}}\makeatother V: aṭṭhaniyā. Taṁ atikkāmayato, chedanakaṁ pācittiyaṁ.



\pdfbookmark[3]{Pācittiya 88}{pac88}
\subsubsection*{\hyperref[exp88]{Pācittiya 88: Tūlonaddhasikkhāpadaṁ}}
\label{pac88}

\linkdest{endnote-body}
\linkdest{endnote-body}
\linkdest{endnote-body}
Yo pana bhikkhu mañcaṁ vā pīṭhaṁ\makeatletter\hyperlink{endnote-appendix}\Hy@raisedlink{\hypertarget{endnote-body}{}{\pagenote{%
		\hypertarget{endnote-appendix}{\hyperlink{endnote-body}{}}}}}\makeatother V: pithaṁ. vā tūl'onaddhaṁ\makeatletter\hyperlink{endnote-appendix}\Hy@raisedlink{\hypertarget{endnote-body}{}{\pagenote{%
		\hypertarget{endnote-appendix}{\hyperlink{endnote-body}{}}}}}\makeatother C, UP, V, Ra: tul-. kārāpeyya, uddālanakaṁ\makeatletter\hyperlink{endnote-appendix}\Hy@raisedlink{\hypertarget{endnote-body}{}{\pagenote{%
		\hypertarget{endnote-appendix}{\hyperlink{endnote-body}{}}}}}\makeatother Bh Pm 1 & 2, Ra, Pg: uddāḷanakaṁ. pācittiyaṁ.



\pdfbookmark[3]{Pācittiya 89}{pac89}
\subsubsection*{\hyperref[exp89]{Pācittiya 89: Nisīdanasikkhāpadaṁ}}
\label{pac89}

\linkdest{endnote-body}
\linkdest{endnote-body}
\linkdest{endnote-body}
Nisīdanaṁ pana\makeatletter\hyperlink{endnote-appendix}\Hy@raisedlink{\hypertarget{endnote-body}{}{\pagenote{%
		\hypertarget{endnote-appendix}{\hyperlink{endnote-body}{}}}}}\makeatother Bh Pm 1 & 2, C, G, V, W, Mi Se, Sannē: nisīdanam-pana. bhikkhunā kārayamānena pamāṇikaṁ kāretabbaṁ. Tatr'idaṁ pamāṇaṁ,\makeatletter\hyperlink{endnote-appendix}\Hy@raisedlink{\hypertarget{endnote-body}{}{\pagenote{%
		\hypertarget{endnote-appendix}{\hyperlink{endnote-body}{}}}}}\makeatother  V: tatrīdaṁ. dīghaso dve vidatthiyo sugatavidatthiyā, tiriyaṁ diyaḍḍhaṁ,\makeatletter\hyperlink{endnote-appendix}\Hy@raisedlink{\hypertarget{endnote-body}{}{\pagenote{%
		\hypertarget{endnote-appendix}{\hyperlink{endnote-body}{}}}}}\makeatother V: diyaḍhaṁ. dasā vidatthi. Taṁ atikkāmayato, chedanakaṁ pācittiyaṁ.



\pdfbookmark[3]{Pācittiya 90}{pac90}
\subsubsection*{\hyperref[exp90]{Pācittiya 90: Kaṇḍuppaṭicchādisikkhāpadaṁ}}
\label{pac90}

\linkdest{endnote-body}
\linkdest{endnote-body}
Kaṇḍupaṭicchādiṁ\makeatletter\hyperlink{endnote-appendix}\Hy@raisedlink{\hypertarget{endnote-body}{}{\pagenote{%
		\hypertarget{endnote-appendix}{\hyperlink{endnote-body}{}}}}}\makeatother Dm: kaṇḍuppaṭicchādiṁ. Bh Pm 1 & 2, C, G: -cchādim-pana.  pana bhikkhunā kārayamānena pamāṇikā kāretabbā. Tatr'idaṁ\makeatletter\hyperlink{endnote-appendix}\Hy@raisedlink{\hypertarget{endnote-body}{}{\pagenote{%
		\hypertarget{endnote-appendix}{\hyperlink{endnote-body}{}}}}}\makeatother V: tatrīdaṁ. pamāṇaṁ, dīghaso catasso vidatthiyo sugatavidatthiyā, tiriyaṁ dve vidatthiyo. Taṁ atikkāmayato, chedanakaṁ pācittiyaṁ.



\pdfbookmark[3]{Pācittiya 91}{pac91}
\subsubsection*{\hyperref[exp91]{Pācittiya 91: Vassikasāṭikāsikkhāpadaṁ}}
\label{pac91}

\linkdest{endnote-body}
\linkdest{endnote-body}
\linkdest{endnote-body}
\linkdest{endnote-body}
Vassikasāṭikaṁ\makeatletter\hyperlink{endnote-appendix}\Hy@raisedlink{\hypertarget{endnote-body}{}{\pagenote{%
		\hypertarget{endnote-appendix}{\hyperlink{endnote-body}{}}}}}\makeatother G, Mi Se v.l. (porānapotthake, marammapotthake): -sāṭikā. It is possible that originally this rule and the previous one read
-cchādī/-cchādi and -sāṭikā, i.e., nominative feminines (as found in the padabhājana). The sentence is passive and the patient
is therefore in the nominative. pana\makeatletter\hyperlink{endnote-appendix}\Hy@raisedlink{\hypertarget{endnote-body}{}{\pagenote{%
		\hypertarget{endnote-appendix}{\hyperlink{endnote-body}{}}}}}\makeatother C, Sannē: -sāṭikam-pana.  bhikkhunā kārayamānena pamāṇikā kāretabbā. Tatr'idaṁ\makeatletter\hyperlink{endnote-appendix}\Hy@raisedlink{\hypertarget{endnote-body}{}{\pagenote{%
		\hypertarget{endnote-appendix}{\hyperlink{endnote-body}{}}}}}\makeatother V: tatrīdaṁ. pamāṇaṁ, dīghaso cha vidatthiyo sugatavidatthiyā, tiriyaṁ aḍḍhateyyā.\makeatletter\hyperlink{endnote-appendix}\Hy@raisedlink{\hypertarget{endnote-body}{}{\pagenote{%
		\hypertarget{endnote-appendix}{\hyperlink{endnote-body}{}}}}}\makeatother V: aḍhateyya.. Taṁ atikkāmayato, chedanakaṁ pācittiyaṁ.



\pdfbookmark[3]{Pācittiya 92}{pac92}
\subsubsection*{\hyperref[exp92]{Pācittiya 92: Nandasikkhāpadaṁ}}
\label{pac92}

\linkdest{endnote-body}
\linkdest{endnote-body}
\linkdest{endnote-body}
\linkdest{endnote-body}
Yo pana bhikkhu sugatacīvarappamāṇaṁ\makeatletter\hyperlink{endnote-appendix}\Hy@raisedlink{\hypertarget{endnote-body}{}{\pagenote{%
		\hypertarget{endnote-appendix}{\hyperlink{endnote-body}{}}}}}\makeatother G: sugatacīvaram-pamāṇaṁ. cīvaraṁ kārāpeyya atirekaṁ vā, chedanakaṁ pācittiyaṁ. Tatr'idaṁ\makeatletter\hyperlink{endnote-appendix}\Hy@raisedlink{\hypertarget{endnote-body}{}{\pagenote{%
		\hypertarget{endnote-appendix}{\hyperlink{endnote-body}{}}}}}\makeatother V: tatrīdaṁ. sugatassa sugatacīvarappamāṇaṁ,\makeatletter\hyperlink{endnote-appendix}\Hy@raisedlink{\hypertarget{endnote-body}{}{\pagenote{%
		\hypertarget{endnote-appendix}{\hyperlink{endnote-body}{}}}}}\makeatother G: sugatacīvaram-pamāṇaṁ. dīghaso nava vidatthiyo sugatavidatthiyā, tiriyaṁ cha vidatthiyo. Idaṁ sugatassa sugatacīvarappamāṇaṁ.\makeatletter\hyperlink{endnote-appendix}\Hy@raisedlink{\hypertarget{endnote-body}{}{\pagenote{%
		\hypertarget{endnote-appendix}{\hyperlink{endnote-body}{}}}}}\makeatother C, W, Dm, Um, Mi Se v.l.:  pamāṇan-ti. D:  pamāṇaṁ ti. G (In a later faint correction.): sugatacīvaram-pamāṇan-ti. This
quotation mark ti here seems to be a remnant from the quotation of the rule in the Suttavibhaṅga. In the Suttavibhaṅga all
rules end in ti as they were spoken by the Buddha, while the Pātimokkha is recited by other monks.

\linkdest{endnote-body}
\begin{center}
	Rājavaggo\makeatletter\hyperlink{endnote-appendix}\Hy@raisedlink{\hypertarget{endnote-body}{}{\pagenote{%
		\hypertarget{endnote-appendix}{\hyperlink{endnote-body}{}}}}}\makeatother All editions, except Vibh Ce, have: ratanavagga. The Vibh Ce reading has been chosen here as it is found in the Parivāra,
Vin V 27; see the section on chapter-division in the Introduction. The Sikkhāpada-uddāna at the end of Bh Pm 1 and 2 (see
below) also has rājavagga in its summary of the Pācittiya section-titles. navamo
\end{center}



\medskip

\begin{center}
	Uddiṭṭhā kho āyasmanto dvenavuti pācittiyā dhammā.

	\smallskip

	Tatth'āyasmante pucchāmi: Kacci'ttha parisuddhā?\\
	Dutiyam'pi pucchāmi: Kacci'ttha parisuddhā?\\
	Tatiyam'pi pucchāmi: Kacci'ttha parisuddhā?

	\smallskip

\linkdest{endnote-body}
	Parisuddh'etth'āyasmanto, tasmā tuṇhī, evam'etaṁ dhārayāmi.\makeatletter\hyperlink{endnote-appendix}\Hy@raisedlink{\hypertarget{endnote-body}{}{\pagenote{%
		\hypertarget{endnote-appendix}{\hyperlink{endnote-body}{}}}}}\makeatother Dm, UP, Ra, Um: dhārayāmī ti.
\end{center}

\begin{outro}
	Dvenavuti pācittiyā dhammā niṭṭhitā
\end{outro}

\clearpage

